
\chapter{A Mapping from Role-Based Models to Context Description Logics}
\label{cha:mapping}

In the last chapter we introduced a family of description logics which are capable of expressing
contextual knowledge. 

 

How to use cDLs in practice?

Semantics, even when durchdacht und präzise, hard to comprehend.

Motivation for looking into cDLs were role-based systems.

automatic mapping from such a system desirable.

Formal representation of role-based system necessary.

Chosen the family of CROMs introduced by Kühn.

Also say what limitations occur, what cannot be represented in cDL

\todo[inline]{paragraph about related work -> formalizing UML diagrams}

\section{Introducing \texorpdfstring{$\Sigma$}{Σ}-CROM}
\label{sec:sigma-crom}


In this section we will present a syntactical variant of the \emph{Compartment Role Object Model}
introduced in \textbf{[Ku-SLE15]}\todo{citation}.

\begin{definition}[$\Sigma$-Compartment Role Object Model]
  Let \NTsf, \RTsf, \CTsf and \RSTsf be finite and mutually disjoint sets of Natural Types, Role
  Types, Compartment Types, and Relationship Types, respectively.  Let $\Sigma$ be the tuple
  $(\NTsf, \RTsf, \CTsf,\RSTsf)$.  A \emph{$\Sigma$-Compartment Role Object Model ($\Sigma$-CROM)}
  \Mmc is a tuple \MM where
  \begin{enumerate}
  \item $\Pmc$ is an arbitrary but fixed partition of \RTsf with $|\Pmc|=|\CTsf|$,
  % \item $\Pmc=(P_{1},\dots,P_{n}), n=|\CTsf|$ is an arbitrary but fixed partition of \RTsf,
  \item $\Smc\subseteq\bigcup_{P\in\Pmc}P\times P$ is a irreflexive binary relation with
    $|\Smc|=|\RSTsf|$,
  % \item \Smc is a irreflexive binary relation
  %   $\Smc=\{S_{1},\dots,S_{k}\}\subseteq\bigcup_{1\leq i\leq n}P_{i}\times P_{i}$, $k=|\RSTsf|$,
  \item $\fills\subseteq(\NTsf\cup\CTsf)\times\RTsf$ is a right-total binary relation,
  \item $\parts:\CTsf\to\Pmc$ is a bijection, and
  \item $\rel:\RSTsf\to\Smc$ is a bijection. \qedhere
  \end{enumerate}
\end{definition}

Furthermore, we say that \emph{$t$ fills \rt} if $(t,\rt)\in\fills$, \emph{\rt participates in \ct}
if $\rt\in\parts(\ct)$, \emph{$\rt_{1}$ and $\rt_{2}$ are related via \rst} if
$(\rt_{1},\rt_{2})=\rel(\rst)$, and \emph{$\rt_{1}$ is the domain of \rst ($\dom(\rst)$) and
  $\rt_{2}$ is the range of \rst ($\ran(\rst)$)} if $(\rt_{1},\rt_{2})\in\rel(\rst)$.

Note here, that reasoning about role-based models does not include checking \emph{well-formed\-ness}
as defined in Definition 1 of \textbf{[??]}, since that is a pure syntactical check. Therefore the
above definition already ensures that the CROM is well-formed.  Since \fills is a right-total
relation, for each role type there exists a natural type or compartment type that fills it.  As each
element of \Pmc is a non-empty subset of \RTsf, the image of \parts does not contain the empty set.
Since \Pmc is a partition each role type participates in only one compartment type.  Due to the
irreflexivity of the image of \rel there exists no $\rst \in \RSTsf$ and $\rt \in \RTsf$ such that
$\rel(\rst) = (\rt,\rt)$. Furthermore, since \Smc is a subset of $\bigcup_{P\in\Pmc}P\times P$, the
pair of role types in the image of \rel always participate in the same compartment type.





\subsection{Type Level}
\label{sec:type-level}




\subsection{Instance Level}
\label{sec:instance-level}




\subsection{Constraint Level}
\label{sec:constraint-level}









%%% Local Variables:
%%% mode: latex
%%% TeX-master: "../thesis"
%%% End:

%  LocalWords:  logics Kühn
