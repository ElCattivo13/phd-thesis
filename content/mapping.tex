
\chapter{A Mapping from Role-Based Models to Context Description Logics}
\label{cha:mapping}

In the last chapter we introduced a family of description logics which are capable of expressing
contextual knowledge. This provides the needed expressiveness to be able to reason on role-based
models. But due to the rather elaborate semantics both of role-based models and of contextualized
description logics it will be tedious and error-prone to construct the DL ontology which exactly
captures a role-based model. Hence, an automated mapping from role-based models into DL would be
desirable. Therefore a formal representation of role-based system with a well-defined semantics is
necessary.

In this chapter we will present a mapping from such role-based models into contextualized DLs in
order to automate this step.
%
To avoid the confusion with the notations, from now on we distinguish between \emph{\rosiroles} as
in \rosirole-based systems and \emph{roles} as used in description logics whenever we feel it is
necessary. Otherwise, we drop that distinction if it is clear from the context.
%
In Section~\ref{sec:sigma-crom} we introduce a syntactical variant of the \emph{Compartment Role
  Object Model (CROM)}~\cite{KBG-SLE15}.  Then, after discussing requirements for the logical
formalism and analyze possible candidates for the translation in
Section~\ref{sec:requirements-and-CDLs}, we present the actual mapping in
Section~\ref{sec:representating-role-based-models}. We close this chapter in
Section~\ref{sec:going-beyond-crom} with possible features of role-based systems that go beyond CROM
but still can be expressed in an ontology.



\section{A Formal Role-Based Modeling Language}
\label{sec:sigma-crom}


In this section we will present a syntactical variant of the \emph{Compartment Role Object Model}
defined in \cite{KBG-SLE15}. This variant is simply introduced for easier
explanation of the mapping to description logics.

\subsection{Type and Instance Level}
\label{sec:type-instance-level}

In CROM, we can model role-based systems via different kinds of concepts: \emph{natural types},
\emph{\rosirole types}, \emph{compartment types} and \emph{relationship types}. These differ in the
three ontological properties \emph{rigidity}, \emph{foundedness} and \emph{identity}.

\Rosirole types are rigid, non-founded and have
%
\todo{alles noch zu sehr abhängig von Section 2.2}




% Instances of these
% types are called \emph{naturals}, \emph{\rosiroles}, \emph{compartments} and \emph{relationships},
% respectively. Naturals denote non-founded and context-independent objects




\begin{definition}[Compartment Role Object Model]\label{def:scrom}
  Let \NT, \RT, \CT and \RST be finite and mutually disjoint sets of Natural Types, Role
  Types, Compartment Types, and Relationship Types, respectively.  The tuple
  $\Sigma = (\NT, \RT, \CT,\RST)$ is the \emph{vocabulary}.
  A \emph{Compartment Role Object Model \Mmc over $\Sigma$ ($\Sigma$-CROM)} is a tuple \MM where
  \begin{enumerate}
  \item $\fills\subseteq(\NT\cup\CT)\times\RT$ is a right-total binary relation,
  \item $\parts:\CT\to\Pmf$ is a bijection where \Pmf is an arbitrary but fixed partition of
    \RT, and
  \item $\rel:\RST\to\Smc$ is a bijection where $\Smc\subseteq\bigcup_{P\in\Pmf}P\times P$ is an
    irreflexive binary relation. \qedhere
  \end{enumerate}
\end{definition}

\noindent
In the above definition, \fills specifies which rigid type is allowed to play which role
type. Expressing in which compartment type a certain role type can be played is stated
in \parts. Finally, \rel defines the domain and range of relationship types.
%
In the rest of this thesis, we use the following phrases:
\begin{itemize}
\item \emph{$T$ fills \rt} if $(T,\rt)\in\fills$,
\item \emph{\rt participates in \ct} if $\rt\in\parts(\ct)$,
\item \emph{\rst participates in \ct} if $\rel(\rst) = (\rt[1],\rt[2])$ with $\{\rt[1], \rt[2]\}\subseteq\parts(\ct)$,
\item \emph{\rt[1] and \rt[2] are related via \rst} if $(\rt[1],\rt[2]) = \rel(\rst)$, and
\item \emph{\rt[1] is the domain of \rst ($\dom(\rst)$) and \rt[2] is the range of \rst
    ($\ran(\rst)$)} if $(\rt[1],\rt[2])\in\rel(\rst)$.
\end{itemize}
To show that our definition satisfies Equations~(1) to~(5) of~\cite{KBG-SLE15}, we exploit the
different assumptions made in the definition.  Since \fills is a right-total relation, for each role
type there must exist a natural type or compartment type that fills it.  As each element of \Pmf is
a non-empty subset of \RT, the range of \parts does not contain the empty set.  Since \Pmf is a
partition each role type participates in exactly one compartment type.  Due to the irreflexivity of
\Smc there exists no $\rst \in \RST$ and $\rt \in \RT$ such that $\rel(\rst) =
(\rt,\rt)$. Furthermore, since \Smc is a subset of $\bigcup_{P\in\Pmf}P\times P$, the pair of role
types in the range of \rel always participate in the same compartment type. Hence, a \SCROM is
always well-formed.

Note here, that reasoning about role-based models does not include checking \emph{well-formed\-ness}
as defined in Definition 1 of~\cite{KBG-SLE15}, since that is a pure syntactical check.  Next, we
introduce instances based on role-based models.


\begin{definition}[Compartment Role Object Instance, Satisfiability]\label{def:scroi}
  Let $\Sigma = (\NT, \RT,$ $\CT,\RST)$ be a vocabulary.  Then, a
  \emph{Compartment Role Object Instance \I over $\Sigma$ (\SCROI)} is a tuple
  $\I=(\Gamma^{\I},\type,\plays,\links)$, where
  \begin{itemize}
  \item $\Gamma^{\I}$ is a non-empty domain,
  \item $\type:\Gamma^{\I}\to\NT\cup\RT\cup\CT$ is a total function,
  \item $\plays \subseteq \Osf\times\Csf\times\Rsf$ is a ternary relation, and
  \item $\links: (\RST\times\Csf) \to \powerset{\Rsf\times\Rsf}$ is a total function
  \end{itemize}
  with 
  \begin{align*}
    \Nsf & \coloneqq \{d \in \Gamma^{\I} \mid \type(d) \in \NT\} \text{ (\emph{naturals})}, \\
    \Rsf & \coloneqq \{d \in \Gamma^{\I} \mid \type(d) \in \RT\} \text{ (\emph{roles})}, \\
    \Csf & \coloneqq \{d \in \Gamma^{\I} \mid \type(d) \in \CT\} \text{ (\emph{compartments}) and} \\
    \Osf & \coloneqq \Nsf \cup \Csf \text{ (\emph{objects})}.
  \end{align*}
  % being the sets of \emph{naturals}, \emph{roles}, \emph{compartments} and
  % \emph{objects}, respectively. 
  Furthermore, %
  the set $T^{\I}$ of all elements of type $T$, %
  the set \Osf[\I,c] of all objects playing a role in $c$, %
  the set \Osf[\I,c,\rt] of all objects playing an \rt-role in $c$, and %
  the set \Rsf[\I,c] of all roles played in $c$ %
  are defined as follows:
  \begin{align*}
    T^{\I}     & \coloneqq \{d \in \Gamma \mid \type(d) = T\},\\
    \Osf[\I,c] & \coloneqq \{o \in \Osf \mid \text{there is some $r$ with $(o,c,r) \in \plays$}\}, \text{ and}\\
    \Osf[\I,c,\rt] & \coloneqq \{o \in \Osf \mid \text{there is some $r$ with $(o,c,r) \in \plays$
                     and $r\in\rt^{\I}$}\}\\
    \Rsf[\I,c] & \coloneqq \{r \in \Rsf \mid \text{there is some $o$ with $(o,c,r) \in \plays$}\}.
  \end{align*}
  %
  A \SCROI{} \I \emph{satisfies} a \SCROM{} \Mmc, denoted by $\I\models\Mmc$, if it has the following
  properties:
  \begin{enumerate}
  \item The \plays-relation respects \fills, i.e.~for each tuple $(o,c,r)\in\plays$ the type of $o$
    fills the type of $r$:
    \begin{align*}
      \{(o,r)\mid(o,\cdot,r)\in\plays\}\ \subseteq\ \{(o,r)\mid\text{there exists
        $(T,\rt)\in\fills$ s.t. $o\in T^{\I}$, $r\in \rt^{\I}$}\}.
    \end{align*}
  \item The \plays-relation respects \parts, i.e.~for each tuple $(o,c,r)\in\plays$ the type of $r$
    participates in the type of $c$:
    \begin{align*}
      & \{(c,r)\mid(\cdot,c,r)\in\plays\}\\
      & \quad \subseteq\ \{(c,r)\mid \text{there exists $\ct \in \CT$, $\rt \in \RT$ s.t. $c \in
        \ct^{\I}$, $r \in \rt^{\I}$, $\rt \in \parts(\ct)$}\}.
    \end{align*}
  \item Each object can only play one role of each role type in each compartment:
    \begin{align*}
      \{(o,c,r),(o,c,r')\}\subseteq\plays \textImplies \type(r)\neq\type(r').
    \end{align*}
  \item Each role is played by exactly one object in exactly one compartment:
    \begin{align*}
      |\{(o,c)\mid(o,c,r)\in\plays\}| = 1 \quad\text{for all $r \in \Rsf$}.
    \end{align*}
  \item Roles occurring in the image of \links are played in the associated compartment, i.e.\ for
    each $(r_{1},r_{2})\in\links(\rst,c)$ there exists objects that play $r_{1}$ and $r_{2}$ in $c$:
    \begin{align*}
      \{r_{1} \mid (r_{1},\cdot) \in \links(\cdot,c)\} \cup \{r_{2} \mid (\cdot,r_{2}) \in \links(\cdot,c)\} \ \subseteq\ \{r\mid(\cdot,c,r)\in\plays\}.
    \end{align*}
  \item The \links-function respects \rel, i.e.\ for each $(r_{1}, r_{2}) \in \links(\rst, \cdot)$ the
    types of $r_{1}$ and $r_{2}$ are related via \rst:
    \begin{align*}
      (r_{1},r_{2}) \in \links(\rst, \cdot) \textImplies \rel(\rst)=(\type(r_{1}),\type(r_{2})).
    \end{align*}
  \end{enumerate}

  A \SCROM{} \Mmc is \emph{satisfiable} if there exists any \SCROI{} \I such that $\I\models\Mmc$. 
\end{definition}

We say that $r$ is an $\rt$-role and $c$ is a $\ct$-compartment if, respectively,
$\type(r) = \rt \in \RT$ and $\type(c) = \ct \in \CT$.  Furthermore, \emph{$o$ plays $r$ in $c$} and
$o$ is the \emph{player} of $r$ if $(o,c,r) \in \plays$, and \emph{$r_{1}$ is linked to $r_{2}$ via
  $\rst$ in $c$} if $(r_{1}, r_{2}) \in \links(\rst, c)$.

Again, it is easy to verify that our definition of a \SCROI satisfying a \SCROM is equivalent to the
definition of~\cite{KBG-SLE15}.
Since the \plays-relation respects the \fills- and the \parts-relation, if $o$ plays an \rt-role in
a \ct-compartment, then the type of $o$ must fill \rt and \rt must participate in \ct. If $o$ plays
$r_{1}$ and $r_{2}$ in $c$ then $r_{1}$ and $r_{2}$ must have different types, because each object
can only play one role of each type in the same compartment. Moreover, for each role there exists
exactly one object that plays it and one compartment where it is played in. In contrast to the
formalisation in~\cite{KBG-SLE15} we omit $\varepsilon$-roles, but roles that are linked via a
relationship type are both played in the same compartment and must be instances of the correct types
since \links respects \rel.  Thus, equations (6) to (11) of~\cite{KBG-SLE15} are satisfied.

Before we investigate how the information about a \SCROM can be encoded in a description logic
ontology, we have to discuss the main reasoning tasks for role-based models. The arguably most
apparent question is, given a \SCROM{} \Mmc and a \SCROI{} \I, whether \I is compliant with
\Mmc. But as this task rather belongs to the area of model checking, we will not focus in this
thesis on that problem.
%
Instead, given a \SCROM{} \Mmc, it is much more interesting whether there exists any \SCROI that is
compliant with \Mmc.  Additionally we often want to know for a specific \SCROM{} \Mmc whether there
exists a compliant \SCROI that fulfills certain assertions, e.g.\ that a role of a certain type is
played.  To express these assertional knowledge, we introduce a so-called \SCROA, a set of
assertions which should also be satisfied by an \SCROI.

\begin{definition}[$\Sigma$-Compartment Role Object Assertions] \label{def:scroa} Let
  $\Sigma = (\NT, \RT, \CT, \RST)$ be a vocabulary and let \INDC and \IND be two non-empty, disjoint
  sets of meta and object individual names disjoint from $\Sigma$.  A \emph{Compartment Role Object
    Assertion over $\Sigma$} is of the form
  \begin{itemize}
  \item $T(c)$ with $T\in\CT$ and $c\in\INDC$ (\emph{meta type assertion}),
  \item $T(a,c)$ with $T \in \NT \cup \CT \cup \RT$, $a\in\IND$ and $c\in\INDC$ (\emph{object type assertion}),
  \item $\playass(a_{1}, c, a_{2})$ with $a_{1}, a_{2} \in \IND$ and $c\in\INDC$ (\emph{plays assertion}), or
  \item $\linkass(\rst, c, a_{1}, a_{2})$ with $\rst \in \RST$ and  $a_{1}, a_{2} \in \IND$ and
    $c\in\INDC$ (\emph{link assertion}). 
  \end{itemize}

  A \emph{set of Compartment Role Object Assertions \A over $\Sigma$ (\SCROA)} is a set of such
  assertions.
  %
  We extend the \SCROI{} \I to additionally map individual names to domain elements, e.g.\
  $a\in\IND$ and $c\in\INDC$ are
  mapped to a domain elements $a^{\I} \in \Gamma^{\I}$ and $c^{\I} \in \Gamma^{\I}$. A \SCROI{} \I \emph{satisfies an assertion
    $\alpha$}, denoted by $\I\models\alpha$, if
  \begin{itemize}
  \item for $\alpha$ of the form $T(c)$: $c^{\I} \in T^{\I}$,
  \item for $\alpha$ of the form $T(a, c)$: $a^{\I} \in T^{\I} \cap (\Osf\cup\Rsf[\I,c])$,
  \item for $\alpha$ of the form $\playass(a_{1}, c, a_{2}, )$:
    $(a_{1}^{\I}, c^{\I}, a_{2}^{\I}) \in \plays$, and
  \item for $\alpha$ of the form $\linkass(\rst, c, a_{1}, a_{2})$: there exist $r_{1}, r_{2}$ with
    $(a_{1}, c, r_{1})\in\plays$, $(a_{2}, c, r_{2})\in\plays$, and
    $(r_{1}^{\I}, r_{2}^{\I}) \in \links(\rst, c^{\I})$.
  \end{itemize}

  A \SCROI{} \I \emph{satisfies \A}, denoted by $\I\models\A$ if it satisfies all assertions
  in \A.
\end{definition}

Note here that the link assertion asserts for two objects that they play roles which are related via
\rst, and not that the objects themselves are related.
%
Moreover, without any assertions or further restrictions as introduced in
Section~\ref{sec:constraint-level} there always exists a trivial CROI with the singleton set
$\Gamma = \{o\}$, the type of $o$ being some natural type, and \plays and \links being empty sets
that satisfies \Mmc.


\subsection{Constraint Level}
\label{sec:constraint-level}

When modelling a domain of interest, not only the type of an object defines whether that object is
allowed to play a certain role. These additional constraints can be divided into four groups.

The first category of constraints are \emph{role constraints}, e.g. roles that mutually exclude each
other or playing one role implies playing another role. More general these constraints are
formalized with  so-called \emph{role groups}. These consist of a set of role
types (or again role groups) and a lower and an upper bound. An object fulfills a role group if it
plays at least the lower and at most the upper bound of roles from the set of role types.

\begin{definition}[Syntax of Role Groups]\label{def:syntax-role-groups}
  Let \RT be a set of role types. The set of \emph{role groups over \RT} is the smallest such that
  \begin{itemize}
  \item if $\rt \in \RT$, then \rt is an (\emph{atomic}) role group, and
  \item if $A_{1}$, \dots, $A_{n}$ are role groups, $k,\ell \in \nat$, then $(\{A_{1}, \dots, A_{n}\},
    k,\ell)$ is a (\emph{complex}) role group.
  \end{itemize}
  \emph{Atoms} of a role group $A$ are defined as:
  \begin{align*}
    \atom(A) & \coloneqq
               \begin{cases}
                 \{\rt\} & \text{if $A = \rt \in \RT$}\\
                 \bigcup_{i=1}^{n} \atom(A_{i}) & \text{if $A = (\{A_{1}, \dots, A_{n}\},k,\ell)$}.
               \end{cases}
  \end{align*}
  Role groups that occur within other role groups are called \emph{nested}.
\end{definition}

The semantics of a role group are based on a \SCROI and are locally evaluated for each domain
element and each compartment.  The interpretation function $\cdot^{\I,c,o}$ calculates recursively
whether an object fulfills the role group.

\begin{definition}[Semantics of Role Groups]\label{def:semantics-role-groups}
  Given a \SCROI{} \I, the semantics of a role group $A$ is defined for an object $o \in \Osf$ in
  $c \in \Csf$ as follows:
  \begin{align*}
    A^{\I,c,o} & \coloneqq 
                 \begin{cases}
                   1 & \text{if $A = \rt\in\RT$ and $o$ plays an \rt-role in $c$,} \\
                   1 & \text{if $A = (\{B_{1},\dots,B_{n}\},k,\ell)$ and $k \leq \sum_{i=1}^{k}
                     B_{i}^{\I,c,o} \leq \ell$, and}\\
                   0 & \text{otherwise.}
                 \end{cases}
  \end{align*}

  If $A^{\I,c,o} = 1$, we say that $o$ fulfills $A$ in $c$.
\end{definition}

\noindent
Basic role constraints, for example as defined in~\cite{RiGr-OOPLSLA98}, i.e.\ role implication,
role equivalence and role prohibition, can be expressed with role groups as well as much more
complex ones. In fact, any propositional formula can be emulated with role groups.

\begin{proposition}
  Let $\varphi$ be some propositional formula. Then, there exists a role group $A_{\varphi}$ such
  that $\varphi$ is satisfiable if and only if $A_{\varphi}$ can be fulfilled.
\end{proposition}
\begin{proof}
  We define $A_{\varphi}$ inductively as follows:
  
  \vspace{\topsep}
  \begin{tabular}{@{ if }l@{\quad then\quad }l}
    $\varphi = p$ & $A_{\varphi} \coloneqq \rt[p]$,\\
    $\varphi = \lnot \psi$ & $A_{\varphi} \coloneqq (\{A_{\psi}\}, 0, 0)$,\\
    $\varphi = \psi_{1}\land\psi_{2}$ & $A_{\varphi} \coloneqq (\{A_{\psi_{1}}, A_{\psi_{2}}\}, 2, 2)$, and\\
    $\varphi = \psi_{1}\lor\psi_{2}$ & $A_{\varphi} \coloneqq (\{A_{\psi_{1}}, A_{\psi_{2}}\}, 1, 2)$.
  \end{tabular}
  \vspace{\topsep}
  
  \noindent
  Next, we establish a 1-to-1-relation between a valuation $\rho$ for $\varphi$ and a
  $\Sigma$-CROI{} \I. For every propositional variable $P_{i}$ occurring in $\varphi$, we introduce
  a role type \rt[i] and assume that $o$ plays an \rt[i]-role iff $\rho(P_{i}) = \true$. By
  induction, it follows that $\rho(\varphi) = \true$ iff $o$ fulfills $A_{\varphi}$.
\end{proof}


The next category of constraints are \emph{occurrence constraints}. These state how often a role
type or role group must at least or at most be played in a compartment. Therefore, we introduce the
notion of a \emph{cardinality}, a pair $(k,\ell)\in\nat\times\natinf$ with $k\leq\ell$. We usually
denote cardinalities by $(k..\ell)$. Since role types are also atomic role groups, it suffices to
specify occurrence constraints for role groups.
%
Similar to multiplicities specified for associations in UML class diagrams, we specify
\emph{cardinal constraints} for relationship types. They express how often a role of certain type
must be related via a relationship type to some other role type.

Last but not least, the category of \emph{intra-relationship type constraints} imposes constraints on the
players of roles which are related via a relationship type. For example, stating that the
relationship type $\mathsf{isAncestorOf}$ between the role types $\mathsf{Parent}$ and
$\mathsf{Child}$ is transitive assures the existence of an respective link between a grandparent and
a grandchild. Note here, that the transitivity is evaluated over the players and not the roles
themselves.

\begin{definition}[Constraint Set]
  Let $\Sigma = (\NT, \RT, \CT, \RST)$ be a vocabulary, let \RG be the set of role groups over \RT
  and let $\Card \coloneqq \nat \times \natinf$ be the set of cardinalities.  Then, a
  \emph{$\Sigma$-Compartment Role Object Constraint Set (\SCROC) \Cmc} is a tuple $\CC$ where
  \begin{itemize}[topsep=5pt]
  \item $\occur : \CT \to \powerset{\Card \times \RG}$,
  \item $\card : \RST \to \Card \times \Card$, and
  \item $\intra : \RST \to \powerset{\Emc}$ with \Emc being a set of functions of the form
    $e : \powerset{A\times B}\to \{\true, \false\}$ for arbitrary sets $A$, $B$.
  \end{itemize}
  are total functions.  The set of all non-nested role groups that appear in \occur is the set of
  \emph{top-level role groups \RG*}.
  %
  A \SCROC{} \Cmc is \emph{compliant} to a \SCROM, if all atoms of a role group that is in the occurrence
  constraints of a compartment type participate in that compartment type.
  %
  A \SCROI{} \I \emph{satisfies \Cmc}, denoted by $\I\models\Cmc$, if it has the following
  properties:
  \begin{enumerate}

  \item All occurrence constraints are respected, i.e.\ if $(k..\ell,A) \in \occur(\ct)$, then in every
    \ct-compartment there must exist at least $k$ and at most $\ell$ objects that fulfill role group
    $A$:
    \begin{align*}
      ((k,\ell),A) \in \occur(\ct) \textImplies  \ct^{\I} \subseteq \left\{c \in \Csf \mmid k \leq \sum\nolimits_{o\in\Osf[\I,c]}
      A^{\I,c,o} \leq \ell \right\}
    \end{align*}
  \item All top-level role groups must be satisfied, i.e.\ if an object $o$ plays an \rt-role and
    \rt is an atom of a top-level role group A, then $o$ must fulfill $A$:
    \begin{align*}
      & (o,c,r) \in \plays \text{, } r \in \rt^{\I} \text{ and } \rt \in \atom(A) \textImplies
      A^{\I,c,o} = 1\text{,} \\
      & \hspace{8cm} \text{for all $o \in \Osf$, $A \in \RG*$}.
    \end{align*}
  \item All cardinality constraints are respected, i.e.\ every role that is played in a compartment
    $c$ and whose type is either the domain or the range of a relationship type \rst with
    $\card(\rst) = (i..j, k..\ell)$ must have at least $k$ and at most $\ell$ \rst-successors in $c$ or at
    least $i$ and at most $j$ \rst-predecessors in $c$, respectively:
    \begin{align*}
      & (\cdot,c,r) \in \plays \text{, } r \in \rt[1]^{\I} \text{, } \rel(\rst) = (\rt[1], \cdot) \text{ and
        } \card(\rst)=(\cdot,k..\ell)\\
      & \qquad \textImplies k \leq | \Succ(\rst,r,c) | \leq \ell
    \end{align*}
    \begin{align*}
      & (\cdot,c,r) \in \plays \text{, } r \in \rt[2]^{\I} \text{, } \rel(\rst) = (\cdot, \rt[2]) \text{ and
        } \card(\rst)=(i..j,\cdot)\\
      & \qquad \textImplies i \leq | \Pred(\rst,r,c) | \leq j
    \end{align*}

  \item All intra-relationship type constraints are respected, i.e.\ every function $\fsf \in
    \intra(\rst)$, evaluated over the players of the roles related via \rst, must return \true:
    \begin{align*}
      & \fsf \in \intra(\rst) \textImplies \fsf(\links^{*}(\rst,c)) = \true, \\
      & \qquad\text{for all $c$ in $\ct^{\I}$ s.t.\ \rst participates in \ct}.
    \end{align*}
  \end{enumerate}

  The auxiliary functions used above are defined as follows:
  \begin{align*}
    \Succ(\rst,r,c) & \coloneqq \{r' \in \Rsf \mid (r,r') \in \links(\rst,c)\}, \\
    \Pred(\rst,r,c) & \coloneqq \{r' \in \Rsf \mid (r',r) \in \links(\rst,c)\}\text{, and} \\
    \links^{*}(\rst,c) & \coloneqq \{ (o_{1}, o_{2}) \mid (r_{1}, r_{2}) \in \links(\rst,c) \text{
                         and } (o_{1}, c, r_{1}), (o_{2}, c, r_{2}) \in \plays\}.
                         \qedhere
  \end{align*}

\end{definition}

The definition of satisfying a constraint model in~\cite{KBG-SLE15} is, neglecting
$\varepsilon$-roles, exactly reflected in the above definition.

At last we combine all parts of the role-based system into one \emph{constrained
  $\Sigma$-compartment Object Role Model}

\begin{definition}[Constrained $\Sigma$-Compartment Role Object Model, Satisfiability Problem]
  \label{def:constrained-sigma-crom}
  Let $\Sigma = (\NT, \RT, \CT, \RST)$ be a vocabulary, let \Mmc be a \SCROM, let \Amc be a \SCROA
  and let \Cmc be a \SCROC.
  %
  Then, a \emph{Constrained $\Sigma$-Compartment Role Object Model \SCCROM} is the tuple
  $\Kmc = (\Mmc, \Amc, \Cmc)$.

  The \emph{satisfiability problem} for \SCCROM{s} is the problem of deciding for a given \SCCROM{}
  $\Kmc = (\Mmc, \Amc, \Cmc)$ whether there exists a \SCROI that satisfies \Mmc, \Amc and \Cmc.
\end{definition}



\section{Requirements for Logical Formalism}
\label{sec:requirements-and-CDLs}

Basically every large software project starts with modelling the application domain in UML. While
UML is the de facto standard as modelling language, it has several drawbacks. First of all, it has
no formally defined semantics~\cite{FrEL-CSI98}. While UML meta-models capture the precise syntax of
concepts used for modelling, they do little for answering questions of how to interpret non-trivial
UML diagrams.  For example, if one wants to map UML class diagrams to a logical formalism to be able
to reason about these diagrams, one takes certain assumptions about the intended meaning of the
occurring elements. Then a domain expert modelling some application has to trust that the logician
constructed the mapping with the same semantics in mind. Nevertheless, over the last years several
approaches for formal frameworks to reason on UML
arose~\cite{Eva-WIFT98,CaCG-ISMIS02,StMS-UML03,SiSJ-OBJ04,BeCG-AI05,SiBH-IJSEKE08,AhNa-ICET10}, from
which we can adapt some ideas, e.g. to model attributes of a UML class with DL roles and how
multiplicities of associations are modelled in~\cite{CaCG-ISMIS02}.

Besides that, UML lacks expressive power to model context-dependent domains. There also exist some
work on extending UML in that direction~\cite{ShB-ICMB05}, but here the semantics are even more ambiguous.
Therefore, we focus on role-based models that are based on CROM since it has both a well-defined,
formal semantics and the means to formulate the necessary concepts accordingly.

The only drawback might be the expressive restrictions of ``classical'' description logics as
explained in the introduction. Standard DLs cannot formalize contextual knowledge in a proper way
which is crucial for role-based systems.  In the recent years many different approaches and
extensions of DLs have been proposed~\cite{BoGH-ISWC03, BoGH-WS04, BeAF-ADVIS06, BaVS-ModOnt09, BaKP-JWS12,
  CePe-IJCAR14, CePe-JAR17}.  However, many were tailored to different goals, for example to support
context-specific reuse of ontologies or enable probabilistic reasoning. Additionally, these
approaches have a quite different intuition of what a context exactly is. In most cases it is simply
a finite set of names.  Serafini et al.~\cite{SeHo-JWS12} defines contexts by a set of
attribute-value declarations, one for each \emph{dimension}, e.g.\ time, topic or location to name a
few. There is only one context for each dimensional vector whereas there can exist many compartments
of one compartment type.  Furthermore, except a coverage relation, there is hardly any other
knowledge about the contexts expressible, e.g. some relational structure between them.  Hence, these
approaches seem not appropriate to model CROMs.

A far more promising formalism is the earlier mentioned description logic of context
\klarALC~\cite{KG-JELIA10, KG16}. Klarman et al.\ follow the ideas of McCarthy's formalization of
contexts~\cite{McC-ACM87,McC-IJCAI93}, i.e.\ that contexts are formal objects, that they have
properties and can be described, and that they are organized in a relational structure.  This
results in a very expressive, two-dimensional description logic with strong interactions between
object and context level allowing to express information within a context that is valid in some
other context.  While transcending object knowledge through contexts is very important in
the general application of contextualized knowledge, it is complexity-wise very costly. Especially
in the presence of rigid roles, the consistency problem becomes undecidable.

By restricting that interaction to a top-down view of contexts, we can retain decidability. From the
meta or context level, we can impose axioms that must hold within a certain context, but on object
level, i.e.\ within a context, we cannot ``look outside''.
%
When formalizing compartments, this is exactly the needed expressiveness. For each compartment type
we want to specify the constraints which must hold within a compartment of that type.
%
This analysis led to the construction of \LMLO.
In Section~\ref{sec:adding-cont-concepts}, we showed that \ALCALC is indeed a sublogic of \klarALC.
Now, we will present how a CROM can be represented in \LMLO.


\section{Representing Role-Based Models}
\label{sec:representating-role-based-models}

We would like to emphasize here once again that checking well-formedness of a \SCROM{}~\Mmc and
compliance of a constraint set with \Mmc is not considered here, because these are purely
syntactical checks, no reasoning is necessary. Furthermore, also checking whether a given \SCROI is
valid is not the considered task. Much more interesting is whether there exists such a \SCROI at
all. Additionally, we can test if certain axioms are entailed or whether specific role types can be
played. Thus, the main objective is, given a \SCCROM{} \Kmc, to construct an \LMLO-ontology
$\Bmf_{\Kmc}$ such that $\Bmf_{\Kmc}$ is consistent iff \Kmc is satisfiable.

The general idea is to model compartment types as concepts on meta level and an object playing a
\rosirole, the relationship types as well as all the constraints within a compartment type on object
level. Here, we introduce o-concepts for natural types and \rosirole types and a special object role
\plays. The fills relation is transformed into corresponding domain and range axioms for \plays.

Here, we made a first design decision on how to express playing a \rosirole within a
compartment. There are two possibilities, depicted in Figure~\ref{fig:two-ways-to-play-roles}, which
we actually already showed in Example~\ref{ex:nfl-with-contexts}. On the one hand, we can introduce
an o-concept \rt and elements playing an \rt-\rosirole are in the extension of \rt (like Mike
McCarthy playing the \rosirole of a coach). Here, an object $o$ and a \rosirole $r$ with
$(o,\cdot,r)\in\plays$ would be mapped to a single element $d$ in the object interpretation for that
compartment. On the other hand, we can introduce the o-concept \rt as well, but additionally an
object role \plays and elements playing an \rt-\rosirole have a \plays-successor that is in the extension
of \rt (like Aaron Rodgers playing the \rosirole of a quarterback).  This is closer to the semantics
of \SCROI{s} but introduces new object domain elements for each \rosirole that is played.  Still, we
chose the latter variant since later on we need to count the number of \rosiroles to assure
occurrence and cardinality constraints. In DLs this can be done via qualified number restrictions.

\begin{figure}
  \centering
  \begin{tikzpicture}
    \node[node,label={[align=left]270:\textit{MikeMcCarthy},\\ $\mathsf{Coach}$}] (mmc) at (-2,0){};
    \node[node,label={[align=left]270:\textit{AaronRodgers}}] (ar) at (2,0){};
    \node[node,label={[align=right]south:$\mathsf{Quarterback}$}] (qb) at (5,0){};
    \draw[edge] (ar) to[bend left=10] node{\plays} (qb);
    \draw[thin] (0,-1.5) to (0,1);
  \end{tikzpicture}
  \caption{Possibilities to formalize objects playing \rosiroles.}
  \label{fig:two-ways-to-play-roles}
\end{figure}

Generally, in a \SCROM compartments are also allowed to play \rosiroles in other compartments. Within
these other compartments a compartment playing a \rosirole does not behave differently than a
natural. Hence, we have compartment types both as meta concept names, i.e.\ as contexts in which
other objects play \rosiroles, and as object concept types, i.e.\ as objects that play \rosiroles in a
context. In a \SCROM there is a one-to-one relationship since the compartment as \rosirole-player and as
context is the same object. In our formalism we cannot establish that tight connection, but we can
assure the existence of contexts of a certain compartment type via a meta role \nested if objects of
that type play \rosiroles. Since we cannot restrict the number of existing contexts in a nested DL
interpretation, this is sufficient for the satisfiability problem. To distinguish between
compartment types as contexts and as objects, we call the former simply \emph{contexts} and the latter
\emph{o-compartments} and introduce a copy \CT* of all compartment types \CT.

Relationship types are intuitively modelled as object roles. Here, it might be more natural to span these
between the played \rosiroles instead of the players. But due to the one-to-one correspondence
between players and played \rosiroles, we can also construct the relationship types between the
players. In doing so, we can avoid the use of role value maps, which would cause the consistency
problem to become undecidable \cite{Sch89}, to formalize intra relationship type constraints. Even so, we
can only support such constraints that are expressible in the underlying DL.

\Rosirole groups are handled like \rosiroles with an additional axiom stating that ``playing'' a
\rosirole group is equivalent to fulfilling the constraints specified in that \rosirole
group. Furthermore, if an object plays an atom of a top-level \rosirole group, that object must
fulfill the \rosirole group.
%
For the occurrence constraints we introduce a fresh individual name \occurCounter and a object role
\counts and enforce that each played role or fulfilled role group is connected to this
counter. Thereby we can use qualified number restrictions to encode the occurrence constraints.
%
For cardinal constraints we also utilize number restrictions.

% Ontologically, natural types would be captured as rigid concepts and their attributes, if present, as
% rigid roles, since that information does not change contexts-dependently. In our setting, rigidity
% has no influence on the consistency, and neglecting it decreases the computational complexity
% exponentially in the size of the input. Additionally, the case with rigid names can be handled quite
% similar and we come back to it in Section~\ref{sec:going-beyond-crom}.

To sum up, we consider the object signature $\Osig=(\OC,\OR,\OI)$ and the meta signature
$\Msig=(\MC,\MR,\MI)$ such that
\begin{itemize}
\item $\CT \subseteq \MC$ since every compartment type is a meta concept,
\item $\nested \in \MR$ to assure the existence of compartments that play \rosiroles,
\item $\NT\cup\CT* \subseteq \OCR$ since every natural type and every o-compartment type are rigid
  object concepts,
\item $\RT\subseteq\OCF$ since every \rosirole type is a flexible object concept
\item $\plays\in\ORF$ to express the \plays-relation, 
\item $\RST \subseteq \ORF$ since every relationship type is an object role,
\item $\occurCounter \in \OI$ and $\counts\in\ORF$ to express the occurrence constraints, and
\item $\INDC\in\MI$ and $\IND\in\OI$ to interpret individual names on their respective level. 
\end{itemize}
Additionally, we introduce the rigid object concept $\Ant \in \OCR$ for all objects eligible of
playing \rosiroles, i.e.\ naturals and o-compartments, and the flexible object concepts
$\Art\in\OCF$ for all \rosiroles and $\Arg\in\OCF$ for all instances of \rosirole groups since we
will consider \rosirole groups of the constraint set similar to \rosiroles.

\subsection{Knowledge about the vocabulary  \texorpdfstring{\ensureboldmath{\Sigma}}{Sigma}}

At first, we express general knowledge about occurring types which is independent of the specific
\SCROM.
%
Every context belongs to exactly one compartment type.
\begin{align}
  \top & \sqsubseteq \bigsqcup_{\ct\in\CT} \ct \label{eq:compartments-coverage}\\
  \ct[1] \sqcap \ct[2] & \sqsubseteq \bot 
      \text{\qquad\qquad for all $\ct[1], \ct[2] \in \CT$, $\ct[1]\neq\ct[2]$}\label{eq:compartments-disjoint}
\end{align}

\noindent
In every context, every natural or o-compartment and every \rosirole belongs to exactly one type.
\begin{align}
  \top & \sqsubseteq \oax{ \Ant \equiv \bigsqcup_{\nt\in\NT}\nt \sqcup \bigsqcup_{\ct*\in\CT*}\ct*}\label{eq:naturals-coverage}\\
  \top & \sqsubseteq \oax{ \Art \equiv \bigsqcup\limits_{\rt\in\RT}\rt} \label{eq:roles-coverage}\\
  \top & \sqsubseteq \bigsqcap_{\substack{T_{1},\,T_{2}\in\NT\cup\CT*\cup\RT,\\ T_{1} \neq T_{2}}}
      \oax{T_{1} \sqcap T_{2} \sqsubseteq \bot} \label{eq:roles-disjoint}
\end{align}

\noindent
On object level, an element can either be a \rosirole, a natural or o-compartment, an
instance of a \rosirole group or the individual \occurCounter.
\begin{align}
  %\top & \sqsubseteq \oax{\top \sqsubseteq \Ant \sqcup \Art \sqcup \Arg \sqcup \{\occurCounter\}}\\
  \begin{split}
  \top & \sqsubseteq \oax{\Ant \sqcap \Art \sqsubseteq \bot} 
      \sqcap \oax{\Ant \sqcap \Arg \sqsubseteq \bot}
      \sqcap \oax{\Art \sqcap \Arg \sqsubseteq \bot} \\
    & \qquad \sqcap \oax{\lnot(\Ant \sqcup \Art \sqcup \Arg)(\occurCounter)}
  \end{split}\label{eq:obj-disjoint}
\end{align}

\noindent
Next, every natural or o-compartment can only play one \rt-\rosirole in each context and
each \rosirole must be played by some object.
\begin{align}
  \top & \sqsubseteq \bigsqcap_{\rt\in\RT} \oax{\Ant \sqsubseteq \atmost{1}{\plays}{\rt}}\label{eq:roletype-played-max-once} \\
  \top & \sqsubseteq \oax{\Art \sqsubseteq \atleast{1}{\plays^{-}}{\top} 
         \sqcap \atmost{1}{\plays^{-}}{\top}}\label{eq:role-is-played-exactly-once}
\end{align}

\noindent
We formalize a general domain and range restriction for \plays. Only naturals or o-compart\-ments
can play something, and only \rosiroles or instances of \rosirole groups can be played.
\begin{align}
  \top & \sqsubseteq \oax{ \exists\plays.\top \sqsubseteq \Ant}\label{eq:plays-domain} \\
  \top & \sqsubseteq \oax{\top \sqsubseteq \forall\plays.(\Art\sqcup\Arg)}\label{eq:plays-range}
\end{align}

\noindent
Finally, if an o-compartment plays a \rosirole in some context, the o-compartment must also exist as
context.
\begin{align}
  \lnot\oax{\ct* \sqcap \exists\plays.\top\sqsubseteq\bot} & \sqsubseteq \exists\nested.\ct
      \text{\hspace{3em} for all $\ct* \in \CT*$}\label{eq:nested-cts}
\end{align}

\subsection{Knowledge about the \texorpdfstring{\SCROM{} \Mmc}{Sigma-CROM M}}

With the general knowledge about the vocabulary $\Sigma$ being set up, we can look into
specific knowledge for a given \SCROM{} \MM.

The \fills-relation restricts which natural or compartment types are allowed to play which \rosirole
types. Hence, elements that play a certain \rosirole type can only be naturals or o-compartments of
types which fill that \rosirole type.
\begin{align}
  \top & \sqsubseteq \bigsqcap_{\rt\in\RT}\oax{\exists\plays.\rt \sqsubseteq 
      \bigg( \bigsqcup\nolimits_{(T,\rt)\in\fills} T \bigg)} \label{eq:fills}
\end{align}
Note here, that Equation~\eqref{eq:fills} is sufficient in the sense that in conjunction with
Equations~\eqref{eq:roles-coverage} and~\eqref{eq:plays-range} it entails that all \plays-successors
of naturals or o-compartments of a specific type are either instances of a \rosirole type that are
filled by that type or instances of a \rosirole group.
\begin{align*}
  \top & \sqsubseteq \bigsqcap_{T\in\NT\cup\CT*}\oax{T \sqsubseteq \forall\plays.
      \bigg(\Arg \sqcup \bigsqcup\nolimits_{(T,\rt)\in\fills} \rt \bigg)}.
\end{align*}

\noindent Since in a valid \SCROI the \plays-relation respects \parts, only \rt-\rosiroles with
$\rt\in\parts(\ct)$ exist in a \ct-context.
\begin{align}
  \ct & \sqsubseteq \oax{\Art \sqsubseteq \bigsqcup_{\rt\in\parts(\ct)}\rt}
      \text{\hspace{3em} for all $\ct\in\CT$} \label{eq:roles-in-compartment}
\end{align}

\noindent Analogous to \fills restricting the domain and range of \plays, the \rel-function
restricts them for each relationship type.
\begin{align}
  \begin{split}
    \top & \sqsubseteq \oax{\exists\rst.\top\sqsubseteq\exists\plays.\rt[1]} \sqcap
         \oax{\top\sqsubseteq\forall\rst.(\exists\plays.\rt[2])}\\
       & \hspace{7em}\text{for all $\rst\in\RST$ and $\rel(\rst) = (\rt[1],\rt[2])$}
  \end{split}\label{eq:rst-domain-range}
\end{align}

\noindent Note here, that due to equations~\eqref{eq:compartments-coverage},
\eqref{eq:roles-coverage},~\eqref{eq:roles-disjoint} and~\eqref{eq:roles-in-compartment} and the
fact that \parts' codomain is a partition of \RT, in any context that is not in \ct there are no
\rosiroles of a type the participates in \ct , e.g. the following axiom is entailed for all
$\ct\in\CT$:
\begin{align*}
  \lnot\ct & \sqsubseteq \oax{\bigsqcup_{\rt\in\parts(\ct)}\rt\sqsubseteq\bot}
\end{align*}

\subsection{Knowledge about the \texorpdfstring{\SCROA{} \A}{Sigma-CROA A}}

Next, let \A be a \SCROA. Compartment role object assertions can be formalized as meta
concept assertions.
For any meta type assertion of the form $\ct(c)\in\A$ with $\ct\in\CT$ and $c\in\INDC$ the individual $a$ must be a context of
type \ct.
\begin{gather}
  \ct(c)\label{eq:meta-type-assertion}
\end{gather}

\noindent
For any object type assertion $T(a,c)\in\A$ with $T\in\NT\cup\CT\cup\RT$, $a\in\IND$ and $c\in\INDC$ the individual
$a$ is a natural, an o-compartment or \rosirole
  that belongs to the concept $T$ in context $c$.
\begin{gather}
  \oax{T(a)}(c)\label{eq:obj-type-assertion}
\end{gather}

\noindent
A \plays assertion $\playass(a_{1}, c, a_{2}) \in \A$ states that $a_{1}$ plays $a_{2}$
  in $c$.
\begin{gather}
  \oax{\plays(a_{1},a_{2})}(c)\label{eq:plays-assertion}
\end{gather}

\noindent
A \links assertion $\linkass(\rst, c, a_{1}, a_{2}) \in \A$ states that $a_{1}$ and $a_{2}$ play
\rosiroles which are related in $c$ via \rst. Due to the way we modelled relationship types, this is
simply the following axiom: 
\begin{gather}
  \oax{\rst(a_{1}, a_{2})}(c)\label{eq:links-assertion} 
\end{gather}
% This is slightly more complicated as we mapped the
%   relationship types between the players:
% \begin{gather}
%   (\oax{\rst(o_{1}, o_{2})}\sqcap\oax{\plays(o_{1}, a_{1})}\sqcap\oax{\plays(o_{2},
%     a_{2})})(c)\label{eq:links-assertion} 
% \end{gather}
%where w.l.o.g.\ $o_{1}, o_{2}\in\OI\setminus\IND$ are fresh individual names.

\subsection{Knowledge about the \texorpdfstring{\SCROC{} \Cmc}{Sigma-CROC C}}

Let \Cmc be a \SCROC, let $\RG(\Cmc)$ denote the set of all complex \rosirole groups occurring in
\Cmc and let $\RG*(\Cmc) \subseteq \RG(\Cmc)$ denote the set of all complex top-level \rosirole
groups. Analogous to \rosiroles, \rosirole groups are disjoint, every instance of a \rosirole group
must be played by some object and every object can either fulfill or not fulfill a \rosirole group.
\begin{align}
  \top & \sqsubseteq \oax{ \Arg \equiv \bigsqcup_{\rg\in\RG(\Cmc)}\rg} 
      \sqcap \bigsqcap_{\substack{\rg[1],\,\rg[2]\in\RG(\Cmc),\\ \rg[1]\neq\rg[2]}} 
      \oax{\rg[1] \sqcap \rg[2] \sqsubseteq \bot}
      \label{eq:rolegroups-coverage-and-disjoint}\\
  \top & \sqsubseteq \oax{\Arg \sqsubseteq \atleast{1}{\plays^{-}}{\top} 
      \sqcap \atmost{1}{\plays^{-}}{\top}}\label{eq:rgs-exactly-one-inverse-plays} \\
  \top & \sqsubseteq \bigsqcap_{\rg\in\RG(\Cmc)} \oax{\Ant \sqsubseteq \atmost{1}{\plays}{\rg}}\label{eq:obj-plays-max-one-rg}
\end{align}

\noindent
Complex \rosirole groups are treated like abstract \rosiroles. An object can ``play'' an instance of
a \rosirole group. This is equivalent to fulfilling that role group. So, the object must ``play''
the required number of containing \rosirole groups.
%
Furthermore, if an object plays a \rosirole whose type is an atom of a top-level \rosirole group,
the object must also fulfill that \rosirole group.
\begin{align}
  \top & \sqsubseteq \bigsqcap_{\substack{\rg\in\RG(\Cmc),\\\rg = (\{A_{1}, \dots, A_{n}\},k,\ell)}}
      \oax{\exists\plays.\rg \equiv (\atleast{k}{\plays}{(A_{1} \sqcup \dots \sqcup A_{n})}) 
      \sqcap (\atmost{\ell}{\plays}{(A_{1} \sqcup \dots \sqcup A_{n})})}\label{eq:rg-declaration}\\
  \top & \sqsubseteq \bigsqcap_{\rg\in\RG*(\Cmc)}
      \oax{\exists\plays.\bigg(\bigsqcup\nolimits_{\rt\in\atom(\rg)}\rt\bigg)\sqsubseteq \exists\plays.\rg}\label{eq:rg-must-play}
\end{align}

\noindent 
Since for the occurrence constraints we only consider objects that play any \rosirole in this
compartment, we assure that fulfilling a \rosirole group implies playing some \rosirole.
\begin{align}
  \top & \sqsubseteq \oax{\exists\plays.\Arg\sqsubseteq\exists\plays.\Art}\label{eq:rgplayer-plays-role}
\end{align}

\noindent
To capture the occurrence constraints we use an object individual name \occurCounter and introduce
an \counts-role from that individual to all \rosiroles and \rosirole group instances. Thus, the
occurrence constraints can be enforced with the help of qualified number restrictions.
\begin{align}
  \top & \sqsubseteq \oax{\Art \sqcup \Arg \sqsubseteq \atleast{1}{\counts^{-}}{\{\occurCounter\}} 
      \sqcap \atmost{1}{\counts^{-}}{\{\occurCounter\}}}\label{eq:counts-decl}\\
  \begin{split}
    \ct & \sqsubseteq \oax{(\atleast{k}{\counts}{\rg})(\occurCounter)}
    \sqcap \oax{(\atmost{\ell}{\counts}{\rg})(\occurCounter)}\\
    & \text{\rlap{\hspace{5em} for all $(k..\ell,\rg)\in\occur(\ct)$, for all $\ct\in\CT$}}
  \end{split}\label{eq:occurrence-constraints}
\end{align}

\noindent Cardinal constraints restrict the number of \rosiroles that are related to \rosirole via a
relationship type. \Rosiroles whose type is the domain or range of a relationship type \rst for
which there exists a cardinal constraint must have the correct amount of \rst successors or
predecessors, respectively.

\begin{align}
  \top & \sqsubseteq \bigsqcap_{\substack{\rst\in\RST, \\ \rel(\rst) = (\rt[1], \rt[2]), \\
  \card(\rst) = (i..j, k..\ell)}}
  \begin{aligned}[t]
    & \oax{\exists\plays.\rt[1] \sqsubseteq \atleast{k}{\rst}{\top} \sqcap \atmost{\ell}{\rst}{\top}} \\
    & \qquad\qquad \sqcap \oax{\exists\plays.\rt[2] \sqsubseteq \atleast{i}{\rst^{-}}{\top} \sqcap \atmost{j}{\rst^{-}}{\top}}
  \end{aligned}
   \label{eq:last-equation}
\end{align}

\noindent
As mentioned above, because we spanned the relationship types between the players and not between the
\rosiroles, we can model at least some of the intra-relationship type constraints, exactly those
that are expressible in the object logic. In \SHOIQ, these would be transitivity and symmetry. Thus,
we add the following role axioms to the object RBox \RO:
\begin{gather}
  \trans{\rst}  \text{\qquad for all $\transsf\in\intra(\rst)$}\label{eq:rbox-1}\\
  \rst \sqsubseteq \rst^{-} \text{\qquad for all $\symm\in\intra(\rst)$}\label{eq:rbox-2}
\end{gather}
where $\transsf$ and $\symm$ are functions that return $\true$ if and only if the binary relation,
respectively, is transitive or symmetric.  But it is important to note, due to the restriction that
only simple roles are allowed to appear in number restrictions, we cannot impose any cardinal
constraints on the relationship type \rst if \rst is supposed to be transitive.


Before we establish the desired relation between the CROM and the ontology, we do a short
analysis of the required expressiveness of the meta and the object logic. To state the disjoint union
axioms we need at least \ALC. As we do not need anything more on meta level, we can
fix \LM to be \ALC. On object level, we need qualified number restrictions and inverse roles to
assure that every \rosirole is played exactly once. 
%Hence, the minimal object DL is \ALCIQ. 
If we have a single occurrence constraint, we have to add nominals. Transitivity axioms or role
hierarchies are only required if some intra-relationship type constraint specifies a transitive or
symmetric relationship type, respectively. Table~\ref{tab:required-dls} shows a summary of the required DLs.

\begin{table}
  \caption{Summary of required DLs for \LM and \LO}
  \label{tab:required-dls}  \centering
  \begin{tabularx}{0.8\linewidth}{lXl}
    \toprule
    \LO & minimal                                  & \ALCIQ\\
        & with occurrence constraints              & \ALCOIQ\\
        & with intra-relationship type constraints & \SHOIQ\\
    \midrule
    \LM & & \ALC\\
    \bottomrule
  \end{tabularx}
\end{table}

Given a \SCCROM{} \Kmc, we obtain the \ALCSHOIQ ontology
$\Bmf_{\Kmc} = (\Bmc_{\Kmc}, \emptyset,\RO)$ where $\Bmc_{\Kmc}$ is the conjunction of all meta
axioms from~\eqref{eq:compartments-coverage} to~\eqref{eq:last-equation} and \RO is the set of all
role axioms~\eqref{eq:rbox-1} and~\eqref{eq:rbox-2}.

\begin{theorem}
  Let $\Kmc = (\Mmc, \Amc, \Cmc)$ be a \SCCROM. Then, \Kmc is satisfiable iff $\Bmf_{\Kmc}$ is consistent.
\end{theorem}
\begin{proof}
  Assume that \Kmc is satisfiable and let \IK denote a \SCROI that satisfies \Mmc, \Amc and
  \Cmc.  We define the set
  $\Delta_{\RT,c}$ of all \rosiroles played in $c$,
  the set $\Delta_{\RG,c}$ of all \rosirole group instances fulfilled in $c$,  and the nested interpretation \JJ as follows:
  \begin{align*}
    \Delta_{\RT,c} & \coloneqq \{r\in\Rsf[\IK] \mid (\cdot,c,r) \in\plays\} \text{\qquad for all $c\in\Csf[\IK]$}\\
    \Delta_{\RG,c} & \coloneqq \{(x,c,y) \in \RG\times\Csf[\IK]\times\Osf[\IK] \mid x^{\IK,c,y}=1\} \text{\qquad for all $c\in\Csf[\IK]$}\\[1ex]
    \Cbb & \coloneqq \Csf[\IK] \\
    \ct^{\J} & \coloneqq \ct^{\IK} \text{\qquad for all $\ct\in\CT$} \\
    \nested^{\J} & \coloneqq \{(c,o) \mid o,c\in\Csf[\IK], (o,c,\cdot)\in\plays\} \\[1ex]
    \Delta^{\J} & \coloneqq \Nsf[\IK] \cup \Csf[\IK] \cup \Rsf[\IK] \cup \bigcup_{c\in\Csf[\IK]}\Delta_{\RG,c} \cup \{d_{\occurCounter}\}\\
    \nt^{\I_{c}} & \coloneqq \nt^{\IK} \text{\qquad for all $\nt\in\NT$} \\
    \ct*^{\I_{c}} & \coloneqq \ct^{\IK} \text{\qquad for all $\ct\in\CT$} \\
    \rt^{\I_{c}} & \coloneqq \rt^{\IK} \cap \Delta_{\RT,c} \text{\qquad for all $\rt\in\RT$} \\
    \rg^{\I_{c}} & \coloneqq \{(\rg,c,y) \mid \rg^{\IK,c,y}=1\} \text{\qquad for all $\rg\in\RG(\Cmc)$}\\
    \Art^{\I_{c}} & \coloneqq \Delta_{\RT,c}\\
    \Ant^{\I_{c}} & \coloneqq \Osf[\IK]\\
    \Arg^{\I_{c}} & \coloneqq \Delta_{\RG,c}\\
    \rst^{\I_{c}} & \coloneqq \{(o_{1},o_{2})\mid\text{ there are $r_{1}$, $r_{2}$ s.t.\
                    $(o_{1},c,r_{1}),(o_{2},c,r_{2})\in\plays$ and}\\
                   & \hspace{3cm}\text{$(r_{1},r_{2})\in\links(\rst,c)$}\}\\
    \plays^{\I_{c}} & \coloneqq \{(o,r) \mid (o,c,r)\in\plays\}\\
                   &\hspace{3cm} \cup \{(o,r{\kern-0.1em}g) \mid r{\kern-0.1em}g =
                     (x,c,o)\in\Delta_{\RG,c}\text{ with $x^{\IK,c,o}=1$}\}\\
    \counts^{\I_{c}} & \coloneqq \{(d_{\occurCounter},y) \mid y\in\Delta_{\RT,c}\cup\Delta_{\RG,c}\}\\
    \occurCounter^{\I_{c}} & \coloneqq d_{\occurCounter}\\
    c^{\J} & \coloneqq c^{\IK} \text{\qquad for $c\in\INDC$}\\
    a^{\I_{c}} & \coloneqq a^{\IK} \text{\qquad for $a\in\IND$}
  \end{align*}
  It is straight forward to show that \J is a model of $\Bmf_{\Kmc}$. Note, that \J respects the
  rigid names since all \nt, \ct*, \Ant and individual names are interpreted the same in every
  world~$c$.
  %
  Axioms~\eqref{eq:compartments-coverage},
  \eqref{eq:compartments-disjoint},
  \eqref{eq:naturals-coverage},
  \eqref{eq:roles-disjoint},
  \eqref{eq:obj-disjoint},
  \eqref{eq:plays-domain},
  \eqref{eq:plays-range},
  \eqref{eq:meta-type-assertion} to~\eqref{eq:obj-plays-max-one-rg},
  \eqref{eq:counts-decl}
  %
  are modelled by construction of \J. Since
  $\Delta_{\RT,c}\subseteq\bigcup_{\rt\in\RT}\rt^{\IK} = \Rsf[\IK]$, \J models \eqref{eq:roles-coverage}.
  %
  Axioms~\eqref{eq:roletype-played-max-once} and~\eqref{eq:role-is-played-exactly-once} are modelled due
  to 3. and 4. of Definition~\ref{def:scroi}. 
  %
  Assume that $c_{1}\in(\lnot\oax{\ct* \sqcap \exists\plays.\top\sqsubseteq\bot})^{\J}$. Hence,
  $\I_{c_{1}}\not\models\ct* \sqcap \exists\plays.\top\sqsubseteq\bot$. Thus, there exists
  $c_{2}\in(\ct* \sqcap \exists\plays.\top)^{\I_{c_{1}}}$. Therefore,
  $(c_{2},c_{1},\cdot)\in\plays$, $(c_{1},c_{2})\in\nested^{\J}$ and $c_{2}\in\ct^{\J}$. Overall, \J
  models Axiom~\eqref{eq:nested-cts}.
  %
  Since \plays respects \fills and \parts and \links respects \rel, \eqref{eq:fills},
  \eqref{eq:roles-in-compartment} and \eqref{eq:rst-domain-range} are satisfied.
  %
  Axioms~\eqref{eq:rg-declaration} to~\eqref{eq:rgplayer-plays-role} are satisfied due to the
  semantics of \rosirole groups and the fact that all top-level \rosirole groups are satisfied.
  %
  \IK respecting occurrence and cardinal constraints ensures that \J
  models~\eqref{eq:occurrence-constraints} and~\eqref{eq:last-equation}. 
  %
  Finally, the respective intra-relationship constraints imply the satisfaction of \RO.
  
  Conversely, let \JJ denote a model of $\Bmc_{\Kmc}$. W.l.o.g. we can assume that all
  o-compartments that exist also play some \rosiroles. Otherwise we could simply delete them, and
  still have a model. Let $\Delta_{\CT}\subseteq\Delta^{\J}$ denote the set of all o-compartments,
  i.e. $\Delta_{\CT} \coloneqq \bigcup_{\ct*\in\CT*} \ct*^{\I_{\hat{c}}}$.
  %
  Due to \eqref{eq:nested-cts}, there exists a mapping $\mu: \Delta_{\CT} \to \Cbb$ which maps
  o-compartments to contexts of respective type.  W.l.o.g, we can assume that there exist
  sufficiently many contexts $c\in\Cbb$ to assure that $\mu$ preserves the occurrence and
  cardinality constraints, since we could introduce copies of $c$ if necessary.  We define the
  \SCROI{} \I as follows:
  \begin{align*}
    \Gamma^{\I} & \coloneqq \Cbb^{\J} \cup \bigcup_{\nt\in\NT}\nt^{\I_{\hat{c}}} \cup \bigcup_{c\in\Cbb} \Art^{\I_{c}} \text{\qquad for
                  some $\hat{c}\in\Cbb$}\\
    \type(d) & \coloneqq  
               \begin{cases}
                 T & \text{if $T\in\CT$ and $d\in T^{\J}$}\\
                 T & \text{if $T\in\NT\cup\RT$ and $d\in T^{\I_{c}}$ for some $c$}
               \end{cases}\\
    (\mu(o),c,r) \in \plays & \text{\quad iff \quad $(o,r)\in\plays^{\I_{c}}$}\\
    (r_{1}, r_{2}) \in \links(\rst,c) & \text{\quad iff \quad $(o_{1},r_{1}), (o_{2},r_{2}) \in
                                        \plays^{\I_{c}}$ and $(o_{1}, o_{2})\in\rst^{\I_{c}}$}
  \end{align*}
  \I is well-defined and analogous to above one can go through the axioms step by step and show that
  \I satisfies \Kmc.
\end{proof}

Besides checking satisfiability in general, we can also address the following other questions a
domain analyst would be interested in:
\begin{enumerate}
\item[(Q1)] Is a specific compartment type \ct instantiable, i.e.\ does there exist some \SCROI{} \I
  s.t.\ $c \in \ct^{\I}$?
\item[(Q2)] Is the \rosirole type \rt playable, i.e.\ does there exist some \SCROI{} \I s.t. there is
  some $r \in \rt^{\I}$?
\item[(Q3)] Can two \rosiroles be linked via the relationship type \rst, i.e. does there exist some
  \SCROI{} \I s.t. there are some $r_{1}, r_{2}\in \Rsf$, $c\in \Csf$ with
  $\links(\rst,c)=(r_{1}, r_{2})$?
\item[(Q4)] Can more precise constraints be entailed, i.e.\ do there exist some cardinalities in \Cmc
  which can never be reached?
\item[(Q5)] Is some partial knowledge about an instance satisfiable, i.e.\ does there exist some
  \SCROI{} \I based this partial knowledge?
\end{enumerate}

To answer all these questions, we utilize \SCROA{s} as introduced earlier. For~(Q1), we add the meta
type assertion
\begin{gather*}
  \ct(a) 
\end{gather*}
to \A and check $\Bmf_{(\Mmc,\A,\Cmc)}$ for consistency. To answer~(Q2), an object type assertion of
the form
\begin{gather*}
  \rt(a,c), 
\end{gather*}
added to \A, is sufficient. W.l.o.g. we assume that $a$ and $c$ do not occur in
$\Bmf_{(\Mmc,\A,\Cmc)}$ before. Due to the other axioms already specified, the \rosirole must also
be played.
%
Similarly, for~(Q3) we add the link assertion
\begin{gather*}
  \linkass(\rst, c, a_{1}, a_{2}).
\end{gather*}
To test whether constraints are sharp, we assert the opposite and check for inconsistency. Here, we
will directly add axioms to $\Bmf_{(\Mmc,\A,\Cmc)}$. If, for example, there must exist at least $n$
roles of type \rt in \ct, we add
\begin{gather*}
  \ct \sqsubseteq \oax{(\atmost{n}{\counts}{\rt})(\occurCounter)} \land \ct(a)
\end{gather*}
to $\Bmf_{(\Mmc,\A,\Cmc)}$ which states that there must also exist at most $n$ \rt-roles. Then,
inconsistency of $\Bmf_{(\Mmc,\A,\Cmc)}$ would imply that there must indeed exist $n+1$ \rt-roles in
\ct.

For~Q5, we assume that we know certain facts about the domain. This could be the existence of a
compartment, a role or a link between two roles. These can all be formulated via a \SCROA, with
individual names adjusted accordingly.  If adding all respective assertions still preserves
consistency of $\Bmf_{(\Mmc,\A,\Cmc)}$, then there exists some \SCROI that satisfies \Mmc, \A and
\Cmc and respects these facts.


\section{Going beyond \texorpdfstring{$\Sigma$}{Sigma}-CROMs}
\label{sec:going-beyond-crom}

In \todo{cite sle14}, Kühn et al. presented a feature model for role-based modelling languages of which
the above defined CROM is one instance. Although the basic ideas of the mapping will persist, a
detailed analysis which features can easily be supported, is necessary. At first we will look into
inheritance.

Let us assume, that additionally to a CROM we also have an irreflexive, asymmetric, functional
inheritance relation $\prec_{\nt}$ over the natural types. Inheritance is intuitively captured as a
GCI:
\begin{align}
  \label{eq:1}
  \top \sqsubseteq & \oax{\nt[a] \sqsubseteq \nt[b]} \text{ for all $\nt[a]\prec_{\nt}\nt[b]$}.
\end{align}
Then, obviously we have to adjust axiom~\eqref{eq:roles-disjoint} since natural types do not need to
be disjoint anymore. This implies that if some natural type \nt fills a role type \rt, then every
subtype of \nt also fills \rt.  Similarly, inheritance for compartment types could be handled.
Here, one has to make sure what exactly the intended semantics of compartment inheritance is
supposed to be. In our setting, a subtype would be a specialization in the sense that all axioms and
constraints that hold for the supertype also hold for the subtype and there could exist additional
axioms or constraints.

Other features might not be relevant in our case, as they have no influence on the
satisfiability. These are especially features about behaviour and dynamic aspects. 
%
Among the rest, most features can easily be supported with only little change of the axioms
required, like deep \rosiroles, i.e. \rosiroles are allowed to play \rosiroles, or whether a same
\rosirole type can be played several times.


% \subsection{Adding Inheritance}
% \label{sec:adding-inheritance}

% Let us assume, that additionally to a CROM we also have an inheritance relation $\prec_{\nt}$ over
% the natural types.

% \todo[inline]{introduction paragraph}

% \todo[inline]{natural type inheritance - what is important in this setting}

% In the last sections we showed how contextualized description logics can represent \rosirole-based models

% \subsection{Further Features}
% \label{sec:further-constraints}


% \subsection{Going beyond CROM}
% \label{sec:temporal-aspects}

To our opinion, two kinds of constraints that might be important in practice were not considered until
now: Constraints based on attributes of players and temporal constraints.

The first ones are quite easy to express in \LMLO. Besides some technical changes on the above
axioms, for each attribute $\mathsf{att_{i}}$ of a natural type or o-compartment we introduce the rigid
role $\mathsf{att_{i}}\in\ORR$ since we assume that attributes of a rigid type are also rigid.
Here, it might even be worth to consider an object logic with concrete domains as for example
$\SHOIQ(\Dmc)$ \todo{cite}. We did not investigate these extensions of DLs, but we conjecture that
all results of Chapter~\ref{cha:context-dls} can be transferred without much effort.

Utilizing rigid types which are modelled with more detail, we can also introduce more fine-grained
constraints. For example, the \fills-relation could be specified to not simply require a player of a
certain \rosirole type to be of some specific natural type but rather to demand that certain
attributes must exist. To be allowed to ``play'' the \rosirole of an employee, a person must have a
tax ID. Going a step further, we could also restrict \rosirole-playing based on the values of the
attributes. In the context of USA, a person can only play president if he is a ``natural born citizen'' and at
least 35 years old:
\begin{align*}
  \mathsf{USA} \sqsubseteq\oax{\exists\plays.\mathsf{President} \sqsubseteq \mathsf{Person} \sqcap
  \exists \mathsf{natural\_born\_citizen}.\{\true\} \sqcap \exists\mathsf{age \geq_{35}}}.
\end{align*}

Last but not least we can express additional complex constraints with arbitrary \LMLO axioms. For
example,
\begin{align*}
  \oax{\exists\rst.(\exists\rst.\top)\sqsubseteq\bot} \sqsubseteq \oax{\exists\rst.\top\sqsubseteq\bot}
\end{align*}
states that if there are no chains of length two for an relationship type \rst, then there will not
be any \rst at all in that particular compartment.


\cleardoublepage

%%% Local Variables:
%%% mode: latex
%%% TeX-master: "../thesis"
%%% reftex-default-bibliography: ("../references.bib")
%%% End:

%  LocalWords:  logics Kühn irreflexivity assertional satisfiability iff cardinalities CROM UML de
%  LocalWords:  facto ontologies DL nominals intra irreflexive GCI
