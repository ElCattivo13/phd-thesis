
\chapter{A Mapping from Role-Based Models to Context Description Logics}
\label{cha:mapping}

In the last chapter we introduced a family of description logics which are capable of expressing
contextual knowledge. 

 

How to use cDLs in practice?

Semantics, even when durchdacht und präzise, hard to comprehend.

Motivation for looking into cDLs were role-based systems.

automatic mapping from such a system desirable.

Formal representation of role-based system necessary.

Chosen the family of CROMs introduced by Kühn.

Also say what limitations occur, what cannot be represented in cDL

\todo[inline]{paragraph about related work -> formalizing UML diagrams}

\section{A Formal Role-Based Modeling Language}
\label{sec:sigma-crom}


In this section we will present a syntactical variant of the \emph{Compartment Role Object Model}
defined in \textbf{[Ku-SLE15]}\todo{citation}. This variant is simply introduced for easier
explanation of the mapping to description logics.

\subsection{Type and Instance Level}
\label{sec:type-instance-level}

We omit the graphical notation and 

\todo[inline]{some more information}



\begin{definition}[$\Sigma$-Compartment Role Object Model]
  Let \NTsf, \RTsf, \CTsf and \RSTsf be finite and mutually disjoint sets of Natural Types, Role
  Types, Compartment Types, and Relationship Types, respectively.  The tuple
  $\Sigma = (\NTsf, \RTsf, \CTsf,\RSTsf)$ is a \todo{or better alphabet?}\emph{vocabulary}.
  A \emph{$\Sigma$-Compartment Role Object Model ($\Sigma$-CROM)} \Mmc is a tuple \MM where
  \begin{enumerate}
  \item $\fills\subseteq(\NTsf\cup\CTsf)\times\RTsf$ is a right-total binary relation,
  \item $\parts:\CTsf\to\Pmc$ is a bijection where \Pmc is an arbitrary but fixed partition of
    \RTsf, and
  \item $\rel:\RSTsf\to\Smc$ is a bijection where $\Smc\subseteq\bigcup_{P\in\Pmc}P\times P$ is an
    irreflexive binary relation. \qedhere
  \end{enumerate}
\end{definition}

Furthermore, we say that \emph{$t$ fills \rt} if $(t,\rt)\in\fills$, \emph{\rt participates in \ct}
if $\rt\in\parts(\ct)$, \emph{$\rt_{1}$ and $\rt_{2}$ are related via \rst} if
$(\rt_{1},\rt_{2})=\rel(\rst)$, and \emph{$\rt_{1}$ is the domain of \rst ($\dom(\rst)$) and
  $\rt_{2}$ is the range of \rst ($\ran(\rst)$)} if $(\rt_{1},\rt_{2})\in\rel(\rst)$.

Note here, that reasoning about role-based models does not include checking \emph{well-formed\-ness}
as defined in Definition 1 of \textbf{[??]}, since that is a pure syntactical check. Therefore the
above definition already ensures that the CROM is well-formed.  Since \fills is a right-total
relation, for each role type there exists a natural type or compartment type that fills it.  As each
element of \Pmc is a non-empty subset of \RTsf, the image of \parts does not contain the empty set.
Since \Pmc is a partition each role type participates in only one compartment type.  Due to the
irreflexivity of \Smc there exists no $\rst \in \RSTsf$ and $\rt \in \RTsf$ such that
$\rel(\rst) = (\rt,\rt)$. Furthermore, since \Smc is a subset of $\bigcup_{P\in\Pmc}P\times P$, the
pair of role types in the image of \rel always participate in the same compartment type.

\todo[inline]{introduction to instance level}


\begin{definition}[$\Sigma$-Compartment Role Object Instance, Compliance]
  Let $\Sigma = (\NTsf, \RTsf,$ \todo{kann man so was machen???}$\CTsf,\RSTsf)$ be a vocabulary and
  let \MM be a $\Sigma$-CROM.  Then a \emph{$\Sigma$-Compartment Role Object Instance
    ($\Sigma$-CROI)} \I for \Mmc is a tuple $\I=(\Gamma,\type,\plays,\links)$, where
  \begin{itemize}
  \item $\Gamma$ is a non-empty domain,
  \item $\type:\Gamma\to\NTsf\cup\RTsf\cup\CTsf$ is a total function,
  \item $\plays\subseteq(\Nsf\cup\Csf)\times\Csf\times\Rsf$ is a ternary relation, and
  \item $\links\subseteq\RSTsf\times\Csf\times\Rsf\times\Rsf$ is a quaternary relation.
  \end{itemize}
  with 
  \begin{align*}
  \Nsf & \coloneqq \{d \in \Gamma \mid \type(d) \in \NTsf\}, \\
  \Rsf & \coloneqq \{d \in \Gamma \mid \type(d) \in \RTsf\}, \text{ and} \\
  \Csf & \coloneqq \{d \in \Gamma \mid \type(d) \in \CTsf\} 
  \end{align*}
  being the sets of \emph{naturals}, \emph{roles} and \emph{compartments}. For
  $T \in \NTsf\cup\RTsf\cup\CTsf$, $T^{\I}$ is the set of all elements of that type, i.e.\
  $T^{\I} \coloneqq \{d \in \Gamma \mid \type(d) = T\}$.

  A $\Sigma$-CROI \I is \emph{compliant} with \Mmc, denoted by $\I\models\Mmc$ if it has the following
  properties:
  \begin{enumerate}
  \item The \plays-relation respects \fills, i.e.~for each tuple $(o,c,r)\in\plays$ the type of $o$
    fills the type of $r$:
    \begin{align*}
      \{(o,r)\mid(o,\cdot,r)\in\plays\}\subseteq\{(o,r)\mid\text{there exists
        $(T_{1},T_{2})\in\fills$ s.t. $o\in T_{1}^{\I}$, $r\in T_{2}^{\I}$}\}.
    \end{align*}
  \item The \plays-relation respects \parts, i.e.~for each tuple $(o,c,r)\in\plays$ the type of $r$
    participates in the type of $c$:
    \begin{align*}
      & \{(c,r)\mid(\cdot,c,r)\in\plays\}\\
      & \qquad \subseteq\{(c,r)\mid \text{there exists $T_{1}\in\CTsf$, $T_{2}\in\RTsf$ s.t. $c\in
        T_{1}^{\I}$, $r\in T_{2}^{\I}$, $T_{2}\in\parts(T_{1})$}\}.
    \end{align*}
  \item Each object can play in each compartment only one role of each role type:
    \begin{align*}
      \{(o,c,r),(o,c,r')\}\subseteq\plays \text{ implies } \type(r)\neq\type(r').
    \end{align*}
  \item Each role is played by exactly one object in exactly one compartment:
    \begin{align*}
      |\{(o,c)\mid(o,c,r)\in\plays\}| = 1 \text{ for all $r \in \Rsf$}.
    \end{align*}
  \item Roles occurring in an element of \links are played in the compartment occurring in that
    element, i.e.\ for each tuple $(\rst,c,r_{1},r_{2})\in\links$ there exists objects that play
    $r_{1}$ and $r_{2}$ in $c$:
    \begin{align*}
      \{r\mid(\cdot,c,r,\cdot)\in\links\text{ or }(\cdot,c,\cdot,r)\in\links\}\subseteq\{r\mid(\cdot,c,r)\in\plays\}.
    \end{align*}
  \item The \links-relation respects \rel, i.e.\ for each tuple $(rst,c,r_{1},r_{2})\in\links$ the
    types of $r_{1}$ and $r_{2}$ are related via $rst$:
    \begin{align*}
      (rst,c,r_{1},r_{2})\in\links &  \text{ implies } \rel(rst)=(\type(r_{1}),\type(r_{2})). \qedhere
    \end{align*}
  \end{enumerate}
\end{definition}

Before we discuss how the information about a \SCROM can be encoded in a description logic ontology,
we discuss the main reasoning tasks for this


\subsection{Constraint Level}
\label{sec:constraint-level}


\section{Adding Inheritance}
\label{sec:adding-inheritance}



\section{Going beyond CROM}
\label{sec:going-beyond-crom}








%%% Local Variables:
%%% mode: latex
%%% TeX-master: "../thesis"
%%% End:

%  LocalWords:  logics Kühn irreflexivity
