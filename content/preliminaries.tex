
\chapter{Preliminiaries}
\label{cha:preliminiaries}

In this chapter, we introduce 

\todo[inline]{some introduction}


\section{Description Logics}
\label{sec:description-logics}

\todo[inline]{a few words to begin with}

\subsection{Description Logic Concepts}
\label{sec:dl-concepts}


The basic building blocks in description logics are so-called \emph{concepts}. As described

used as description, classify a set of objects
\todo[inline]{what are concepts and for what are they good}

Note that in the following definitions we refer to the triple \Nsig explicitly although it is
usually left implicit in standard definitions.  This turns out to be useful in chapter
\ref{cha:context-dls} as we need to distinguish between the symbols used in the meta level and the
object level.  Sometimes we omit \Nsig, however, if it is clear from the context.
%

\begin{definition}[Syntax of \Nsig-roles and \Nsig-concepts]
  \label{def:syntax-concepts}
  Let \NC, \NR, \NI be non-empty, pairwise disjoint sets of \emph{concept names}, \emph{role names},
  and \emph{individual names}. Then, the triple $\Nsig\coloneqq(\NC,\NR,\NI)$ is the
  \emph{signature}. An \emph{\Nsig-role} $r$ is either a role name, i.e.~$r\in\NR$, or it is of the
  form $s^{-}$ with $s\in\NR$ (\emph{inverse role}).

  The set of \emph{\Nsig-concepts} is the smallest set, such that
  \begin{itemize}
  \item for all $A\in\NC$, $A$ is a \Nsig-concept (\emph{atomic concept}), and
  \item if $C$ and $D$ are \Nsig-concepts, $r$ is a \Nsig-role, $a\in\NI$, then $\lnot C$ (\emph{negation}),
    $C\sqcap D$ (\emph{conjunction}), $\exists r.C$ (\emph{existential restriction}), $\{a\}$
    (\emph{nominal}) and $\atleast{n}{r}{C}$ (\emph{at-least restriction}) are \Nsig-concepts. \qedhere
  \end{itemize}
\end{definition}

As usual in description logics, we use the following abbreviations:
\begin{itemize}
\item $C\sqcup D$ (\emph{disjunction}) for $\lnot(\lnot C \sqcap \lnot D)$,
\item $\top$ (\emph{top}) for $A \sqcup \lnot A$ where $A\in\NC$ is arbitrary but fixed,
\item $\bot$ (\emph{bottom}) for $\lnot\top$,
\item $\forall r.C$ (\emph{value restriction}) for $\lnot(\exists r.\lnot C)$, and
\item $\atmost{n}{r}{C}$ (\emph{at-most restriction}) for $\lnot(\atleast{n+1}{r}{C})$.
\end{itemize}


The semantics of description logic concepts are defined in a model-theoretic way using the notion of
interpretations.

\begin{definition}[\Nsig-interpretation, Semantics of \Nsig-concepts]
  \label{def:n-interpretation}
  Let $\Nsig\coloneqq(\NC,\NR,\NI)$ be the signature. Then, an \emph{\Nsig-interpretation \I} is a pair
  $(\Delta^{\I}, \cdot^{\I})$ where the \emph{domain} $\Delta^{\I}$ is a non-empty set and
  the \emph{interpretation function} $\cdot^{\I}$ maps \todo{sth. about UNA?}
  \begin{itemize}
  \item every concept name $A\in\NC$ to the set $A^{\I}\subseteq\Delta^{\I}$,
  \item every role name $r\in\NR$ to the binary relation
    $r^{\I}\subseteq\Delta^{\I}\times\Delta^{\I}$, and
  \item every individual name $a\in\NI$ to the element $a^{\I}\in\Delta^{\I}$.
  \end{itemize}

  Now, that function is extended for inverse roles and complex concepts as follows:
  \begin{itemize}
  \item $(s^{-})^{\I} \coloneqq \left\{(d,c)\in\Delta^{\I}\times\Delta^{\I}\mmid (c,d)\in
      s^{-}\right\}$,
  \item $(\lnot C)^{\I} \coloneqq \Delta^{\I} \setminus C^{\I}$,
  \item $(C \sqcap D)^{\I} \coloneqq C^{\I} \cap D^{\I}$,
  \item $(\exists r.C)^{\I} \coloneqq \{d \in \Delta^\I \mid \text{there is an}\ e \in C^\I \
    \text{with}\ (d,e)\in r^\I\}$,
  \item $\{a\}^{\I} \coloneqq \{a^{\I}\}$, and
  \item $(\atleast{n}{r}{C})^{\I} \coloneqq \{d \in \Delta^{\I} \mid \sharp\{e\in C^{\I}\mid (d,e)\in r^{\I}\}\ge n\}$.
  \end{itemize}
  where $\sharp S$ denotes the cardinality of the set $S$.
\end{definition}

With these notions at hand, we can already define some first 

\todo[inline]{some text and example of a complex concept with an interpretation.}
some text

\begin{example}
  
\end{example}




\subsection{Boolean Knowledge Bases}
\label{sec:dl-axioms}

\todo[inline]{introducing words, why BKB, what are they made of, what are axioms}

\begin{definition}[Syntax of axioms over \Nsig, BKBs over \Nsig]
  Let $\Nsig\coloneqq(\NC,\NR,\NI)$ be the signature. Then, if $C$ and $D$ are \Nsig-concepts, $r$
  and $s$ are a \Nsig-roles, and $\{a,b\}\subseteq\NI$, then $C \sqsubseteq D$ (\emph{general concept inclusion,
    GCI}), $C(a)$ (\emph{concept assertion}), $r(a,b)$ (\emph{role assertion}), $r \sqsubseteq s$
  (\emph{role inclusion}), and $\mathsf{trans}(r)$ (\emph{transitivity axiom}) are
  \emph{axioms over \Nsig}.

  Moreover, an \emph{RBox \Rmc over \Nsig} is a finite set of role inclusions over \Nsig and
  transitivity axioms over \Nsig. A \emph{Boolean axiom formula over \Nsig} is defined inductively
  as follows:
  \begin{itemize}
  \item every GCI over \Nsig is a Boolean axiom formula over \Nsig,
  \item every concept and role assertion over \Nsig is a Boolean axiom formula over \Nsig,
  \item if $\Bmc_{1}$, $\Bmc_{2}$ are Boolean axiom formulas over \Nsig, then so are $\lnot\Bmc_{1}$
    (axiom negation) and $\Bmc_{1}\land\Bmc_{2}$ (axiom conjunction), and
  \item nothing else is a Boolean axiom formula over \Nsig.
  \end{itemize}

  Finally, a \emph{Boolean knowledge base (BKB) over \Nsig} is a pair $\Bmf = (\Bmc, \Rmc)$, where
  \Bmc is a Boolean axiom formula over \Nsig and \Rmc is an RBox over \Nsig. An \emph{ontology over
    \Nsig} is a BKB over \Nsig, where only axiom conjunction and no axiom negation is allowed in the
  Boolean axiom formula.
\end{definition}

Often an ontology $\Omc = (\Bmc, \Rmc)$ is written as a triple $\Omc = (\Tmc, \Amc, \Rmc)$ where
\Tmc (\emph{TBox}) is the set of all GCIs occurring in \Bmc and \Amc (\emph{ABox}) is the set of all
assertion axioms occurring in \Bmc.  Again as usual in logics, we use $\Bmc_1\lor\Bmc_2$ (axiom
disjunction) as abbreviation for $\lnot(\lnot\Bmc_1\land\lnot\Bmc_2)$.

\begin{definition}[Semantics of axioms over \Nsig, BKBs over \Nsig]
  \label{def:semantics-of-axioms}
  An \Nsig-interpretation \I is a model of
  \begin{itemize}
  \item the GCI $C \sqsubseteq D$ over \Nsig if $C^{\I} \subseteq D^{\I}$,
  \item the concept assertion $C(a)$ over \Nsig if $a^{\I} \in C^{\I}$,
  \item the role assertion $r(a,b)$ over \Nsig if $(a^\I,b^\I)\in r^\I$,
  \item the role inclusion $r \sqsubseteq s$ over \Nsig if $r^\I \subseteq s^\I$, and
  \item the transitivity axiom $\mathsf{trans}(r)$ over \Nsig if $r^\I=(r^\I)^+$, where $\cdot^{+}$
    denotes the transitive closure of a binary relation.
  \end{itemize}

  This is extended to Boolean axiom formulas over~\Nsig inductively as follows:
  \begin{itemize}
  \item \I is a model of $\lnot\B_1$ if it is not a model of~$\B_1$, and
  \item \I is a model of $\B_1\land\B_2$ if it is a model of both $\B_1$ and~$\B_2$.
  \end{itemize}

  We write $\I\models\alpha$ and $\I\models\Bmc$ if \I is a model of the axiom~$\alpha$ over~\Nsig
  or \I is a model of the Boolean axiom formula~\B, respectively. Furthermore, \I is a model of an
  RBox~\Rmc over~\Nsig (written $\I\models\Rmc$) if it is a model of each axiom in \Rmc.

  Finally, \I is a model of the BKB $\Bmf=(\B,\Rmc)$ over~\Nsig (written $\I\models\Bmf$) if it is a
  model of both~\B and~\Rmc.  We call~\Bmf \emph{consistent} if it has a model.  The
  \emph{consistency problem} is the problem of deciding whether a given BKB is consistent.
\end{definition}

\todo[inline]{some paragraph for reasoning tasks and why only consider consistency problem}

Note that besides the consistency problem there are several other reasoning tasks for description
logics.  The entailment problem, for instance, is the problem of deciding, given a BKB~\Bmf and an
axiom~$\beta$, whether \Bmf \emph{entails} $\beta$, i.e.~whether all models of~\Bmf are also models
of~$\beta$.
%
The consistency problem, however, is fundamental in the sense that most other standard decision
problems (reasoning tasks) can be polynomially reduced to it (in the presence of axiom negation).
For the entailment problem, note that it can be reduced to the \emph{in}consistency problem: $\Bmf=(\B,\Rmc)$
entails $\beta$ iff $(\B\land\lnot\beta,\Rmc)$ is inconsistent.  Hence, we focus in this thesis only
on the consistency problem.



\todo[inline]{paragraph about names of DLs, what is \EL, \ALC, \SHOIQ, ...}

\todo[inline]{paragraph about simple roles, restrictions for \SHQ ...}

In the rest of this thesis, we make this restriction to the syntax of \SHQ and all its
extensions.

This restriction is also the reason why there are no Boolean combinations of
role inclusions and transitivity axioms allowed in the RBox~\Rmc over~\Nsig in
the above definition.  Otherwise the notion of a simple role w.r.t.~\Rmc
involves reasoning.  Consider, for instance, the Boolean combination of axioms
$(\mathsf{trans}(r)\lor\mathsf{trans}(s))\land r\sqsubseteq s$.  It should be
clear that $s$ is not simple, but this is no longer a pure syntactic check.

\todo[inline]{few words about next example}

\begin{example}
  
\end{example}

\subsection{Complexity results for specific description logics}
\label{sec:compl-results-spec}

\todo[inline]{paragraph about expTime of \ALC, BKBs for SHOQ (Marcels thesis), ...}

\clearpage

\section{Roles}
\label{sec:rosiroles}

In this section 

Bachman

japanisches Paper

rigid

foundedness

identity




%%% Local Variables:
%%% mode: latex
%%% TeX-master: "../thesis"
%%% reftex-default-bibliography: ("../references.bib")
%%% End:
