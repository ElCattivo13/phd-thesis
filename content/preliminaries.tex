
\chapter{Preliminiaries}
\label{cha:preliminiaries}

In this chapter, we introduce 

\todo[inline]{Missing text}


\section{Description Logics}
\label{sec:description-logics}

\todo[inline]{missing text}

\subsection{Description Logic Concepts}
\label{sec:dl-concepts}


The basic building blocks in description logics are so-called \emph{concepts}. As described

used as description, classify a set of objects


\begin{definition}[Syntax of \Nsig-concepts]
  \label{def:syntax-concepts}
  Let \NC, \NR, \NI be non-empty, pairwise disjoint sets of \emph{concept names}, \emph{role names},
  and \emph{individual names}. Furthermore let $\Nsig\coloneqq(\NC,\NR,\NI)$. A \emph{\Nsig-role} $r$ is
  either a role name, i.e.~$r\in\NR$, or it is of the form $s^{-}$ with $s\in\NR$ (\emph{inverse role}).

  The set of \emph{\Nsig-concepts} is the smallest set, such that
  \begin{itemize}
  \item for all $A\in\NC$, $A$ is a \Nsig-concept (\emph{atomic concept}), and
  \item if $C$ and $D$ are \Nsig-concepts, $r$ is a \Nsig-role, $a\in\NI$, then $\lnot C$ (\emph{negation}),
    $C\sqcap D$ (\emph{conjunction}), $\exists r.C$ (\emph{existential restriction}), $\{a\}$
    (\emph{nominal}) and $\atleast{n}{r}{C}$ (\emph{at-least restriction}) are \Nsig-concepts. \qedhere
  \end{itemize}
\end{definition}

As usual in description logics, we use the following abbreviations:
\begin{itemize}
\item $C\sqcup D$ (\emph{disjunction}) for $\lnot(\lnot C \sqcap \lnot D)$,
\item $\top$ (\emph{top}) for $A \sqcup \lnot A$ where $A\in\NC$ is arbitrary but fixed,
\item $\bot$ (\emph{bottom}) for $\lnot\top$,
\item $\forall r.C$ (\emph{value restriction}) for $\lnot(\exists r.\lnot C)$, and
\item $\atmost{n}{r}{C}$ (\emph{at-most restriction}) for $\lnot(\atleast{n+1}{r}{C})$.
\end{itemize}


The semantics of description logic concepts are defined in a model-theoretic way using the notion of
interpretations.

\begin{definition}[\Nsig-interpretation, Semantics of \Nsig-concepts]
  \label{def:n-interpretation}
  Let $\Nsig\coloneqq(\NC,\NR,\NI)$. Then, an \emph{\Nsig-interpretation \I} is a pair
  $(\Delta^{\I}, \cdot^{\I})$ where the \emph{domain} $\Delta^{\I}$ is a non-empty set and
  the \emph{interpretation function} $\cdot^{\I}$ maps \todo{evtl noch was zur UNA}
  \begin{itemize}
  \item every concept name $A\in\NC$ to the set $A^{\I}\subseteq\Delta^{\I}$,
  \item every role name $r\in\NR$ to the binary relation
    $r^{\I}\subseteq\Delta^{\I}\times\Delta^{\I}$, and
  \item every individual name $a\in\NI$ to the element $a^{\I}\in\Delta^{\I}$. \qedhere
  \end{itemize}

  Now, that function is extended for inverse roles and concepts as follows:
  \begin{itemize}
  \item $(s^{-})^{\I} \coloneqq \left\{(d,c)\in\Delta^{\I}\times\Delta^{\I}\mmid (c,d)\in
      s^{-}\right\}$,
  \item $(\lnot C)^{\I} \coloneqq \Delta^{\I} \setminus C^{\I}$,
  \end{itemize}
\end{definition}


some text

\begin{example}
  
\end{example}




\subsection{Description Logic Axioms}
\label{sec:dl-axioms}



\clearpage

\section{Roles}
\label{sec:rosiroles}

In this section 

Bachman

japanisches Paper

rigid

foundedness

identity




%%% Local Variables:
%%% mode: latex
%%% TeX-master: "../thesis"
%%% End:
