\chapter{The Contextualised Description Logic \texorpdfstring{\LMLO}{LM[LO]}}
\label{cha:context-dls}

In this chapter, after introducing \emph{description logics (DLs)} as a well-established logical
formalism for knowledge representation, we present a novel family of contextualised description
logics which we use to reason on complex domain models. Firstly we present some basic notions of DLs
which we use throughout this thesis in Section~\ref{sec:preliminaries}.
%
After defining the syntax and semantics of contextualised DLs in Section~\ref{sec:syn-seman-cdl}, we
analyze the computational complexity of the consistency problem in
Section~\ref{sec:complexity-consis-problem}. Particular consideration is needed in the case of \EL,
which is discussed in Section~\ref{sec:case-el}. At last, we show that introducing an additional
concept constructor directly leads to undecidability in Section~\ref{sec:adding-cont-concepts}.

\section{Preliminaries}
\label{sec:preliminaries}

Description logics are a family of knowledge representation formalisms. As already outlined in the
introduction, DLs allow to represent application domains in a well-structured way. In this section,
we present the notations, definitions and known results which are used in this thesis.
%
For a more thorough introduction into description logics we refer the reader to
\cite{DLhandbook-07}.

\subsection{Description Logic Concepts}
\label{sec:dl-concepts}

As shown in Section~\ref{sec:intro-description-logics}, DL concepts describe sets of
elements. Concepts are build from \emph{concept names}, \emph{role names} and \emph{individual
  names} using concept and role constructors.  Note that in the following definitions we refer to
the specific description logic \Lmc and the triple $\Nsig \coloneqq (\NC, \NR, \NI)$ explicitly
although it is usually left implicit in standard definitions.  This turns out to be useful in
Chapter~\ref{cha:context-dls} as we need to distinguish between the DL and symbols used in the meta
level and the object level.  Sometimes we omit \Lmc and \Nsig, however, if they are irrelevant or
clear from the context.

\begin{definition}[Syntax of \Lmc-roles over \Nsig and \Lmc-concepts over \Nsig]
  \label{def:syntax-concepts}
  Let~\Lmc be a DL and let~\NC, \NR, \NI be countably infinite, pairwise disjoint sets of
  \emph{concept names}, \emph{role names}, and \emph{individual names}. Then, the triple
  $\Nsig\coloneqq(\NC,\NR,\NI)$ is a \emph{signature}. An \emph{\Lmc-role $r$ over \Nsig} is either
  a role name, i.e.~$r\in\NR$, or it is of the form $s^{-}$ with $s\in\NR$ (\emph{inverse role}).

  The set of \emph{\Lmc-concepts over \Nsig} is the smallest set such that
  \begin{itemize}
  \item for all $A\in\NC$, then $A$ is an \Lmc-concept over \Nsig (\emph{atomic concept}), and
  \item if $C$ and $D$ are \Lmc-concepts over \Nsig, $r$ is a \Lmc-role over \Nsig and $a\in\NI$, then
    $\lnot C$ (\emph{negation}), $C\sqcap D$ (\emph{conjunction}), $\exists r.C$ (\emph{existential
      restriction}), $\{a\}$ (\emph{nominal}) and $\atleast{n}{r}{C}$ (\emph{at-least restriction})
    are \Lmc-concepts over \Nsig. \qedhere
  \end{itemize}
\end{definition}

\noindent Non-atomic concepts are also called \emph{complex concepts}. As usual in description
logics, we use the following abbreviations:
\begin{itemize}
\item $C\sqcup D$ (\emph{disjunction}) for $\lnot(\lnot C \sqcap \lnot D)$,
\item $\top$ (\emph{top}) for $A \sqcup \lnot A$ where $A\in\NC$ is arbitrary but fixed,
\item $\bot$ (\emph{bottom}) for $\lnot\top$,
\item $\forall r.C$ (\emph{value restriction}) for $\lnot(\exists r.\lnot C)$, and
\item $\atmost{n}{r}{C}$ (\emph{at-most restriction}) for $\lnot(\atleast{n+1}{r}{C})$.
\end{itemize}

\noindent
The semantics of description logic concepts are defined in a model-theoretic way using the notion of
interpretations.

\begin{definition}[\Nsig-interpretation, Semantics of concepts over \Nsig]
  \label{def:n-interpretation}
  Let $\Nsig\coloneqq(\NC,\NR,\NI)$ be the signature. Then, an \emph{\Nsig-interpretation \I} is a pair
  $(\Delta^{\I}, \cdot^{\I})$ where the \emph{domain} $\Delta^{\I}$ is a non-empty set and
  the \emph{interpretation function} $\cdot^{\I}$ maps
  \begin{itemize}
  \item every concept name $A\in\NC$ to a set $A^{\I}\subseteq\Delta^{\I}$,
  \item every role name $r\in\NR$ to a binary relation
    $r^{\I}\subseteq\Delta^{\I}\times\Delta^{\I}$, and
  \item every individual name $a\in\NI$ to an element $a^{\I}\in\Delta^{\I}$ such that different
    individual names are mapped to different elements, i.e.\ for $a,b\in\NI$ it holds that
    $a^{\I}\neq b^{\I}$ if $a\neq b$.
  \end{itemize}
  %
  Now, that function is extended to inverse roles and complex concepts as follows:
  \begin{itemize}
  \item $(s^{-})^{\I} \coloneqq \left\{(d,c)\in\Delta^{\I}\times\Delta^{\I}\mmid (c,d)\in
      s^{-}\right\}$,
  \item $(\lnot C)^{\I} \coloneqq \Delta^{\I} \setminus C^{\I}$,
  \item $(C \sqcap D)^{\I} \coloneqq C^{\I} \cap D^{\I}$,
  \item $(\exists r.C)^{\I} \coloneqq \{d \in \Delta^\I \mid \text{there is an}\ e \in C^\I \
    \text{with}\ (d,e)\in r^\I\}$,
  \item $\{a\}^{\I} \coloneqq \{a^{\I}\}$, and
  \item $(\atleast{n}{r}{C})^{\I} \coloneqq \{d \in \Delta^{\I} \mid \sharp\{e\in C^{\I}\mid (d,e)\in r^{\I}\}\ge n\}$.
  \end{itemize}
  where $\sharp S$ denotes the cardinality of the set $S$.
\end{definition}

\noindent
For any $x\in\NC\cup\NR\cup\NI$, $x^{\I}$ is called the \emph{extension of $x$}.  Note that in the
above definition of an interpretation, we adopt the so-called \emph{unique name assumption} stating
that every individual name is interpreted as a distinct element. By doing so, we emphasize that an
individual name is meant to be the identity of an individual, rather than just a tag as it is
usually used in the context of semantic web.

Now, we can look at a first example using the notions just defined.

\begin{example}\label{ex:concept-nfl}
  Consider the following complex concept $C$:
  \begin{align*}
    & \mathsf{NFL\_{}Team} \sqcap \lnot \mathsf{AFC} \sqcap
    \atleast{1}{\mathsf{playsFor}^{-}}{(\exists\mathsf{position}.\mathsf{Quarterback})} \\
    & \qquad\sqcap \exists\mathsf{coaches}^{-}.\{\text{\textit{MikeMcCarthy}}\}
  \end{align*}
  It describes all NFL teams which are not in the AFC, have at least one person playing for them at
  quarterback position and are coached by Mike McCarthy.

  Figure~\ref{fig:example-concept} depicts an interpretation in which the Green Bay Packers are in the extension of the above concept.
\end{example}
\begin{figure}
  \centering
  \begin{tikzpicture}
    \node[node, label={[align=center]180:\textit{GreenBayPackers},\\\textsf{NFL\_Team}, \textsf{NFC}}] (gbp) at (0,0) {};
    \node[node, label={[align=center]270:\textit{MikeMcCarthy}}] (mmc) at (4,1.0) {};
    \node[node, label={[align=center]270:\textit{AaronRodgers}, \\\textsf{NFL\_PLayer}}] (ar) at (4,-1.0) {};
    \node[node, label={[align=center]270:\textsf{Quarterback}}] (qb) at (8,-1.0) {};
    %
    \draw[edge] (mmc) to[bend right=10,swap] node{$\mathsf{coaches}$} (gbp);
    \draw[edge] (ar) to[bend left=10] node{$\mathsf{playsFor}$} (gbp);
    \draw[edge] (ar) to[bend left=10] node{$\mathsf{position}$} (qb);
  \end{tikzpicture}
  \caption{An Interpretation \I such that $\text{\textit{GreenBayPackers}}^{\I} \in C^{\I}$ and
    $\I\models\Omc$ with $C$ from Example~\ref{ex:concept-nfl} and \Omc from Example~\ref{ex:bkb-nfl}.}
  \label{fig:example-concept}
\end{figure}



\subsection{Boolean Knowledge Bases}
\label{sec:dl-axioms}

With the notion of concepts at hand, we can formulate \emph{axioms} to capture domain knowledge in a
so-called \emph{Boolean knowledge base (BKB)}. Each BKB consists of a Boolean combination of certain
axioms and an RBox which states the general knowledge about roles.

\begin{definition}[Syntax of \Lmc-axioms over \Nsig and \Lmc-BKBs over \Nsig]
  Let \Lmc be a DL and
  let $\Nsig\coloneqq(\NC,\NR,\NI)$ be the signature. Then, if $C$ and $D$ are \Lmc-concepts over \Nsig, $r$
  and $s$ are a \Lmc-roles over \Nsig, and $\{a,b\}\subseteq\NI$, then
  \begin{itemize}
  \item $C \sqsubseteq D$ (\emph{general concept inclusion, GCI}),
  \item $C(a)$ (\emph{concept assertion}),
  \item $r(a,b)$ (\emph{role assertion}),
  \item $r \sqsubseteq s$ (\emph{role inclusion}), and
  \item $\mathsf{trans}(r)$ (\emph{transitivity axiom})
  \end{itemize}
  are \emph{\Lmc-axioms over \Nsig}.
  %
  Moreover, an \emph{\Lmc-RBox \Rmc over \Nsig} is a finite set of \Lmc-role inclusions over \Nsig and
  \Lmc-transitivity axioms over \Nsig. A \emph{Boolean \Lmc-axiom formula over \Nsig} is defined inductively
  as follows:
  \begin{itemize}
  \item every \Lmc-GCI over \Nsig is a Boolean \Lmc-axiom formula over \Nsig,
  \item every \Lmc-concept and \Lmc-role assertion over \Nsig is a Boolean \Lmc-axiom formula over \Nsig,
  \item if $\Bmc_{1}$, $\Bmc_{2}$ are Boolean \Lmc-axiom formulas over \Nsig, then so are $\lnot\Bmc_{1}$
    (axiom negation) and $\Bmc_{1}\land\Bmc_{2}$ (axiom conjunction), and
  \item nothing else is a Boolean \Lmc-axiom formula over \Nsig.
  \end{itemize}
  %
  Finally, a \emph{Boolean \Lmc-knowledge base (\Lmc-BKB) over \Nsig} is a pair
  $\Bmf = (\Bmc, \Rmc)$, where \Bmc is a Boolean \Lmc-axiom formula over \Nsig and \Rmc is an
  \Lmc-RBox over \Nsig. An \emph{\Lmc-ontology over \Nsig} is an \Lmc-BKB over \Nsig, where only
  axiom conjunction and no axiom negation is allowed in the Boolean axiom formula.
\end{definition}

\noindent
Again as usual in description logics, we use $C \equiv D$ (\emph{concept equivalence}) as abbreviation
for $(C \sqsubseteq D) \land (D \sqsubseteq C)$ and $\Bmc_1\lor\Bmc_2$ (\emph{axiom disjunction}) as
abbreviation for $\lnot(\lnot\Bmc_1\land\lnot\Bmc_2)$.
%
Often an ontology $\Omc = (\Bmc, \Rmc)$ is written as a triple $\Omc = (\Tmc, \Amc, \Rmc)$ where
\Tmc (\emph{TBox}) is the set of all GCIs occurring in \Bmc and \Amc (\emph{ABox}) is the set of all
assertion axioms occurring in \Bmc.


\begin{definition}[Semantics of axioms over \Nsig, BKBs over \Nsig]
  \label{def:semantics-of-axioms}
  An \Nsig-interpretation \I is a model of
  \begin{itemize}
  \item the GCI $C \sqsubseteq D$ over \Nsig if $C^{\I} \subseteq D^{\I}$,
  \item the concept assertion $C(a)$ over \Nsig if $a^{\I} \in C^{\I}$,
  \item the role assertion $r(a,b)$ over \Nsig if $(a^\I,b^\I)\in r^\I$,
  \item the role inclusion $r \sqsubseteq s$ over \Nsig if $r^\I \subseteq s^\I$, and
  \item the transitivity axiom $\mathsf{trans}(r)$ over \Nsig if $r^\I=(r^\I)^+$, where $\cdot^{+}$
    denotes the transitive closure of a binary relation.
  \end{itemize}
  %
  This is extended to Boolean axiom formulas over~\Nsig inductively as follows:
  \begin{itemize}
  \item \I is a model of $\lnot\B_1$ if it is not a model of~$\B_1$, and
  \item \I is a model of $\B_1\land\B_2$ if it is a model of both $\B_1$ and~$\B_2$.
  \end{itemize}
  %
  We write $\I\models\alpha$ and $\I\models\Bmc$ if \I is a model of the axiom~$\alpha$ over~\Nsig
  or \I is a model of the Boolean axiom formula~\B, respectively. Furthermore, \I is a model of an
  RBox~\Rmc over~\Nsig (written $\I\models\Rmc$) if it is a model of each axiom in \Rmc.

  Finally, \I is a model of the BKB $\Bmf=(\B,\Rmc)$ over~\Nsig (written $\I\models\Bmf$) if it is a
  model of both~\B and~\Rmc.  We call~\Bmf \emph{consistent} if it has a model.  The
  \emph{consistency problem} is the problem of deciding whether a given BKB is consistent.
\end{definition}

\noindent
Note that besides the consistency problem there are several other reasoning tasks for description
logics.  The entailment problem, for instance, is the problem of deciding, given a BKB~\Bmf and an
axiom~$\beta$, whether \Bmf \emph{entails} $\beta$, i.e.~whether all models of~\Bmf are also models
of~$\beta$.
%
The consistency problem, however, is fundamental in the sense that most other standard reasoning tasks can be polynomially reduced to it in the presence of axiom negation.
For example, the entailment problem can be reduced to the \emph{in}consistency problem: $\Bmf=(\B,\Rmc)$
entails $\beta$ iff $(\B\land\lnot\beta,\Rmc)$ is inconsistent.  Hence, we focus in this thesis only
on the consistency problem.

To show an example of an ontology, we continue contentwise with American football.

\begin{example}\label{ex:bkb-nfl}
  Consider the following ontology $\Omc = (\Bmc,\emptyset)$ with $\Bmc =$
  \begin{gather*}
    \mathsf{NFC}(\text{\textit{GreenBayPackers}}) \quad\land\\
    \mathsf{playsFor}(\text{\textit{AaronRodgers}}, \text{\textit{GreenBayPackers}})\quad\land\\
    \begin{aligned}
      \exists\mathsf{playsFor}.\mathsf{NFL\_Team} & \sqsubseteq \mathsf{NFL\_Player}\quad\land\\
      \mathsf{NFL\_Team} & \equiv \mathsf{NFC} \sqcup \mathsf{AFC}\quad\land\\
      \mathsf{NFC} \sqcap \mathsf{AFC} & \sqsubseteq \bot.
    \end{aligned}
  \end{gather*}
  The first two axioms assert that the Green Bay Packers are in the NFC and that Aaron Rodgers plays
  for Green Bay. The third axiom states that everybody who plays for an NFL team is an NFL
  player. Finally, the last two axioms define NFL teams as a disjoint union of the NFC and the AFC.

  The ontology \Omc is consistent and Figure~\ref{fig:example-concept} depicts a model of \Omc. Note
  here, that it is not coincidentally that Aaron Rodgers is in the extension of
  $\mathsf{NFL\_Player}$, since \Omc entails
  \begin{gather*}
    \mathsf{NFL\_Player}(\text{\textit{AaronRodgers}}). \qedhere
  \end{gather*}
\end{example}

\subsection{Specific Description Logics}
\label{sec:specific-description-logics}

The specific description logics differ in the available concept and role constructors to formulate
concepts and axioms, and also in the available axioms in a knowledge base.

The prototypical description logic is \ALC, the \emph{attributive language with complement}. There
are no inverse roles and only negation, conjunction and existential restriction are allowed as
concept constructors. Furthermore, only GCIs, concept and role assertions are allowed as
axioms. Hence, only role names are \ALC-roles and an \ALC-RBox is always the empty set.  The DL \ALC
is the smallest propositionally closed DL~\cite{ScSm-AIJ91}.
%
Adding a letter to \ALC stands for certain constructors or axioms that are additionally allowed. For
example, \ALCI additionally allows inverse roles in complex concepts.  By the naming convention of
DLs, specific letters denote a concept or role constructor or a type of axioms that is allowed in
that DL:
\begin{itemize}
\item \Omc means nominals,
\item \Imc means inverse roles,
\item \Qmc means at-least restrictions,
\item \Hmc means role inclusions, and
\item \Smc means transitivity axioms.
\end{itemize}
%
\ALC with additional transitivity axioms is called \Smc instead of $\mathcal{ALCS}$, due to its
connection to the modal logic $\mathsf{S4}$. Thus, \SHOIQ, for example, is the DL that allows all
constructors and axioms which are introduced above. Besides extensions of \ALC, there also exist many
sublogics of \ALC of which we only consider \EL in this thesis. The sub-Boolean description logic
\EL is the fragment of \ALC where only conjunction, existential restriction, and the top concept
(which cannot be expressed as an abbreviation anymore due to the lack of negation) are
admitted.

In~\cite{HoST-IGPL00}, it is shown for \SHQ that allowing arbitrary roles in number restrictions leads to
undecidability of the consistency problem. Decidability can be regained by restricting roles used in
number restrictions to simple roles. To define what a \emph{simple} role is, for a given BKB
$\Bmf = (\Bmc,\Rmc)$, we introduce $\transClosure$ as the transitive-reflexive closure of
$\sqsubseteq$ on
$\Rmc \cup \{\mathsf{Inv}(r)\sqsubseteq\mathsf{Inv}(s) \mid r\sqsubseteq s\in\Rmc\}$ where
$\mathsf{Inv}(r)$ is defined as
\begin{align*}
  \mathsf{Inv}(r) \coloneqq
  \begin{cases}
    r^{-} & \text{ if $r\in\NR$, and} \\
    s    & \text{ if $r$ is an inverse role with $r=s^{-}$}
  \end{cases}
\end{align*}
and $r \equiv_{\Rmc} s$ as abbreviation for $r \transClosure s$ and $s \transClosure r$. A role $r$
is \emph{transitive w.r.t.\ \Rmc} if for some $s$ with $r \equiv_{\Rmc} s$, we have
$\trans{s}\in\Rmc$ or $\trans{\mathsf{Inv}(s)}\in\Rmc$. A role $r$ is called \emph{simple w.r.t.\
  \Rmc} if it is neither transitive nor has any transitive sub-role, i.e. there is no~$s$ such that
$s\transClosure r$ and~$s$ is transitive w.r.t.\ \Rmc.

In the rest of this thesis, we make this restriction to the syntax of \SHQ and all its
extensions.
%
This restriction is also the reason why there are no Boolean combinations of
role inclusions and transitivity axioms allowed in an RBox~\Rmc over~\Nsig in
the above definition.  Otherwise, the notion of a simple role w.r.t.~\Rmc
involves reasoning.  Consider, for instance, the Boolean combination of axioms
$(\mathsf{trans}(r)\lor\mathsf{trans}(s))\land r\sqsubseteq s$.  It should be
clear that~$s$ is not simple, but this is no longer a pure syntactic check.

The complexity of the consistency problem for DL ontologies is well-investigated. Is is
\ExpTime-complete for any DL between \ALC and \SHOQ and \NExpTime-complete for \SHOIQ. The lower
bound for \ALC was shown in~\cite{Sch-IJCAI91}, the upper bound for \SHOQ in~\cite{Tob-PhD01}. For
\SHOIQ the lower and upper bound were proven in~\cite{Tob-JAIR00} and~\cite{Tob-PhD01},
respectively. While for BKBs the complexity class stays the same, this is much less explored. It is
in \ExpTime for \SHOQ~\cite{Lip-PhD14} and it remains in \NExpTime for \SHOIQ as a consequence of
Theorem~2 in \cite{Pra-JLLI05}.

For \EL-ontologies, the consistency problem is trivial since no contradictions can be expressed and,
thus, every \EL-ontology is consistent. On the other hand, we do not consider \EL-BKBs as it seems
very unnatural to admit axiom negation while denying concept negation at the same time.

\chapter{The Contextualized Description Logic LMLO}
\label{cha:context-dls}

\todo[inline]{somewhere a few paragraphs about existing cdls and why they are not feasible in my
  setting.}


Some introduction

As mentioned in the introduction classical description logics lack the expressive power to capture
knowledge about \ldots

mention somewhere what \LM and \LO are.

\todo[inline]{introductory paragraphs to this chapter}


\section{Syntax and Semantics of Contextualized Description Logics}
\label{sec:syn-seman-cdl}

conceptually we start with an description logic \LM on the meta level. Here we can state our
knowledge about contexts. We enrich this with \ldots 

\todo[inline]{basic ideas how the syntax works}


Throughout this chapter, let $\Msig = (\MC, \MR, \MI)$ and $\Osig = (\OC, \OR, \OI)$ denote the
signatures for \LM and \LO. Thus, we call \MC, \MR, \MI, \OC, \OR and \OI, respectively, the set of
\emph{meta concept}, \emph{role} and \emph{individual names} and \emph{object concept}, \emph{role}
and \emph{individual names}.

\begin{definition}[Syntax of \LMLO]\label{def:syntax-cdls}
  A \emph{concept of the object logic~\DLinner (o-concept)} is an \DLinner-concept over~\Osig.  An
  \emph{o-axiom} is an \DLinner-GCI over~\Osig, an \DLinner-concept assertion over~\Osig, or an
  \DLinner-role assertion over~\Osig.

  The set of \emph{concepts of the meta logic~\DLouter (m-concepts)} is the smallest set such that
  \begin{itemize}
  \item for all $A\in\MC$, $A$ is a meta concept (\emph{\todo{here I'm struggling with names. I need
        some distinction between ``pure meta concept'' and ``referring meta concepts'' as I need the
        second name very often}???}),
  \item for all o-axioms $\alpha$, \oalpha is a meta concept (\emph{???}), and
  \item all complex concepts that can be built with the concept constructors allowed in \LM are meta
    concepts.
  \end{itemize}
  
  An \emph{m-axiom} is \todo{what exactly is an m-axiom?}

  Analogously a \emph{Boolean m-axiom formula} is defined inductively as follows:
  \begin{itemize}
  \item every m-axiom is a Boolean m-axiom formula,
  \item if $\B_1,\B_2$ are Boolean m-axiom formulas, then so are $\lnot\B_1$ and $\B_1\land\B_2$,
    and
  \item nothing else is a Boolean m-axiom formula.
  \end{itemize}
    %
  Finally, a \emph{Boolean \LMLO-knowledge base (\LMLO-BKB)} is a triple $\Bmf=(\B,\RO,\RM)$ where
  \RO is an \LO-RBox over~\Osig, \RM an \LM-RBox over~\Msig, and \B is a Boolean m-axiom formula. An
  \emph{\LMLO-ontology} is an \LMLO-BKB, where only axiom conjunction and no axiom negation is
  allowed in the Boolean m-axiom formula.
\end{definition}

For the same reasons as mentioned in section \ref{sec:description-logics}, role inclusions
over~\Osig and transitivity axioms over~\Osig are not allowed to constitute m-concepts.  However, we
fix an RBox~\RO over~\Osig that contains such o-axioms and holds in \emph{all} contexts.  The same
applies to role inclusions over~\Msig and transitivity axioms over~\Msig, which are only allowed to
occur in a RBox~\RM over~\Msig.

Again, we use the usual abbreviations (for disjunctions etc.) for m-concepts and
Boolean m-axiom formulas.

The semantics of \LMLO is defined by the notion of \emph{nested interpretations}.  These consist of
\Osig-interpretations for the specific contexts and an \Msig-interpretation for the relational
structure between them.  We assume that all contexts speak about the same non-empty domain
(\emph{constant domain assumption}). \todo{here something about constant domain assumption,
  references.}

As argued earlier, sometimes it is desired that concepts or roles in the object logic are
interpreted the same in all contexts. Therefore we introduce \emph{rigid names}. Let
$\OCR\subseteq\OC$ be the set of \emph{rigid object concept names} and $\ORR\subseteq\OR$ be the set
of \emph{rigid object role names}.
%
Often, we refer to \OCR and \ORR simply as \emph{rigid concepts} and \emph{rigid roles} as there is
no such notion on the meta level.
%
We call concept names and role names in $\OC\setminus\OCR$ and $\OR\setminus\ORR$ \emph{flexible}.
%
Moreover, we assume that individuals of the object logic are always interpreted the same in all
contexts (\emph{rigid individual assumption}). \todo{reference to RIA}

\begin{definition}[Nested interpretation]\label{def:nested-interpretation}
  A \emph{nested interpretation} is a tuple \todo{not nice!}\\ \JJ, where \Cbb is a non-empty set (called
  \emph{contexts}) and $(\Cbb,\cdot^\J)$ is an \Msig-interpretation.
  %
  Moreover, for every $c\in\Cbb$, $\I_c\coloneqq(\Delta^{\J},\cdot^{\I_c})$ is an \Osig-interpretation
  such that we have for all $c,c'\in\Cbb$ that $x^{\I_{c}}=x^{\I_{c'}}$ for every
  $x\in\OI\cup\OCR\cup\ORR$.
\end{definition}

We are now ready to define the semantics of \LMLO.

\begin{definition}[Semantics of \LMLO]
  Let \JJ be a nested interpretation.  The mapping $\cdot^\J$ is extended to o-axioms
  \todo{referenced meta concept} as follows: $\oalpha^\J:=\{c\in\Cbb\mid\I_c\models\alpha\}$.

    Moreover, \J is a model of the m-axiom $\beta$ if $(\Cbb,\cdot^\J)$ is a model
    of $\beta$.  This is extended to Boolean m-axiom formulas inductively as
    follows:
    \begin{itemize}
        \item \J is a model of $\lnot\B_1$ if it is not a model of $\B_1$, and
        \item \J is a model of $\B_1\land\B_2$ if it is a model of both $\B_1$
            and $\B_2$.
    \end{itemize}
    We write $\J\models\B$ if \J is a model of the Boolean m-axiom formula~\B.
    Furthermore, \J is a model of~\RM (written $\J\models\RM$) if
    $(\Cbb,\cdot^\J)$ is a model of~\RM, and \J is a model of~\RO (written
    $\J\models\RO$) if $\I_{c}$ is a model of~\RO for all $c\in\Cbb$.
    
    Finally, \J is a model of the \con{\DLouter}{\DLinner}-BKB \BB (written
    $\J\models\Bmf$) if \J is a model of~\B, \RO, and~\RM.  We call~\Bmf
    \emph{consistent} if it has a model.

    The \emph{consistency problem in \LMLO} is the problem of deciding whether a given
    \LMLO-BKB is consistent.
\end{definition}



\todo[inline]{some words about the next example.}

\begin{example}
  
\end{example}

\section{Complexity of the Consistency Problem}
\label{sec:complexity-consis-problem}

\todo[inline]{somewhere a few words about top down view, can't go up, because of that its possible
  to divide reasoning problem into two subtasks, later on }

\todo[inline]{results later}

\todo[inline]{paragraph that we will handle lower bounds in 3.3 where we talk about EL}

For the upper bounds, let in the following \BB be an \LMLO-BKB.  We proceed
similar to what was done for \ALC-LTL in~\cite{BaGL-KR08,BaGL-ToCL12} (and
\SHOQ-LTL in~\cite{Lip-PhD14}) and reduce the consistency problem to two separate
decision problems.

For the first problem, we consider the so-called \emph{outer abstraction}, which is the \LM-BKB
over~\Msig obtained by replacing each \todo{referring m-concept?} m-concept of the form \oalpha
occurring in~\B by a fresh concept name such that there is a 1--1 relationship between them.

\begin{definition}[Outer abstraction]
  Let \BB be an \LMLO-BKB.  Let \bsf be the bijection mapping every m-concept of the form \oalpha
  occurring in~\B to the concept name $A_{\oalpha}\in\MC$, where we assume w.l.o.g.\ that
  $A_{\oalpha}$ does not occur in~\B.
  \begin{enumerate}
  \item The \LM-concept $C^{\bsf}$ over \Msig is obtained from the m-concept~$C$ by replacing every occurrence of
    \oalpha by $\bsf(\oalpha)$.
  \item The Boolean \LM-axiom formula~\Bb over \Msig is obtained from~\B by replacing every
    m-concept $C$ occurring in~\B with~$C^{\bsf}$.  We call the \DLouter-BKB $\Bmfb=(\Bb,\RM)$ the
    \emph{outer abstraction of~\Bmf}.
        \item Given \JJ, its \emph{outer abstraction} is the
            \Msig-interpretation $\Jb=(\Cbb,\cdot^{\Jb})$ where
            \begin{itemize}
                \item for every $x\in\MR\cup\MI\cup(\MC\setminus\ran(\bsf))$, we
                    have $x^{\Jb}=x^\J$, and
                \item for every $A\in\ran(\bsf)$, we have
                    $A^{\Jb}=(\bsf^{-1}(A))^\J$,
            \end{itemize}
            where $\ran(\bsf)$ denotes the image of~\bsf. \qedhere
    \end{enumerate}
\end{definition}

For simplicity, for $\Bmf'=(\B',\RO,\RM)$ where $\B'$ is a subformula of~\B, we
denote by $(\Bmf')^\bsf$ the outer abstraction of~$\Bmf'$ that is obtained by
restricting \bsf to the m-concepts occurring in~$\B'$.
%
Now let us consider the following small example.


\begin{example}\label{ex:outer-abstraction}
  Let $\Bmf_{\text{ex}} = (\B_{\text{ex}},\emptyset,\emptyset)$ with $\B_{\text{ex}}\coloneqq
  C\sqsubseteq(\oax{A\sqsubseteq\bot})\ \land\ (C\sqcap\oax{A(a)})(c)$ be an \ALCALC-BKB.  Then,
  \bsf maps $\oax{A\sqsubseteq\bot}$ to $A_{\oax{A\sqsubseteq\bot}}$ and $\oax{A(a)}$ to
  $A_{\oax{A(a)}}$.  Thus, we have that
  \begin{align*}
    \Bmf_{\text{ex}}^\bsf\coloneqq \Big(C\sqsubseteq(A_{\oax{A\sqsubseteq\bot}})\ \land\ (C\sqcap
    A_{\oax{A(a)}})(c),\ \emptyset\Big)
  \end{align*}
  is the outer abstraction of $\Bmf_\text{ex}$.
\end{example}

The following lemma makes the relationship between \Bmf and its outer abstraction
\Bmfb explicit.  It is proved by induction on the structure of~\B.

\begin{lemma}\label{lem:interpretation-outer-abstraction}
  Let \J be a nested interpretation such that \J is a model of \RO.  Then, \J is a model of \Bmf iff
  $\Jb$ is a model of~\Bmfb.
\end{lemma}

\begin{proof}
  Since $r^{\J}=r^{\Jb}$ for all \LM-role $r$ over \Msig, we have that \J is a model of \RM iff \Jb is a model of
  \RM. Thus, it is only left to show that for any m-axiom $\gamma$ occurring in \B, it holds that
  $\J \models \gamma$ iff $\Jb \models \gamma^{\bsf}$.

  \begin{claim}
    For any $x \in \Cbb$ it holds that $x \in C^{\J}$ iff
    $x \in (C^{\bsf})^{\Jb}$.
  \end{claim}

  \begin{claimproof}
    We prove the claim by induction on the structure of $C$: \todo{müssen hier noch inverse rollen
      betrachtet werden? eigentlich nicht.}

    \begin{tabularx}{\linewidth}{@{}l@{ }X@{}}
      $C = A \in \MC\!\setminus\!\ran(\bsf)$: 
      & $x \in A^{\J}$ 
        iff $x \in (A^{\bsf})^{\Jb}$ by definition of $\Jb$ and since $A = A^{\bsf}$ 
      \\[1ex]
      $C = \oalpha$:
      & $x \in \oalpha^{\J}$
        iff $x \in (A_{\oalpha})^{\Jb}$
        iff $x \in (\oalpha^{\bsf})^{\Jb}$
      \\[1ex] 
      $C = \lnot D$:
      & $x \in (\lnot D)^{\J}$ 
        iff $x \notin D^{\J}$ 
        iff, by induction hypothesis, $x \notin (D^{\bsf})^{\Jb}$ 
        iff $x \in (\lnot D^{\bsf})^{\Jb}$ 
        iff $x \in ((\lnot D)^{\bsf})^{\Jb}$ 
      \\[1ex]
      $C = D \sqcap E$: 
      & $x \in (D \sqcap E)^{\J}$
        iff $x \in D^{\J}$ and $x \in E^{\J}$ 
        iff, by induction hypothesis, $x \in (D^{\bsf})^{\Jb}$ and $x \in
        (E^{\bsf})^{\Jb}$
        iff $x \in (D^{\bsf} \sqcap E^{\bsf})^{\Jb}$
        iff $x \in ((D \sqcap E)^{\bsf})^{\Jb}$ 
      \\[1ex]
      $C = \exists r.D$: 
      & $x \in (\exists r.D)^{\J}$
        iff there exists $y \in \Cbb$ \suth $(x,y) \in r^{\J}$ and $y \in D^{\J}$
        iff there exists $y \in \Cbb$ \suth $(x,y) \in r^{\Jb}$ and $y \in (D^{\bsf})^{\Jb}$
        iff $x \in (\exists r.D^{\bsf})^{\Jb}$ 
        iff $x \in ((\exists r.D)^{\bsf})^{\Jb}$ 
      \\[1ex]
      $C = \{a\}$:
      & $x\in\{a\}^{\J}$ 
        iff $x\in(\{a\}^{\bsf})^{\Jb}$ by definition of $\Jb$ and since $\{a\} = \{a\}^{\bsf}$ 
      \\[1ex]
      $C =\ \atleast{n}{r}{D}$:
      & $x \in (\atleast{n}{r}{D})^{\J}$
        iff there are at least $n$ elements $y \in \Cbb$ s.t.\ $(x,y) \in r^{\J}$ and $y \in D^{\J}$
        iff there are at least $n$ elements $y \in \Cbb$ s.t.\ $(x,y) \in r^{\Jb}$ and $y \in (D^{\bsf})^{\Jb}$
        iff $x \in (\atleast{n}{r}{D^{\bsf}})^{\Jb}$
        iff $x \in ((\atleast{n}{r}{D})^{\bsf})^{\Jb}$ 
    \end{tabularx}

    \vspace{-2.0\baselineskip}
  \end{claimproof}

  If $\gamma$ is of the form $C \sqsubseteq D$, we have that $\J \models C
  \sqsubseteq D$ iff $x \in C^{\J}$ implies $x \in D^{\J}$
  iff (by claim) $x
  \in (C^{\bsf})^{\Jb}$ implies $x \in (D^{\bsf})^{\Jb}$ iff
  $\Jb \models C^{\bsf} \sqsubseteq D^{\bsf}$.

  If $\gamma$ is of the form $C(a)$, we have that $\J \models C(a)$ iff
  $a^{\J} \in C^{\J}$ iff (by claim) $a^{\Jb} \in
  (C^{\bsf})^{\Jb}$ iff $\Jb \models C^{\bsf}(a)$.

  If $\gamma$ is of the form $r(a,b)$, we have that $\J \models r(a,b)$ iff
  $(a^{\J}, b^{\J}) \in r^{\J}$ iff $(a^{\Jb},
  b^{\Jb}) \in r^{\Jb}$ iff $\Jb \models
  r(a,b)$.

  If \B is of the form $\lnot\B_{1}$, we have that $\J\models\B$ iff not
  $\J\models\B_{1}$ iff not $\Jb\models\Bb_1$ iff $\Jb\models\Bb$.

  If \B is of the form $\B_{1}\land\B_{2}$, we have that $\J\models\B$ iff $\J\models\B_{1}$ and
  $\B_{2}$ iff $\Jb\models\Bb_{1}$ and $\Jb\models\Bb_{2}$ iff $\Jb\models\Bb$.

  Since $\J\models\RO$, $\J\models\RM$ iff $\Jb\models\RM$ and $\J\models\B$ iff $\Jb\models\Bb$, we
  have $\J\models\Bmf$ iff $\Jb\models\Bmfb$.
\end{proof}

Note that this lemma yields that consistency of~\Bmf implies consistency of~\Bmfb.  Thus, the
consistency of~\Bmfb is a necessary condition for the consistency of~\Bmf.  However, it is not
sufficient since the converse does not hold as the following example shows.

\begin{example}\label{ex:outer-abstraction-continued}
  Consider again $\Bmf_\text{ex}$ of Example~\ref{ex:outer-abstraction}.
  %
  Take any \Msig-interpretation $\Hmc=(\Delta^{\Hmc},\cdot^\Hmc)$ with $\Delta^{\Hmc}=\{e\}$,
  $d^\Hmc=e$, and $C^\Hmc = A_{\oax{A\sqsubseteq\bot}}^\Hmc = A_{\oax{A(a)}}^\Hmc = \{e\}$.

  Clearly, \Hmc is a model of~$\Bmf_\text{ex}^{\bsf}$.  But there is no nested interpretation~\JJ
  with $\J\models\Bmf_\text{ex}$ since this would imply $\Cbb=\Delta^{\Hmc}$, and that $\I_e$ is a model of
  both $A\sqsubseteq\bot$ and $A(a)$, which is not possible.
\end{example}

The above example illustrates that there exist implicit restrictions on the interpretation of the
meta level as certain combinations of concept names in $\ran(\bsf)$ are not allowed.  Therefore, we
need to ensure that these are not treated independently.  For expressing such a restriction on the
model~\Hmc of~\Bmfb, we adapt a notion of~\cite{BaGL-KR08,BaGL-ToCL12}. It is also worth noting that
this problem occurs also in much less expressive DLs such as \ELbot (i.e.~\EL extended with the
bottom concept).

\begin{definition}[\mbox{\Nsig-interpretation (weakly) respects $(\Umc,\Ymc)$}]
  \label{def:int-respects-D} 
  Let $\Umc\subseteq\NC$, let $\Ymc\subseteq\powerset{\Umc}$ and let \II be an \Nsig-interpretation.
  The \emph{type of $d\in\Delta^{\I}$ w.r.t.~\Umc} is defined as
  $\mathsf{type}_{\Umc}(d) \coloneqq \{A \in \Umc \mid d \in A^{\I}\}$.  The interpretation \I
  \emph{respects} $(\Umc,\Ymc)$ if $\Zmc = \Ymc$ where
  \begin{align*}
    \Zmc & \coloneqq\{Y\subseteq\Umc\mid\text{there is some $d\in\Delta^\I$ with
           $\mathsf{type}_{\Umc}(d) = Y$}\}
  \end{align*}

    It \emph{weakly respects} $(\Umc,\Ymc)$ if $\Zmc \subseteq \Ymc$.
\end{definition}

\todo[inline]{a few words about the meaning of the definition, explanation. Types usual definition}


The second decision problem that we use for deciding consistency is needed to make sure that such a
set of concept names is admissible in the following sense.

\begin{definition}[Admissibility]\label{def:admissibility}
  Let $\Xmc=\{X_1,~\dots,\ X_k\}\subseteq\powerset{\ran(\bsf)}$.  We call \Xmc \emph{admissible} if
  there exist \Osig-interpretations $\I_1=(\Delta,\cdot^{\I_1})$,~\dots,
  $\I_k=(\Delta,\cdot^{\I_k})$ such that
  \begin{itemize}
  \item $x^{\I_i}=x^{\I_j}$ for all $x\in\OI\cup\OCR\cup\ORR$ and all $i,j\in\{1,\dots,k\}$, and
  \item every $\I_i$, $1\le i\le k$, is a model of the \LO-BKB $\Bmf_{X_{i}}= (\B_{X_i},\RO)$
    over~\Osig where
    \begin{align*}
      \B_{X_i}:=\bigwedge_{\bsf(\oalpha)\in X_i}\alpha\ \land
      \bigwedge_{\bsf(\oalpha)\in\ran(\bsf)\setminus X_i}\lnot\alpha.
    \end{align*}
  \end{itemize}
  \vspace{-1.7\baselineskip}
\end{definition}

Note that any subset $\Xmc'\subseteq\Xmc$ is admissible if \Xmc is admissible.
%
Intuitively, the sets $X_i$ in an admissible set \Xmc consist of concept names \todo{referring concept
  names} such that the corresponding o-axioms \enquote{fit together}.  Consider again
Example~\ref{ex:outer-abstraction-continued}.  Clearly, the set
$\{A_{\oax{A\sqsubseteq\bot}},A_{\oax{A(a)}}\}\in\powerset{\ran(\bsf)}$ \emph{cannot} be contained
in any admissible set~\Xmc.  

The next definition captures the above mentioned restriction on the model~\Hmc
of~\Bmfb.

\todo[inline]{explanation of outer consistency}

\begin{definition}[Outer consistency]\label{def:outer-consistency}
  Let $\Xmc \subseteq \powerset{\ran(\bsf)}$.  We call the \DLouter-BKB~\Bmfb over \Msig \emph{outer
    consistent w.r.t.~\Xmc} if there exists a model of~\Bmfb that weakly respects~$(\ran(\bsf),\Xmc)$.
\end{definition}

The next two lemmas show that the consistency problem in \LMLO can be decided by checking whether
there is an admissible set~\Xmc and the outer abstraction of the given \LMLO-BKB is outer consistent
w.r.t.~\Xmc.

\begin{lemma}\label{lem:model-equivalent-to-admissible}
  For every \Msig-interpretation \HH, the following two statements are equivalent:
  \begin{enumerate}
  \item There exists a model~\J of~\Bmf with $\Jb=\Hmc$.
  \item \Hmc is a model of~\Bmfb and the set $\{X_d\mid d\in\Gamma\}$ is admissible, where $X_d$ is
    defined as $X_{d}:=\{A\in\ran(\bsf)\mid d\in A^\Hmc\}$.
  \end{enumerate}
\end{lemma}

\begin{proof}
  (1 $\Rightarrow$ 2): Let \JJ be a model of~\Bmf with $\Jb=\Hmc$.  Since $\Jb=\Hmc$, we have that
  $\Cbb=\Delta^{\Hmc}$.  By Lemma~\ref{lem:interpretation-outer-abstraction}, we have that \Hmc is a
  model of~\Bmfb.
    %
  Moreover, since \bsf is a bijection between m-concepts of the form \oalpha occurring in~\Bmf and
  concept names of~\MC, we have that $\ran(\bsf)$ is finite, and thus also the set
  $\Xmc \coloneqq \{X_d\mid d \in \Delta^{\Hmc} \} \subseteq \powerset{\ran(\bsf)}$ is finite.  Let
  $\Xmc = \{Y_1, \dots, Y_k\}$.  Since $\Cbb = \Delta^{\Hmc}$, there exists an index function
  $\nu\colon\Cbb\to\{1,\dots,k\}$ such that $X_c = Y_{\nu(c)}$ for every $c\in\Cbb$, i.e.
  \begin{align*}
    Y_{\nu(c)} & = \bigl\{\bsf(\oalpha)\mid\text{\oalpha occurs in~\Bmf and}\
                 c\in\oalpha^\Hmc\bigr\} \\
               & =  \bigl\{\bsf(\oalpha)\mid\text{\oalpha occurs in~\Bmf and}\ \I_c\models\alpha\bigr\}.
  \end{align*}
  Conversely, for every $\mu\in\{1,\dots,k\}$, there is an element $c\in\Cbb$ such that
  $\nu(c)=\mu$.
    % 
  The \Osig-interpretations for showing admissibility of~\Xmc are obtained as follows.  Take
  $c_1,\dots,c_k \in \Cbb$ such that $\nu(c_1) = 1$,~\dots, $\nu(c_k) = k$.  Now, for every~$i$,
  $1 \leq i \leq k$, we define the \Osig-interpretation $\Gmc_i:=(\Delta,\cdot^{\I_{c_i}})$.
  Clearly, we have that $Gmc_i\models\B_{Y_i}$ and since $\J\models\RO$, we have that
  $Gmc_i\models\Bmf_{Y_i}$.  Moreover, the definition of a nested interpretation yields that
  $x^{\Gmc_i}=x^{\Gmc_j}$ for all $x\in\OI\cup\OCR\cup\ORR$ and all $i,j \in \{1,\dots,k\}$.  Hence,
  the \Osig-interpretations $\Gmc_1, \dots, \Gmc_k$ attest admissibility of~\Xmc.

  (2 $\Rightarrow$ 1): Assume that \HH is a model of~\Bmfb and that the set
  $\Xmc \coloneqq \{X_d\mid d\in\Gamma\}$ is admissible.  Again, since $\ran(\bsf)$ is finite, we
  have that $\Xmc \subseteq \powerset{\ran(\bsf)}$ is finite.  Let $\Xmc = \{Y_1,\dots,Y_k\}$.
  Since \Xmc is admissible, there are \Osig-interpretations $\Gmc_1=(\Delta,\cdot^{\Gmc_1})$,~\dots,
  $\Gmc_k=(\Delta,\cdot^{\Gmc_k})$ such that $\Gmc_i\models\Bmf_{Y_i}$ and $x^{\Gmc_i}=x^{\Gmc_j}$
  for all $x\in\OI\cup\OCR\cup\ORR$ and all $i,j\in\{1,\dots,k\}$.
    %
  Furthermore, there exists an index function $\nu\colon\Gamma\to\{1,\dots,k\}$ such that
  $Y_{\nu(d)}=X_d$ for every $d\in\Gamma$.
    %
  We define a nested interpretation \JJ as follows:
  \begin{itemize}
  \item $\Cbb \coloneqq \Delta^{\Hmc}$;
  \item $x^\J \coloneqq x^\Hmc$ for every $x\in\MC\cup\MR\cup\MI$; and
  \item $x^{\I_c}:=x^{\Gmc_{\nu(c)}}$ for every $x \in \OC \cup \OR \cup \OI$ and every $c \in \Cbb$.
  \end{itemize}
    %
  By construction of \J, we have that $x^{\Jb} = x^\Hmc$ for every
  $x \in \MR \cup \MI \cup (\MC \setminus \ran(\bsf))$.
    %
  Let $A \in \ran(\bsf)$, and let $\bsf^{-1}(A) = \oalpha$.  We have for every $d \in \Gamma = \Cbb$
  that $d\in A^{\Jb}$ iff $d\in(\bsf^{-1}(A))^\J$ iff $d \in \oalpha^\J$ iff $\I_d \models \alpha$
  iff $\Gmc_{\nu(d)}\models\alpha$ iff $\bsf(\oalpha)=A\in Y_{\nu(d)}$ (since
  $\Gmc_{\nu(d)}\models\B_{Y_{\nu(d)}}$) iff $A\in X_d$ iff $d\in A^\Hmc$.
    %
  Hence, we have $\Jb=\Hmc$.
    %
  Since \Hmc is a model of~\Bmfb and, by construction of \J, \J is a model of \RO, we have by
  Lemma~\ref{lem:interpretation-outer-abstraction} that \J is a model of~\Bmf.

  \todo[inline]{noch nicht drüber gelesen, hier müssen sicher noch einige Formelzeichen angepasst
    werden. Und ist er verständlich???}
\end{proof}

The following lemma is a consequence of the previous one.

\todo[inline]{a few more words? :/}

\begin{lemma}\label{lem:admissible-and-outerConsistent}
    The \LMLO-BKB~\Bmf is consistent iff there is a set
    $\Xmc=\{X_1,\dots,X_k\}\subseteq\powerset{\ran(\bsf)}$ such that
    \begin{enumerate}
        \item \Xmc is admissible, and
        \item \Bmfb is outer consistent w.r.t.~\Xmc.
    \end{enumerate}
\end{lemma}

\begin{proof}
  \onlyifdirection Let \J be a model of \Bmf, and let $\Jb=(\Cbb,\cdot^{\Jb})$.  By
  Lemma~\ref{lem:model-equivalent-to-admissible}, we have that $\Jb$ is a model of~\Bmfb, and the
  set $\Xmc \coloneqq \{X_c \mid c \in \Cbb\}$ is admissible.  By construction, $\Jb$ weakly
  respects $(\ran(\bsf),\Xmc)$, and hence \Bmfb is outer consistent w.r.t.~\Xmc.
    
  \ifdirection Let $\Xmc = \{X_1,\dots,X_k\}\subseteq\powerset{\ran(b)}$ such that \Xmc is
  admissible and \Bmfb is outer consistent w.r.t.~\Xmc.  Hence there is a model
  $\Gmc=(\Cbb,\cdot^\Gmc)$ of~\Bmfb that weakly respects $(\ran(\bsf),\Xmc)$.
    %
  We define $\Xmc' \coloneqq \{Y_c \mid c\in\Cbb\}$, where
  $Y_c \coloneqq \{A \in \ran(b) \mid c \in A^\Gmc\}$.  Since \Gmc weakly respects
  $(\ran(\bsf),\Xmc)$ and $c \in (C_{\ran(b),Y_c})^\Gmc$ for every $c \in \Cbb$, we have that
  $\Xmc' \subseteq \Xmc$.  Since \Xmc is admissible, this yields admissibility of~$\Xmc'$.
  Lemma~\ref{lem:model-equivalent-to-admissible} yields now consistency of~\Bmf.
  %
  \todo[inline]{noch nicht drübergelesen!}
\end{proof}


\todo[inline]{some text}

\begin{lemma}\label{lem:shoq-outer-consisteny-exptime}
  Deciding whether a \cSHOQ-BKB \Bmfb is outer consistent w.r.t.~\Xmc can be done in time
  exponential in the size of~\Bmfb and linear in size of~\Xmc.
\end{lemma}

\begin{proof}
  It is enough to show that deciding whether~\Bmfb has a model that weakly respects
  $(\ran(\bsf),\Xmc)$ can be done in time exponential in the size of~\Bmfb and linear in the size
  of~\Xmc.  It is not hard to see that we can adapt the notion of a quasimodel respecting a pair
  $(\Umc,\Ymc)$ of~\cite{Lip-PhD14} to a quasimodel \emph{weakly} respecting $(\Umc,\Ymc)$.  Indeed,
  one just has to drop Condition~(i) in Definition~3.25 of~\cite{Lip-PhD14}.  Then, the proof of
  Lemma~3.26 there can be adapted such that our claim follows.  This is done by dropping one check
  in Step~4 of the algorithm of~\cite{Lip-PhD14}.

    \todo[inline]{complete rewrite necessary}
\end{proof}

\begin{lemma}\label{lem:shoiq-outer-consisteny-exptime}
  Deciding whether a \cSHOIQ-BKB \Bmfb is outer consistent w.r.t.~\Xmc can be non-deterministically
  done in time exponential in the size of~\Bmfb and linear in size of~\Xmc.
\end{lemma}

\missingproof


\subsection{Consistency without rigid names}
\label{sec:cons-without-rigid}

\subsection{Consistency with rigid concept and role names}
\label{sec:cons-with-rigid}




\subsection{Consistency with only rigid concept names}
\label{sec:cons-with-only}



\section{The Case of \EL}
\label{sec:case-el}

For the sake of completeness 



\missingproof


%%% Local Variables:
%%% mode: latex
%%% TeX-master: "../thesis"
%%% reftex-default-bibliography: ("../references.bib")
%%% End:



\section{Adding Contextualised Concepts}
\label{sec:adding-cont-concepts}


In this section we discuss a possible extension to our contextualized description logic.  We start
with a little comparison to temporal description logics. Both DL-LTL-structures and nested DL
interpretations use a \emph{possible worlds semantics}. Single time points or contexts are
represented in a meta dimension and for each such a meta element, or possible world, there exists
one DL interpretation on the object level. Important for both the expressivity and the complexity of
reasoning problems is how these two dimensions can interact. Syntactically, it is a question where
these \emph{meta} operators, i.e.\ temporal or contextual, can be used.

\begin{table}
  \caption{Classification of different two-dimensional temporal and contextual description logics
    (\cite{LuWZ-TIME08,BaGL-ToCL12,KG-JELIA10}) }
  \centering
  \begin{tabular}{lM{0.1\linewidth}M{0.25\linewidth}M{0.17\linewidth}N}
    \toprule
    \multicolumn{2}{c}{\multirow{2}{5cm}[-1ex]{\centering Temporal or contextual operators}}
    & \multicolumn{2}{c}{in front of/around axioms}\\
    & & yes & no &\\[10pt]
    \midrule
    & yes & temporal $\text{LTL}_{\ALC}$, \klarALC, \LMLOplus & $\text{LTL}_{\ALC}$, \hspace{2cm}$\emptyset$ &\\[15pt]
    \cmidrule{2-4}
    \multirow{-2}{*}[2.4ex]{
    \centering inside concepts
    }& no & \ALC-LTL, \hspace{2cm}\LMLO & \ALC &\\[15pt]
    \bottomrule
  \end{tabular}
  \label{tab:classification-tdl-cdl}
\end{table}

Table~\ref{tab:classification-tdl-cdl} gives an overview over some two-dimensional DLs. This is not
a complete overview, but it illustrates some common properties about the complexity of the
consistency problem. Note first that neither having meta operators in front of axioms nor having
meta operators inside concepts is strictly more expressive than the other. Meta operators in front
of axioms can handle general knowledge that holds in some or all worlds.  On the other hand, meta
operators inside object concepts allow to access the extension of concepts in other worlds.
%
We illustrate this by an example taken from \cite{LuWZ-TIME08}.
\begin{gather*}
  \Diamond\Box(\mathsf{European\_country} \sqsubseteq \mathsf{EU\_country})\\
  \mathsf{Independent\_country} \sqsubseteq \Box\mathsf{Independent\_country}
\end{gather*}
The first temporalised GCI states that eventually, i.e.\ at some time point $t$ in future, all
European countries will forever be EU members, i.e.\ in every time point after $t$. The second axiom
says that the extension of the concept $\mathsf{Independent\_country}$ does not decrease.

In \LMLO, we only have the contextual operator $\oax{\cdot}$ around axioms. We can express that
certain axioms must hold in some worlds, but we cannot access the extension of a concept from
another world, i.e.~we cannot express the set of domain elements which belong to some concept in
another context.  The idea is to overcome this lack of expressive power by introducing a new
contextual operator inside object concepts.

Before extending \LMLO, we analyse some general behaviour of two-dimensional DLs.  The first common
property that we want to focus on is the computational complexity if rigid roles are present.
If meta operators are allowed within object concepts, the consistency problem becomes
undecidable. This holds for all logics in the first row of
Table~\ref{tab:classification-tdl-cdl}. We prove this negative result for \LMLOplus below. If meta operators
are only allowed in front of axioms, we may retain decidability, but at the cost that the
consistency problem is one exponential harder.
%
If no rigid names are allowed, the complexity of the consistency problem increases for
logics with meta operators both in front of axioms and in object concepts: from ExpTime-completeness for \ALC
to \ExpSpace-completeness for temporal $\text{LTL}_{\ALC}$ TBoxes and to \TwoExpTime-completeness
for \klarALC knowledge bases. If only one kind of meta operators is allowed, the complexity
class stays the same, i.e.\ the
consistency problem is ExpTime-complete in \ALCALC, \ALC-LTL, and $\text{LTL}_{\ALC}$.

A setting in which only rigid concepts, but no rigid roles are allowed, is only interesting if meta
operators are not allowed inside object concepts. Otherwise, rigid concepts can easily be
simulated. In $\text{LTL}_{\ALC}$, this can be done by adding $C\sqsubseteq\Box C$ and
$\lnot C\sqsubseteq\Box\lnot C$ to the TBox.  We show below how rigid concepts can be simulated in
\LMLOplus.
%
The contextualized description logic \LMLOplus is an extension of \LMLO in which we additionally allow
contextualized object concepts. Therefore, we update the definition of o-concepts from
Def. \ref{def:syntax-lmlo}.  

\begin{definition}[Object concepts of \LMLOplus]
The set of \emph{concepts of the object logic \LO (o-concepts)} is the smallest set such that
\begin{itemize}
\item for all $A\in\OC$, $A$ is an object concept,
\item if $D$ is an object concept, $\mathbf{C}\in\MC$, $\mathbf{r}\in\MR$, then $\ocont{C}[r]{D}$ is
  also an object concept, and
\item all complex concepts that can be built with the concept constructors allowed in \LO are
  object concepts.
\end{itemize}

Furthermore, for a nested interpretation \JJ the mapping $\cdot^{\I_{c}}$ is extended to
$\ocont{C}[r]{D}$ as follows: $(\ocont{C}[r]{D})^{\I_{c}} \coloneqq \{d \in \Delta^{\J} \mid \text{there is
some $c'\in\Cbb$ s.t.\ $(c,c')\in r^{\J}$ and $d \in D^{\I_{c'}}$}\}$.
\end{definition}

Following customs of modal logic, we use $\ocont*{C}[r]{D}$ as an abbreviation for
$\lnot\ocont{C}[r]{(\lnot D)}$. Intuitively, in a context $c$ the concept $\ocont{C}[r]{D}$ denotes
the set of all object domain elements that belong to the concept $D$ in some other context $c'$
which belongs to the meta concept $\mathbf{C}$ and is related to $c$ via $\mathbf{r}$. An object
domain element is in the extension of concept $\ocont*{C}[r]{D}$ in context $c$, if it belongs to $D$ in all
contexts $c'$ that belong to the meta concept $\mathbf{C}$ and are related to $c$ via $\mathbf{r}$.

Thus, within a context we can talk about object elements that belong to some object concept in some
other context. This is somehow similar to $\Next C$ in $\text{LTL}_{\ALC}$, which denotes the set of all
elements that are in $C$ \emph{in the next time point}.
\begin{example}\label{ex:alcalc-plus}
  Going back to Example~\ref{ex:nfl-with-contexts}, the following meta concept assertion states that someone who
  plays quarterback for the Green Bay Packers must work as coach in a junior training camp that is organized by
  Green Bay:
  \begin{align*}
    & \oax{\exists\plays.\mathsf{Quarterback} \sqsubseteq
    \ocont{\mathsf{JuniorFootballClinic}}[\mathsf{organizes}]{\mathsf{Coach}}}(\text{\textit{GreenBayPackers}}).
  \end{align*}
  Note here, that Green Bay Packers and the junior training camp are two different contexts and that
  this cannot be expressed in \LMLO.
\end{example}

The contextualized description logic \ALCALCplus without rigid names is a syntactical variant of
\klarALC~\cite{KG-JELIA10,KG16}. Consequently, the consistency problem in \ALCALCplus has
the same complexity.

\begin{theorem}\label{thm:alcalcplus-without-rigid-twoexptime}
  The consistency problem in \ALCALCplus is \TwoExpTime-complete if $\OCR = \emptyset$ and
  $\ORR = \emptyset$.
\end{theorem}

\begin{proof}[Sketch]
  We can prove the theorem by a mutual reduction of an \klarALC and an \ALCALCplus knowledge base.
  Without introducing the complete syntax of \klarALC, we show how to map an \klarALC ontology into
  \ALCALCplus.

  Table~\ref{tab:syntax-klarALC} shows in the upper part the two special constructors for object
  concepts available in \klarALC. The middle part provides the syntax and semantics of \emph{object
    formulas} which in turn constitute the \emph{object ontology axioms}, shown in lower part. The
  rightmost column defines the mapping $\tau$ which translates terms from \klarALC to
  \ALCALCplus. The following example shows how an object ontology axiom is mapped.
  \begin{align*}
    \mathbf{C}:\langle \mathbf{C} \rangle_{\mathbf{r}}(\langle \mathbf{C}\rangle_{\mathbf{r}}D
    \sqsubseteq A) 
    \qquad\leadsto\qquad
    C\sqsubseteq\exists r.\left(C\sqcap\oax{\ocont{C}[r]{D}}\right)
  \end{align*}

  \begin{table}[t]    
    \caption{Syntax and semantics of \klarALC, and the mapping $\tau$ to \ALCALCplus}
    \centering
    \begin{tabularx}{0.96\linewidth}{ll@{ iff }X@{}l}
      \toprule
      Syntax   & \multicolumn{2}{l}{Semantics} 
               & mapping $\tau(x)$ \\
      \midrule
      $\langle \mathbf{C} \rangle_{\mathbf{r}}D$ 
               & \multicolumn{2}{l}{
                 $\cdot^{\I_{c}} = \{d\in\Delta\mid\text{there is $c'$ s.t.\
                 $(c,c')\in\mathbf{r}^{\J}$, $c'\in C^{\J}$, $d\in D^{\I_{c'}}$}\}$} 
               & $\ocont{C}[r]{D}$ \\
      $[\mathbf{C}]_{\mathbf{r}}D$ 
               & \multicolumn{2}{l}{
                 $\cdot^{\I_{c}} = \{d\in\Delta\mid\text{$(c,c')\in\mathbf{r}^{\J}$ and $c' \in
                 C^{\J}$ imply $d\in D^{\I_{c'}}$}\}$}
               & $\ocont*{C}[r]{D}$ \\
      \midrule
      $B \sqsubseteq D$  & $\I_{c}\models B \sqsubseteq D$
               & $B^{\I_{c}}\subseteq D^{\I_{c}}$ 
               & $\oax{B \sqsubseteq D}$\\
      $D(a)$   & $\I_{c}\models D(a)$
               & $a^{\I_{c}}\in D^{\I_{c}}$
               & $\oax{D(a)}$\\
      $s(a,b)$ & $\I_{c}\models s(a,b)$
               & $(a^{\I_{c}}, b^{\I_{c}})\in s^{\I_{c}}$
               & $\oax{s(a,b)}$\\
      $\lnot\varphi$     & $\I_{c}\models\lnot\varphi $
               & $\I_{c}\not\models\varphi$
               & $\lnot\tau(\varphi)$\\
      $\varphi\land\psi$    & $\I_{c}\models\varphi\land\psi $
               & $\I_{c}\models\varphi$ and $\I_{c}\models\psi$
               & $\tau(\varphi) \sqcap \tau(\psi)$\\
      $\langle \mathbf{C}\rangle_{\mathbf{r}}\varphi$ 
               & $\I_{c}\models\langle\mathbf{C}\rangle_{\mathbf{r}}\varphi$
               & \text{there is $c'\in \mathbf{C}^{\J}$ s.t.\ $(c,c')\in\mathbf{r}^{\J}$, $\I_{c'}\models\varphi$}
               & $\exists r.(C\sqcap\tau(\varphi))$\\
      $[\mathbf{C}]_{\mathbf{r}}\varphi$ 
               & $\I_{c}\models [\mathbf{C}]_{\mathbf{r}}\varphi$
               & \text{every $c'\in \mathbf{C}^{\J}$, $(c,c')\in\mathbf{r}^{\J}$ implies $\I_{c'}\models\varphi$ }
               & $\forall r.(\lnot C \sqcup \tau(\varphi))$\\
      \midrule
      $\mathbf{a} : \varphi$ 
               & $\J\models \mathbf{a} : \varphi$
               & $\I_{c}\models\varphi$ with $c=\mathbf{a}^{\J}$
               & $(\tau(\varphi))(a)$\\
      $\mathbf{C} : \varphi$
               & $\J\models \mathbf{C} : \varphi$ 
               & $\I_{c}\models\varphi$ for every $c$ with $c\in\mathbf{C}^{\J}$
               & $C \sqsubseteq \tau(\varphi)$\\
      \bottomrule
    \end{tabularx}
    \label{tab:syntax-klarALC}
  \end{table}

  An \klarALC ontology $\Kmc = (\Cmc, \Omc)$ consists of a context ontology \Cmc, which is in fact a
  standard \ALC ontology, and of an object ontology. Given \Kmc, let us define the \ALCALC ontology
  $\Bmc_{\Kmc} \coloneqq \Cmc \land \tau(\Omc)$. It is easy to verify that a nested interpretation
  \J is a model of \Kmc if and only if it is a model of $\Bmc_{\Kmc}$.

  Conversely, for an \ALCALC ontology \Bmc, we take the outer abstraction \Bb as context ontology
  and for each \oalpha in \Bmc, we add $(\mathbf{A_{\oalpha}} : \alpha)$ and
  $(\lnot\mathbf{A_{\oalpha}} : \lnot\alpha)$ to the object ontology~$\Omc_{\Bmc}$. Again, it is
  easy to show that~\J models~\Bmc iff~\J models $\Kmc_{\Bmc} = (\Bb,\Omc_{\Bmc})$.
\end{proof}



The more interesting setting with rigid roles behaves much worse. One can easily show that the consistency
problem becomes undecidable.

\begin{theorem}\label{thm:elalcplus-with-rigid-undecidable}
  The consistency problem in \ELALCplus is undecidable if $\ORR \neq \emptyset$.
\end{theorem}

\begin{proof}
  Similar to the idea of \cite{LuWZ-TIME08}, we proof the claim by reduction of a well-known
  undecidable problem, namely the \emph{domino problem} \cite{Ber-66}: given a triple
  $\Dmc = (D, H, V)$ with a set of domino types $D=\{d_{1}, \dots, d_{n}\}$, a horizontal
  compatibility relation $H \subseteq D \times D$ and a vertical compatibility relation
  $V \subseteq D \times D$, decide whether there exists a solution to cover the
  $\nat\times\nat$-grid with these domino types respecting the compatibility relations, i.e.\ does
  there exist a \emph{tiling} $\mathsf{t}:\nat\times\nat\to D$ s.t.\
  $(\mathsf{t}(i,j),\mathsf{t}(i+1,j))\in H$ and $(\mathsf{t}(i,j),\mathsf{t}(i,j+1))\in V$?

  We encode this problem in \ELALCplus with a single rigid role $v\in\ORR$. Let $\Bmc_{\Dmc}$ be the
  conjunction of the following meta axioms. Each context represents a \emph{column} of the
  grid. Using a meta role $h \in \MR$ and a rigid object role $v \in \ORR$, we ensure the existence
  of a grid:
  \begin{align*}
    \top & \sqsubseteq \exists h.\top \tag{$\alpha_{1}$}\\
    \top & \sqsubseteq \oax{\top \sqsubseteq \exists v.\top} \tag{$\alpha_{2}$}
  \end{align*}
  Let $A_{0}, \dots, A_{n} \in \OC$ be object concept names representing the given domino types. To
  ensure that a single domino type is assigned to each grid position, we use
  \begin{align*}
    \top & \sqsubseteq \oax{\top \sqsubseteq (A_{1} \sqcup \dots \sqcup A_{n})
           \sqcap \bigsqcap_{1 \leq i < j \leq n} \lnot(A_{i} \sqcap A_{j})}.
           \tag{$\alpha_{3}$}
  \end{align*}
  To enforce the compatibility relations, we use
  \begin{align*}
    \top & \sqsubseteq \bigsqcap_{i=1}^{n} 
           \oax{A_{i} \sqsubseteq \forall v.(\bigsqcup_{(d_{i},d_{j})\in V} A_{j})}, \text{ and} \tag{$\alpha_{4}$}\\
    \top & \sqsubseteq \bigsqcap_{i=1}^{n} 
           \oax{A_{i} \sqsubseteq \bigsqcup_{(d_{i},d_{j})\in H} \ocont*{\top}[h]{A_{j}}}.\tag{$\alpha_{5}$}
  \end{align*}

  \begin{claim}
    $\Bmc_{\Dmc}$ is consistent iff \Dmc has a
    solution.
  \end{claim}

  \begin{claimproof}
    Assume that $\mathsf{t}$ is a solution for \Dmc. Then, based on $\mathsf{t}$ we define the
    nested interpretation
    $\J_{\mathsf{t}} = (\nat, \cdot^{\J_{\mathsf{t}}}, \nat, (\cdot^{\I_{x}})_{x\in\nat})$ with
    \begin{align*}
      h^{\J_{\mathsf{t}}} & \coloneqq \{(k,k+1) \mid k \geq 0\}\\
      v^{\I_{x}} & \coloneqq \{(l, l+1) \mid l \geq 0\} \qquad \text{for all $n \geq 0$}\\
      %
      A_{i}^{\I_{x}} & \coloneqq \{ y \in \nat \mid \mathsf{t}(x,y) = d_{i}\}
    \end{align*}
    By definition, $\J_{\mathsf{t}}$ models $\alpha_{1}$ and $\alpha_{2}$. For each object domain
    element $y \in \nat$ and each $\I_{x}$, $x \geq 0$, we have that $y \in {A_{i}}^{\I_{x}}$ and
    $y \notin {A_{j}}^{\I_{x}}$, $1 \leq j \leq n, i\neq j$, for $\mathsf{t}(x,y)=d_{i}$. Hence,
    $\J_{\mathsf{t}} \models \alpha_{3}$. By definition, $y+1$ is the only $v$-successor of $y$. If
    $y \in A_{i}^{\I_{x}}$ we know that
    $y+1 \in \smash{\big(\bigsqcup_{(d_{i},d_{j})\in V} A_{j}\big)^{\I_{x}}}$ because $\mathsf{t}$
    is a solution and $(\mathsf{t}(x,y),\mathsf{t}(x,y+1))\in V$. Analogously, $x+1$ is the only
    $h$-successor of $x$ and if $y \in A_{i}^{\I_{x}}$, we know that
    $y \in \smash{\big(\bigsqcup_{(d_{i},d_{j})\in H} A_{j}\big)^{\I_{x+1}}}$ because $\mathsf{t}$
    is a solution and $(\mathsf{t}(x,y),\mathsf{t}(x+1,y))\in H$. Hence,
    $\J_{\mathsf{t}} \models \alpha_{4} \land \alpha_{5}$. Thus, $\Bmc_{\Dmc}$ is consistent.

    Assume that \JJ is a model of $\Bmc_{\Dmc}$. Let $\Psf_{\Msf}$ and $\Psf_{\Osf}$ be infinite
    paths $\Psf_{\Msf} = c_{0}c_{1}\dots$ and $\Psf_{\Osf} = o_{0}o_{1}\dots$ with $c_{i} \in \Cbb$,
    $(c_{i}, c_{i+1}) \in h^{\J}$, $o_{i} \in \Delta^{\J}$ and $(o_{i}, o_{i+1}) \in v^{\I_{c}}$ for
    some $c \in \Cbb$. Such paths exists, because $\J\models\alpha_{1}$, $\J\models\alpha_{2}$ and
    $v$ is a rigid role. We define the nested interpretation
    $\J_{\Psf} \coloneqq (\{c_{i} \mid i \geq 0\}, \cdot^{\J_{\Psf}}, \{o_{i} \mid i \geq 0\},
    (\cdot^{\I_{\Psf,c_{i}}})_{i \geq 0})$, where $\cdot^{\J_{\Psf}}$ is the restriction of
    $\cdot^{\J}$ to the domain ${c_{i} \mid i \geq 0}$, and $\cdot^{\I_{\Psf,c_{i}}}$ is the restriction of
    $\cdot^{\I_{c_{i}}}$ to the domain ${o_{i} \mid i \geq 0}$.

    By construction, $\J_{\Psf}$ is a model of $\alpha_{1}$ and $\alpha_{2}$. Since $\alpha_{3}$ to
    $\alpha_{5}$ do not contain any existential or at-least restrictions, the restriction of the
    meta and the object domain preserves the entailment relation, and
    $\J_{\Psf}\models\Bmc_{\Dmc}$. We define the tiling $\mathsf{t}$ as follows:
    \begin{align*}
      t(x,y) = d_{i} \quad\text{ if $o_{y} \in A_{i}^{\I_{\Psf,c_{x}}}$}.
    \end{align*}
    The tiling $\mathsf{t}$ is a total function and well-defined, due to $\alpha_{3}$. Let
    $o_{j} \in {A_{k_{1}}}^{\!\!\!\I_{\Psf,c_{i}}}$,
    $o_{j+1} \in {A_{k_{2}}}^{\!\!\!\I_{\Psf,c_{i}}}$ and
    $o_{j} \in {A_{k_{3}}}^{\I_{\Psf,c_{i+1}}}$. Thus, we have $\mathsf{t}(i,j) = d_{k_{1}}$,
    $\mathsf{t}(i,j+1) = d_{k_{2}}$ and $\mathsf{t}(i+1,j) = d_{k_{3}}$. By $\alpha_{4}$ we know
    that
    $o_{j} \in \smash{\forall v.\big(\bigsqcup_{(d_{k_{1}},d_{j})\in V}
      A_{j}\big)^{\I_{\Psf,c_{i}}}}$ and, hence, $(d_{k_{1}}, d_{k_{2}})\in V$. Analogously by
    $\alpha_{5}$, we get $(d_{k_{1}}, d_{k_{3}})\in H$. Thus, \Dmc has a solution.
  \end{claimproof}

  Thus, deciding whether $\Bmc_{\Dmc}$ is consistent is undecidable.
\end{proof}

In Section~\ref{sec:complexity-consis-problem}, we discussed three different settings depending on
whether rigid concept and role names are admitted. We already obtained the complexity results for
\LMLOplus for Setting~(i), i.e.~no rigid names are allowed, and for Setting~(iii), i.e.~rigid roles
are allowed.  In LMLOplus, however, a Setting~(ii) that allows rigid concept names but no rigid
roles, coincides with Setting~(i), since rigid concepts can be easily simulated.

To simulate rigid concepts in \LMLOplus, let \BB be an \LMLOplus ontology. We first have to
simulate a universal role $u$ on the meta level. We can assume w.l.o.g.\
that $u\in\MR$ does not occur in \Bmf. We obtain $\Bmf_{u}$ from \Bmf by adding the following axioms
to \Bmf:
\begin{itemize}
\item $u^{-} \sqsubseteq u$,
%\item $\trans{u}$,
\item $r \sqsubseteq u$, for all $r\in\MR$ occurring in \Bmf,
\item $u(a,b)$ for all $a,b\in\MI$ occurring in \Bmf. 
\end{itemize}
Note that $\Bmf_{u}$ is of size polynomial in the size of \Bmf and that we need at least \ALCHI as
\LM to simulate the universal role.
%
Assume there exists an unnamed context in a model of \Bmf, i.e.~$c\in\Cbb$ such that there is
no~$a\in\MI$ occurring in~\Bmf with~$a^{\J}=c$, and $c$ is not connected to any named context by some
path. Then, we can always remove that unnamed context from the interpretation and still have a model.
Hence, we can assume that every model of~$\Bmf_{u}$ is connected and that every context can be
reached by any other context via a path of $u$-edges.
%
Here, we restrict~\Bmf to~\LMLOplus-ontologies, since in an \LMLOplus-BKB, a negated meta GCI can
enforce the existence of an unnamed context that is not necessarily connected to the rest of the model.


With the help of $u$ and contextualized concepts, we can express that $A\in\OC$ is rigid by the
following two axioms:
\begin{align*}
  \top & \sqsubseteq \oax{A \sqsubseteq \ocont*{\top}[u]{A}} \\
  \top & \sqsubseteq \oax{\lnot A \sqsubseteq \ocont*{\top}[u]{\lnot A}}
\end{align*}
Due to these axioms, the extension of $A$ in a context $c$ are exactly these elements which are in
$A$ in every context $c'$ that is related to $c$ via $u$. Thus, the extension of $A$ is equal in
every context, i.e.\ $A$ is rigid.

Hence, the complexity of the consistency problem in \LMLOplus is independent of whether rigid
concepts are allowed or not.


%%% Local Variables:
%%% mode: latex
%%% TeX-master: "../thesis"
%%% reftex-default-bibliography: ("../references.bib")
%%% End:

%  LocalWords:  logics DL LTL expressivity temporalized ontologies


\cleardoublepage

%%% Local Variables:
%%% mode: latex
%%% TeX-master: "../thesis"
%%% reftex-default-bibliography: ("../references.bib")
%%% End:

%  LocalWords:  logics sublogics ontologies DL BKB proven nominals
