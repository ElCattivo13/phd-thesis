\chapter{Preliminiaries}
\label{cha:preliminiaries}

In this chapter, we present some basic notions which we use throughout this thesis. We first
introduce \emph{description logics (DLs)} as logical formalism that we use to reason on complex
domain models. In Section~\ref{sec:rosiroles}, we give an ontological overview of the notion of a
\emph{\rosiroles}.
%
Specific definitions, such as the \emph{restricted type} of an element, are given in the respective
chapters.



\section{Description Logics}
\label{sec:description-logics}

Description logics are a family of knowledge representation formalisms. As already outlined in the
introduction, DLs allow to represent application domains in a well-structured way. In this section,
we present the notations, definitions and known results which are used in this thesis.
%
For a more thorough introduction into description logics we refer the reader to
\cite{DLhandbook-07}.

\subsection{Description Logic Concepts}
\label{sec:dl-concepts}

As shown in Section~\ref{sec:intro-description-logics}, DL concepts describe a set of
elements. Concepts are build from \emph{concept names}, \emph{role names} and \emph{individual
  names} using concept and role constructors.  Note that in the following definitions we refer to
the description logic \Lmc and the triple $\Nsig \coloneqq (\NC, \NR, \NI)$ explicitly although it
is usually left implicit in standard definitions.  This turns out to be useful in
Chapter~\ref{cha:context-dls} as we need to distinguish between the DL and symbols used in the meta
level and the object level.  Sometimes we omit \Lmc and \Nsig, however, if it is irrelevant or if it
is clear from the context.
%

\begin{definition}[Syntax of \Lmc-roles over \Nsig and \Lmc-concepts over \Nsig]
  \label{def:syntax-concepts}
  Let \Lmc be a DL and
  let \NC, \NR, \NI be countably infinite, pairwise disjoint sets of \emph{concept names}, \emph{role names},
  and \emph{individual names}. Then, the triple $\Nsig\coloneqq(\NC,\NR,\NI)$ is the
  \emph{signature}. An \emph{\Lmc-role over \Nsig} $r$ is either a role name, i.e.~$r\in\NR$, or it is of the
  form $s^{-}$ with $s\in\NR$ (\emph{inverse role}).

  The set of \emph{\Lmc-concepts over \Nsig} is the smallest set, such that
  \begin{itemize}
  \item for all $A\in\NC$, $A$ is an \Lmc-concept over \Nsig (\emph{atomic concept}), and
  \item if $C$ and $D$ are \Lmc-concepts over \Nsig, $r$ is a \Lmc-role over \Nsig, $a\in\NI$, then
    $\lnot C$ (\emph{negation}), $C\sqcap D$ (\emph{conjunction}), $\exists r.C$ (\emph{existential
      restriction}), $\{a\}$ (\emph{nominal}) and $\atleast{n}{r}{C}$ (\emph{at-least restriction})
    are \Lmc-concepts over \Nsig. \qedhere
  \end{itemize}
\end{definition}

\noindent Non-atomic concepts are also called \emph{complex concepts}. As usual in description
logics, we use the following abbreviations:
\begin{itemize}
\item $C\sqcup D$ (\emph{disjunction}) for $\lnot(\lnot C \sqcap \lnot D)$,
\item $\top$ (\emph{top}) for $A \sqcup \lnot A$ where $A\in\NC$ is arbitrary but fixed,
\item $\bot$ (\emph{bottom}) for $\lnot\top$,
\item $\forall r.C$ (\emph{value restriction}) for $\lnot(\exists r.\lnot C)$, and
\item $\atmost{n}{r}{C}$ (\emph{at-most restriction}) for $\lnot(\atleast{n+1}{r}{C})$.
\end{itemize}

\noindent
The semantics of description logic concepts are defined in a model-theoretic way using the notion of
interpretations.

\begin{definition}[\Nsig-interpretation, Semantics of concepts over \Nsig]
  \label{def:n-interpretation}
  Let $\Nsig\coloneqq(\NC,\NR,\NI)$ be the signature. Then, an \emph{\Nsig-interpretation \I} is a pair
  $(\Delta^{\I}, \cdot^{\I})$ where the \emph{domain} $\Delta^{\I}$ is a non-empty set and
  the \emph{interpretation function} $\cdot^{\I}$ maps
  \begin{itemize}
  \item every concept name $A\in\NC$ to the set $A^{\I}\subseteq\Delta^{\I}$,
  \item every role name $r\in\NR$ to the binary relation
    $r^{\I}\subseteq\Delta^{\I}\times\Delta^{\I}$, and
  \item every individual name $a\in\NI$ to the element $a^{\I}\in\Delta^{\I}$ such that different
    individual names are mapped to different elements, i.e.\ for $a,b\in\NI$ it holds that
    $a^{\I}\neq b^{\I}$ if $a\neq b$.
  \end{itemize}
  %
  Now, that function is extended for inverse roles and complex concepts as follows:
  \begin{itemize}
  \item $(s^{-})^{\I} \coloneqq \left\{(d,c)\in\Delta^{\I}\times\Delta^{\I}\mmid (c,d)\in
      s^{-}\right\}$,
  \item $(\lnot C)^{\I} \coloneqq \Delta^{\I} \setminus C^{\I}$,
  \item $(C \sqcap D)^{\I} \coloneqq C^{\I} \cap D^{\I}$,
  \item $(\exists r.C)^{\I} \coloneqq \{d \in \Delta^\I \mid \text{there is an}\ e \in C^\I \
    \text{with}\ (d,e)\in r^\I\}$,
  \item $\{a\}^{\I} \coloneqq \{a^{\I}\}$, and
  \item $(\atleast{n}{r}{C})^{\I} \coloneqq \{d \in \Delta^{\I} \mid \sharp\{e\in C^{\I}\mid (d,e)\in r^{\I}\}\ge n\}$.
  \end{itemize}
  where $\sharp S$ denotes the cardinality of the set $S$.
\end{definition}

\noindent
For any $x\in\NC\cup\NR\cup\NI$, $x^{\I}$ is called the \emph{extension of $x$}.  Note that in the
above definition of an interpretation, we adopt the so-called \emph{unique name assumption} stating
that every individual is interpreted as a distinct element. By doing so, we emphasize that an
individual name is meant to be the identity of an individual, rather than just a tag as it is
usually used in the context of semantic web.

Now, we can look at a first example using the notions just defined.

\begin{example}\label{ex:concept-nfl}
  Consider the following complex concept $C$:
  \begin{align*}
    & \mathsf{NFL\_{}Team} \sqcap \lnot \mathsf{AFC} \sqcap
    \atleast{1}{\mathsf{playsFor}^{-}}{(\exists\mathsf{position}.\mathsf{Quarterback})} \\
    & \qquad\sqcap \exists\mathsf{coaches}^{-}.\{\text{\textit{MikeMcCarthy}}\}
  \end{align*}
  It describes all NFL teams which are not in the AFC, have at least one person playing for them at
  quarterback position and are coached by Mike McCarthy.

  Figure~\ref{fig:example-concept} depicts an interpretation in which the Green Bay Packers are in the extension of the above concept.
\end{example}
\begin{figure}
  \centering
  \begin{tikzpicture}
    \node[node, label={[align=center]180:\textit{GreenBayPackers},\\\textsf{NFL\_Team}, \textsf{NFC}}] (gbp) at (0,0) {};
    \node[node, label={[align=center]270:\textit{MikeMcCarthy}}] (mmc) at (4,1.0) {};
    \node[node, label={[align=center]270:\textit{AaronRodgers}, \\\textsf{NFL\_PLayer}}] (ar) at (4,-1.0) {};
    \node[node, label={[align=center]270:\textsf{Quarterback}}] (qb) at (8,-1.0) {};
    %
    \draw[edge] (mmc) to[bend right=10,swap] node{$\mathsf{coaches}$} (gbp);
    \draw[edge] (ar) to[bend left=10] node{$\mathsf{playsFor}$} (gbp);
    \draw[edge] (ar) to[bend left=10] node{$\mathsf{position}$} (qb);
  \end{tikzpicture}
  \caption{An Interpretation \I such that $\text{\textit{GreenBayPackers}}^{\I} \in C^{\I}$ and
    $\I\models\Omc$ with $C$ from Example~\ref{ex:concept-nfl} and \Omc from Example~\ref{ex:bkb-nfl}.}
  \label{fig:example-concept}
\end{figure}



\subsection{Boolean Knowledge Bases}
\label{sec:dl-axioms}

With the notion of concepts at hand, we can formulate \emph{axioms} to capture domain knowledge in a
so-called \emph{Boolean knowledge base (BKB)}. Each BKB consists of a Boolean combination of certain
axioms and an RBox which states the general knowledge about roles.

\begin{definition}[Syntax of \Lmc-axioms over \Nsig and \Lmc-BKBs over \Nsig]
  Let \Lmc be a DL and
  let $\Nsig\coloneqq(\NC,\NR,\NI)$ be the signature. Then, if $C$ and $D$ are \Lmc-concepts over \Nsig, $r$
  and $s$ are a \Lmc-roles over \Nsig, and $\{a,b\}\subseteq\NI$, then
  \begin{itemize}
  \item $C \sqsubseteq D$ (\emph{general concept inclusion, GCI}),
  \item $C(a)$ (\emph{concept assertion}),
  \item $r(a,b)$ (\emph{role assertion}),
  \item $r \sqsubseteq s$ (\emph{role inclusion}), and
  \item $\mathsf{trans}(r)$ (\emph{transitivity axiom})
  \end{itemize}
  are \emph{\Lmc-axioms over \Nsig}.
  %
  Moreover, an \emph{\Lmc-RBox \Rmc over \Nsig} is a finite set of \Lmc-role inclusions over \Nsig and
  \Lmc-transitivity axioms over \Nsig. A \emph{Boolean \Lmc-axiom formula over \Nsig} is defined inductively
  as follows:
  \begin{itemize}
  \item every \Lmc-GCI over \Nsig is a Boolean \Lmc-axiom formula over \Nsig,
  \item every \Lmc-concept and \Lmc-role assertion over \Nsig is a Boolean \Lmc-axiom formula over \Nsig,
  \item if $\Bmc_{1}$, $\Bmc_{2}$ are Boolean \Lmc-axiom formulas over \Nsig, then so are $\lnot\Bmc_{1}$
    (axiom negation) and $\Bmc_{1}\land\Bmc_{2}$ (axiom conjunction), and
  \item nothing else is a Boolean \Lmc-axiom formula over \Nsig.
  \end{itemize}
  %
  Finally, a \emph{Boolean \Lmc-knowledge base (\Lmc-BKB) over \Nsig} is a pair
  $\Bmf = (\Bmc, \Rmc)$, where \Bmc is a Boolean \Lmc-axiom formula over \Nsig and \Rmc is an
  \Lmc-RBox over \Nsig. An \emph{\Lmc-ontology over \Nsig} is an \Lmc-BKB over \Nsig, where only
  axiom conjunction and no axiom negation is allowed in the Boolean axiom formula.
\end{definition}

\noindent
Again as usual in description logics, we use $C \equiv D$ (\emph{concept equivalence}) as abbreviation
for $(C \sqsubseteq D) \land (D \sqsubseteq C)$ and $\Bmc_1\lor\Bmc_2$ (\emph{axiom disjunction}) as
abbreviation for $\lnot(\lnot\Bmc_1\land\lnot\Bmc_2)$.
%
Often an ontology $\Omc = (\Bmc, \Rmc)$ is written as a triple $\Omc = (\Tmc, \Amc, \Rmc)$ where
\Tmc (\emph{TBox}) is the set of all GCIs occurring in \Bmc and \Amc (\emph{ABox}) is the set of all
assertion axioms occurring in \Bmc.


\begin{definition}[Semantics of axioms over \Nsig, BKBs over \Nsig]
  \label{def:semantics-of-axioms}
  An \Nsig-interpretation \I is a model of
  \begin{itemize}
  \item the GCI $C \sqsubseteq D$ over \Nsig if $C^{\I} \subseteq D^{\I}$,
  \item the concept assertion $C(a)$ over \Nsig if $a^{\I} \in C^{\I}$,
  \item the role assertion $r(a,b)$ over \Nsig if $(a^\I,b^\I)\in r^\I$,
  \item the role inclusion $r \sqsubseteq s$ over \Nsig if $r^\I \subseteq s^\I$, and
  \item the transitivity axiom $\mathsf{trans}(r)$ over \Nsig if $r^\I=(r^\I)^+$, where $\cdot^{+}$
    denotes the transitive closure of a binary relation.
  \end{itemize}
  %
  This is extended to Boolean axiom formulas over~\Nsig inductively as follows:
  \begin{itemize}
  \item \I is a model of $\lnot\B_1$ if it is not a model of~$\B_1$, and
  \item \I is a model of $\B_1\land\B_2$ if it is a model of both $\B_1$ and~$\B_2$.
  \end{itemize}
  %
  We write $\I\models\alpha$ and $\I\models\Bmc$ if \I is a model of the axiom~$\alpha$ over~\Nsig
  or \I is a model of the Boolean axiom formula~\B, respectively. Furthermore, \I is a model of an
  RBox~\Rmc over~\Nsig (written $\I\models\Rmc$) if it is a model of each axiom in \Rmc.

  Finally, \I is a model of the BKB $\Bmf=(\B,\Rmc)$ over~\Nsig (written $\I\models\Bmf$) if it is a
  model of both~\B and~\Rmc.  We call~\Bmf \emph{consistent} if it has a model.  The
  \emph{consistency problem} is the problem of deciding whether a given BKB is consistent.
\end{definition}

\noindent
Note that besides the consistency problem there are several other reasoning tasks for description
logics.  The entailment problem, for instance, is the problem of deciding, given a BKB~\Bmf and an
axiom~$\beta$, whether \Bmf \emph{entails} $\beta$, i.e.~whether all models of~\Bmf are also models
of~$\beta$.
%
The consistency problem, however, is fundamental in the sense that most other standard decision
problems (reasoning tasks) can be polynomially reduced to it (in the presence of axiom negation).
For the entailment problem, note that it can be reduced to the \emph{in}consistency problem: $\Bmf=(\B,\Rmc)$
entails $\beta$ iff $(\B\land\lnot\beta,\Rmc)$ is inconsistent.  Hence, we focus in this thesis only
on the consistency problem.

To show an example of an ontology, we continue contentwise with American football.

\begin{example}\label{ex:bkb-nfl}
  Consider the following ontology $\Omc = (\Bmc,\emptyset)$ with $\Bmc =$
  \begin{gather*}
    \mathsf{NFC}(\text{\textit{GreenBayPackers}}) \quad\land\\
    \mathsf{playsFor}(\text{\textit{AaronRodgers}}, \text{\textit{GreenBayPackers}})\quad\land\\
    \begin{aligned}
      \exists\mathsf{playsFor}.\mathsf{NFL\_Team} & \sqsubseteq \mathsf{NFL\_Player}\quad\land\\
      \mathsf{NFL\_Team} & \equiv \mathsf{NFC} \sqcup \mathsf{AFC}\quad\land\\
      \mathsf{NFC} \sqcap \mathsf{AFC} & \sqsubseteq \bot.
    \end{aligned}
  \end{gather*}
  The first two axioms assert that the Green Bay Packers are in the NFC and that Aaron Rodgers plays
  for Green Bay. The third axiom states that everybody who plays for an NFL team is an NFL
  player. Finally, the last two axioms define NFL teams as a disjoint union of the NFC and the AFC.

  The ontology \Omc is consistent and Figure~\ref{fig:example-concept} depicts a model of \Omc. Note
  here, that it is not coincidentally that Aaron Rodgers is in the extension of
  $\mathsf{NFL\_Player}$ since \Omc entails
  \begin{gather*}
    \mathsf{NFL\_Player}(\text{\textit{AaronRodgers}}). \qedhere
  \end{gather*}
\end{example}

\subsection{Specific Description Logics}
\label{sec:specific-description-logics}

The specific description logics differ in the concept and role constructors they allow to formulate
concepts and axioms, and also in the axioms that are allowed to use in a knowledge base.

The prototypical description logic is \ALC, the \emph{attributive language with complement}, which
does not allow any inverse roles and only negation, conjunction and existential restriction are
allowed as concept constructors. Furthermore, only GCIs, concept and role assertions are allowed as
axioms. Hence, only role names are \ALC-roles and an \ALC-RBox is always the empty set.  The DL \ALC
is the smallest propositionally closed DL~\cite{ScSm-AIJ91}.
%
Adding a letter to \ALC stands for certain constructors or axioms that are additionally allowed. For
example, \ALCI additionally allows inverse roles in complex concepts.  By the naming convention of
DLs, specific letters denote a concept or role constructor or a type of axioms that is allowed in
that DL:
\begin{itemize}
\item \Omc means nominals,
\item \Imc means inverse roles,
\item \Qmc means at-least restrictions,
\item \Hmc means role inclusions, and
\item \Smc means transitivity axioms.
\end{itemize}
%
Only, \ALC with additional transitivity axioms is called \Smc instead of $\mathcal{ALCS}$ due to its
connection to the modal logic $\mathsf{S4}$. Thus, \SHOIQ, for example, is the DL that allows all
constructors and axioms which were introduced above. Besides extensions of \ALC, there also exist many
sublogics of \ALC of which in this thesis we only consider \EL. The sub-Boolean description logic
\EL is the fragment of \ALC where only conjunction, existential restriction, and the top concept
(which cannot be expressed as an abbreviation anymore due to the lack of negation) are
admitted.

In~\cite{HoST-IGPL00}, it was shown for \SHQ that allowing arbitrary roles in number restrictions lead to
undecidability of the consistency problem. Decidability can be regained by restricting roles used in
number restrictions to simple roles. To define what a simple role is, for a given BKB
$\Bmf = (\Bmc,\Rmc)$, we introduce $\transClosure$ as the transitive-reflexive closure of
$\sqsubseteq$ on
$\Rmc \cup \{\mathsf{Inv}(r)\sqsubseteq\mathsf{Inv}(s) \mid r\sqsubseteq s\in\Rmc\}$ where
$\mathsf{Inv}(r)$ is defined as
\begin{align*}
  \mathsf{Inv}(r) \coloneqq
  \begin{cases}
    r^{-} & \text{ if $r\in\NR$, and} \\
    s    & \text{ if $r$ is an inverse role with $r=s^{-}$}
  \end{cases}
\end{align*}
and $r \equiv_{\Rmc} s$ as abbreviation for $r \transClosure s$ and $s \transClosure r$. A role $r$
is \emph{transitive w.r.t.\ \Rmc} if for some $s$ with $r \equiv_{\Rmc} s$, we have
$\trans{s}\in\Rmc$ or $\trans{\mathsf{Inv}(s)}\in\Rmc$. A role $r$ is called \emph{simple w.r.t.\
  \Rmc} if it is neither transitive nor has any transitive sub-role, i.e. there is no $s$ such that
$s\transClosure r$ and $s$ is transitive w.r.t.\ \Rmc.

In the rest of this thesis, we make this restriction to the syntax of \SHQ and all its
extensions.
%
This restriction is also the reason why there are no Boolean combinations of
role inclusions and transitivity axioms allowed in the RBox~\Rmc over~\Nsig in
the above definition.  Otherwise the notion of a simple role w.r.t.~\Rmc
involves reasoning.  Consider, for instance, the Boolean combination of axioms
$(\mathsf{trans}(r)\lor\mathsf{trans}(s))\land r\sqsubseteq s$.  It should be
clear that $s$ is not simple, but this is no longer a pure syntactic check.

The complexity of the consistency problem for DL ontologies is well-investigated. Is is
\ExpTime-complete for any DL between \ALC and \SHOQ and \NExpTime-complete for \SHOIQ. The lower
bound for \ALC was shown in~\cite{Sch-IJCAI91}, the upper bound for \SHOQ in~\cite{Tob-PhD01}. For
\SHOIQ the lower and upper bound were proven in~\cite{Tob-JAIR00} and~\cite{Tob-PhD01},
respectively. While for BKBs the complexity class stays the same, this is much less explored. It is
in \ExpTime for \SHOQ~\cite{Lip-PhD14} and it remains in \NExpTime for \SHOIQ as a consequence of
Theorem~2 in \cite{Pra-JLLI05}.

For \EL ontologies, the consistency problem is trivial since no contradictions can be expressed and, thus,
every \EL ontology is consistent. On the other hand, we do not consider \EL-BKBs as it seems very unnatural to
admit axiom negation while denying concept negation at the same time.

%\clearpage

\section{Roles}
\label{sec:rosiroles}

In this section 

Bachman

japanisches Paper

rigid

foundedness

identity

\todo{cite features}

\begin{table}[ht]
  \caption{Classifying features for roles}
  \small
  \centering
  \begin{tabularx}{0.8\linewidth}{rX} 
    \toprule
    1.  & Roles have properties and behaviour.\\
    2.  & Roles depend on relationships.\\
    3.  & Objects may play different roles simultaneously.\\
    4.  & Objects may play the same role (type) several times.\\
    5.  & Objects may acquire and abandon roles dynamically.\\
    6.  & The sequence of role acquisition and removal may be restricted.\\
    7.  & Unrelated objects can play the same role.\\
    8.  & Roles can play roles.\\
    9.  & Roles can be transferred between objects.\\
    10. & The state of an object can be role-specific.\\
    11. & Features of an object can be role-specific.\\
    12. & Roles restrict access.\\
    13. & Different roles may share structure and behavior.\\
    14. & An object and its roles share identity.\\
    15. & An object and its roles have different identities.\\
    \midrule
    16. & Relationships between roles can be constrained.\\
    17. & There may be constraints between relationships.\\
    18. & Roles can be grouped and constrained together.\\
    19. & Roles depend on Compartments.\\
    20. & Compartments have properties and behaviors.\\
    21. & A Role can be part of several Compartments.\\
    22. & Compartments may play roles like objects.\\
    23. & Compartments may play roles which are part of themselves.\\
    24. & Compartments can contain other compartments.\\
    25. & Different compartments may share structure and behavior.\\
    26. & Compartments have their own identity.\\
    \bottomrule
  \end{tabularx}
  \label{tab:role-features}
\end{table}

\blindtext

inherently

\blindtext

\blindtext

\blindtext


%%% Local Variables:
%%% mode: latex
%%% TeX-master: "../thesis"
%%% reftex-default-bibliography: ("../references.bib")
%%% End:

%  LocalWords:  logics sublogics ontologies DL BKB proven
