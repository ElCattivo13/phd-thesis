\chapter{The Contextualized Description Logic \texorpdfstring{\LMLO}{LM[LO]}}
\label{cha:context-dls}

\todo[inline]{somewhere a few paragraphs about existing cdls and why they are not feasible in my
  setting.}


Some introduction

As mentioned in the introduction classical description logics lack the expressive power to capture
knowledge about \ldots

mention somewhere what \LM and \LO are.

\todo[inline]{introductory paragraphs to this chapter}


\section{Syntax and Semantics of Contextualized Description Logics}
\label{sec:syn-seman-cdl}

conceptually we start with an description logic \LM on the meta level. Here we can state our
knowledge about contexts. We enrich this with \ldots 

\todo[inline]{basic ideas how the syntax works}


Throughout this chapter, let $\Msig = (\MC, \MR, \MI)$ and $\Osig = (\OC, \OR, \OI)$ denote the
signatures for \LM and \LO. Thus, we call \MC, \MR, \MI, \OC, \OR and \OI, respectively, the set of
\emph{meta concept}, \emph{role} and \emph{individual names} and \emph{object concept}, \emph{role}
and \emph{individual names}.

\begin{definition}[Syntax of \LMLO]\label{def:syntax-cdls}
  A \emph{concept of the object logic~\LO (o-concept)} is an \LO-concept over~\Osig.  An
  \emph{o-axiom} is an \LO-GCI over~\Osig, an \LO-concept assertion over~\Osig, or an
  \LO-role assertion over~\Osig.

  The set of \emph{concepts of the meta logic~\LM (m-concepts)} is the smallest set such that
  \begin{itemize}
  \item for all $A\in\MC$, $A$ is a meta concept (\emph{\todo{here I'm struggling with names. I need
        some distinction between ``pure meta concept'' and ``referring meta concepts'' as I need the
        second name very often}???}),
  \item for all o-axioms $\alpha$, \oalpha is a meta concept (\emph{???}), and
  \item all complex concepts that can be built with the concept constructors allowed in \LM are meta
    concepts.
  \end{itemize}
  
  An \emph{m-axiom} is \todo{what exactly is an m-axiom?}

  Analogously a \emph{Boolean m-axiom formula} is defined inductively as follows:
  \begin{itemize}
  \item every m-axiom is a Boolean m-axiom formula,
  \item if $\B_1,\B_2$ are Boolean m-axiom formulas, then so are $\lnot\B_1$ and $\B_1\land\B_2$,
    and
  \item nothing else is a Boolean m-axiom formula.
  \end{itemize}
    %
  Finally, a \emph{Boolean \LMLO-knowledge base (\LMLO-BKB)} is a triple $\Bmf=(\B,\RO,\RM)$ where
  \RO is an \LO-RBox over~\Osig, \RM an \LM-RBox over~\Msig, and \B is a Boolean m-axiom formula. An
  \emph{\LMLO-ontology} is an \LMLO-BKB, where only axiom conjunction and no axiom negation is
  allowed in the Boolean m-axiom formula.
\end{definition}

For the same reasons as mentioned in section \ref{sec:description-logics}, role inclusions
over~\Osig and transitivity axioms over~\Osig are not allowed to constitute m-concepts.  However, we
fix an RBox~\RO over~\Osig that contains such o-axioms and holds in \emph{all} contexts.  The same
applies to role inclusions over~\Msig and transitivity axioms over~\Msig, which are only allowed to
occur in a RBox~\RM over~\Msig.

Again, we use the usual abbreviations (for disjunctions etc.) for m-concepts and
Boolean m-axiom formulas.

The semantics of \LMLO is defined by the notion of \emph{nested interpretations}.  These consist of
\Osig-interpretations for the specific contexts and an \Msig-interpretation for the relational
structure between them.  We assume that all contexts speak about the same non-empty domain
(\emph{constant domain assumption}).

\todo[inline]{here something more about constant domain assumption, references. Argument about
  varying domains and how they can be reduced to CDA. Maybe with example, maybe a bit later after
  semantics are defined.}

As argued earlier, sometimes it is desired that concepts or roles in the object logic are
interpreted the same in all contexts. Therefore we introduce \emph{rigid names}. Let
$\OCR\subseteq\OC$ be the set of \emph{rigid object concept names} and $\ORR\subseteq\OR$ be the set
of \emph{rigid object role names}.
%
Often, we refer to \OCR and \ORR simply as \emph{rigid concepts} and \emph{rigid roles} as there is
no such notion on the meta level.
%
We call concept names and role names in $\OC\setminus\OCR$ and $\OR\setminus\ORR$ \emph{flexible}.
%
Moreover, we assume that individuals of the object logic are always interpreted the same in all
contexts (\emph{rigid individual assumption}). \todo{reference to RIA, LTL-ALC paper}

\todo[inline]{Why does the RIA makes sense. Only in combination with UNA. Maybe already the link to
  OWL which does not use UNA, example}

\begin{definition}[Nested interpretation]\label{def:nested-interpretation}
  A \emph{nested interpretation} is a tuple \todo{not nice!}\\ \JJ, where \Cbb is a non-empty set
  (called \emph{contexts}) and $(\Cbb,\cdot^\J)$ is an \Msig-interpretation.
  %
  Moreover, for every $c\in\Cbb$, $\I_c\coloneqq(\Delta^{\J},\cdot^{\I_c})$ is an
  \Osig-interpretation such that we have for all $c,c'\in\Cbb$ that $x^{\I_{c}}=x^{\I_{c'}}$ for
  every $x\in\OI\cup\OCR\cup\ORR$.
\end{definition}

We are now ready to define the semantics of \LMLO.

\begin{definition}[Semantics of \LMLO]
  Let \JJ be a nested interpretation.  The mapping $\cdot^\J$ is extended to o-axioms
  \todo{referenced meta concept} as follows: $\oalpha^\J:=\{c\in\Cbb\mid\I_c\models\alpha\}$.

    Moreover, \J is a model of the m-axiom $\beta$ if $(\Cbb,\cdot^\J)$ is a model
    of $\beta$.  This is extended to Boolean m-axiom formulas inductively as
    follows:
    \begin{itemize}
        \item \J is a model of $\lnot\B_1$ if it is not a model of $\B_1$, and
        \item \J is a model of $\B_1\land\B_2$ if it is a model of both $\B_1$
            and $\B_2$.
    \end{itemize}
    We write $\J\models\B$ if \J is a model of the Boolean m-axiom formula~\B.
    Furthermore, \J is a model of~\RM (written $\J\models\RM$) if
    $(\Cbb,\cdot^\J)$ is a model of~\RM, and \J is a model of~\RO (written
    $\J\models\RO$) if $\I_{c}$ is a model of~\RO for all $c\in\Cbb$.
    
    Finally, \J is a model of the \LMLO-BKB \BB (written
    $\J\models\Bmf$) if \J is a model of~\B, \RO, and~\RM.  We call~\Bmf
    \emph{consistent} if it has a model.

    The \emph{consistency problem in \LMLO} is the problem of deciding whether a given
    \LMLO-BKB is consistent.
\end{definition}



\todo[inline]{some words about the next example.}

\begin{example}
  
\end{example}

\section{Complexity of the Consistency Problem}
\label{sec:complexity-consis-problem}

{\renewcommand{\arraystretch}{1.1}
\begin{table}[t]
  \centering
  \caption{The complexity results for the consistency problem in \LMLO}
  \label{tab:compl-results-no-rigid-names}
  \begin{tabularx}{1.0\linewidth}[t]{clC@{\ }C@{\ }C@{\ }C}
    \toprule
    & \diagbox[width=2cm]{\LM}{\LO}
    & \tikzmark{X1l} \hfill \EL    \hfill \tikzmark{X1r}
    & \tikzmark{X2l} \hfill \ALC   \hfill \tikzmark{X2r}
    & \tikzmark{X3l} \hfill \SHOQ  \hfill \tikzmark{X3r}
    & \tikzmark{X4l} \hfill \SHOIQ \hfill \tikzmark{X4r}\\
    \midrule \noalign{\hbox{\tikzmark{rawY1}}}
    \multirow{4}{*}{\rotatebox[origin=c]{90}{Setting (i)}}
    & \EL    & constant \\ \noalign{\hbox{\tikzmark{rawY2}}}
    & \ALC   &          \\ \noalign{\hbox{\tikzmark{rawY3}}}
    & \SHOQ  &          \\ \noalign{\hbox{\tikzmark{rawY4}}}
    & \SHOIQ &          \\ \noalign{\hbox{\tikzmark{rawY5}}}
    \midrule \noalign{\hbox{\tikzmark{rawY6}}}
    \multirow{4}{*}{\rotatebox[origin=c]{90}{Setting (ii)}}
    & \EL    & constant \\ \noalign{\hbox{\tikzmark{rawY7}}}
    & \ALC   &          \\ \noalign{\hbox{\tikzmark{rawY8}}}
    & \SHOQ  &          \\ \noalign{\hbox{\tikzmark{rawY9}}}
    & \SHOIQ &          \\ \noalign{\hbox{\tikzmark{rawY10}}}
    \midrule \noalign{\hbox{\tikzmark{rawY11}}}
    \multirow{4}{*}{\rotatebox[origin=c]{90}{Setting (iii)}}
    & \EL    & constant \\ \noalign{\hbox{\tikzmark{rawY12}}}
    & \ALC   &          \\ \noalign{\hbox{\tikzmark{rawY13}}}
    & \SHOQ  &          \\ \noalign{\hbox{\tikzmark{rawY14}}}
    & \SHOIQ &          \\ \noalign{\hbox{\tikzmark{rawY15}}}
    \bottomrule
  \end{tabularx}
  \caption*{Settings: (i) no rigid names are allowed, i.e.\ $\OCR=\ORR=\emptyset$. (ii) only rigid
    concepts are allowed, i.e.\ $\OCR\neq\emptyset$ and $\ORR=\emptyset$. (iii) rigid concepts and
    rigid roles are allowed, i.e.\ $\OCR\neq\emptyset$ and $\ORR\neq\emptyset$.}
\end{table}
}
\begin{tikzpicture}[remember picture, overlay]
  % \foreach \x in {1,...,5}
  %   \foreach \y in {1,...,5}
  %     \node[node] at (X\x|-Y\y) {}; 
  %
  \pgfmathsetmacro{\eps}{0.05}
  %
  \coordinate (Y1) at ($(rawY1)-(0,\eps)$);
  \coordinate (Y2) at ($(rawY2)-(0,\eps)$);
  \coordinate (Y3) at ($(rawY3)+(0,0)$);
  \coordinate (Y4a) at ($(rawY4)+(0,\eps)$);
  \coordinate (Y4b) at ($(rawY4)-(0,\eps)$);
  \coordinate (Y5) at ($(rawY5)+(0,\eps)$);
  \fill[ExpTime] (X2l|-Y1) -- (X2l|-Y2) -- (X1l|-Y2) -- (X1l|-Y4a) -- (X3r|-Y4a) -- (X3r|-Y1) -- cycle;
  \fill[NExpTime] (X4l|-Y1) -- (X4l|-Y4b) -- (X1l|-Y4b) -- (X1l|-Y5) -- (X4r|-Y5) -- (X4r|-Y1) -- cycle;
  \node at ($0.5*(X1l|-Y3)+0.5*(X3r|-Y3)+(0.9,0.2)$) {\ExpTime-complete};
  \node[align=center] at ($0.5*(X4l|-Y3)+0.5*(X4r|-Y3)$) {\NExpTime-\\complete};
  %
  \coordinate (Y6) at ($(rawY6)-(0,\eps)$);
  \coordinate (Y7) at ($(rawY7)-(0,\eps)$);
  \coordinate (Y8) at ($(rawY8)+(0,0)$);
  \coordinate (Y10) at ($(rawY10)+(0,\eps)$);
  \fill[NExpTime] (X2l|-Y6) -- (X2l|-Y7) -- (X1l|-Y7) -- (X1l|-Y10) -- (X4r|-Y10) -- (X4r|-Y6) -- cycle;
  \node at ($0.5*(X1l|-Y10)+0.5*(X4r|-Y6)+(0.5,-0.1)$) {\NExpTime-complete};
  %
  \coordinate (Y11) at ($(rawY11)-(0,\eps)$);
  \coordinate (Y12) at ($(rawY12)-(0,\eps)$);
  \coordinate (Y13) at ($(rawY13)+(0,0)$);
  \coordinate (Y14a) at ($(rawY14)+(0,\eps)$);
  \coordinate (Y14b) at ($(rawY14)-(0,\eps)$);
  \coordinate (Y15) at ($(rawY15)+(0,\eps)$);
  \fill[2ExpTime] (X2l|-Y11) -- (X2l|-Y14a) -- (X3r|-Y14a) -- (X3r|-Y11) -- cycle;
  \fill[NExpTime] (X1l|-Y12) -- (X1l|-Y15) -- (X1r|-Y15) -- (X1r|-Y12) -- cycle;
  % \draw[draw=gray!60, pattern=north east lines, pattern color=gray!50] (X1l|-Y14b) -- (X1l|-Y15) --
  % (X1r|-Y15) -- (X1r|-Y14b) -- cycle;
  \fill[N2ExpTime] (X4l|-Y11) -- (X4l|-Y14b) -- (X2l|-Y14b) -- (X2l|-Y15) -- (X4r|-Y15) -- (X4r|-Y11) -- cycle;
  \node[align=center] at ($0.5*(X1l|-Y15)+0.5*(X1r|-Y12)$) {\NExpTime-\\complete};
  \node at ($0.5*(X2l|-Y14a)+0.5*(X3r|-Y11)$) {\TwoExpTime-complete};
  \node[align=center] at ($0.5*(X4l|-Y13)+0.5*(X4r|-Y13)$) {\TwoExpTime-hard\\ and\\ in \TwoNExpTime};
  %
\end{tikzpicture}

\todo[inline]{somewhere a few words about top down view, can't go up, because of that its possible
  to divide reasoning problem into two subtasks, later on }

\todo[inline]{results later}

\todo[inline]{paragraph that we will handle lower bounds in 3.3 where we talk about EL}

For the upper bounds, let in the following \BB be an \LMLO-BKB.  We proceed
similar to what was done for \ALC-LTL in~\cite{BaGL-KR08,BaGL-ToCL12} (and
\SHOQ-LTL in~\cite{Lip-PhD14}) and reduce the consistency problem to two separate
decision problems.

For the first problem, we consider the so-called \emph{outer abstraction}, which is the \LM-BKB
over~\Msig obtained by replacing each \todo{referring m-concept?} m-concept of the form \oalpha
occurring in~\B by a fresh concept name such that there is a 1--1 relationship between them.

\begin{definition}[Outer abstraction]
  Let \BB be an \LMLO-BKB.  Let \bsf be the bijection mapping every m-concept of the form \oalpha
  occurring in~\B to the concept name $A_{\oalpha}\in\MC$, where we assume w.l.o.g.\ that
  $A_{\oalpha}$ does not occur in~\B.
  \begin{enumerate}
  \item The \LM-concept $C^{\bsf}$ over \Msig is obtained from the m-concept~$C$ by replacing every occurrence of
    \oalpha by $\bsf(\oalpha)$.
  \item The Boolean \LM-axiom formula~\Bb over \Msig is obtained from~\B by replacing every
    m-concept $C$ occurring in~\B with~$C^{\bsf}$.  We call the \LM-BKB $\Bmfb=(\Bb,\RM)$ the
    \emph{outer abstraction of~\Bmf}.
        \item Given \JJ, its \emph{outer abstraction} is the
            \Msig-interpretation $\Jb=(\Cbb,\cdot^{\Jb})$ where
            \begin{itemize}
                \item for every $x\in\MR\cup\MI\cup(\MC\setminus\ran(\bsf))$, we
                    have $x^{\Jb}=x^\J$, and
                \item for every $A\in\ran(\bsf)$, we have
                    $A^{\Jb}=(\bsf^{-1}(A))^\J$,
            \end{itemize}
            where $\ran(\bsf)$ denotes the image of~\bsf. \qedhere
    \end{enumerate}
\end{definition}

For simplicity, for $\Bmf'=(\B',\RO,\RM)$ where $\B'$ is a subformula of~\B, we
denote by $(\Bmf')^\bsf$ the outer abstraction of~$\Bmf'$ that is obtained by
restricting \bsf to the m-concepts occurring in~$\B'$.
%
Now let us consider the following small example.


\begin{example}\label{ex:outer-abstraction}
  Let $\Bmf_{\text{ex}} = (\B_{\text{ex}},\emptyset,\emptyset)$ with $\B_{\text{ex}}\coloneqq
  C\sqsubseteq(\oax{A\sqsubseteq\bot})\ \land\ (C\sqcap\oax{A(a)})(c)$ be an \ALCALC-BKB.  Then,
  \bsf maps $\oax{A\sqsubseteq\bot}$ to $A_{\oax{A\sqsubseteq\bot}}$ and $\oax{A(a)}$ to
  $A_{\oax{A(a)}}$.  Thus, we have that
  \begin{align*}
    \Bmf_{\text{ex}}^\bsf\coloneqq \Big(C\sqsubseteq(A_{\oax{A\sqsubseteq\bot}})\ \land\ (C\sqcap
    A_{\oax{A(a)}})(c),\ \emptyset\Big)
  \end{align*}
  is the outer abstraction of $\Bmf_\text{ex}$.
\end{example}

The following lemma makes the relationship between \Bmf and its outer abstraction
\Bmfb explicit.  It is proved by induction on the structure of~\B.

\begin{lemma}\label{lem:interpretation-outer-abstraction}
  Let \J be a nested interpretation such that \J is a model of \RO.  Then, \J is a model of \Bmf iff
  $\Jb$ is a model of~\Bmfb.
\end{lemma}

\begin{proof}
  Since $r^{\J}=r^{\Jb}$ for all \LM-role $r$ over \Msig, we have that \J is a model of \RM iff \Jb is a model of
  \RM. Thus, it is only left to show that for any m-axiom $\gamma$ occurring in \B, it holds that
  $\J \models \gamma$ iff $\Jb \models \gamma^{\bsf}$.

  \begin{claim}
    For any $x \in \Cbb$ it holds that $x \in C^{\J}$ iff
    $x \in (C^{\bsf})^{\Jb}$.
  \end{claim}

  \begin{claimproof}
    We prove the claim by induction on the structure of $C$: \todo{müssen hier noch inverse rollen
      betrachtet werden? eigentlich nicht.}

    \begin{tabularx}{\linewidth}{@{}l@{ }X@{}}
      $C = A \in \MC\!\setminus\!\ran(\bsf)$: 
      & $x \in A^{\J}$ 
        iff $x \in (A^{\bsf})^{\Jb}$ by definition of $\Jb$ and since $A = A^{\bsf}$ 
      \\[1ex]
      $C = \oalpha$:
      & $x \in \oalpha^{\J}$
        iff $x \in (A_{\oalpha})^{\Jb}$
        iff $x \in (\oalpha^{\bsf})^{\Jb}$
      \\[1ex] 
      $C = \lnot D$:
      & $x \in (\lnot D)^{\J}$ 
        iff $x \notin D^{\J}$ 
        iff, by induction hypothesis, $x \notin (D^{\bsf})^{\Jb}$ 
        iff $x \in (\lnot D^{\bsf})^{\Jb}$ 
        iff $x \in ((\lnot D)^{\bsf})^{\Jb}$ 
      \\[1ex]
      $C = D \sqcap E$: 
      & $x \in (D \sqcap E)^{\J}$
        iff $x \in D^{\J}$ and $x \in E^{\J}$ 
        iff, by induction hypothesis, $x \in (D^{\bsf})^{\Jb}$ and $x \in
        (E^{\bsf})^{\Jb}$
        iff $x \in (D^{\bsf} \sqcap E^{\bsf})^{\Jb}$
        iff $x \in ((D \sqcap E)^{\bsf})^{\Jb}$ 
      \\[1ex]
      $C = \exists r.D$: 
      & $x \in (\exists r.D)^{\J}$
        iff there exists $y \in \Cbb$ \suth $(x,y) \in r^{\J}$ and $y \in D^{\J}$
        iff there exists $y \in \Cbb$ \suth $(x,y) \in r^{\Jb}$ and $y \in (D^{\bsf})^{\Jb}$
        iff $x \in (\exists r.D^{\bsf})^{\Jb}$ 
        iff $x \in ((\exists r.D)^{\bsf})^{\Jb}$ 
      \\[1ex]
      $C = \{a\}$:
      & $x\in\{a\}^{\J}$ 
        iff $x\in(\{a\}^{\bsf})^{\Jb}$ by definition of $\Jb$ and since $\{a\} = \{a\}^{\bsf}$ 
      \\[1ex]
      $C =\ \atleast{n}{r}{D}$:
      & $x \in (\atleast{n}{r}{D})^{\J}$
        iff there are at least $n$ elements $y \in \Cbb$ s.t.\ $(x,y) \in r^{\J}$ and $y \in D^{\J}$
        iff there are at least $n$ elements $y \in \Cbb$ s.t.\ $(x,y) \in r^{\Jb}$ and $y \in (D^{\bsf})^{\Jb}$
        iff $x \in (\atleast{n}{r}{D^{\bsf}})^{\Jb}$
        iff $x \in ((\atleast{n}{r}{D})^{\bsf})^{\Jb}$ 
    \end{tabularx}

    \vspace{-2.0\baselineskip}
  \end{claimproof}

  If $\gamma$ is of the form $C \sqsubseteq D$, we have that $\J \models C
  \sqsubseteq D$ iff $x \in C^{\J}$ implies $x \in D^{\J}$
  iff (by claim) $x
  \in (C^{\bsf})^{\Jb}$ implies $x \in (D^{\bsf})^{\Jb}$ iff
  $\Jb \models C^{\bsf} \sqsubseteq D^{\bsf}$.

  If $\gamma$ is of the form $C(a)$, we have that $\J \models C(a)$ iff
  $a^{\J} \in C^{\J}$ iff (by claim) $a^{\Jb} \in
  (C^{\bsf})^{\Jb}$ iff $\Jb \models C^{\bsf}(a)$.

  If $\gamma$ is of the form $r(a,b)$, we have that $\J \models r(a,b)$ iff
  $(a^{\J}, b^{\J}) \in r^{\J}$ iff $(a^{\Jb},
  b^{\Jb}) \in r^{\Jb}$ iff $\Jb \models
  r(a,b)$.

  If \B is of the form $\lnot\B_{1}$, we have that $\J\models\B$ iff not
  $\J\models\B_{1}$ iff not $\Jb\models\Bb_1$ iff $\Jb\models\Bb$.

  If \B is of the form $\B_{1}\land\B_{2}$, we have that $\J\models\B$ iff $\J\models\B_{1}$ and
  $\B_{2}$ iff $\Jb\models\Bb_{1}$ and $\Jb\models\Bb_{2}$ iff $\Jb\models\Bb$.

  Since $\J\models\RO$, $\J\models\RM$ iff $\Jb\models\RM$ and $\J\models\B$ iff $\Jb\models\Bb$, we
  have $\J\models\Bmf$ iff $\Jb\models\Bmfb$.
\end{proof}

Note that this lemma yields that consistency of~\Bmf implies consistency of~\Bmfb.  Thus, the
consistency of~\Bmfb is a necessary condition for the consistency of~\Bmf.  However, it is not
sufficient since the converse does not hold as the following example shows.

\begin{example}\label{ex:outer-abstraction-continued}
  Consider again $\Bmf_\text{ex}$ of Example~\ref{ex:outer-abstraction}.
  %
  Take any \Msig-interpretation $\Hmc=(\Delta^{\Hmc},\cdot^\Hmc)$ with $\Delta^{\Hmc}=\{e\}$,
  $d^\Hmc=e$, and $C^\Hmc = A_{\oax{A\sqsubseteq\bot}}^\Hmc = A_{\oax{A(a)}}^\Hmc = \{e\}$.

  Clearly, \Hmc is a model of~$\Bmf_\text{ex}^{\bsf}$.  But there is no nested interpretation~\JJ
  with $\J\models\Bmf_\text{ex}$ since this would imply $\Cbb=\Delta^{\Hmc}$, and that $\I_e$ is a model of
  both $A\sqsubseteq\bot$ and $A(a)$, which is not possible.
\end{example}

The above example illustrates that there exist implicit restrictions on the interpretation of the
meta level as certain combinations of concept names in $\ran(\bsf)$ are not allowed.  Therefore, we
need to ensure that these are not treated independently.  For expressing such a restriction on the
model~\Hmc of~\Bmfb, we adapt a notion of~\cite{BaGL-KR08,BaGL-ToCL12}. It is also worth noting that
this problem occurs also in much less expressive DLs such as \ELbot (i.e.~\EL extended with the
bottom concept).

\begin{definition}[\Umc-type, \mbox{\Nsig-interpretation (weakly) respects $(\Umc,\Ymc)$}]
  \label{def:int-respects-D}
  Let \II be an \Nsig-interpretation, let $\Umc\subseteq\NC$ and let $\Ymc\subseteq\powerset{\Umc}$.
  %
  The \emph{\Umc-type of $d\in\Delta^{\I}$ in \I} is defined as
  $\mathsf{type}_{\Umc}^{\I}(d) \coloneqq \{A \in \Umc \mid d \in A^{\I}\}$.  The interpretation \I
  \emph{respects} $(\Umc,\Ymc)$ if $\Zmc = \Ymc$ where
  \begin{align*}
    \Zmc & \coloneqq\{Y\subseteq\Umc\mid\text{there is some $d\in\Delta^\I$ with
           $\mathsf{type}_{\Umc}^{\I}(d) = Y$}\}
  \end{align*}

    It \emph{weakly respects} $(\Umc,\Ymc)$ if $\Zmc \subseteq \Ymc$.
\end{definition}

\todo[inline]{a few words about the meaning of the definition, explanation. Types usual definition}


The second decision problem that we use for deciding consistency is needed to make sure that such a
set of concept names is admissible in the following sense.

\begin{definition}[Admissibility]\label{def:admissibility}
  Let $\Xmc=\{X_1,~\dots,\ X_k\}\subseteq\powerset{\ran(\bsf)}$.  We call \Xmc \emph{admissible} if
  there exist \Osig-interpretations $\I_1=(\Delta,\cdot^{\I_1})$,~\dots,
  $\I_k=(\Delta,\cdot^{\I_k})$ such that
  \begin{itemize}
  \item $x^{\I_i}=x^{\I_j}$ for all $x\in\OI\cup\OCR\cup\ORR$ and all $i,j\in\{1,\dots,k\}$, and
  \item every $\I_i$, $1\le i\le k$, is a model of the \LO-BKB $\Bmf_{X_{i}}= (\B_{X_i},\RO)$
    over~\Osig where
    \begin{align*}
      \B_{X_i}:=\bigwedge_{\bsf(\oalpha)\in X_i}\alpha\ \land
      \bigwedge_{\bsf(\oalpha)\in\ran(\bsf)\setminus X_i}\lnot\alpha.
    \end{align*}
  \end{itemize}
  \vspace{-1.7\baselineskip}
\end{definition}

Note that any subset $\Xmc'\subseteq\Xmc$ is admissible if \Xmc is admissible.
%
Intuitively, the sets $X_i$ in an admissible set \Xmc consist of concept names \todo{referring concept
  names} such that the corresponding o-axioms \enquote{fit together}.  Consider again
Example~\ref{ex:outer-abstraction-continued}.  Clearly, the set
$\{A_{\oax{A\sqsubseteq\bot}},A_{\oax{A(a)}}\}\in\powerset{\ran(\bsf)}$ \emph{cannot} be contained
in any admissible set~\Xmc.  

The next definition captures the above mentioned restriction on the model~\Hmc
of~\Bmfb.

\todo[inline]{explanation of outer consistency}

\begin{definition}[Outer consistency]\label{def:outer-consistency}
  Let $\Xmc \subseteq \powerset{\ran(\bsf)}$.  We call the \LM-BKB~\Bmfb over \Msig \emph{outer
    consistent w.r.t.~\Xmc} if there exists a model of~\Bmfb that weakly respects~$(\ran(\bsf),\Xmc)$.
\end{definition}

The next two lemmas show that the consistency problem in \LMLO can be decided by checking whether
there is an admissible set~\Xmc and the outer abstraction of the given \LMLO-BKB is outer consistent
w.r.t.~\Xmc.

\begin{lemma}\label{lem:model-equivalent-to-admissible}
  For every \Msig-interpretation \HH, the following two statements are equivalent:
  \begin{enumerate}
  \item There exists a model~\J of~\Bmf with $\Jb=\Hmc$.
  \item \Hmc is a model of~\Bmfb and the set $\{X_d\mid d\in\Gamma\}$ is admissible, where $X_d$ is
    defined as $X_{d}:=\{A\in\ran(\bsf)\mid d\in A^\Hmc\}$.
  \end{enumerate}
\end{lemma}

\begin{proof}\todo{noch nicht drüber gelesen, hier müssen sicher noch einige Formelzeichen angepasst
    werden. Und ist er verständlich???}
  (1 $\Rightarrow$ 2): Let \JJ be a model of~\Bmf with $\Jb=\Hmc$.  Since $\Jb=\Hmc$, we have that
  $\Cbb=\Delta^{\Hmc}$.  By Lemma~\ref{lem:interpretation-outer-abstraction}, we have that \Hmc is a
  model of~\Bmfb.
    %
  Moreover, since \bsf is a bijection between m-concepts of the form \oalpha occurring in~\Bmf and
  concept names of~\MC, we have that $\ran(\bsf)$ is finite, and thus also the set
  $\Xmc \coloneqq \{X_d\mid d \in \Delta^{\Hmc} \} \subseteq \powerset{\ran(\bsf)}$ is finite.  Let
  $\Xmc = \{Y_1, \dots, Y_k\}$.  Since $\Cbb = \Delta^{\Hmc}$, there exists an index function
  $\nu\colon\Cbb\to\{1,\dots,k\}$ such that $X_c = Y_{\nu(c)}$ for every $c\in\Cbb$, i.e.
  \begin{align*}
    Y_{\nu(c)} & = \bigl\{\bsf(\oalpha)\mid\text{\oalpha occurs in~\Bmf and}\
                 c\in\oalpha^\Hmc\bigr\} \\
               & =  \bigl\{\bsf(\oalpha)\mid\text{\oalpha occurs in~\Bmf and}\ \I_c\models\alpha\bigr\}.
  \end{align*}
  Conversely, for every $\mu\in\{1,\dots,k\}$, there is an element $c\in\Cbb$ such that
  $\nu(c)=\mu$.
    % 
  The \Osig-interpretations for showing admissibility of~\Xmc are obtained as follows.  Take
  $c_1,\dots,c_k \in \Cbb$ such that $\nu(c_1) = 1$,~\dots, $\nu(c_k) = k$.  Now, for every~$i$,
  $1 \leq i \leq k$, we define the \Osig-interpretation $\Gmc_i:=(\Delta,\cdot^{\I_{c_i}})$.
  Clearly, we have that $Gmc_i\models\B_{Y_i}$ and since $\J\models\RO$, we have that
  $Gmc_i\models\Bmf_{Y_i}$.  Moreover, the definition of a nested interpretation yields that
  $x^{\Gmc_i}=x^{\Gmc_j}$ for all $x\in\OI\cup\OCR\cup\ORR$ and all $i,j \in \{1,\dots,k\}$.  Hence,
  the \Osig-interpretations $\Gmc_1, \dots, \Gmc_k$ attest admissibility of~\Xmc.

  (2 $\Rightarrow$ 1): Assume that \HH is a model of~\Bmfb and that the set
  $\Xmc \coloneqq \{X_d\mid d\in\Gamma\}$ is admissible.  Again, since $\ran(\bsf)$ is finite, we
  have that $\Xmc \subseteq \powerset{\ran(\bsf)}$ is finite.  Let $\Xmc = \{Y_1,\dots,Y_k\}$.
  Since \Xmc is admissible, there are \Osig-interpretations $\Gmc_1=(\Delta,\cdot^{\Gmc_1})$,~\dots,
  $\Gmc_k=(\Delta,\cdot^{\Gmc_k})$ such that $\Gmc_i\models\Bmf_{Y_i}$ and $x^{\Gmc_i}=x^{\Gmc_j}$
  for all $x\in\OI\cup\OCR\cup\ORR$ and all $i,j\in\{1,\dots,k\}$.
    %
  Furthermore, there exists an index function $\nu\colon\Gamma\to\{1,\dots,k\}$ such that
  $Y_{\nu(d)}=X_d$ for every $d\in\Gamma$.
    %
  We define a nested interpretation \JJ as follows:
  \begin{itemize}
  \item $\Cbb \coloneqq \Delta^{\Hmc}$;
  \item $x^\J \coloneqq x^\Hmc$ for every $x\in\MC\cup\MR\cup\MI$; and
  \item $x^{\I_c}:=x^{\Gmc_{\nu(c)}}$ for every $x \in \OC \cup \OR \cup \OI$ and every $c \in \Cbb$.
  \end{itemize}
    %
  By construction of \J, we have that $x^{\Jb} = x^\Hmc$ for every
  $x \in \MR \cup \MI \cup (\MC \setminus \ran(\bsf))$.
    %
  Let $A \in \ran(\bsf)$, and let $\bsf^{-1}(A) = \oalpha$.  We have for every $d \in \Gamma = \Cbb$
  that $d\in A^{\Jb}$ iff $d\in(\bsf^{-1}(A))^\J$ iff $d \in \oalpha^\J$ iff $\I_d \models \alpha$
  iff $\Gmc_{\nu(d)}\models\alpha$ iff $\bsf(\oalpha)=A\in Y_{\nu(d)}$ (since
  $\Gmc_{\nu(d)}\models\B_{Y_{\nu(d)}}$) iff $A\in X_d$ iff $d\in A^\Hmc$.
    %
  Hence, we have $\Jb=\Hmc$.
    %
  Since \Hmc is a model of~\Bmfb and, by construction of \J, \J is a model of \RO, we have by
  Lemma~\ref{lem:interpretation-outer-abstraction} that \J is a model of~\Bmf.
\end{proof}

The following lemma is a direct consequence of the previous one. It states that we can split the
consistency check into two subtasks.

\begin{lemma}\label{lem:admissible-and-outerConsistent}%
  The \LMLO-BKB~\Bmf is consistent iff there is a set
  $\Xmc\subseteq\powerset{\ran(\bsf)}$ such that
  \begin{enumerate}
  \item \Xmc is admissible, and
  \item \Bmfb is outer consistent w.r.t.~\Xmc.
  \end{enumerate}
\end{lemma}
\begin{proof}
  \todo{not proof-read yet!}
  \onlyifdirection Let \J be a model of \Bmf, and let $\Jb=(\Cbb,\cdot^{\Jb})$.  By
  Lemma~\ref{lem:model-equivalent-to-admissible}, we have that $\Jb$ is a model of~\Bmfb, and the
  set $\Xmc \coloneqq \{X_c \mid c \in \Cbb\}$ is admissible.  By construction, $\Jb$ weakly
  respects $(\ran(\bsf),\Xmc)$, and hence \Bmfb is outer consistent w.r.t.~\Xmc.
    
  \ifdirection Let $\Xmc = \{X_1,\dots,X_k\}\subseteq\powerset{\ran(b)}$ such that \Xmc is
  admissible and \Bmfb is outer consistent w.r.t.~\Xmc.  Hence there is a model
  $\Gmc=(\Cbb,\cdot^\Gmc)$ of~\Bmfb that weakly respects $(\ran(\bsf),\Xmc)$.
    %
  We define $\Xmc' \coloneqq \{Y_c \mid c\in\Cbb\}$, where
  $Y_c \coloneqq \{A \in \ran(b) \mid c \in A^\Gmc\}$.  Since \Gmc weakly respects
  $(\ran(\bsf),\Xmc)$ and $c \in (C_{\ran(b),Y_c})^\Gmc$ for every $c \in \Cbb$, we have that
  $\Xmc' \subseteq \Xmc$.  Since \Xmc is admissible, this yields admissibility of~$\Xmc'$.
  Lemma~\ref{lem:model-equivalent-to-admissible} yields now consistency of~\Bmf.
\end{proof}



Before we can analyse the complexity of the consistency problem in \LMLO, we need to state two
complexity results for the consistency problem of \SHOQ-BKBs and \SHOIQ-BKBs. For the former we follow the idea of~\cite{Lip-PhD14}.
%
The former is a corollary of Theorem~\ref{thm:shoiq-bkb-consistency-nexptime} which is proven
is Section \ref{sec:consistency-shoiq-bkb}.



\begin{lemma}\label{lem:shoq-outer-consisteny-exptime}
  Deciding whether a \SHOQ-BKB \Bmfb is outer consistent w.r.t.~\Xmc can be done in time
  exponential in the size of~\Bmfb and linear in size of~\Xmc.
\end{lemma}
\begin{proof}\todo{complete rewrite necessary, why linear in X}
  It is enough to show that deciding whether~\Bmfb has a model that weakly respects
  $(\ran(\bsf),\Xmc)$ can be done in time exponential in the size of~\Bmfb and linear in the size
  of~\Xmc.  

  Here, we will adapt the ideas of~\cite{Lip-PhD14}.  It is not hard to see that we can adapt the
  notion of a quasimodel respecting a pair $(\Umc,\Ymc)$ of~\cite{Lip-PhD14} to a quasimodel
  \emph{weakly} respecting $(\Umc,\Ymc)$.  Indeed, one just has to drop Condition~(i) in
  Definition~3.25 of~\cite{Lip-PhD14}.  Then, the proof of Lemma~3.26 of~\cite{Lip-PhD14} can be
  adapted such that our claim follows.  This is done by dropping one check in Step~4 of the
  algorithm of~\cite{Lip-PhD14}.
\end{proof}



\begin{lemma}\label{lem:shoiq-outer-consisteny-nexptime}
  Deciding whether a \SHOIQ-BKB \Bmfb is outer consistent w.r.t.~\Xmc can be non-deterministically
  done in time exponential in the size of~\Bmfb and linear in size of~\Xmc.
\end{lemma}
\begin{proof}
  Let \Umc be the set $\ran(\bsf)$. Due to Theorem~\ref{thm:shoiq-bkb-consistency-nexptime},
  deciding whether there exists a model that weakly respects $(\ran(\bsf),\Xmc)$ can be
  non-deterministically done in time exponential in the size of \Bmf and linear in the size of \Xmc.
\end{proof}



\subsection{Consistency without rigid names}
\label{sec:cons-without-rigid}
In this section, we consider the case where no rigid concept names or role names
are allowed. So we fix $\OCR=\ORR=\emptyset$.

\begin{theorem}\label{thm:shoqshoq-without-rigid-exptime}
  The consistency problem in \SHOQSHOQ is in \ExpTime if $\OCR=\ORR=\emptyset$.
\end{theorem}

\begin{proof}
  Let \Bmf be a \SHOQSHOQ-BKB and \Bmfb its outer abstraction.  We can decide consistency of~\Bmf
  using Lemma~\ref{lem:admissible-and-outerConsistent}.  We define
  $\Xmc \coloneqq \{ X \subseteq \ran(\bsf) \mid \Bmf_{X}\ \text{is consistent}\}$ where $\Bmf_{X}$
  is defined as in Definition~\ref{def:admissibility}.
  %
  We first show that $\Xmc = \{X_1, \dots, X_k\}$ is admissible.  Let $\I_i$ be a model
  of~$\Bmf_{X_i}$, which exists since $\Bmf_{X_i}$ is consistent.  Since \NI is countably infinite,
  interpretations must respect the UNA and due to the Löwenheim-Skolem
  theorem, we can assume that all models $\I_i$, $1 \leq i \leq k$, have a countably infinite
  domain. Thus, w.l.o.g.\ we can assume that all models have the same domain~$\Delta$.  Furthermore,
  we can assume that individual names are interpreted the same.  Since $\OCR=\ORR=\emptyset$, the
  set~\Xmc fulfills all conditions of Definition~\ref{def:admissibility} for admissibility.

  Thus, if \Bmfb is outer consistent w.r.t.~\Xmc, then we have by
  Lemma~\ref{lem:admissible-and-outerConsistent} that \Bmf is consistent.
  %
  Conversely, assume that \Bmf is consistent.  Then, by
  Lemma~\ref{lem:admissible-and-outerConsistent}, there is an admissible set
  $\Xmc' \subseteq \powerset{\ran(\bsf)}$ and \Bmfb is outer consistent w.r.t.~$\Xmc'$.  Since \Xmc
  is the maximal admissible subset of $\powerset{\ran(\bsf)}$, we have $\Xmc' \subseteq \Xmc$.  If
  \Bmfb is outer consistent w.r.t.~$\Xmc'$, it is also outer consistent w.r.t.~\Xmc.  Hence, \Bmf is
  consistent iff \Bmfb is outer consistent w.r.t.~\Xmc, which yields a decision procedure for the
  consistency problem in \SHOQSHOQ.

  It remains to analyze the complexity.  There are exponentially many
  $\Xmc \in \powerset{\ran(\bsf)}$, but each \SHOQ-BKB~$\Bmf_{X}$ can be constructed in time
  polynomial in the size of~\Bmf.  We can decide consistency of~$\Bmf_{X}$ in time
  exponential~\cite{Lip-PhD14}.  Thus, the set~\Xmc can be constructed in time exponential in the size
  of~\Bmf and it is of exponential size.  Due to Lemma \ref{lem:shoq-outer-consisteny-exptime},
  deciding whether \Bmfb is outer consistent w.r.t.~\Xmc can be done in time exponential in the size
  of~\Bmfb and linear in the size of~\Xmc.  Thus, overall we can decide the consistency problem in
  exponential time.
\end{proof}

\begin{theorem}\label{thm:shoiqshoiq-without-rigid-exptime}
  If $\OCR=\ORR=\emptyset$, the consistency problem in \SHOIQSHOIQ is in \NExpTime.
\end{theorem}
\begin{proof}
  Let \Bmf be a \cSHOIQ-BKB and \Bmfb is outer abstraction.  Analogous to the proof of
  Theorem~\ref{thm:shoqshoq-without-rigid-exptime} we construct the maximal admissible subset~\Xmc
  of $\powerset{\ran(\bsf)}$.  Again, \Bmf is consistent iff \Bmfb is outer consistent w.r.t.~\Xmc.

  In contrast to the proof of Theorem~\ref{thm:shoqshoq-without-rigid-exptime} we can decide
  consistency of~$\Bmf_{X}$ non-deter\-ministically in time exponential in the size of~$\Bmf_{X}$
  (Theorem~\ref{thm:shoiq-bkb-consistency-nexptime}).  Thus, the set~\Xmc can be constructed
  non-deterministically in time exponential in the size of~\Bmf.  Also due to
  Lemma~\ref{lem:shoiq-outer-consisteny-nexptime}, deciding whether \Bmfb is outer consistent
  w.r.t.~\Xmc can be non-deterministically done in time exponential in the size of \Bmfb and linear
  in the size of~\Xmc.  
\end{proof}

Together with the lower bounds shown in
Theorems~\ref{thm:alcel-lower-no-rigid},~\ref{thm:shoiqel-lower-no-rigid}
and~\ref{thm:elshoiq-lower-no-rigid} we obtain \ExpTime-completeness for the consistency problem in
\LMLO for \LM and \LO being DLs between \ALC and \SHOQ, and \NExpTime-completeness for \LMLO where
either \LM or \LO is \SHOIQ, if $\OCR=\ORR=\emptyset$.

\subsection{Consistency with rigid concept and role names}
\label{sec:cons-with-rigid}

In this section, we consider the case where rigid concept and role names are present.  So we fix
$\OCR\ne\emptyset$ and $\ORR\ne\emptyset$.

\begin{theorem}\label{thm:shoqshoq-with-rigid-names-twoexptime}
  The consistency problem in \SHOQSHOQ is in \TwoExpTime if $\OCR\ne\emptyset$ and $\ORR\ne\emptyset$.
\end{theorem}

\begin{proof}
  Let \BB be a \SHOQSHOQ-BKB and \BBb its outer abstraction.  We can decide consistency of~\Bmf
  using Lemma~\ref{lem:admissible-and-outerConsistent}.  For that, we enumerate all sets
  $\Xmc \subseteq \powerset{\ran(\bsf)}$, which can be done in time doubly exponential in~\Bmf.  For
  each of these sets $\Xmc = \{X_1,\dots,X_k\}$, we check whether \Bmfb is outer consistent
  w.r.t.~\Xmc, which can be done in time exponential in the size of~\Bmfb and linear in the size
  of~\Xmc.
  
  Then, we check \Xmc for admissibility using the renaming technique \todo{define separately?}
  of~\cite{BaGL-KR08,BaGL-ToCL12}.  For every~$i$, $1\le i\le k$, every flexible concept name~$A$
  occurring in~\Bb, and every flexible role name~$r$ occurring in~\Bb or~\RO, we introduce
  copies~$A^{(i)}$ and~$r^{(i)}$.  The \SHOQ-BKB~$\Bmf_{X_i}^{(i)} = (\B_{X_i}^{(i)},\RO^{(i)})$
  over~\Osig is obtained from~$\Bmf_{X_i}$ (see Definition~\ref{def:admissibility}) by replacing
  every occurrence of a flexible name~$x$ by $x^{(i)}$.  We define
  \begin{align*}
    \Bmf_\Xmc \coloneqq \left(\bigwedge\nolimits_{1\le i\le k}\B_{X_i}^{(i)}, \bigcup\nolimits_{1\le i\le k} \RO^{(i)}\right).
  \end{align*}
  It is not hard to verify \todo{do short argumentation here} (using arguments of~\cite{Lip-PhD14})
  that \Xmc is admissible iff $\Bmf_\Xmc$ is consistent.  Note that $\Bmf_\Xmc$ is of size at most
  exponential in~\Bmf and can be constructed in exponential time.  Moreover, consistency
  of~$\Bmf_\Xmc$ can be decided in time exponential in the size of~$\Bmf_\Xmc$~\cite{Lip-PhD14}, and
  thus in time doubly exponential in the size of~\Bmf.
\end{proof}

\begin{theorem}\label{thm:shoiqshoiq-with-rigid-names-ntwoexptime}
  The consistency problem in \SHOIQSHOIQ is in \TwoNExpTime if $\OCR\ne\emptyset$ and $\ORR\ne\emptyset$.
\end{theorem}

\begin{proof}
  Let \BB be a \SHOIQSHOIQ-BKB and \BBb its outer abstraction. We proceed the same as in the proof
  of Theorem~\ref{thm:shoqshoq-with-rigid-names-twoexptime}. Enumerating all sets
  $\Xmc \subseteq \powerset{\ran(\bsf)}$ can still be done in time doubly exponential in the size of
  \Bmf, but checking for each \Xmc whether \Bmfb is outer consistent w.r.t.~\Xmc can be done
  non-deterministically in time exponential in the size of \Bmfb and linear in the size of \Xmc.

  To check \Xmc for admissibility we again use the renaming technique
  of~\cite{BaGL-KR08,BaGL-ToCL12} and by the same arguments as above \Xmc is admissible iff $\Bmf_\Xmc$ is
  consistent.  Again, $\Bmf_\Xmc$ is of size at most exponential in~\Bmf and can be constructed in
  exponential time, but consistency of~$\Bmf_\Xmc$ can be decided non-deterministically in time
  exponential in the size of~$\Bmf_\Xmc$ (Theorem~\ref{thm:shoiq-bkb-consistency-nexptime}) and thus
  non-deterministically in time doubly exponential in the size of \Bmf.
\end{proof}

Together with the lower bounds shown in Section~\ref{sec:case-el}, we obtain
\TwoExpTime-completeness for the consistency problem in \LMLO
for \LM and \LO being DLs between \ALC and \SHOQ if $\OCR\neq\emptyset$
and $\ORR\neq\emptyset$. \todo{update for \LMLO with \SHOIQ}

\subsection{Consistency with only rigid concept names}
\label{sec:cons-with-only}

In this section, we consider the case where rigid concept are present, but rigid role names are not
allowed.  So we fix $\OCR \neq \emptyset$ but $\ORR=\emptyset$.

\commented{
\begin{theorem}\label{thm:shoqshoq-with-rigid-concepts-nexptime}
  The consistency problem in \SHOQSHOQ is in \NExpTime if $\OCR \neq \emptyset$ and $\ORR=\emptyset$.
\end{theorem}

\begin{proof}
  Let \BB be a \SHOQSHOQ-BKB and \BBb its outer abstraction.  We can decide consistency of~\Bmf
  using Lemma~\ref{lem:admissible-and-outerConsistent}. We first non-deterministically guess the set
  $\Xmc = \{X_{1},\ldots,X_{k}\} \subseteq \powerset{\ran(\bsf)}$, which is of size at most
  exponential in \Bmf. Due to Lemma~\ref{lem:shoq-outer-consisteny-exptime} we can check whether
  \Bmfb is outer consistent w.r.t.~\Xmc in time exponential in the size of~\Bmfb and linear in the
  size of~\Xmc.
  
  It remains to check \Xmc for admissibility. For that let $\OCR(\B) \subseteq \OCR$ and
  $\OI(\B) \subseteq \OI$ be the sets of all rigid concept names and individual names occurring in
  \B, respectively. As done in~\cite{BaGL-KR08,BaGL-ToCL12} we non-deterministically guess a set
  $\Ymc \subseteq \powerset{\OCR(\B)}$ and a mapping $\kappa\colon\OI(\B)\to\Ymc$ which also can be
  done in time exponential in the size of \Bmf. Using the same arguments as
  in~\cite{BaGL-KR08,BaGL-ToCL12} we can show that \Xmc is admissible iff
  \begin{align*}
    \widehat{\Bmf}_{X_{i}}\coloneqq\left(\B_{X_{i}} \land \bigwedge_{a\in\OI(\B)} \left(\bigsqcap_{A\in\kappa(a)} A \sqcap
    \bigsqcap_{A\in\OCR(\B)\setminus\kappa(a)} \lnot A\right)(a),\ \RO\right)
  \end{align*}
  has a model that respects $(\OCR(\B),\Ymc)$, for all $1 \leq i \leq k$. The \SHOQ-BKB
  $\widehat{\Bmf}_{X_{i}}$ is of size polynomial in the size of \B and can be constructed in time
  exponential in the size of \B. We can check if $\widehat{\Bmf}_{X_{i}}$ has a model that respects
  $(\OCR(\B),\Ymc)$ in time exponential in the size of $\widehat{\Bmf}_{X_{i}}$
  \cite{BaGL-KR08,BaGL-ToCL12}, and thus exponential in the size of \Bmf.
\end{proof}
}


\begin{theorem}\label{thm:shoiqshoiq-with-rigid-concepts-nexptime}
  The consistency problem in \SHOIQSHOIQ is in \NExpTime if $\OCR \neq \emptyset$ and $\ORR=\emptyset$.
\end{theorem}

\begin{proof}
  Let \BB be a \SHOIQSHOIQ-BKB and \BBb its outer abstraction.  We can decide consistency of~\Bmf
  using Lemma~\ref{lem:admissible-and-outerConsistent}. We first non-deterministically guess the set
  $\Xmc = \{X_{1},\ldots,X_{k}\} \subseteq \powerset{\ran(\bsf)}$, which is of size at most
  exponential in \Bmf. Due to Lemma~\ref{lem:shoiq-outer-consisteny-nexptime} we can check whether
  \Bmfb is outer consistent w.r.t.~\Xmc non-deterministically in time exponential in the size
  of~\Bmfb and linear in the size of~\Xmc.
  
  It remains to check \Xmc for admissibility. For that let $\OCR(\B) \subseteq \OCR$ and
  $\OI(\B) \subseteq \OI$ be the sets of all rigid concept names and individual names occurring in
  \B, respectively. As done in~\cite{BaGL-KR08,BaGL-ToCL12} we non-deterministically guess a set
  $\Ymc \subseteq \powerset{\OCR(\B)}$ and a mapping $\kappa\colon\OI(\B)\to\Ymc$ which also can be
  done in time exponential in the size of \Bmf. Using the same arguments as
  in~\cite{BaGL-KR08,BaGL-ToCL12} we can show that \Xmc is admissible iff
  \begin{align*}
    \widehat{\Bmf}_{X_{i}}\coloneqq\left(\B_{X_{i}} \land \bigwedge_{a\in\OI(\B)} \left(\bigsqcap_{A\in\kappa(a)} A \sqcap
    \bigsqcap_{A\in\OCR(\B)\setminus\kappa(a)} \lnot A\right)(a),\ \RO\right)
  \end{align*}
  has a model that respects $(\OCR(\B),\Ymc)$, for all $1 \leq i \leq k$. The \SHOIQ-BKB
  $\widehat{\Bmf}_{X_{i}}$ is of size polynomial in the size of \B and can be constructed in time
  exponential in the size of \B. We can check if $\widehat{\Bmf}_{X_{i}}$ has a model that respects
  $(\OCR(\B),\Ymc)$ non-deterministically in time exponential in the size of
  $\widehat{\Bmf}_{X_{i}}$ \cite{BaGL-KR08,BaGL-ToCL12}, and thus non-deterministically in time
  exponential in the size of \Bmf.
\end{proof}

Together with the lower bound shown in Theorem~\ref{thm:alcel-rigid-concepts-nexptime-hard}, we
obtain \NExpTime-complete\-ness for the consistency problem in \LMLO for \LM and \LO being DLs between
\ALC and \SHOIQ if $\OCR \neq \emptyset$ and $\ORR = \emptyset$.

Summing up the results, we obtain the following corollary.

\begin{corollary}
  For all \LM, \LO between \ALC and \SHOQ, the consistency problem in \LMLO is
  \begin{itemize}
  \item \ExpTime-complete if $\OCR=\emptyset$ and $\ORR=\emptyset$,
  \item \NExpTime-complete if $\OCR\ne\emptyset$ and $\ORR=\emptyset$, and
  \item \TwoExpTime-complete if $\OCR\ne\emptyset$ and $\ORR\ne\emptyset$.
    % otherwise.
  \end{itemize}
\end{corollary}

\section{The Case of \texorpdfstring{\EL}{EL}}
\label{sec:case-el} 

In this section, we give some complexity results for context DLs
\LMLO where \LM or \LO are \EL.
%
In Section~\ref{sec:dlinner-el}, we consider \LMEL where \LM
is between \ALC and \SHOQ.  Then, in Section~\ref{sec:dlouter-el}, we consider
the remaining context DLs \ELLO where \LO is either \EL \todo{\ELEL?}or
between \ALC and \SHOQ.

\subsection{The Contextualized Description Logics \texorpdfstring{\LMEL}{LM[EL]}}
\label{sec:dlinner-el}

In this section, we consider \LMEL where \LM is between \ALC
and \SHOQ.
%
The lower bounds already hold for \ALCEL.

\begin{theorem}\label{thm:alcel-lower-no-rigid}
  The consistency problem in \ALCEL is \ExpTime-hard if $\OCR = \ORR = \emptyset$.
\end{theorem}

\begin{proof}
    Deciding whether a given conjunction of \ALC-axioms~\B is consistent is
    \ExpTime-hard~\cite{Sch-IJCAI91}.  Obviously, \B is also an \ALCEL-BKB.
\end{proof}

For the cases of rigid names, the lower bounds of \NExpTime are obtained by a
careful reduction of the satisfiability problem in the temporalized DL
\EL-LTL~\cite{BoTh-IJCAI15,BoTh-LTCS-15-07}, which is a fragment of \ALC-LTL introduced
in~\cite{BaGL-KR08,BaGL-ToCL12}.
%
For the sake of completeness, we recall the basic definitions of \Lmc-LTL here,
where \Lmc is a DL.

\begin{definition}[Syntax of \Lmc-LTL]
  \emph{\Lmc-LTL-formulas over \Osig} are defined by induction:
  \begin{itemize}
  \item if $\alpha$ is an \Lmc-axiom over \Osig, then $\alpha$ is an \Lmc-LTL-formula, and
  \item if $\phi,\psi$ are \Lmc-LTL-formulas over \Osig, then so are $\lnot\phi$ (negation),
    $\phi\land\psi$ (conjunction), $\phi\until\psi$ (until), $\Next\phi$ (next), and
  \item nothing else is an \Lmc-LTL-formula. \qedhere
  \end{itemize}
\end{definition}

As usual in temporal logics, we use the following abbreviations: 
\begin{itemize}
\item $\phi\lor\psi$ (disjunction) for $\lnot(\lnot\phi\land\lnot\psi)$,
\item $\mathsf{true}$ (tautology) for $A(a)\lor\lnot A(a)$ where $A\in\OC$ is arbitrary but fixed,
\item $\Diamond\phi$ (eventually) for $\mathsf{true}\until\phi$, and
\item $\Box\phi$ (always) for $\lnot\Diamond\lnot\phi$.
\end{itemize}

The semantics of \Lmc-LTL is based on DL-LTL-structures.  These are sequences of
\Osig-inter\-pre\-tations over the same non-empty domain that additionally respect
rigid names and the rigid individual assumption.

\begin{definition}[DL-LTL-structure]
    A \emph{DL-LTL-structure over \Osig} is a sequence $\Imf=(\I_i)_{i \geq 0}$ of
    \Osig-interpretations $(\Delta,\cdot^{\I_i})$ such that
    $x^{\I_{i}} = x^{\I_{j}}$ holds for all $x\in\OCR\cup\ORR\cup\OI$, $i,j>0$.
\end{definition}

We are now ready to define the semantics of \Lmc-LTL.

\begin{definition}[Semantics of \Lmc-LTL]
  The validity of an \Lmc-LTL-formula~$\phi$ in a DL-LTL-structure $\Imf=(\I_i)_{i\ge 0}$ at
  time~$i\ge 0$, denoted by $\Imf,i\models\varphi$, is defined inductively:

  \vspace{\topsep}
  \begin{tabular}{l@{\quad}l@{\quad}l}
    $\Imf,i\models\alpha$        & iff & $\I_{i}\models\alpha$ where $\alpha$ is an \ALC-axiom over \Osig,\\
    $\Imf,i\models\phi\land\psi$      & iff & $\Imf,i\models\phi$ and $\Imf,i\models\psi$, \\
    $\Imf,i\models\lnot\phi$       & iff & not $\Imf,i\models\phi$, \\
    $\Imf,i\models\Next\phi$   & iff & $\Imf,i+1\models\phi$, \\
    $\Imf,i\models\phi\until\psi$ & iff & there is $k\geq i$ such that $\Imf,k\models\psi$ and \\
                      &     & $\Imf,j\models\phi$ for all $j$ with $i\leq j < k$.
  \end{tabular}

  \vspace{\topsep} We call an \Lmc-LTL-structure \Imf a \emph{model of~$\phi$} if
  $\Imf,0\models\phi$.  The \emph{satisfiability problem in \Lmc-LTL} is the question whether a
  given \Lmc-LTL-formula~$\phi$ has a model.
\end{definition}

In~\cite{BoTh-IJCAI15,BoTh-LTCS-15-07}, it is shown that the satisfiability problem in \EL-LTL is
\NExpTime-hard as soon as rigid concept names are available.  We reduce the satisfiability problem
in \EL-LTL to the consistency problem in \ALCEL to obtain the lower bounds of \NExpTime, where we
use the fact that the lower bounds of~\cite{BoTh-IJCAI15,BoTh-LTCS-15-07} hold already for a
syntactically restricted fragment of \EL-LTL.

\begin{theorem}\label{thm:alcel-rigid-concepts-nexptime-hard}
  The consistency problem in \ALCEL is \NExpTime-hard if $\OCR \neq \emptyset$ and
  $\ORR = \emptyset$.
\end{theorem}

\begin{proof}
  In fact, the lower bounds hold for \EL-LTL-formulas of the form $\Box\phi$ where $\phi$ is an
  \EL-LTL-formula that contains only \Next as temporal operator~\cite{BoTh-LTCS-15-07}.

  Let $\Box\phi$ be such an \EL-LTL-formula over~\Osig.  Now, we obtain the m-concept $C_\phi$
  from~$\phi$ by replacing \EL-axioms~$\alpha$ by \oalpha, $\land$ by $\sqcap$, and subformulas of
  the form $\Next\psi$ by $\forall t.\psi\sqcap\exists t.\psi$, where $t\in\MR$ is arbitrary but
  fixed.

  \begin{claim}
    $\Box\phi$ is satisfiable iff $\B=\top\sqsubseteq C_\phi\sqcap\exists t.\top$ is consistent.
  \end{claim}

  \begin{claimproof}
    First, assume that $\Box\phi$ is satisfiable.
    Take any DL-LTL-structure $\Imf=(\Delta,\cdot^{\I_i})_{i\ge 0}$ with $\Imf,0\models\Box\phi$.
    We define the nested interpretation \JJ as follows:
    \begin{align*}
      \Cbb & \coloneqq \{c_i \mid i \geq 0\},\\
      \Delta^{\J} & \coloneqq \Delta,\\
      \cdot^{\I_{c_{i}}} & \coloneqq \cdot^{\I_{i}},\\
      t^\J & \coloneqq \{(c_{i},c_{i+1}) \mid i \geq 0\}.
    \end{align*}
    
    We now show that for every $i\ge 0$, we have $\Imf,i\models\phi$ iff $c_i\in C_\phi^\J$ by
    induction on the structure of~$\phi$:
    
    \begin{tabularx}{\linewidth}{@{}l@{ }X@{}}
      $\phi = \alpha:$ & $\Imf,i\models\phi$ 
               iff $\I_{i} \models \alpha$ 
               iff $\I_{c_{i}} \models \alpha$
               iff $c_{i} \in \oax{\alpha}^{\J} = C_{\phi}^{\J}$, \\[1ex]
      $\phi = \lnot \psi$: &  $\Imf,i\models\phi$ 
               iff $\Imf,i\not\models\psi$ 
               iff $c_i\notin C_\psi^\J$ 
               iff $c_i\in(\lnot C_\psi)^\J=C_\phi^\J$, \\[1ex]
      $\phi = \psi_1\land\psi_2:$ & $\Imf,i\models\phi$ 
               iff $\Imf,i\models\psi_{1}$ and $\Imf,i\models\psi_{2}$ 
               iff $c_{i} \in C_{\psi_{1}}^{\J}$ and $c_{i} \in C_{\psi_{2}}^{\J}$ \\
             & \hphantom{$\Imf,i\models\phi$} iff $c_{i} \in (C_{\psi_{1}} \sqcap C_{\psi_{2}})^{\J}
               = C_{\phi}^{\J}$, and\\[1ex]
      $\phi = \Next\psi$: & $\Imf,i\models\phi$
               iff $\Imf,i+1\models\psi$
               iff $c_{i+1} \in C_{\psi}^{\J}$
               iff $c_{i} \in (\forall t.C_{\psi} \sqcap \exists t.C_{\psi})^{\J} = C_{\phi}^{\J}$,
    \end{tabularx}
    \noindent
    where $\alpha$ is an \EL-axiom over~\Osig.
    %
    It follows that $\Imf,0\models\Box\phi$ iff $\J\models\top\sqsubseteq C_\phi$.  Furthermore,
    since $(c_{i},c_{i+1})\in t^\J$, we have $c_{i}\in(\exists t.\top)^\J$.  Thus,
    $\J\models\top\sqsubseteq\exists t.\top$.

    For the if-direction, take any nested interpretation \JJ that is a model of
    $\top\sqsubseteq C_\phi\sqcap\exists t.\top$.  Let $\Psf$ be an infinite path
    $\Psf=c_0 c_1 \dots$ with $c_i\in\Cbb$ and $(c_{i},c_{i+1}) \in t^\J$ for every $i \geq 0$.
    Such a path exists, because $\J\models\top\sqsubseteq\exists t.\top$.  We define the nested
    interpretation
    $\J_{\Psf}\coloneqq(\{c_i\mid i\ge 0\},\cdot^{\J_{\Psf}},\Delta^{\J},(\cdot^{\I_{c_{i}}})_{i
      \geq 0})$ where $\cdot^{\J_{\Psf}}$ is the restriction of $\cdot^{\J}$ to the domain
    $\{c_i\mid i\ge 0\}$.
        
    By construction we have that $\J_{\Psf}\models\top\sqsubseteq\exists t.\top$.  We show by a
    simple case distinction that $\J_{\Psf}\models\top\sqsubseteq C_\phi$.
    %
    If $C_\phi$ does not contain any role name $r\in\MR$, the restriction on the set of worlds
    preserves the entailment relation.
    %
    Otherwise, $C_\phi$ is of the form $\forall t.C_\psi\sqcap\exists t.C_\psi$.  Since
    $\J_{\Psf}\models\top\sqsubseteq\exists t.\top$, $\J_{\Psf}\models\top\sqsubseteq C_\psi$, and
    there is only one $t$-successor, we have $\J_{\Psf}\models\top\sqsubseteq C_\phi$.  Hence,
    $\J_{\Psf}\models\top\sqsubseteq C_{\phi}\sqcap\exists t.\top$.

    We define the DL-LTL-structure \Imf over~\Osig as
    $\Imf\coloneqq(\Delta^{\J},\cdot^{\I_i})_{i\ge 0}$ where
    $\cdot^{\I_{i}}\coloneqq\cdot^{\I_{c_{i}}}$.
    %
    Again, we show that for every $i \geq 0$, that we have $c_i \in C_{\phi}^{\J_{\Psf}}$ iff
    $\Imf,i\models\phi$ by induction on the structure of~$\phi$:

    \begin{tabularx}{\linewidth}{@{}l@{ }X@{}}
      $\phi = \alpha:$ & $c_{i} \in C_{\phi}^{\J_{\Psf}} = \oax{\alpha}^{\J_{\Psf}}$
               iff $\I_{c_{i}} \models \alpha$
               iff $\I_{i} \models \alpha$ 
               iff $\Imf,i \models \phi$,\\[1ex]
      $\phi = \lnot \psi$: &  $c_{i} \in C_{\phi}^{\J_{\Psf}} = (\lnot C_{\psi})^{\J_{\Psf}}$
               iff $c_{i} \notin C_{\psi}^{\J_{P}}$
               iff $\Imf,i\not\models\psi$
               iff $\Imf,i\models\phi$,\\[1ex]
      $\phi = \psi_1\land\psi_2:$ & $c_{i} \in C_{\phi}^{\J_{\Psf}} = (C_{\psi_{1}} \sqcap C_{\psi_{2}})^{\J_{\Psf}}$
               iff $c_{i} \in C_{\psi_{1}}^{\J_{\Psf}}$ and $c_{i} \in C_{\psi_{1}}^{\J_{\Psf}}$ 
               iff $\Imf,i\models\psi_{1}$ and $\Imf,i\models\psi_{2}$ \\
             & \hphantom{$c_{i} \in C_{\phi}^{\J_{\Psf}}$} iff $\Imf,i\models\phi$, and\\[1ex]
      $\phi = \Next\psi$: & $c_{i} \in C_{\phi}^{\J_{\Psf}} = (\forall t.C_{\psi} \sqcap \exists t.C_{\psi})^{\J_{\Psf}} $
               iff $c_{i+1} \in C_{\psi}^{\J_{\Psf}}$
               iff $\Imf,i+1\models\psi$
               iff $\Imf,i\models\phi$,
    \end{tabularx}
    %
    where $\alpha$ is an \EL-axiom over~\Osig.  It follows that
    $\J_{P}\models\top\sqsubseteq C_\phi$ iff $\Imf,0\models\Box\phi$.
    \end{claimproof}

    This claim yields the lower bound of \NExpTime for the consistency problem
    in \ALCEL if $\OCR\ne\emptyset$.
\end{proof}

Next, we prove the upper bound of \NExpTime for the consistency problem in the
case of rigid names.

\begin{theorem}\label{thm:shoqel-upper-bound}
  The consistency problem in \SHOQEL is in \NExpTime if $\OCR \neq \emptyset$ and
  $\ORR \neq \emptyset$.
\end{theorem}

\begin{proof}
  We again use Lemma~\ref{lem:admissible-and-outerConsistent}.  First, we non-deterministically
  guess a set $\Xmc \subseteq \powerset{\ran(\bsf)}$ and construct the \EL-BKB $\B_\Xmc$ over~\Osig
  as in the proof of Theorem~\ref{thm:shoqshoq-with-rigid-names-twoexptime}, which is actually a
  conjunction of \EL-literals over~\Osig, i.e.~of (negated) \EL-axioms over~\Osig.  The following
  claim shows that consistency of~$\B_\Xmc$ can be reduced to consistency of a conjunction of
  \ELObot-axioms over~\Osig, where \ELObot is the extension of \EL with nominals and the bottom
  concept.

  \begin{claim}
    For every conjunction of \EL-literals \B over~\Osig, there exists an equisatisfiable conjunction
    $\B'$ of \ELObot-axioms over~\Osig.
  \end{claim}

  \begin{claimproof}
    Let $\B$ be a conjunction of \EL-literals over~\Osig, i.e.
    \begin{align*}
      \B & = \alpha_{1} \land \dots \land \alpha_{n} \land \lnot\beta_{1} \land \dots \land \lnot\beta_{m} \\
    \intertext{where $\alpha_{i}$, $1\leq i \leq n$, $\beta_{j}$, $1\leq j \leq m$ are \EL-axioms over~\Osig. We define $\B'$ as follows:}
      \B' & = \alpha_{1} \land \dots \land \alpha_{n} \land \gamma_{1} \land \dots \land \gamma_{m}, \\
    \intertext{where} 
      \gamma_{i} & \coloneqq
        \begin{cases}
          C(a_{i}) \land D'(a_{i}) \land D \sqcap D' \sqsubseteq \bot & \text{if $\beta_{i} = C \sqsubseteq D$,}\\
          A'(a) \land A \sqcap A' \sqsubseteq \bot & \text{if $\beta_{i} = A(a)$, and}\\
          \{a\} \sqcap \exists r.\{b\} \sqsubseteq \bot & \text{if $\beta_{i} = r(a,b)$}
        \end{cases}
    \end{align*}
    with $A',D'$ being fresh concept names and $a_{i}$ being fresh individual names.  It is easy to
    see that if an \Osig-interpretation \I is a model of
    $\lnot\beta_{1}\land\dots\land\lnot\beta_{m}$, there exists an extension of \I that is a model
    of $\gamma_{1}\land\dots\land\gamma_{m}$.  Conversely, if an \Osig-interpretation $\I'$ is a
    model of $\gamma_{1}\land\dots\land\gamma_{m}$, it is also a model of
    $\lnot\beta_{1}\land\dots\land\lnot\beta_{m}$.  Hence \B and $\B'$ are equisatisfiable.
    \end{claimproof}

    By this claim and the fact that consistency of conjunctions of \ELObot-axioms can be decided in
    polynomial time~\cite{BaBL-IJCAI05}, we obtain our claimed upper bound.
\end{proof}

Summming up the results of this section, we obtain the following corollary.

\begin{corollary}
  For all \LM between \ALC and \SHOQ, the consistency problem in \LMEL is
    \begin{itemize}
        \item \ExpTime-complete if $\OCR=\emptyset$ and $\ORR=\emptyset$, and
        \item \NExpTime-complete otherwise.
    \end{itemize}
\end{corollary}

\begin{proof}
  The lower bounds follow from Theorems~\ref{thm:alcel-lower-no-rigid}
  and~\ref{thm:alcel-rigid-concepts-nexptime-hard}.  The upper bound of \ExpTime in the case
  $\OCR=\ORR=\emptyset$ follows immediately from Theorem~\ref{thm:shoqshoq-without-rigid-exptime},
  whereas the upper bound of \NExpTime follows from Theorem~\ref{thm:shoqel-upper-bound}.
\end{proof}

\subsection{The Contextualized Description Logics \texorpdfstring{\ELLO}{EL[LO]}}
\label{sec:dlouter-el}

In this section, we consider \ELLO where \LO is either \EL or between \ALC and \SHOQ.
%
Instead of considering \ELLO-BKBs, we allow only \emph{conjunctions} of m-axioms.  \todo{short
  explanation} Then the consistency problem becomes trivial in the case of \ELEL since all
\ELEL-BKBs are consistent, as \EL lacks to express contradictions.
%
This restriction, however, does not yield a better complexity in the cases of \ELLO, where \LO is
between \ALC and \SHOQ.

First, we show the lower bounds for the consistency problem in \ELALC.  We again distinguish the
three cases of which names are allowed to be rigid.

\begin{theorem}\label{thm:el-lower-exp}
  The consistency problem in \ELALC is \ExpTime-hard if $\OCR = \ORR = \emptyset$.
\end{theorem}

\begin{proof}
  Deciding whether a given conjunction $\B = \alpha_{1} \land \dots \land \alpha_{n}$ of \ALC-axioms
  is consistent is \ExpTime-hard\cite{Sch-IJCAI91}.  Obviously, \B is consistent iff the \ELALC-BKB
  $(\oax{\alpha_{1}} \sqcap \dots \sqcap \oax{\alpha_{n}})(c)$ is consistent, where $c\in\MI$.
\end{proof}

For the case of rigid role names, we have lower bounds of \TwoExpTime.

\begin{theorem}\label{thm:el-lower-2exp}
  The consistency problem in \ELALC is \TwoExpTime-hard if $\OCR \neq \emptyset$ and
  $\ORR \neq \emptyset$.
\end{theorem}

\begin{proof}
  \todo{beweis noch einmal nachvollziehen} To show the lower bound, we adapt the proof ideas
  of~\cite{BaGL-KR08,BaGL-ToCL12}, and reduce the word problem for exponentially space-bounded
  alternating Turing machines (i.e.~is a given word~$w$ accepted by the machine~$M$) to the
  consistency problem in \ELALC with rigid roles, i.e.~$\ORR\ne\emptyset$.
  %
  In~\cite{BaGL-KR08,BaGL-ToCL12}, a reduction was provided to show \TwoExpTime-hardness for the
  temporalized DL \ALC-LTL in the presence of rigid roles.
  %
  Here, we mimic the properties of the time dimension that are important for the reduction using a
  role name $t\in\MR$.

  Our \ELALC-BKB is the conjunction of the \ELALC-BKBs introduced below.
  %
  First, we ensure that we never have a \emph{last} time point:
  \begin{gather*}
    \top\sqsubseteq\exists t.\top
  \end{gather*}
  Note that in the corresponding model, we do not enforce a $t$-chain since cycles are not
  prohibited.  This, however, is not important in the reduction.

  The \ALC-LTL-formula obtained in the reduction of~\cite{BaGL-KR08,BaGL-ToCL12} is a conjunction of
  \ALC-LTL-formulas of the form $\Box\phi$, where $\phi$ is an \ALC-LTL-formula.  This makes sure
  that $\phi$ holds in all (temporal) worlds.  For the cases where $\phi$ is an \ALC-axiom, we can
  simply express this by:
  \begin{gather*}
    \top\sqsubseteq\oax{\phi}
  \end{gather*}

  This captures all except for two conjuncts of the \ALC-LTL-formula of the reduction
  of~\cite{BaGL-KR08,BaGL-ToCL12}.  There, a $k$-bit binary counter using concept names
  $A_0',\dots,A_{k-1}'$ was attached to the individual name $a$, which is incremented along the
  temporal dimension.  We can express something similar in \ELALC, but instead of incrementing the
  counter values along a sequence of $t$-successors, we have to go backwards since \EL does allow
  for branching but does not allow for values restrictions, i.e.~we cannot make sure that all
  $t$-successors behave the same.  More precisely, if the counter value~$n$ is attached to~$a$ in
  context~$c$, the value $n+1$ (modulo $2^k-1$) must be attached to~$a$ in \emph{all} of $c$'s
  $t$-predecessors.

  First, we ensure which bits must be flipped:
  \begin{align*}
    \bigwedge_{i<k} \Bigl( \exists t.\bigl(\oax{A_{0}'(a)} \sqcap \dots \sqcap \oax{A_{i-1}'(a)} \sqcap \oax{A_{i}'(a)}\bigr)
    & \ \sqsubseteq\ \oax{(\lnot A_{i}')(a)}\Bigr)\\ 
    \bigwedge_{i<k}\Bigl( \exists t.\bigl(\oax{A_{0}'(a)}\sqcap\ldots\sqcap \oax{A_{i-1}'(a)} \sqcap \oax{(\lnot A_{i}')(a)}\bigr)
    & \ \sqsubseteq\ \oax{A_{i}'(a)}\Bigr)
    \end{align*}

    Next, we ensure that all other bits stay the same:
    \begin{align*}
      \bigwedge_{0<i<k}\ \bigwedge_{j<i}\Bigl(\exists t.\bigl(\oax{(\lnot A_{j}')(a)}\sqcap\oax{A_{i}'(a)}\bigr)
      & \ \sqsubseteq\ \oax{A_{i}'(a)}\Bigr)\\
      \bigwedge_{0<i<k}\ \bigwedge_{j<i}\Bigl(\exists t.\bigl(\oax{(\lnot A_{j}')(a)}\sqcap\oax{(\lnot A_{i}')(a)}\bigr)
      & \ \sqsubseteq\ \oax{(\lnot A_{i}')(a)}\Bigr)
    \end{align*}

    Note that due to the first m-axiom above, we enforce that every context has a $t$-successor.  By
    the other m-axioms, we make sure that we enforce a $t$-chain of length $2^k$.
    %
    As in~\cite{BaGL-KR08,BaGL-ToCL12}, it is not necessary to initialize the counter.  Since we
    decrement the counter along the $t$-chain (modulo $2^k-1$), every value between $0$ and $2^k-1$
    is reached.

    The conjunction of all the \ELALC-BKBs above yields an \ELALC-BKB~\B that is consistent iff $w$
    is accepted by~$M$.
  \end{proof}

Finally, we obtain a lower bound of \NExpTime in the case of rigid concept names
only.

\begin{theorem}\label{thm:el-lower-nexp}
  The consistency problem in \ELALC is \NExpTime-hard if $\OCR \neq \emptyset$ and
  $\ORR = \emptyset$.
\end{theorem}

\begin{proof}
  To show the lower bound, we again adapt the proof ideas of~\cite{BaGL-KR08,BaGL-ToCL12}, and
  reduce an exponentially bounded version of the domino problem to the consistency problem in \ELALC
  with rigid concepts, i.e.~$\OCR\ne\emptyset$ and $\ORR=\emptyset$.
  %
  In~\cite{BaGL-KR08,BaGL-ToCL12}, a reduction was provided to show \NExpTime-hardness for the
  temporalized DL \ALC-LTL in the presence of rigid concepts.
  %
  As in the proof of Theorem~\ref{thm:el-lower-2exp}, we mimic the properties of the time dimension
  that are important for the reduction using a role name $t\in\MR$.
    
  Our \ELALC-BKB is the conjunction of the \ELALC-BKBs introduced below.  We proceed in a similar
  way as in the proof of Theorem~\ref{thm:el-lower-2exp}.
  %
  First, we ensure that we never have a \emph{last} time point:
  \begin{gather*}
    \top\sqsubseteq\exists t.\top
  \end{gather*}
  Note that in the corresponding model, we do not enforce a $t$-chain since cycles are not
  prohibited.  As in the reduction in the proof of Theorem~\ref{thm:el-lower-2exp}, this is not
  important in the reduction here.

  Next, note that since the $\Box$-operator distributes over conjunction, most of the conjuncts of
  the \ALC-LTL-formula of the reduction of~\cite{BaGL-KR08,BaGL-ToCL12} can be rewritten as
  conjunctions of \ALC-LTL-formulas of the form $\Box\alpha$, where $\alpha$ is an \ALC-axiom.  As
  already argued in the proof of Theorem~\ref{thm:el-lower-2exp}, this can equivalently be expressed
  by $\top\sqsubseteq\oalpha$.

  In~\cite{BaGL-KR08,BaGL-ToCL12}, a $(2n+2)$-bit binary counter is employed using concept names
  $Z_0,\dots,Z_{2n+1}$.  This counter is attached to an individual name~$a$, which is incremented
  along the temporal dimension.  This can be expressed in \ELALC as shown in the proof of
  Theorem~\ref{thm:el-lower-2exp}:

  \begin{align*}
    \bigwedge_{i<2n+2} \Bigl( \exists t.\bigl(\oax{Z_{0}(a)} \sqcap\ldots\sqcap \oax{Z_{i-1}(a)} \sqcap \oax{Z_{i}(a)}\bigr)
    & \ \sqsubseteq\ \oax{(\lnot Z_{i})(a)}\Bigr)\\ 
    \bigwedge_{i<2n+2}\Bigl( \exists t.\bigl(\oax{Z_{0}(a)}\sqcap\ldots\sqcap \oax{Z_{i-1}(a)} \sqcap \oax{(\lnot Z_{i})(a)}\bigr)
    & \ \sqsubseteq\ \oax{Z_{i}(a)}\Bigr)\\
    \bigwedge_{0<i<2n+2}\ \bigwedge_{j<i}\Bigl(\exists t.\bigl(\oax{(\lnot Z_{j})(a)}\sqcap\oax{Z_{i}(a)}\bigr)
    & \ \sqsubseteq\ \oax{Z_{i}(a)}\Bigr)\\
    \bigwedge_{0<i<2n+2}\ \bigwedge_{j<i}\Bigl(\exists t.\bigl(\oax{(\lnot Z_{j})(a)}\sqcap\oax{(\lnot Z_{i})(a)}\bigr)
    & \ \sqsubseteq\ \oax{(\lnot Z_{i})(a)}\Bigr)
  \end{align*}

  Note that due to the first m-axiom above, we enforce that every context has a $t$-successor.  By
  the other m-axioms, we make sure that we enforce a $t$-chain of length $2^{2n+2}$.
  %
  As in~\cite{BaGL-KR08,BaGL-ToCL12}, it is not necessary to initialize the counter.  Since we
  decrement the counter along the $t$-chain (modulo $2^{2n+1}$), every value between $0$ and
  $2^{2n+1}$ is reached.

  In~\cite{BaGL-KR08,BaGL-ToCL12}, an \ALC-LTL-formula is used to express that the value of the
  counter in shared by all domain elements belonging to the current (temporal) world.  This is
  expressed using a disjunction, which we can simulate as follows:
  \begin{gather*}
    \bigwedge_{0\le i\le 2n+1} \Bigl(\oax{Z_i(a)}\sqsubseteq\oax{\top\sqsubseteq Z_i}\ \land\
    \oax{(\lnot Z_i)(a)}\sqsubseteq\oax{Z_i\sqsubseteq\bot}\Bigr)
  \end{gather*}

  Next, there is a concept name~$N$, which is required be non-empty in every (temporal) world.  We
  express this using a role name $r\in\OR$:
  \begin{gather*}
    \top\sqsubseteq\oax{(\exists r.N)(a)}
  \end{gather*}

  It is only left to express the following \ALC-LTL-formula of~\cite{BaGL-KR08,BaGL-ToCL12}:
  \begin{gather*}
    \Box\Bigl(\bigvee_{d\in D} (\top\sqsubseteq d')\Bigr)
  \end{gather*}
  For readability, let $D=\{d_1,\dots,d_k\}$.  We use non-convexity of \ALC as follows to express
  this:
  \begin{gather*}
    \top\sqsubseteq\oax{(d_1'\sqcup\dots\sqcup d_k')(a)}\ \land\ \bigwedge_{1\le i\le k}
    \Bigl(\oax{d_i'(a)}\sqsubseteq\oax{\top\sqsubseteq d_i'}\Bigr)
  \end{gather*}

  The conjunction of all the \ELALC-BKBs above yields an \ELALC-BKB~\B that is consistent iff the
  exponentially bounded version of the domino problem has a solution.
\end{proof}

Summing up the results of this section together with the upper bounds of
Section~\ref{sec:complexity-consis-problem}, we obtain the following corollary.

\begin{corollary}
  For all \LO between \ALC and \SHOQ, the consistency problem in \ELLO is
  \begin{itemize}
  \item \ExpTime-complete if $\OCR=\emptyset$ and $\ORR=\emptyset$,
  \item \NExpTime-complete if $\OCR\ne\emptyset$ and $\ORR=\emptyset$, and
  \item \TwoExpTime-complete if $\OCR\ne\emptyset$ and $\ORR\ne\emptyset$.
  \end{itemize}
\end{corollary}

\begin{proof}
  The lower bounds follow from Theorems~\ref{thm:el-lower-exp}, \ref{thm:el-lower-2exp},
  and~\ref{thm:el-lower-nexp}.  The corresponding upper bounds follow from
  Theorems~\ref{thm:shoqshoq-without-rigid-exptime}, \ref{thm:shoqshoq-with-rigid-names-twoexptime},
  and~\ref{thm:shoqshoq-with-rigid-concepts-nexptime}.
\end{proof}

\subsection{What to do with \texorpdfstring{\SHOIQEL}{SHOIQ[EL]} and \texorpdfstring{\ELSHOIQ}{EL[SHOIQ]}}
\label{sec:shoiq-el-and-el-shoiq}

\begin{theorem}\label{thm:shoiqel-lower-no-rigid}
  The consistency problem in \SHOIQEL is \NExpTime-hard if no rigid names are allowed, i.e.\ $\OCR = \ORR = \emptyset$.
\end{theorem}

\begin{proof}
    Deciding whether a given conjunction of \ALCOIQ-axioms~\B is consistent is
    \NExpTime-complete~\cite{Tob-JAIR00}.  Obviously, \B is also an \SHOIQEL-BKB.
\end{proof}

\begin{theorem}\label{thm:elshoiq-lower-no-rigid}
  The consistency problem in \ELSHOIQ is \NExpTime-hard if no rigid names are allowed, i.e.\ $\OCR = \ORR = \emptyset$.
\end{theorem}

\begin{proof}
  Deciding whether a given conjunction $\B = \alpha_{1} \land \dots \land \alpha_{n}$ of \ALCOIQ-axioms
  is consistent is \NExpTime-complete~\cite{Tob-JAIR00}.  Obviously, \B is consistent iff the \ELSHOIQ-BKB
  $(\oax{\alpha_{1}} \sqcap \dots \sqcap \oax{\alpha_{n}})(c)$ is consistent, where $c\in\MI$.
\end{proof}


\section{Adding Contextualized Concepts}
\label{sec:adding-cont-concepts}


In this section we discuss a possible extension to our contextualized description logic.  We start
with a little comparison to temporal description logics. Both DL-LTL-structures and nested DL
interpretations use a \emph{possible worlds semantics}. Single time points or contexts are
represented in a meta dimension and for each such a meta element, or possible world, there exists
one DL interpretation on the object level. Important for both the expressivity and the complexity of
reasoning problems is how these two dimensions can interact. Syntactically, it is a question where
these \emph{meta} operators, i.e.\ temporal or contextual, can be used.

\begin{table}
  \caption{Classification of different two-dimensional temporal and contextual description logics
    (\cite{LuWZ-TIME08,BaGL-ToCL12,KG-JELIA10}) }
  \centering
  \begin{tabular}{lM{0.1\linewidth}M{0.25\linewidth}M{0.17\linewidth}N}
    \toprule
    \multicolumn{2}{c}{\multirow{2}{5cm}[-1ex]{\centering Temporal or contextual operators}}
    & \multicolumn{2}{c}{in front of/around axioms}\\
    & & yes & no &\\[10pt]
    \midrule
    & yes & temporal $\text{LTL}_{\ALC}$, \klarALC, \LMLOplus & $\text{LTL}_{\ALC}$, \hspace{2cm}$\emptyset$ &\\[15pt]
    \cmidrule{2-4}
    \multirow{-2}{*}[2.4ex]{
    \centering inside concepts
    }& no & \ALC-LTL, \hspace{2cm}\LMLO & \ALC &\\[15pt]
    \bottomrule
  \end{tabular}
  \label{tab:classification-tdl-cdl}
\end{table}

Table~\ref{tab:classification-tdl-cdl} gives an overview over some two-dimensional DLs. This is not
a complete overview, but it illustrates some common properties about the complexity of the
consistency problem. Note first that neither having meta operators in front of axioms nor having
meta operators inside concepts is strictly more expressive than the other. Meta operators in front
of axioms can handle general knowledge that holds in some or all worlds.  On the other hand, meta
operators inside object concepts allow to access the extension of concepts in other worlds.
%
We illustrate this by an example taken from \cite{LuWZ-TIME08}.
\begin{gather*}
  \Diamond\Box(\mathsf{European\_country} \sqsubseteq \mathsf{EU\_country})\\
  \mathsf{Independent\_country} \sqsubseteq \Box\mathsf{Independent\_country}
\end{gather*}
The first temporalised GCI states that eventually, i.e.\ at some time point $t$ in future, all
European countries will forever be EU members, i.e.\ in every time point after $t$. The second axiom
says that the extension of the concept $\mathsf{Independent\_country}$ does not decrease.

In \LMLO, we only have the contextual operator $\oax{\cdot}$ around axioms. We can express that
certain axioms must hold in some worlds, but we cannot access the extension of a concept from
another world, i.e.~we cannot express the set of domain elements which belong to some concept in
another context.  The idea is to overcome this lack of expressive power by introducing a new
contextual operator inside object concepts.

Before extending \LMLO, we analyse some general behaviour of two-dimensional DLs.  The first common
property that we want to focus on is the computational complexity if rigid roles are present.
If meta operators are allowed within object concepts, the consistency problem becomes
undecidable. This holds for all logics in the first row of
Table~\ref{tab:classification-tdl-cdl}. We prove this negative result for \LMLOplus below. If meta operators
are only allowed in front of axioms, we may retain decidability, but at the cost that the
consistency problem is one exponential harder.
%
If no rigid names are allowed, the complexity of the consistency problem increases for
logics with meta operators both in front of axioms and in object concepts: from ExpTime-completeness for \ALC
to \ExpSpace-completeness for temporal $\text{LTL}_{\ALC}$ TBoxes and to \TwoExpTime-completeness
for \klarALC knowledge bases. If only one kind of meta operators is allowed, the complexity
class stays the same, i.e.\ the
consistency problem is ExpTime-complete in \ALCALC, \ALC-LTL, and $\text{LTL}_{\ALC}$.

A setting in which only rigid concepts, but no rigid roles are allowed, is only interesting if meta
operators are not allowed inside object concepts. Otherwise, rigid concepts can easily be
simulated. In $\text{LTL}_{\ALC}$, this can be done by adding $C\sqsubseteq\Box C$ and
$\lnot C\sqsubseteq\Box\lnot C$ to the TBox.  We show below how rigid concepts can be simulated in
\LMLOplus.
%
The contextualized description logic \LMLOplus is an extension of \LMLO in which we additionally allow
contextualized object concepts. Therefore, we update the definition of o-concepts from
Def. \ref{def:syntax-lmlo}.  

\begin{definition}[Object concepts of \LMLOplus]
The set of \emph{concepts of the object logic \LO (o-concepts)} is the smallest set such that
\begin{itemize}
\item for all $A\in\OC$, $A$ is an object concept,
\item if $D$ is an object concept, $\mathbf{C}\in\MC$, $\mathbf{r}\in\MR$, then $\ocont{C}[r]{D}$ is
  also an object concept, and
\item all complex concepts that can be built with the concept constructors allowed in \LO are
  object concepts.
\end{itemize}

Furthermore, for a nested interpretation \JJ the mapping $\cdot^{\I_{c}}$ is extended to
$\ocont{C}[r]{D}$ as follows: $(\ocont{C}[r]{D})^{\I_{c}} \coloneqq \{d \in \Delta^{\J} \mid \text{there is
some $c'\in\Cbb$ s.t.\ $(c,c')\in r^{\J}$ and $d \in D^{\I_{c'}}$}\}$.
\end{definition}

Following customs of modal logic, we use $\ocont*{C}[r]{D}$ as an abbreviation for
$\lnot\ocont{C}[r]{(\lnot D)}$. Intuitively, in a context $c$ the concept $\ocont{C}[r]{D}$ denotes
the set of all object domain elements that belong to the concept $D$ in some other context $c'$
which belongs to the meta concept $\mathbf{C}$ and is related to $c$ via $\mathbf{r}$. An object
domain element is in the extension of concept $\ocont*{C}[r]{D}$ in context $c$, if it belongs to $D$ in all
contexts $c'$ that belong to the meta concept $\mathbf{C}$ and are related to $c$ via $\mathbf{r}$.

Thus, within a context we can talk about object elements that belong to some object concept in some
other context. This is somehow similar to $\Next C$ in $\text{LTL}_{\ALC}$, which denotes the set of all
elements that are in $C$ \emph{in the next time point}.
\begin{example}\label{ex:alcalc-plus}
  Going back to Example~\ref{ex:nfl-with-contexts}, the following meta concept assertion states that someone who
  plays quarterback for the Green Bay Packers must work as coach in a junior training camp that is organized by
  Green Bay:
  \begin{align*}
    & \oax{\exists\plays.\mathsf{Quarterback} \sqsubseteq
    \ocont{\mathsf{JuniorFootballClinic}}[\mathsf{organizes}]{\mathsf{Coach}}}(\text{\textit{GreenBayPackers}}).
  \end{align*}
  Note here, that Green Bay Packers and the junior training camp are two different contexts and that
  this cannot be expressed in \LMLO.
\end{example}

The contextualized description logic \ALCALCplus without rigid names is a syntactical variant of
\klarALC~\cite{KG-JELIA10,KG16}. Consequently, the consistency problem in \ALCALCplus has
the same complexity.

\begin{theorem}\label{thm:alcalcplus-without-rigid-twoexptime}
  The consistency problem in \ALCALCplus is \TwoExpTime-complete if $\OCR = \emptyset$ and
  $\ORR = \emptyset$.
\end{theorem}

\begin{proof}[Sketch]
  We can prove the theorem by a mutual reduction of an \klarALC and an \ALCALCplus knowledge base.
  Without introducing the complete syntax of \klarALC, we show how to map an \klarALC ontology into
  \ALCALCplus.

  Table~\ref{tab:syntax-klarALC} shows in the upper part the two special constructors for object
  concepts available in \klarALC. The middle part provides the syntax and semantics of \emph{object
    formulas} which in turn constitute the \emph{object ontology axioms}, shown in lower part. The
  rightmost column defines the mapping $\tau$ which translates terms from \klarALC to
  \ALCALCplus. The following example shows how an object ontology axiom is mapped.
  \begin{align*}
    \mathbf{C}:\langle \mathbf{C} \rangle_{\mathbf{r}}(\langle \mathbf{C}\rangle_{\mathbf{r}}D
    \sqsubseteq A) 
    \qquad\leadsto\qquad
    C\sqsubseteq\exists r.\left(C\sqcap\oax{\ocont{C}[r]{D}}\right)
  \end{align*}

  \begin{table}[t]    
    \caption{Syntax and semantics of \klarALC, and the mapping $\tau$ to \ALCALCplus}
    \centering
    \begin{tabularx}{0.96\linewidth}{ll@{ iff }X@{}l}
      \toprule
      Syntax   & \multicolumn{2}{l}{Semantics} 
               & mapping $\tau(x)$ \\
      \midrule
      $\langle \mathbf{C} \rangle_{\mathbf{r}}D$ 
               & \multicolumn{2}{l}{
                 $\cdot^{\I_{c}} = \{d\in\Delta\mid\text{there is $c'$ s.t.\
                 $(c,c')\in\mathbf{r}^{\J}$, $c'\in C^{\J}$, $d\in D^{\I_{c'}}$}\}$} 
               & $\ocont{C}[r]{D}$ \\
      $[\mathbf{C}]_{\mathbf{r}}D$ 
               & \multicolumn{2}{l}{
                 $\cdot^{\I_{c}} = \{d\in\Delta\mid\text{$(c,c')\in\mathbf{r}^{\J}$ and $c' \in
                 C^{\J}$ imply $d\in D^{\I_{c'}}$}\}$}
               & $\ocont*{C}[r]{D}$ \\
      \midrule
      $B \sqsubseteq D$  & $\I_{c}\models B \sqsubseteq D$
               & $B^{\I_{c}}\subseteq D^{\I_{c}}$ 
               & $\oax{B \sqsubseteq D}$\\
      $D(a)$   & $\I_{c}\models D(a)$
               & $a^{\I_{c}}\in D^{\I_{c}}$
               & $\oax{D(a)}$\\
      $s(a,b)$ & $\I_{c}\models s(a,b)$
               & $(a^{\I_{c}}, b^{\I_{c}})\in s^{\I_{c}}$
               & $\oax{s(a,b)}$\\
      $\lnot\varphi$     & $\I_{c}\models\lnot\varphi $
               & $\I_{c}\not\models\varphi$
               & $\lnot\tau(\varphi)$\\
      $\varphi\land\psi$    & $\I_{c}\models\varphi\land\psi $
               & $\I_{c}\models\varphi$ and $\I_{c}\models\psi$
               & $\tau(\varphi) \sqcap \tau(\psi)$\\
      $\langle \mathbf{C}\rangle_{\mathbf{r}}\varphi$ 
               & $\I_{c}\models\langle\mathbf{C}\rangle_{\mathbf{r}}\varphi$
               & \text{there is $c'\in \mathbf{C}^{\J}$ s.t.\ $(c,c')\in\mathbf{r}^{\J}$, $\I_{c'}\models\varphi$}
               & $\exists r.(C\sqcap\tau(\varphi))$\\
      $[\mathbf{C}]_{\mathbf{r}}\varphi$ 
               & $\I_{c}\models [\mathbf{C}]_{\mathbf{r}}\varphi$
               & \text{every $c'\in \mathbf{C}^{\J}$, $(c,c')\in\mathbf{r}^{\J}$ implies $\I_{c'}\models\varphi$ }
               & $\forall r.(\lnot C \sqcup \tau(\varphi))$\\
      \midrule
      $\mathbf{a} : \varphi$ 
               & $\J\models \mathbf{a} : \varphi$
               & $\I_{c}\models\varphi$ with $c=\mathbf{a}^{\J}$
               & $(\tau(\varphi))(a)$\\
      $\mathbf{C} : \varphi$
               & $\J\models \mathbf{C} : \varphi$ 
               & $\I_{c}\models\varphi$ for every $c$ with $c\in\mathbf{C}^{\J}$
               & $C \sqsubseteq \tau(\varphi)$\\
      \bottomrule
    \end{tabularx}
    \label{tab:syntax-klarALC}
  \end{table}

  An \klarALC ontology $\Kmc = (\Cmc, \Omc)$ consists of a context ontology \Cmc, which is in fact a
  standard \ALC ontology, and of an object ontology. Given \Kmc, let us define the \ALCALC ontology
  $\Bmc_{\Kmc} \coloneqq \Cmc \land \tau(\Omc)$. It is easy to verify that a nested interpretation
  \J is a model of \Kmc if and only if it is a model of $\Bmc_{\Kmc}$.

  Conversely, for an \ALCALC ontology \Bmc, we take the outer abstraction \Bb as context ontology
  and for each \oalpha in \Bmc, we add $(\mathbf{A_{\oalpha}} : \alpha)$ and
  $(\lnot\mathbf{A_{\oalpha}} : \lnot\alpha)$ to the object ontology~$\Omc_{\Bmc}$. Again, it is
  easy to show that~\J models~\Bmc iff~\J models $\Kmc_{\Bmc} = (\Bb,\Omc_{\Bmc})$.
\end{proof}



The more interesting setting with rigid roles behaves much worse. One can easily show that the consistency
problem becomes undecidable.

\begin{theorem}\label{thm:elalcplus-with-rigid-undecidable}
  The consistency problem in \ELALCplus is undecidable if $\ORR \neq \emptyset$.
\end{theorem}

\begin{proof}
  Similar to the idea of \cite{LuWZ-TIME08}, we proof the claim by reduction of a well-known
  undecidable problem, namely the \emph{domino problem} \cite{Ber-66}: given a triple
  $\Dmc = (D, H, V)$ with a set of domino types $D=\{d_{1}, \dots, d_{n}\}$, a horizontal
  compatibility relation $H \subseteq D \times D$ and a vertical compatibility relation
  $V \subseteq D \times D$, decide whether there exists a solution to cover the
  $\nat\times\nat$-grid with these domino types respecting the compatibility relations, i.e.\ does
  there exist a \emph{tiling} $\mathsf{t}:\nat\times\nat\to D$ s.t.\
  $(\mathsf{t}(i,j),\mathsf{t}(i+1,j))\in H$ and $(\mathsf{t}(i,j),\mathsf{t}(i,j+1))\in V$?

  We encode this problem in \ELALCplus with a single rigid role $v\in\ORR$. Let $\Bmc_{\Dmc}$ be the
  conjunction of the following meta axioms. Each context represents a \emph{column} of the
  grid. Using a meta role $h \in \MR$ and a rigid object role $v \in \ORR$, we ensure the existence
  of a grid:
  \begin{align*}
    \top & \sqsubseteq \exists h.\top \tag{$\alpha_{1}$}\\
    \top & \sqsubseteq \oax{\top \sqsubseteq \exists v.\top} \tag{$\alpha_{2}$}
  \end{align*}
  Let $A_{0}, \dots, A_{n} \in \OC$ be object concept names representing the given domino types. To
  ensure that a single domino type is assigned to each grid position, we use
  \begin{align*}
    \top & \sqsubseteq \oax{\top \sqsubseteq (A_{1} \sqcup \dots \sqcup A_{n})
           \sqcap \bigsqcap_{1 \leq i < j \leq n} \lnot(A_{i} \sqcap A_{j})}.
           \tag{$\alpha_{3}$}
  \end{align*}
  To enforce the compatibility relations, we use
  \begin{align*}
    \top & \sqsubseteq \bigsqcap_{i=1}^{n} 
           \oax{A_{i} \sqsubseteq \forall v.(\bigsqcup_{(d_{i},d_{j})\in V} A_{j})}, \text{ and} \tag{$\alpha_{4}$}\\
    \top & \sqsubseteq \bigsqcap_{i=1}^{n} 
           \oax{A_{i} \sqsubseteq \bigsqcup_{(d_{i},d_{j})\in H} \ocont*{\top}[h]{A_{j}}}.\tag{$\alpha_{5}$}
  \end{align*}

  \begin{claim}
    $\Bmc_{\Dmc}$ is consistent iff \Dmc has a
    solution.
  \end{claim}

  \begin{claimproof}
    Assume that $\mathsf{t}$ is a solution for \Dmc. Then, based on $\mathsf{t}$ we define the
    nested interpretation
    $\J_{\mathsf{t}} = (\nat, \cdot^{\J_{\mathsf{t}}}, \nat, (\cdot^{\I_{x}})_{x\in\nat})$ with
    \begin{align*}
      h^{\J_{\mathsf{t}}} & \coloneqq \{(k,k+1) \mid k \geq 0\}\\
      v^{\I_{x}} & \coloneqq \{(l, l+1) \mid l \geq 0\} \qquad \text{for all $n \geq 0$}\\
      %
      A_{i}^{\I_{x}} & \coloneqq \{ y \in \nat \mid \mathsf{t}(x,y) = d_{i}\}
    \end{align*}
    By definition, $\J_{\mathsf{t}}$ models $\alpha_{1}$ and $\alpha_{2}$. For each object domain
    element $y \in \nat$ and each $\I_{x}$, $x \geq 0$, we have that $y \in {A_{i}}^{\I_{x}}$ and
    $y \notin {A_{j}}^{\I_{x}}$, $1 \leq j \leq n, i\neq j$, for $\mathsf{t}(x,y)=d_{i}$. Hence,
    $\J_{\mathsf{t}} \models \alpha_{3}$. By definition, $y+1$ is the only $v$-successor of $y$. If
    $y \in A_{i}^{\I_{x}}$ we know that
    $y+1 \in \smash{\big(\bigsqcup_{(d_{i},d_{j})\in V} A_{j}\big)^{\I_{x}}}$ because $\mathsf{t}$
    is a solution and $(\mathsf{t}(x,y),\mathsf{t}(x,y+1))\in V$. Analogously, $x+1$ is the only
    $h$-successor of $x$ and if $y \in A_{i}^{\I_{x}}$, we know that
    $y \in \smash{\big(\bigsqcup_{(d_{i},d_{j})\in H} A_{j}\big)^{\I_{x+1}}}$ because $\mathsf{t}$
    is a solution and $(\mathsf{t}(x,y),\mathsf{t}(x+1,y))\in H$. Hence,
    $\J_{\mathsf{t}} \models \alpha_{4} \land \alpha_{5}$. Thus, $\Bmc_{\Dmc}$ is consistent.

    Assume that \JJ is a model of $\Bmc_{\Dmc}$. Let $\Psf_{\Msf}$ and $\Psf_{\Osf}$ be infinite
    paths $\Psf_{\Msf} = c_{0}c_{1}\dots$ and $\Psf_{\Osf} = o_{0}o_{1}\dots$ with $c_{i} \in \Cbb$,
    $(c_{i}, c_{i+1}) \in h^{\J}$, $o_{i} \in \Delta^{\J}$ and $(o_{i}, o_{i+1}) \in v^{\I_{c}}$ for
    some $c \in \Cbb$. Such paths exists, because $\J\models\alpha_{1}$, $\J\models\alpha_{2}$ and
    $v$ is a rigid role. We define the nested interpretation
    $\J_{\Psf} \coloneqq (\{c_{i} \mid i \geq 0\}, \cdot^{\J_{\Psf}}, \{o_{i} \mid i \geq 0\},
    (\cdot^{\I_{\Psf,c_{i}}})_{i \geq 0})$, where $\cdot^{\J_{\Psf}}$ is the restriction of
    $\cdot^{\J}$ to the domain ${c_{i} \mid i \geq 0}$, and $\cdot^{\I_{\Psf,c_{i}}}$ is the restriction of
    $\cdot^{\I_{c_{i}}}$ to the domain ${o_{i} \mid i \geq 0}$.

    By construction, $\J_{\Psf}$ is a model of $\alpha_{1}$ and $\alpha_{2}$. Since $\alpha_{3}$ to
    $\alpha_{5}$ do not contain any existential or at-least restrictions, the restriction of the
    meta and the object domain preserves the entailment relation, and
    $\J_{\Psf}\models\Bmc_{\Dmc}$. We define the tiling $\mathsf{t}$ as follows:
    \begin{align*}
      t(x,y) = d_{i} \quad\text{ if $o_{y} \in A_{i}^{\I_{\Psf,c_{x}}}$}.
    \end{align*}
    The tiling $\mathsf{t}$ is a total function and well-defined, due to $\alpha_{3}$. Let
    $o_{j} \in {A_{k_{1}}}^{\!\!\!\I_{\Psf,c_{i}}}$,
    $o_{j+1} \in {A_{k_{2}}}^{\!\!\!\I_{\Psf,c_{i}}}$ and
    $o_{j} \in {A_{k_{3}}}^{\I_{\Psf,c_{i+1}}}$. Thus, we have $\mathsf{t}(i,j) = d_{k_{1}}$,
    $\mathsf{t}(i,j+1) = d_{k_{2}}$ and $\mathsf{t}(i+1,j) = d_{k_{3}}$. By $\alpha_{4}$ we know
    that
    $o_{j} \in \smash{\forall v.\big(\bigsqcup_{(d_{k_{1}},d_{j})\in V}
      A_{j}\big)^{\I_{\Psf,c_{i}}}}$ and, hence, $(d_{k_{1}}, d_{k_{2}})\in V$. Analogously by
    $\alpha_{5}$, we get $(d_{k_{1}}, d_{k_{3}})\in H$. Thus, \Dmc has a solution.
  \end{claimproof}

  Thus, deciding whether $\Bmc_{\Dmc}$ is consistent is undecidable.
\end{proof}

In Section~\ref{sec:complexity-consis-problem}, we discussed three different settings depending on
whether rigid concept and role names are admitted. We already obtained the complexity results for
\LMLOplus for Setting~(i), i.e.~no rigid names are allowed, and for Setting~(iii), i.e.~rigid roles
are allowed.  In LMLOplus, however, a Setting~(ii) that allows rigid concept names but no rigid
roles, coincides with Setting~(i), since rigid concepts can be easily simulated.

To simulate rigid concepts in \LMLOplus, let \BB be an \LMLOplus ontology. We first have to
simulate a universal role $u$ on the meta level. We can assume w.l.o.g.\
that $u\in\MR$ does not occur in \Bmf. We obtain $\Bmf_{u}$ from \Bmf by adding the following axioms
to \Bmf:
\begin{itemize}
\item $u^{-} \sqsubseteq u$,
%\item $\trans{u}$,
\item $r \sqsubseteq u$, for all $r\in\MR$ occurring in \Bmf,
\item $u(a,b)$ for all $a,b\in\MI$ occurring in \Bmf. 
\end{itemize}
Note that $\Bmf_{u}$ is of size polynomial in the size of \Bmf and that we need at least \ALCHI as
\LM to simulate the universal role.
%
Assume there exists an unnamed context in a model of \Bmf, i.e.~$c\in\Cbb$ such that there is
no~$a\in\MI$ occurring in~\Bmf with~$a^{\J}=c$, and $c$ is not connected to any named context by some
path. Then, we can always remove that unnamed context from the interpretation and still have a model.
Hence, we can assume that every model of~$\Bmf_{u}$ is connected and that every context can be
reached by any other context via a path of $u$-edges.
%
Here, we restrict~\Bmf to~\LMLOplus-ontologies, since in an \LMLOplus-BKB, a negated meta GCI can
enforce the existence of an unnamed context that is not necessarily connected to the rest of the model.


With the help of $u$ and contextualized concepts, we can express that $A\in\OC$ is rigid by the
following two axioms:
\begin{align*}
  \top & \sqsubseteq \oax{A \sqsubseteq \ocont*{\top}[u]{A}} \\
  \top & \sqsubseteq \oax{\lnot A \sqsubseteq \ocont*{\top}[u]{\lnot A}}
\end{align*}
Due to these axioms, the extension of $A$ in a context $c$ are exactly these elements which are in
$A$ in every context $c'$ that is related to $c$ via $u$. Thus, the extension of $A$ is equal in
every context, i.e.\ $A$ is rigid.

Hence, the complexity of the consistency problem in \LMLOplus is independent of whether rigid
concepts are allowed or not.


%%% Local Variables:
%%% mode: latex
%%% TeX-master: "../thesis"
%%% reftex-default-bibliography: ("../references.bib")
%%% End:

%  LocalWords:  logics DL LTL expressivity temporalized ontologies


\section{Consistency of Boolean \texorpdfstring{\SHOIQ}{SHOIQ} knowledge bases}
\label{sec:consistency-shoiq-bkb}

\begin{definition}[Concept Type]
  
\end{definition}

\begin{definition}[Role Type]
  
\end{definition}

\begin{definition}[Model candidate]
  
\end{definition}

\begin{definition}[Quasimodel]
  
\end{definition}

\begin{lemma}
  Let \Bmf be a \SHOIQ-BKB. weakly respects iff quasimodel exists.
\end{lemma}

\begin{theorem}\label{thm:shoiq-bkb-consistency-nexptime}
  Let \Bmf be a \SHOIQ-BKB, let $\Umc \subseteq \NC$ and let $\Ymc\subseteq\powerset{\Umc}$. Then
  deciding whether there exists a model of \Bmf that weakly respects $(\Umc,\Ymc)$ can be
  non-deterministically done in time exponential in the size of \Bmf and linear in the size of \Ymc.
\end{theorem}

\begin{corollary}
  Let \Bmf be a \SHOIQ-BKB. Then, consistency of \Bmf can be non-deterministically decided in time
  exponential in the size of \Bmf.
\end{corollary}

%%% Local Variables:
%%% mode: latex
%%% TeX-master: "../thesis"
%%% reftex-default-bibliography: ("../references.bib")
%%% End:

%  LocalWords:  Löwenheim Skolem Logics logics iff temporalized Quasimodel quasimodel BKB
