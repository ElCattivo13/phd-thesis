
\section{Syntax and Semantics of the Contextualised Description Logic \texorpdfstring{\LMLO}{LM[LO]}}
\label{sec:syn-seman-cdl}

Our approach for contextualised description logics is similar to temporal description logics as
investigated in~\cite{LuWZ-TIME08} in which a temporal logic for the time dimension and a
description logic for the object dimension are combined in order to express information valid at a
certain time. We combine two, possibly different, description logics: one DL \LM for the
\emph{context} or \emph{meta dimension} for knowledge \emph{about} contexts and one DL \LO for the
\emph{object dimension} for knowledge \emph{within} contexts.
%
Considering the semantics, \LMLO can also be seen as a restricted variant of the context description
logic proposed by Klarman et al.~\cite{KG16}.

The contextualised DL \LMLO is a two-dimensional and two-sorted description logic. Syntactically, we
start with a description logic \LM for the meta level to describe the relational structure between
contexts. We add one meta concept constructor that allows to refer to the inner structure of
each context. More precisely, we allow an \LO-axiom to constitute a meta concept which denotes the
set of contexts in which that axiom holds.

Throughout the rest of this thesis, let $\Msig = (\MC, \MR, \MI)$ and $\Osig = (\OC, \OR, \OI)$
denote the signatures for \LM and \LO, respectively. Thus, we call \MC, \MR, \MI, \OC, \OR and \OI,
respectively, the set of \emph{meta concept}, \emph{role} and \emph{individual names} and
\emph{object concept}, \emph{role} and \emph{individual names}.

\begin{definition}[Syntax of \LMLO]\label{def:syntax-lmlo}
  A \emph{concept of the object logic~\LO (o-concept)} is an \LO-concept over~\Osig.  An
  \emph{o-axiom} is an \LO-GCI over~\Osig, an \LO-concept assertion over~\Osig, or an
  \LO-role assertion over~\Osig.

  The set of \emph{concepts of the meta logic~\LM (m-concepts)} is the smallest set such that
  \begin{itemize}
  \item for all $A\in\MC$, $A$ is a \emph{basic} meta concept,
  \item for all o-axioms $\alpha$, \oalpha is a \emph{referring} meta concept, and
  \item all complex concepts that can be built with the concept constructors allowed in \LM are meta
    concepts.
  \end{itemize}
  %
  A \emph{meta general concept inclusion (m-GCI)} is a GCI $C\sqsubseteq D$ where $C$ and $D$
  are m-concepts, a \emph{meta concept assertion} is a concept assertion $C(a)$ where $C$ is a
  m-concept and a \emph{meta role assertion} is simply an \LM-role assertion over \Msig.
  %
  An \emph{m-axiom} is an m-GCI, a meta concept assertion or a meta role assertion.

  As for DLs, a \emph{Boolean m-axiom formula} is inductively defined as follows:
  \begin{itemize}
  \item every m-axiom is a Boolean m-axiom formula,
  \item if $\B_1,\B_2$ are Boolean m-axiom formulas, then so are $\lnot\B_1$ and $\B_1\land\B_2$,
    and
  \item nothing else is a Boolean m-axiom formula.
  \end{itemize}
    %
  Finally, a \emph{Boolean \LMLO-knowledge base (\LMLO-BKB)} is a triple $\Bmf=(\B,\RO,\RM)$, where~\RO is an~\LO-RBox over~\Osig, \RM an \LM-RBox over~\Msig, and~\B is a Boolean m-axiom formula. An
  \emph{\LMLO-ontology} is an \LMLO-BKB, where only axiom conjunction and no axiom negation is
  allowed in the Boolean m-axiom formula.
\end{definition}

\noindent
Essentially, m-GCIs and meta concept assertions are \LM-GCIs over \Msig and \LM-concept assertions
over \Msig in which additionally referring m-concepts are admitted.
%
For the same reasons as mentioned in Section~\ref{sec:preliminaries}, role inclusions
over~\Osig and transitivity axioms over~\Osig are not allowed to constitute m-concepts.  However, we
fix an RBox~\RO over~\Osig that contains such o-axioms and holds in \emph{all} contexts.  The same
applies to role inclusions over~\Msig and transitivity axioms over~\Msig, which are only allowed to
occur in a RBox~\RM over~\Msig.
%
Again, we use the usual abbreviations (for disjunctions etc.) for m-concepts and
Boolean m-axiom formulas.

The semantics of \LMLO is defined by the notion of \emph{nested interpretations}.  These consist of
\Osig-interpretations for the specific contexts and an \Msig-interpretation for the relational
structure between them.  We assume that all contexts speak about the same non-empty domain
(\emph{constant domain assumption}). We show later that this is no real restriction.

As argued earlier, sometimes it is desired that concepts or roles in the object logic are
interpreted the same in all contexts. Therefore we introduce \emph{rigid names}. Let
$\OCR\subseteq\OC$ be the set of \emph{rigid object concept names} and $\ORR\subseteq\OR$ be the set
of \emph{rigid object role names}.  Often, we refer to \OCR and \ORR simply as \emph{rigid concepts}
and \emph{rigid roles}, as there is no such notion on the meta level.  We set
$\OCF\coloneqq\OC\setminus\OCR$ and $\ORF\coloneqq\OR\setminus\ORR$ and call these concept names and
role names \emph{flexible}.  Moreover, following the argument for the UNA, we assume that the
identity of an individual is context-independent and, thus, individual names of the object logic are
always interpreted the same in all contexts (\emph{rigid individual assumption}).

\begin{definition}[Nested interpretation]\label{def:nested-interpretation}
  A \emph{nested interpretation} is a tuple
  $\J=(\Cbb,\cdot^{\J},$ $\Delta^{\J}{\kern-0.2em},(\cdot^{\I_c})_{c\in\Cbb})$, where \Cbb is a
  non-empty set (called set of \emph{contexts}) and $(\Cbb,\cdot^\J)$ is an \Msig-inter\-pre\-ta\-tion.
  %
  Moreover, for every $c\in\Cbb$, $\I_c\coloneqq(\Delta^{\J},\cdot^{\I_c})$ is an
  \Osig-interpretation such that for all $c,c'\in\Cbb$, we have that $x^{\I_{c}}=x^{\I_{c'}}$ for
  every $x\in\OI\cup\OCR\cup\ORR$.
\end{definition}

\noindent
We are now ready to define the semantics of \LMLO.

\begin{definition}[Semantics of \LMLO]
  \label{def:semantics-lmlo}
  Let \JJ be a nested interpretation.  The mapping $\cdot^\J$ is extended to referring meta concepts
  as follows:
  \begin{align*}
    \oalpha^\J:=\{c\in\Cbb\mid\I_c\models\alpha\}.
  \end{align*}
  Moreover, \J is a model of the m-axiom $\beta$ if $(\Cbb,\cdot^\J)$ is a model
    of $\beta$.  This is extended to Boolean m-axiom formulas inductively as
    follows:
    \begin{itemize}
        \item \J is a model of $\lnot\B_1$ if it is not a model of $\B_1$, and
        \item \J is a model of $\B_1\land\B_2$ if it is a model of both $\B_1$
            and $\B_2$.
    \end{itemize}
    We write $\J\models\B$ if \J is a model of the Boolean m-axiom formula~\B.
    Furthermore, \J is a model of~\RM (written $\J\models\RM$) if
    $(\Cbb,\cdot^\J)$ is a model of~\RM, and \J is a model of~\RO (written
    $\J\models\RO$) if $\I_{c}$ is a model of~\RO for all $c\in\Cbb$.
    
    Finally, \J is a model of the \LMLO-BKB \BB (written
    $\J\models\Bmf$) if \J is a model of~\B, \RO, and~\RM.  We call~\Bmf
    \emph{consistent} if it has a model.

    The \emph{consistency problem in \LMLO} is the problem of deciding whether a given
    \LMLO-BKB is consistent.
\end{definition}

\noindent
With these notions, we can refine the examples of
Section~\ref{sec:preliminaries}. 

\begin{example}\label{ex:nfl-with-contexts}
  Considering again Example~\ref{ex:concept-nfl} and Example~\ref{ex:bkb-nfl}, we recognize that
  $\mathsf{NFL\_Team}$ actually is a context. The meta concept $C'$ analogous to $C$ could be:
  \begin{align*}
    & \mathsf{NFL\_{}Team} \sqcap \lnot \mathsf{AFC} \sqcap
    \lnot\oax{\exists\plays.\mathsf{Quarterback}\sqsubseteq\bot} 
    \sqcap \oax{\mathsf{Coach}(\text{\textit{MikeMcCarthy}})},
  \end{align*}
  describing the context of an NFL team that is not in the AFC, has someone playing quarterback and
  has Mike McCarthy as coach.  The contextualised ontology $\Omc'$ could look like the following:
  \begin{gather*}
    (\mathsf{NFC} \sqcap
    \oax{(\exists\plays.\top)(\text{\textit{AaronRodgers}})})(\text{\textit{GreenBayPackers}})
    \quad\land\\ 
    \begin{aligned}
      \mathsf{NFL\_Team} & \sqsubseteq \oax{\exists\plays.\top \sqsubseteq
        \mathsf{NFL\_Player}}\quad\land\\ 
      \mathsf{NFL\_Team} & \equiv \mathsf{NFC} \sqcup \mathsf{AFC}\quad\land\\
      \mathsf{NFC} \sqcap \mathsf{AFC} & \sqsubseteq \bot.
    \end{aligned}
  \end{gather*}
  The first meta concept assertion states that the context of the Green Bay Packers belongs to the
  meta concept NFC and is a context in which Aaron Rodgers plays something. Furthermore, people who
  play something within the context of an NFL team are NFL players, and again, the NFL is a disjoint
  union of the NFC and the AFC. As before, we can entail that Aaron Rodgers is an NFL player, at
  least in the context of the Packers, i.e.\ $\Omc'$ entails
  \begin{gather*}
    \oax{\mathsf{NFL\_Player}(\text{\textit{AaronRodgers}})}(\text{\textit{GreenBayPackers}}).
  \end{gather*}
  A nested interpretation \J, such that Green Bay Packers are in the extension of $C'$, and \J is a
  model of $\Omc'$, is depicted in Figure~\ref{fig:example-nfl-contexts}. Note here that after
  introducing ``Junior Football Clinic'' as another context it would not be clear anymore in which
  context Aaron Rodgers actually plays quarterback if modelled with standard DLs.
\end{example}

\begin{figure}[t]
  \centering
  \begin{tikzpicture}
    \node[rectangle,
          draw,
          rounded corners= 5mm, 
          minimum height = 3.5cm, 
          %minimum width = 6cm,
          label={[align=left, anchor=south west]135:$\mathsf{NFL\_Team}$, $\mathsf{NFC}$, \\\textit{GreenBayPackers}}] (a) at (2.0,0){
            \begin{tikzpicture}
              \node[node,label={[align=left]270:\textit{MikeMcCarthy},\\ $\mathsf{Coach}$}] (mmc) at (0,1.8){};
              \node[node] (kid) at (2.5, 1.5){};
              \node[node,label={[align=left]270:\textit{AaronRodgers}, \\ $\mathsf{NFL\_Player}$}] (ar) at (0,0){};
              \node[node,label={[align=right]south:$\mathsf{Quarterback}$}] (qb) at (3,0){};
              \draw[edge] (ar) to[bend left=10] node{\plays} (qb);
            \end{tikzpicture}
          };
    \node[rectangle,
          draw,
          rounded corners= 5mm,
          minimum height = 3.5cm,
          %minimum width = 6cm,
          label={[anchor=south]90:$\mathsf{Junior Power Pack Football Clinic}$}] (b) at (9.45,0){
            \begin{tikzpicture}
              \node[node,label={270:\textit{MikeMcCarthy}}] (mmc) at (0,1.8){};
              \node[node,label={0:$\mathsf{Kid}$}] (kid) at (2.5,1.5){};
              \node[node,label={[align=left]270:\textit{AaronRodgers},\\ $\mathsf{Coach}$}] (ar) at (0,0){};
              \node[node] (qb) at (3,0){};
              \draw[edge] (ar) to[bend left=20, swap] node[inner sep=0pt]{$\textsf{teaches\_HailMarys}\footnotemark$} (kid);
            \end{tikzpicture}
          };
    \draw[edge] (a.east) to[bend left] node{$\mathsf{organizes}$} (b.west);
  \end{tikzpicture}
  \caption{A Nested interpretation \J such that $\text{\textit{GreenBayPackers}}^{\J} \in C'^{\J}$
    and $\J\models\Omc'$.}
  \label{fig:example-nfl-contexts}
\end{figure}

To argue that the constant domain assumption is no serious restriction, we show that consistency
with varying domains can be polynomially reduced to consistency with constant domains. Here, we can
adapt the ideas of \cite{GKW+-03} and \cite{LuWZ-TIME08}. Let~\Bmf be an \LMLO-BKB and let~$\J_{v}$
denote a nested interpretation with varying domains, i.e.\
$\J_{v} = (\Cbb,\cdot^{\J},(\Delta^{\I_{c}})_{c\in\Cbb},(\cdot^{\I_c})_{c\in\Cbb})$,
where~$(\Delta^{\I_{c}})_{c\in\Cbb}$ is a set of, possibly overlapping, object domains. We introduce
a fresh object concept name~$E$ that expresses the existence of an element in an object
domain. W.l.o.g.\ we assume that all o-GCIs are of the form~$\top \sqsubseteq C$. We obtain~$C_{E}$
from the o-concept~$C$ by replacing every o-subconcept~$\exists r.D$ with~$\exists r.(D \sqcap E)$
and every o-subconcept~$\atleast{n}{r}{D}$ with~$\atleast{n}{r}{(D \sqcap E)}$. We obtain~$\Bmf_{E}$
from~\Bmf by replacing every occurrence of an o-axiom~$\alpha$ in~\Bmf with~$\alpha_{E}$,
where~$\alpha_{E}$ is defined as follows:
\begin{align*}
  \alpha_{E} \coloneqq
  \begin{cases}
    E \sqsubseteq C_{E} & \text{ if $\alpha = \top \sqsubseteq C$,} \\
    (C_{E}\sqcap E)(a) & \text{ if $\alpha = C(a)$, and} \\
    \alpha & \text{otherwise.}
  \end{cases}
\end{align*}
\footnotetext{A Hail Mary pass is a very long forward pass in American football, made in desperation with only a small chance of success.}
%
\begin{proposition}\label{prop:constant-domain-assumption}
  \Bmf is consistent w.r.t. varying domains if and only if $\Bmf_{E}$ is consistent w.r.t. constant
  domains and $\Bmf_{E}$ is of size polynomial in the size of \Bmf.
\end{proposition}

\begin{proof}
  For the `only if' direction, let
  $\J_{v} = (\Cbb,\cdot^{\J},(\Delta^{\I_{c}})_{c\in\Cbb},(\cdot^{\I_c})_{c\in\Cbb})$ be a model of
  \Bmf with varying domains. We construct the nested interpretation \JJ with a constant domain from
  $\J_{v}$ such that $\Delta^{\J}\coloneqq\bigcup_{c\in\Cbb}\Delta^{\I_{c}}$ and
  $E^{\I_{c}} \coloneqq \Delta^{\I_{c}}$. Let $\I'_{c} = (\Delta^{\I_{c}}, \cdot^{\I_{c}})$ and
  $\I_{c} = (\Delta^{\J}, \cdot^{\I_{c}})$ .
  \begin{claim}
    $\{c\in\Cbb \mid \I'_{c}\models \alpha\} = \{c\in\Cbb \mid \I_{c}\models \alpha_{E}\}$
  \end{claim}
  \begin{claimproof}
    We prove the claim by showing $\I'_{c}\models\alpha$ iff $\I_{c}\models\alpha_{E}$ for the
    different cases of $\alpha$:

    \noindent
    \begin{tabularx}{\linewidth}{lX}
      $\alpha = \top \sqsubseteq C$: & $\I'_{c}\models \top \sqsubseteq C$ 
                                       iff $\Delta^{\I_{c}} = C^{\I_{c}}$
                                       iff\,\footnote{Exchanging $C$ with $C_{E}$ can be done here, as the concept
    $E$ is equivalent to $\top$ for $\I'_{c}$.} $E^{\I_{c}} = C_{E}^{\I_{c}}$
                                       iff $\I_{c} \models E \sqsubseteq (C_{E} \sqcap E) = \alpha_{E}$.\\
      $\alpha = C(a)$: & $\I'_{c}\models C(a)$ 
                        iff $a^{\I_{c}}\in C^{\I_{c}}$. Since $E^{\I_{c}}=\Delta^{\I_{c}}$, we also have
                         $a^{\I_{c}}\in E^{\I_{c}}$ and, hence, $\I_{c} \models (C \sqcap E)(a)$.\\
      $\alpha = r(a,b)$: & We have $\alpha = \alpha_{E}$ and, since $a^{\I_{c}}, b^{\I_{c}}\in\Delta^{\I_{c}}$,
    $\I'_{c} \models \alpha$ iff $\I_{c} \models \alpha_{E}$.
    \end{tabularx}
    
    \vspace*{-1.0\baselineskip}
  \end{claimproof}
  %
  \noindent Since we do not change the meta interpretation and $\oax{\alpha}^{\J_{v}} =
  \oax{\alpha_{E}}^{\J}$, we get that $\J\models\Bmf_{E}$.

  For the `if' direction, let \JJ be a model of $\Bmf_{E}$. Then, we define
  $\J_{v} = (\Cbb,\cdot^{\J},(\Delta^{\I'_{c}})_{c\in\Cbb},(\cdot^{\I'_c})_{c\in\Cbb})$ with
  $\Delta^{\I'_{c}}\coloneqq E^{\I_{c}}$ and $A^{\I'_{c}} \coloneqq A^{\I_{c}} \cap E^{\I_{c}}$.  By
  similar arguments as above and due to the restriction in existential and at-least restrictions for
  $r$-successors being in $E$, we have $\oax{\alpha}^{\J_{v}} = \oax{\alpha_{E}}^{\J}$ and therefore
  $\J_{v}\models\Bmf$.
\end{proof}

\noindent
The above proposition, however, does not imply that \Bmf is consistent w.r.t.\ constant domains if
and only if it is consistent w.r.t.\ varying domains. For example, with a rigid concept name
$A\in\OCR$ the BKB
$\Bmf = (\oax{\top \sqsubseteq A} \sqcap\ \exists t.\oax{A \sqsubseteq \exists r.\lnot A})(c)$ is consistent w.r.t.\ varying domains, but not w.r.t.\ constant domains. Nonetheless,
$\Bmf_{E}$ is consistent w.r.t.\ constant domains and checking consistency of \Bmf w.r.t.\ varying
domains can be done by considering $\Bmf_{E}$.  Thus, for the rest of the thesis, we concentrate on
constant domains.

\section{Complexity of the Consistency Problem in \texorpdfstring{\LMLO}{LM[LO]}}
\label{sec:complexity-consis-problem}

After we introduced the syntax and semantics of our contextualised description logics, in this
section, we investigate the computational complexity of the consistency problem.  Therefore, we
consider three different settings dependent on whether rigid concepts and rigid roles are admitted.
In Setting~(i) no rigid names are allowed at all.  In Setting~(ii) rigid concepts are
allowed, but no rigid roles.  Setting~(iii) then allows rigid roles. There, it is not necessary
to distinguish whether rigid concepts are admitted or not, since rigid concepts can be emulated via
rigid roles. For this, one simply replaces the rigid concept~$A$ by~$\exists r_{A}.\top$ where~$r_{A}$ is
a rigid role which does not occur in the original knowledge base.

Our results for the complexity of the consistency problem are listed in
Table~\ref{tab:compl-results-no-rigid-names}. Here, it is worth noting that if no rigid names are
admitted, the complexity class does not increase compared to the consistency problem for the
classical DL, i.e.\ for any \LMLO up to \SHOQSHOQ the consistency problem is \ExpTime-complete, if
\SHOIQ is involved, it is \NExpTime-complete. Furthermore, in the presence of rigid roles we can
retain decidability.  This distinguishes our approach from the semantically similar context DL
introduced by Klarman et al.~\cite{KG-JELIA10,KG16}, where the consistency problem becomes
undecidable when rigid roles occur in the ontology.

Since the lower bounds already hold for the case of \EL, we handle them in Section~\ref{sec:case-el}.
%
For the upper bounds, let in the following \BB be an \LMLO-BKB.  We proceed
similar to what was done for \ALC-LTL in~\cite{BaGL-KR08,BaGL-ToCL12} (and
\SHOQ-LTL in~\cite{Lip-PhD14}) and reduce the consistency problem to two separate
decision problems.

For the first decision problem, we consider the so-called \emph{outer abstraction}, which is the \LM-BKB
over~\Msig obtained by replacing each referring m-concept of the form \oalpha
occurring in~\B by a fresh concept name such that there is a 1--1 relationship between them.

\begin{definition}[Outer abstraction]
  \label{def:outer-abstraction}
  Let \BB be an \LMLO-BKB.  Let \bsf be the bijection mapping every referring m-concept of the form \oalpha
  occurring in~\B to the so-called \emph{abstracted concept name $A_{\oalpha}\in\MC$}, where we assume w.l.o.g.\ that
  $A_{\oalpha}$ does not occur in~\B.
  \begin{enumerate}
  \item The \LM-concept $C^{\bsf}$ over \Msig is obtained from the m-concept~$C$ by replacing every occurrence of
    \oalpha by $\bsf(\oalpha)$.
  \item The Boolean \LM-axiom formula~\Bb over \Msig is obtained from~\B by replacing every
    m-concept~$C$ occurring in~\B with~$C^{\bsf}$.  We call the \LM-BKB $\Bmfb=(\Bb,\RM)$ the
    \emph{outer abstraction of~\Bmf}.
        \item Given \JJ, its \emph{outer abstraction} is the
            \Msig-interpretation $\Jb=(\Cbb,\cdot^{\Jb})$ where
            \begin{itemize}
                \item for every $x\in\MR\cup\MI\cup(\MC\setminus\ran(\bsf))$, we
                    have $x^{\Jb}=x^\J$, and
                \item for every $A\in\ran(\bsf)$, we have
                    $A^{\Jb}=(\bsf^{-1}(A))^\J$,
            \end{itemize}
            where $\ran(\bsf)$ denotes the image of~\bsf. \qedhere
    \end{enumerate}
\end{definition}

For simplicity, for $\Bmf'=(\B',\RO,\RM)$, where $\B'$ is a subformula of~\B, we
denote by $(\Bmf')^\bsf$ the outer abstraction of~$\Bmf'$ that is obtained by
restricting \bsf to the m-concepts occurring in~$\B'$.
%
Now let us consider the following small example.


\begin{example}\label{ex:outer-abstraction}
  Let $\Bmf_{\text{ex}} = (\B_{\text{ex}},\emptyset,\emptyset)$ with $\B_{\text{ex}}\coloneqq
  C\sqsubseteq(\oax{A\sqsubseteq\bot})\ \land\ (C\sqcap\oax{A(a)})(c)$ be an \ALCALC-BKB.  Then,
  \bsf maps $\oax{A\sqsubseteq\bot}$ to $A_{\oax{A\sqsubseteq\bot}}$ and $\oax{A(a)}$ to
  $A_{\oax{A(a)}}$.  Thus, we have that
  \begin{align*}
    \Bmf_{\text{ex}}^\bsf\coloneqq \Big(C\sqsubseteq(A_{\oax{A\sqsubseteq\bot}})\ \land\ (C\sqcap
    A_{\oax{A(a)}})(c),\ \emptyset\Big)
  \end{align*}
  is the outer abstraction of $\Bmf_\text{ex}$.
\end{example}
\TableComplexityResults

\noindent
The following lemma makes the relationship between \Bmf and its outer abstraction
\Bmfb explicit.  It is proved by induction on the structure of~\B.

\begin{lemma}\label{lem:interpretation-outer-abstraction}
  Let \J be a nested interpretation such that \J is a model of \RO.  Then, \J is a model of \Bmf iff
  $\Jb$ is a model of~\Bmfb.
\end{lemma}

\begin{proof}
  Since $r^{\J}=r^{\Jb}$ for all \LM-role $r$ over \Msig, we have that \J is a model of \RM iff \Jb is a model of
  \RM. Thus, it is only left to show that for any m-axiom $\gamma$ occurring in \B, it holds that
  $\J \models \gamma$ iff $\Jb \models \gamma^{\bsf}$.

  \begin{claim}
    For any $x \in \Cbb$ it holds that $x \in C^{\J}$ iff
    $x \in (C^{\bsf})^{\Jb}$.
  \end{claim}

  \begin{claimproof}
    We prove the claim by induction on the structure of $C$: 
    
    \noindent
    \begin{tabularx}{\linewidth}{@{}l@{ }X@{}}
      $C = A \in \MC\!\setminus\!\ran(\bsf)$: 
      & $x \in A^{\J}$ 
        iff $x \in (A^{\bsf})^{\Jb}$ by definition of $\Jb$ and since $A = A^{\bsf}$ 
      \\[1ex]
      $C = \oalpha$:
      & $x \in \oalpha^{\J}$
        iff $x \in (A_{\oalpha})^{\Jb}$
        iff $x \in (\oalpha^{\bsf})^{\Jb}$
      \\[1ex] 
      $C = \lnot D$:
      & $x \in (\lnot D)^{\J}$ 
        iff $x \notin D^{\J}$ 
        iff, by induction hypothesis, $x \notin (D^{\bsf})^{\Jb}$ 
        iff $x \in (\lnot D^{\bsf})^{\Jb}$ 
        iff $x \in ((\lnot D)^{\bsf})^{\Jb}$ 
      \\[1ex]
      $C = D \sqcap E$: 
      & $x \in (D \sqcap E)^{\J}$
        iff $x \in D^{\J}$ and $x \in E^{\J}$ 
        iff, by induction hypothesis, $x \in (D^{\bsf})^{\Jb}$ and $x \in
        (E^{\bsf})^{\Jb}$
        iff $x \in (D^{\bsf} \sqcap E^{\bsf})^{\Jb}$
        iff $x \in ((D \sqcap E)^{\bsf})^{\Jb}$ 
      \\[1ex]
      $C = \exists r.D$: 
      & $x \in (\exists r.D)^{\J}$
        iff there exists $y \in \Cbb$ \suth $(x,y) \in r^{\J}$ and $y \in D^{\J}$
        iff there exists $y \in \Cbb$ \suth $(x,y) \in r^{\Jb}$ and $y \in (D^{\bsf})^{\Jb}$
        iff $x \in (\exists r.D^{\bsf})^{\Jb}$ 
        iff $x \in ((\exists r.D)^{\bsf})^{\Jb}$ 
      \\[1ex]
      $C = \{a\}$:
      & $x\in\{a\}^{\J}$ 
        iff $x\in(\{a\}^{\bsf})^{\Jb}$ by definition of $\Jb$ and since $\{a\} = \{a\}^{\bsf}$ 
      \\[1ex]
      $C =\ \atleast{n}{r}{D}$:
      & $x \in (\atleast{n}{r}{D})^{\J}$
        iff there are at least $n$ elements $y \in \Cbb$ s.t.\ $(x,y) \in r^{\J}$ and $y \in D^{\J}$
        iff there are at least $n$ elements $y \in \Cbb$ s.t.\ $(x,y) \in r^{\Jb}$ and $y \in (D^{\bsf})^{\Jb}$
        iff $x \in (\atleast{n}{r}{D^{\bsf}})^{\Jb}$
        iff $x \in ((\atleast{n}{r}{D})^{\bsf})^{\Jb}$ 
    \end{tabularx}

    \vspace{-1.0\baselineskip}
  \end{claimproof}

  \begin{itemize}
  \item If $\gamma$ is of the form $C \sqsubseteq D$, we have that $\J \models C \sqsubseteq D$ iff
    $x \in C^{\J}$ implies $x \in D^{\J}$ iff (by claim) $x \in (C^{\bsf})^{\Jb}$ implies
    $x \in (D^{\bsf})^{\Jb}$ iff $\Jb \models C^{\bsf} \sqsubseteq D^{\bsf}$.
  \item If $\gamma$ is of the form $C(a)$, we have that $\J \models C(a)$ iff $a^{\J} \in C^{\J}$
    iff (by claim) $a^{\Jb} \in (C^{\bsf})^{\Jb}$ iff $\Jb \models C^{\bsf}(a)$.
  \item If $\gamma$ is of the form $r(a,b)$, we have that $\J \models r(a,b)$ iff
    $(a^{\J}, b^{\J}) \in r^{\J}$ iff $(a^{\Jb}, b^{\Jb}) \in r^{\Jb}$ iff $\Jb \models r(a,b)$.
  \item If \B is of the form $\lnot\B_{1}$, we have that $\J\models\B$ iff not $\J\models\B_{1}$ iff
    not $\Jb\models\Bb_1$ iff $\Jb\models\Bb$.
  \item If \B is of the form $\B_{1}\land\B_{2}$, we have that $\J\models\B$ iff $\J\models\B_{1}$
    and $\B_{2}$ iff $\Jb\models\Bb_{1}$ and $\Jb\models\Bb_{2}$ iff $\Jb\models\Bb$.
  \end{itemize}

  \noindent
  Since $\J\models\RO$, $\J\models\RM$ iff $\Jb\models\RM$ and $\J\models\B$ iff $\Jb\models\Bb$, we
  have $\J\models\Bmf$ iff $\Jb\models\Bmfb$.
\end{proof}

Note that this lemma yields that consistency of~\Bmf implies consistency of~\Bmfb.  Thus, the
consistency of~\Bmfb is a necessary condition for the consistency of~\Bmf.  However, it is not
sufficient since the converse does not hold as the following example shows.

\begin{example}\label{ex:outer-abstraction-continued}
  Consider again $\Bmf_\text{ex}$ of Example~\ref{ex:outer-abstraction}.
  %
  Take any \Msig-interpretation $\Hmc=(\Delta^{\Hmc},\cdot^\Hmc)$ with $\Delta^{\Hmc}=\{e\}$,
  $d^\Hmc=e$, and $C^\Hmc = A_{\oax{A\sqsubseteq\bot}}^\Hmc = A_{\oax{A(a)}}^\Hmc = \{e\}$.

  Clearly, \Hmc is a model of~$\Bmf_\text{ex}^{\bsf}$.  But there is no nested interpretation~\JJ
  with $\J\models\Bmf_\text{ex}$ since this would imply $\Cbb=\Delta^{\Hmc}$, and that $\I_e$ is a model of
  both $A\sqsubseteq\bot$ and $A(a)$, which is not possible.
\end{example}

The above example illustrates that there exist implicit restrictions on the interpretation of the
meta level as certain combinations of concept names in $\ran(\bsf)$ are not allowed.  Therefore, we
need to ensure that these are not treated independently.  For expressing such a restriction on the
model~\Hmc of~\Bmfb, we adapt a notion of~\cite{BaGL-KR08,BaGL-ToCL12}. It is also worth noting that
this problem occurs also in much less expressive DLs such as \ELbot (i.e.~\EL extended with the
bottom concept).

\begin{definition}[\Umc-type, \mbox{\Nsig-interpretation (weakly) respects $(\Umc,\Ymc)$}]
  \label{def:int-respects-D}
  Let \II be an \Nsig-interpretation, let $\Umc\subseteq\NC$ and let $\Ymc\subseteq\powerset{\Umc}$,
  where $\powerset{S}$ denotes the power set of $S$.
  %
  The \emph{\Umc-type of $d\in\Delta^{\I}$ in \I} is defined as
  $\mathsf{type}_{\Umc}^{\I}(d) \coloneqq \{A \in \Umc \mid d \in A^{\I}\}$.  The interpretation \I
  \emph{respects} $(\Umc,\Ymc)$ if $\Zmc = \Ymc$, where
  \begin{align*}
    \Zmc & \coloneqq\{Y\subseteq\Umc\mid\text{there is some $d\in\Delta^\I$ with
           $\mathsf{type}_{\Umc}^{\I}(d) = Y$}\}
  \end{align*}

    It \emph{weakly respects} $(\Umc,\Ymc)$ if $\Zmc \subseteq \Ymc$.
\end{definition}

For $\Umc = \ran(\bsf)$, the \Umc-type of a context~$c$, i.e.\ an element $c\in\Cbb$ in a nested
interpretation, is the set of all abstracted concept names of which $c$ is an instance. In other
words, it describes all o-axioms which hold in that context. The $\ran(\bsf)$-type of $c$ is also
called its \emph{restricted type}.
%
The second decision problem that we use for deciding consistency is needed to make sure that such a
set of abstracted concept names is admissible in the following sense.

\begin{definition}[Admissibility]\label{def:admissibility}
  Let $\Xmc=\{X_1,~\dots,\ X_k\}\subseteq\powerset{\ran(\bsf)}$.  We call \Xmc \emph{admissible} if
  there exist \Osig-interpretations $\I_1=(\Delta,\cdot^{\I_1})$,~\dots,
  $\I_k=(\Delta,\cdot^{\I_k})$ such that
  \begin{itemize}
  \item $x^{\I_i}=x^{\I_j}$ for all $x\in\OI\cup\OCR\cup\ORR$ and all $i,j\in\{1,\dots,k\}$, and
  \item every $\I_i$, $1\le i\le k$, is a model of the \LO-BKB $\Bmf_{X_{i}}= (\B_{X_i},\RO)$
    over~\Osig where
    \begin{align*}
      \B_{X_i}&\coloneqq\bigwedge_{\bsf(\oalpha)\in X_i}\alpha\ \land
      \bigwedge_{\bsf(\oalpha)\in\ran(\bsf)\setminus X_i}\lnot\alpha.
    \end{align*}
  \end{itemize}
  \vspace{-1.7\baselineskip}
\end{definition}

Note that any subset $\Xmc'\subseteq\Xmc$ is admissible if \Xmc is admissible.
%
Intuitively, the sets $X_i$ in an admissible set \Xmc consist of referring meta concepts such that
the corresponding o-axioms \enquote{fit together}.  Consider again
Example~\ref{ex:outer-abstraction-continued}.  Clearly, the set
$\{A_{\oax{A\sqsubseteq\bot}},A_{\oax{A(a)}}\}\in\powerset{\ran(\bsf)}$ \emph{cannot} be contained
in any admissible set~\Xmc.

The next definition captures the above mentioned restriction on the model~\Hmc
of~\Bmfb. The set of restricted types that occur in the meta interpretation must be a subset of some
admissible set.

\begin{definition}[Outer consistency]\label{def:outer-consistency}
  Let $\Xmc \subseteq \powerset{\ran(\bsf)}$.  We call the \LM-BKB~\Bmfb over \Msig \emph{outer
    consistent w.r.t.~\Xmc} if there exists a model of~\Bmfb that weakly respects~$(\ran(\bsf),\Xmc)$.
\end{definition}

The next two lemmas show that the consistency problem in \LMLO can be decided by checking whether
there is an admissible set~\Xmc such that outer abstraction of the given \LMLO-BKB is outer consistent
w.r.t.~\Xmc.

\begin{lemma}\label{lem:model-equivalent-to-admissible}
  For every \Msig-interpretation \HH, the following two statements are equivalent:
  \begin{enumerate}
  \item There exists a model~\J of~\Bmf with $\Jb=\Hmc$.
  \item \Hmc is a model of~\Bmfb and the set $\{X_{d}\mid d\in\Delta^{\Hmc}\}$ is admissible, where $X_{d}$
    is defined as $X_{d} \coloneqq \type_{\ran(\bsf)}^{\Hmc}(d)$.
  \end{enumerate}
\end{lemma}

\begin{proof}
  (1 $\Rightarrow$ 2): Let \JJ be a model of~\Bmf with $\Jb=\Hmc$.  Since $\Jb=\Hmc$, we have that
  $\Cbb=\Delta^{\Hmc}$.  By Lemma~\ref{lem:interpretation-outer-abstraction}, we have that \Hmc is a
  model of~\Bmfb.
    %
  Moreover, since \bsf is a bijection between referring meta concepts of the form \oalpha occurring
  in~\Bmf and abstracted concept names of~\MC, we have that $\ran(\bsf)$ is finite, and thus also
  that the set $\Xmc \coloneqq \{X_{d}\mid d \in \Delta^{\Hmc} \} \subseteq \powerset{\ran(\bsf)}$
  is finite.  Let $\Xmc = \{Y_1, \dots, Y_k\}$.  Since $\Cbb = \Delta^{\Hmc}$, there exists an index
  function $\nu\colon\Cbb\to\{1,\dots,k\}$ such that $X_{c} = Y_{\nu(c)}$ for every $c\in\Cbb$, i.e.
  \begin{align*}
    Y_{\nu(c)} & = \bigl\{\bsf(\oalpha)\mid\text{\oalpha occurs in~\Bmf and $c\in\bsf(\oalpha)^\Hmc$}\bigr\} \\
               & =  \bigl\{\bsf(\oalpha)\mid\text{\oalpha occurs in~\Bmf and}\ \I_c\models\alpha\bigr\}.
  \end{align*}
  Conversely, for every $\mu\in\{1,\dots,k\}$, there is an element $c\in\Cbb$ such that
  $\nu(c)=\mu$.
    % 
  The \Osig-interpretations for showing admissibility of~\Xmc are obtained as follows.  Take
  $c_1,\dots,c_k \in \Cbb$ such that $\nu(c_1) = 1$,~\dots, $\nu(c_k) = k$.  Now, for every~$i$,
  $1 \leq i \leq k$, we define the \Osig-interpretation $\Gmc_i:=(\Delta,\cdot^{\I_{c_i}})$.
  Clearly, we have that $\Gmc_i\models\B_{Y_i}$ and since $\J\models\RO$, we have that
  $\Gmc_i\models\Bmf_{Y_i}$.  Moreover, the definition of a nested interpretation yields that
  $x^{\Gmc_i}=x^{\Gmc_j}$ for all $x\in\OI\cup\OCR\cup\ORR$ and all $i,j \in \{1,\dots,k\}$.  Hence,
  the \Osig-interpretations $\Gmc_1, \dots, \Gmc_k$ attest admissibility of~\Xmc.

  (2 $\Rightarrow$ 1): Assume that \HH is a model of~\Bmfb and that the set
  $\Xmc \coloneqq \{X_d\mid d\in\Delta^{\Hmc}\}$ is admissible.  Again, since $\ran(\bsf)$ is finite, we
  have that $\Xmc \subseteq \powerset{\ran(\bsf)}$ is finite.  Let $\Xmc = \{Y_1,\dots,Y_k\}$.
  Since \Xmc is admissible, there are \Osig-interpretations $\Gmc_1=(\Delta^{\Gmc},\cdot^{\Gmc_1})$,~\dots,
  $\Gmc_k=(\Delta^{\Gmc},\cdot^{\Gmc_k})$ such that $\Gmc_i\models\Bmf_{Y_i}$ and $x^{\Gmc_i}=x^{\Gmc_j}$
  for all $x\in\OI\cup\OCR\cup\ORR$ and all $i,j\in\{1,\dots,k\}$.
    %
  Furthermore, there exists an index function $\nu\colon\Delta^{\Hmc}\to\{1,\dots,k\}$ such that
  $Y_{\nu(d)}=X_d$ for every $d\in\Delta^{\Hmc}$.
    %
  We define a nested interpretation \JJ as follows:
  \begin{itemize}
  \item $\Cbb \coloneqq \Delta^{\Hmc}$,
  \item $x^\J \coloneqq x^\Hmc$ for every $x\in\MC\cup\MR\cup\MI$,
  \item $\Delta^{\J} \coloneqq \Delta^{\Gmc}$, and
  \item $x^{\I_c}:=x^{\Gmc_{\nu(c)}}$ for every $x \in \OC \cup \OR \cup \OI$ and every $c \in \Cbb$.
  \end{itemize}
    %
  By construction of \J, we have that $x^{\Jb} = x^\Hmc$ for every
  $x \in \MR \cup \MI \cup (\MC \setminus \ran(\bsf))$.
    %
  Let $A \in \ran(\bsf)$, and let $\bsf^{-1}(A) = \oalpha$.  We have for every $d \in \Delta^{\Hmc} = \Cbb$
  that $d\in A^{\Jb}$ iff $d\in(\bsf^{-1}(A))^\J$ iff $d \in \oalpha^\J$ iff $\I_d \models \alpha$
  iff $\Gmc_{\nu(d)}\models\alpha$ iff $\bsf(\oalpha)=A\in Y_{\nu(d)}$ (since
  $\Gmc_{\nu(d)}\models\B_{Y_{\nu(d)}}$) iff $A\in X_d$ iff $d\in A^\Hmc$.
    %
  Hence, we have $\Jb=\Hmc$.
    %
  Since \Hmc is a model of~\Bmfb and, by construction of \J, \J is a model of \RO, we have by
  Lemma~\ref{lem:interpretation-outer-abstraction} that \J is a model of~\Bmf.
\end{proof}

\noindent
The following lemma is a direct consequence of the previous one. It states that we can split the
consistency check into two subtasks.

\begin{lemma}\label{lem:admissible-and-outerConsistent}%
  The \LMLO-BKB~\Bmf is consistent iff there is a set
  $\Xmc\subseteq\powerset{\ran(\bsf)}$ such that
  \begin{enumerate}
  \item \Xmc is admissible, and
  \item \Bmfb is outer consistent w.r.t.~\Xmc.
  \end{enumerate}
\end{lemma}
\begin{proof}
  \onlyifdirection Let \J be a model of \Bmf, and let $\Jb=(\Cbb,\cdot^{\Jb})$.  By
  Lemma~\ref{lem:model-equivalent-to-admissible}, we have that $\Jb$ is a model of~\Bmfb, and the
  set $\Xmc \coloneqq \{\type_{\ran(\bsf)}^{\Jb}(c) \mid c \in \Cbb\}$ is admissible.  By construction, $\Jb$ weakly
  respects $(\ran(\bsf),\Xmc)$, and hence \Bmfb is outer consistent w.r.t.~\Xmc.
    
  \ifdirection Let $\Xmc = \{X_1,\dots,X_k\}\subseteq\powerset{\ran(b)}$ such that \Xmc is
  admissible and \Bmfb is outer consistent w.r.t.~\Xmc.  Hence there is a model
  $\Hmc=(\Delta^{\Hmc},\cdot^\Hmc)$ of~\Bmfb that weakly respects $(\ran(\bsf),\Xmc)$.
    %
  We define $\Zmc \coloneqq \{\type_{\ran(\bsf)}^{\Hmc}(c) \mid c\in\Delta^{\Hmc}\}$.  Since \Hmc
  weakly respects $(\ran(\bsf),\Xmc)$, we have that $\Zmc \subseteq \Xmc$.  Since \Xmc is
  admissible, this yields admissibility of~$\Zmc$.  Lemma~\ref{lem:model-equivalent-to-admissible}
  now yields consistency of~\Bmf.
\end{proof}

Before we can analyse the complexity of the consistency problem in \LMLO, we need to state two
complexity results for the consistency problem of \SHOQ-BKBs and \SHOIQ-BKBs. For the former we
follow the idea of~\cite{Lip-PhD14}.
%
The latter is an adaptation of a proof of~\cite{Pra-JLLI05}.

\begin{lemma}\label{lem:shoq-outer-consisteny-exptime}
  Deciding whether a \SHOQ-BKB \Bmfb is outer consistent w.r.t.~\Xmc can be done in time exponential
  in the size of~\Bmfb and linear in size of~\Xmc.
\end{lemma}
\begin{proof}
  It is enough to show that deciding whether~\Bmfb has a model that weakly respects
  $(\ran(\bsf),\Xmc)$ can be done in time exponential in the size of~\Bmfb and linear in the size
  of~\Xmc.

  Here, we will adapt the ideas of~\cite{Lip-PhD14}.  It is not hard to see that we can adapt the
  notion of a quasimodel respecting a pair $(\ran(\bsf),\Xmc)$ of~\cite{Lip-PhD14} to a quasimodel
  \emph{weakly} respecting $(\ran(\bsf),\Xmc)$.
  %
  Condition~(i) in Definition~3.25 of~\cite{Lip-PhD14} states that for every $X\in\Xmc$ there must
  exist a concept type that, restricted to $\ran(\bsf)$, coincides with $X$.
  %
  %\todo{gibt es beim quasi model zu jedem concept type min 1 element mit diesem typ?}
  %
  Hence, dropping Condition~(i) yields that the quasimodel weakly respects $(\ran(\bsf),\Xmc)$.
  Then, the proof of Lemma~3.26 of~\cite{Lip-PhD14} can be adapted such that our claim follows.
  This is done by dropping the check whether Condition~(i) is satisfied in Step~4 of the algorithm
  of~\cite{Lip-PhD14}.

  The only condition for which \Xmc is relevant, is Condition~(h) which is checked in
  Step~2. Clearly, this check can be done in time linear in the size of~\Xmc.
\end{proof}


\begin{lemma}\label{lem:shoiq-outer-consisteny-nexptime}
  Deciding whether a \SHOIQ-BKB \Bmfb is outer consistent w.r.t.~\Xmc can be non-deterministically
  done in time exponential in the size of~\Bmfb and linear in size of~\Xmc.
\end{lemma}
\begin{proof}
  %Let \Umc be the set $\ran(\bsf)$. 
  %
  In~\cite{Pra-JLLI05}, it is shown the deciding satisfiability of a formula $\varphi$ of the two-variable
  fragment of first-order logic with counting quantifiers $\Cmc^{2}$ can be done non-deterministically in time
  exponential in the size of $\varphi$. There, Theorem~2 of~\cite{Pra-JLLI05} can be adapted is such
  a way that the set $\Pi$ of 1-types that occur in a model can be restricted.

  Since \SHOIQ-BKBs are also $\Cmc^{2}$-formulas and \Xmc is a set of 1-types, the claim follows.
\end{proof}



\subsection{Consistency without rigid names}
\label{sec:cons-without-rigid}
In this section, we consider the case where neither rigid concept names nor rigid role names
are allowed. 

\begin{theorem}\label{thm:shoqshoq-without-rigid-exptime}
  The consistency problem in \SHOQSHOQ is in \ExpTime if $\OCR=\ORR=\emptyset$.
\end{theorem}

\begin{proof}
  Let \Bmf be a \SHOQSHOQ-BKB and \Bmfb its outer abstraction.  We can decide consistency of~\Bmf
  using Lemma~\ref{lem:admissible-and-outerConsistent}.  We define
  $\Xmc \coloneqq \{ X \subseteq \ran(\bsf) \mid \Bmf_{X}\ \text{is consistent}\}$ where $\Bmf_{X}$
  is defined as in Definition~\ref{def:admissibility}.
  %
  We first show that $\Xmc = \{X_1, \dots, X_k\}$ is admissible.  Let $\I_i$ be a model
  of~$\Bmf_{X_i}$, which exists since $\Bmf_{X_i}$ is consistent.  Since \NI is countably infinite,
  interpretations must respect the UNA and due to the Löwenheim-Skolem
  theorem, we can assume that all models $\I_i$, $1 \leq i \leq k$, have a countably infinite
  domain. Thus, w.l.o.g.\ we can assume that all models have the same domain~$\Delta$.  Furthermore,
  we can assume that individual names are interpreted the same.  Since $\OCR=\ORR=\emptyset$, the
  set~\Xmc fulfills all conditions of Definition~\ref{def:admissibility} for admissibility.
  %
  Thus, if \Bmfb is outer consistent w.r.t.~\Xmc, then we have by
  Lemma~\ref{lem:admissible-and-outerConsistent} that \Bmf is consistent.
  
  Conversely, assume that \Bmf is consistent.  Then, by
  Lemma~\ref{lem:admissible-and-outerConsistent}, there is an admissible set
  $\Xmc' \subseteq \powerset{\ran(\bsf)}$ and \Bmfb is outer consistent w.r.t.~$\Xmc'$.  Since \Xmc
  is the maximal admissible subset of $\powerset{\ran(\bsf)}$, we have $\Xmc' \subseteq \Xmc$.  If
  \Bmfb is outer consistent w.r.t.~$\Xmc'$, it is also outer consistent w.r.t.~\Xmc.  Hence, \Bmf is
  consistent iff \Bmfb is outer consistent w.r.t.~\Xmc, which yields a decision procedure for the
  consistency problem in \SHOQSHOQ.

  It remains to analyze the complexity.  There are exponentially many
  $\Xmc \in \powerset{\ran(\bsf)}$, but each \SHOQ-BKB~$\Bmf_{X}$ can be constructed in time
  polynomial in the size of~\Bmf.  We can decide consistency of~$\Bmf_{X}$ in time
  exponential~\cite{Lip-PhD14}.  Thus, the set~\Xmc can be constructed in time exponential in the size
  of~\Bmf and it is of exponential size.  Due to Lemma \ref{lem:shoq-outer-consisteny-exptime},
  deciding whether \Bmfb is outer consistent w.r.t.~\Xmc can be done in time exponential in the size
  of~\Bmfb and linear in the size of~\Xmc.  Thus, overall we can decide the consistency problem in
  exponential time.
\end{proof}

\begin{theorem}\label{thm:shoiqshoiq-without-rigid-exptime}
  If $\OCR=\ORR=\emptyset$, the consistency problem in \SHOIQSHOIQ is in \NExpTime.
\end{theorem}
\begin{proof}
  Let \Bmf be a \cSHOIQ-BKB and \Bmfb is outer abstraction.  Analogous to the proof of
  Theorem~\ref{thm:shoqshoq-without-rigid-exptime}, we construct the maximal admissible subset~\Xmc
  of $\powerset{\ran(\bsf)}$.  Again, \Bmf is consistent iff \Bmfb is outer consistent w.r.t.~\Xmc.

  In contrast to the proof of Theorem~\ref{thm:shoqshoq-without-rigid-exptime}, we can decide
  consistency of~$\Bmf_{X}$ non-deter\-ministically in time exponential in the size of~$\Bmf_{X}$.  Thus, the set~\Xmc can be constructed
  non-deterministically in time exponential in the size of~\Bmf.  By
  Lemma~\ref{lem:shoiq-outer-consisteny-nexptime}, deciding whether \Bmfb is outer consistent
  w.r.t.~\Xmc can be non-deterministically done in time exponential in the size of \Bmfb and linear
  in the size of~\Xmc.  Overall we can decide the consistency problem non-deterministically in
  exponential time. 
\end{proof}

As the lower bounds already hold for \LMLO involving \EL, we prove them separately in
Section~\ref{sec:case-el}.  Anticipating the lower bounds shown in
Theorems~\ref{thm:alcel-exp-hard-no-rigid},~\ref{thm:elalc-exp-hard-no-rigid},~\ref{thm:shoiqel-lower-no-rigid}
and~\ref{thm:elshoiq-lower-no-rigid} we obtain \ExpTime-completeness for the consistency problem in
\LMLO for \LM and \LO being DLs between \EL and \SHOQ, excluding \ELEL, and \NExpTime-completeness
for \LMLO where either \LM or \LO is \SHOIQ, if $\OCR=\ORR=\emptyset$.

\subsection{Consistency with rigid role names}
\label{sec:cons-with-rigid}

In this section, we consider the case where rigid role names are present.

\begin{theorem}\label{thm:shoiqshoq-with-rigid-names-twoexptime}
  The consistency problem in \SHOIQSHOQ is in \TwoExpTime if $\ORR\ne\emptyset$.
\end{theorem}

\begin{proof}
  Let \BB be a \SHOIQSHOQ-BKB and \BBb its outer abstraction.  We can decide consistency of~\Bmf
  using Lemma~\ref{lem:admissible-and-outerConsistent}.  For that, we enumerate all sets
  $\Xmc \subseteq \powerset{\ran(\bsf)}$, which can be done in time doubly exponential in~\Bmf.  For
  each of these sets $\Xmc = \{X_1,\dots,X_k\}$, we check whether \Bmfb is outer consistent
  w.r.t.~\Xmc, which can be done non-deterministically in time exponential in the size of~\Bmfb and
  linear in the size of~\Xmc.
  
  Then, we check \Xmc for admissibility using the renaming technique
  of~\cite{BaGL-KR08,BaGL-ToCL12}.  For every~$i$, $1\le i\le k$, every flexible concept name~$A$
  occurring in~\Bb, and every flexible role name~$r$ occurring in~\Bb or~\RO, we introduce
  copies~$A^{(i)}$ and~$r^{(i)}$.  The \SHOQ-BKB~$\Bmf_{X_i}^{(i)} = (\B_{X_i}^{(i)},\RO^{(i)})$
  over~\Osig is obtained from~$\Bmf_{X_i}$ (see Definition~\ref{def:admissibility}) by replacing
  every occurrence of a flexible name~$x$ by $x^{(i)}$.  We define
  \begin{align*}
    \Bmf_\Xmc \coloneqq \left(\bigwedge\nolimits_{1\le i\le k}\B_{X_i}^{(i)}, \bigcup\nolimits_{1\le i\le k} \RO^{(i)}\right).
  \end{align*}
  \vspace{-\baselineskip}
  \begin{claim}
    \Xmc is admissible iff $\Bmf_\Xmc$ is consistent.
  \end{claim}
  \begin{claimproof}
    We prove the claim similarly to what was done in~\cite{Lip-PhD14}.

    If \Xmc is admissible, then there exist \Osig-interpretations
    $\Gmc_{1} = (\Delta^{\Gmc}, \cdot^{\Gmc_{1}})$, \dots,
    $\Gmc_{k} = (\Delta^{\Gmc}, \cdot^{\Gmc_{k}})$. We define the \Osig-interpretation
    $\Gmc = (\Delta^{\Gmc}, \cdot^{\Gmc})$ as follows:
  \begin{align*}
    x^{\Gmc} & \coloneqq x^{\Gmc_{1}} \text{ for all $x \in \OCR \cup \ORR \cup \OI$}\\
    (x^{(i)})^{\Gmc} & \coloneqq x^{\Gmc_{i}} \text{ for all $A\in(\OC\setminus\OCR)\cup(\OR\setminus\ORR)$}
  \end{align*}
  It is not hard to verify that \Gmc is a model of $\Bmf_{\Xmc}$.

  If $\Bmf_{\Xmc}$ is consistent, then there exists a model $\Gmc = (\Delta^{\Gmc},
  \cdot^{\Gmc})$. Analogously, we define the \Osig-interpretations
  $\Gmc_{1} = (\Delta^{\Gmc}, \cdot^{\Gmc_{1}})$, \dots,
  $\Gmc_{k} = (\Delta^{\Gmc}, \cdot^{\Gmc_{k}})$ as follows:
  \begin{align*}
    x^{\Gmc_{i}} \coloneqq
    \begin{cases}
      x^{\Gmc} & \text{ for all $x \in \OCR \cup \ORR \cup \OI$}\\
      (x^{(i)})^{\Gmc} & \text{ for all $A\in(\OC\setminus\OCR)\cup(\OR\setminus\ORR)$}
    \end{cases}
  \end{align*}
  Again, it is not hard to verify that \Xmc is admissible.
  \end{claimproof}

  \noindent
  Note that $\Bmf_\Xmc$ is of size at most
  exponential in~\Bmf and can be constructed in exponential time.  Moreover, consistency
  of~$\Bmf_\Xmc$ can be decided in time exponential in the size of~$\Bmf_\Xmc$~\cite{Lip-PhD14}, and
  thus in time doubly exponential in the size of~\Bmf.
\end{proof}

\begin{theorem}\label{thm:shoiqshoiq-with-rigid-names-ntwoexptime}
  The consistency problem in \SHOIQSHOIQ is in \TwoNExpTime if $\ORR\ne\emptyset$.
\end{theorem}

\begin{proof}
  Let \BB be a \SHOIQSHOIQ-BKB and \BBb its outer abstraction. We proceed the same as in the proof
  of Theorem~\ref{thm:shoiqshoq-with-rigid-names-twoexptime}. Enumerating all sets
  $\Xmc \subseteq \powerset{\ran(\bsf)}$ can still be done in time doubly exponential in the size of
  \Bmf, but checking for each \Xmc whether \Bmfb is outer consistent w.r.t.~\Xmc can be done
  non-deterministically in time exponential in the size of \Bmfb and linear in the size of \Xmc.

  To check \Xmc for admissibility, we again use the renaming technique
  of~\cite{BaGL-KR08,BaGL-ToCL12}, and by the same arguments as above, \Xmc is admissible iff
  $\Bmf_\Xmc$ is consistent.  Again, $\Bmf_\Xmc$ is of size at most exponential in~\Bmf and can be
  constructed in exponential time, but consistency of~$\Bmf_\Xmc$ can be decided
  non-deterministically in time exponential in the size of~$\Bmf_\Xmc$, and thus
  non-deterministically in time doubly exponential in the size of \Bmf.
\end{proof}

\noindent
Together with the lower bound shown in Theorem~\ref{thm:elalc-2exp-hard-rigid-roles}, we obtain
\TwoExpTime-completeness for the consistency problem in any \LMLO between \ELALC and \SHOIQSHOQ if
$\ORR\neq\emptyset$. The consistency problem in \LMSHOIQ for \LM between \EL
and \SHOIQ is \TwoExpTime-hard and in \TwoNExpTime, if $\ORR\neq\emptyset$.

\subsection{Consistency with only rigid concept names}
\label{sec:cons-with-only}

In this section, we consider the case where rigid concept are present, but rigid role names are not
allowed.

\begin{theorem}\label{thm:shoiqshoq-with-rigid-concepts-nexptime}
  The consistency problem in \SHOIQSHOQ is in \NExpTime if $\OCR \neq \emptyset$ and $\ORR=\emptyset$.
\end{theorem}

\begin{proof}
  Let \BB be a \SHOIQSHOQ-BKB and \BBb its outer abstraction.  We can decide consistency of~\Bmf
  using Lemma~\ref{lem:admissible-and-outerConsistent}. We first non-deterministically guess the set
  $\Xmc = \{X_{1},\ldots,X_{k}\} \subseteq \powerset{\ran(\bsf)}$, which is of size at most
  exponential in \Bmf. Due to Lemma~\ref{lem:shoiq-outer-consisteny-nexptime} we can check whether
  \Bmfb is outer consistent w.r.t.~\Xmc non-deterministically in time exponential in the size
  of~\Bmfb and linear in the size of~\Xmc.
  
  It remains to check \Xmc for admissibility. Here we follow the ideas
  of~\cite{BaGL-KR08,BaGL-ToCL12,Lip-PhD14}. Instead of checking the consistency of $\Bmf_{\Xmc}$
  directly, which would yield a double-exponential time bound, we reduce it to~$k$ separate
  consistency checks each of which can be decided in time exponential in the size of
  \Bmc. Therefore, for each individual name $a$ in \Bmf we guess the set of rigid concepts of
  which~$a$ is an instance.
  % 
  More precisely, let $\OCR(\B) \subseteq \OCR$ and $\OI(\B) \subseteq \OI$ be the sets of all rigid
  concept names and individual names occurring in \B, respectively. We non-deterministically guess a
  set $\Ymc \subseteq \powerset{\OCR(\B)}$ and a mapping $\kappa\colon\OI(\B)\to\Ymc$, which also can
  be done in time exponential in the size of \Bmf.

  \begin{claim}
    \Xmc is admissible iff $\widehat{\Bmf}_{X_{i}}$ has a model that respects $(\OCR(\B),\Ymc)$, for
    all $1 \leq i \leq k$, where $\widehat{\Bmf}_{X_{i}}$ is defined as:
    \begin{align*}
    \widehat{\Bmf}_{X_{i}}\coloneqq\left(\B_{X_{i}} \land \bigwedge_{a\in\OI(\B)} \left(\bigsqcap_{A\in\kappa(a)} A \sqcap
    \bigsqcap_{A\in\OCR(\B)\setminus\kappa(a)} \lnot A\right)(a),\ \RO\right).
  \end{align*}
  \end{claim}
  \begin{claimproof}
    If \Xmc is admissible, then there exists an \Osig-interpretation
    $\Gmc_{i} = (\Delta^{\Gmc}, \cdot^{\Gmc_{i}})$ such that $\Gmc_{i}$ is a model of
    $(\Bmc_{X_{i}}, \RO)$. Let \Ymc be the set of all \OCR-types in $\Gmc_{i}$, i.e.\
    $\Ymc = \{\smash{\type_{\OCR}^{\Gmc_{i}}(d)} \mid d\in\Delta^{\Gmc_{i}}\}$ and let $\kappa$ map $d$ to
    its type, i.e.\ $\smash{\kappa(d) = \type_{\OCR}^{\Gmc_{i}}(d)}$. It is easy to see that $\Gmc_{i}$ is
    a model of $\widehat{\Bmf}_{X_{i}}$, and by construction, $\Gmc_{i}$ respects $(\OCR(\B),\Ymc)$.

    Now, assume that there exist \Osig-interpretations
    $\Gmc_{1} = (\Delta^{\Gmc_{1}}, \cdot^{\Gmc_{1}})$, \dots,
    $\Gmc_{k} = (\Delta^{\Gmc_{k}}, \cdot^{\Gmc_{k}})$ s.t.\ $\Gmc_{i}\models\widehat{\Bmf}_{X_{i}}$
    and $\Gmc_{i}$ respects $(\OCR(\B),\Ymc)$, for all $1 \leq i \leq k$.  By the same arguments as
    in the proof of Theorem~\ref{thm:shoqshoq-without-rigid-exptime}, we can assume w.l.o.g.\ that
    all models have a countably infinite domain. Since a disjoint union of $\Gmc_{i}$ with itself is
    again a model of $\widehat{\Bmf}_{X_{i}}$, we can also assume that for each $Y\in\Ymc$ there are
    infinitely many elements $d$ in ${C_{Y}}^{\Gmc_{i}}$ where
    \begin{align*}
    C_{Y}\coloneqq\bigsqcap_{A \in Y}A\sqcap\bigsqcap_{A \in \OCR \setminus Y}\lnot A.
    \end{align*}
    Hence, we can partition the domain $\Delta^{\Gmc_{i}}$ into % countably infinite sets
    \begin{align*}
      \Delta^{\Gmc_{i}} = \bigcup_{Y\in\Ymc} P_{i}(Y) \text{\quad with \quad} P_{i}(Y)\coloneqq \{d\in\Delta^{\Gmc_{i}}\mid d\in C_{Y}\}.
    \end{align*}
    Because of the above assumptions and since $\Gmc_{i}\models
    \bigwedge_{a\in\OI(\B)}C_{\kappa(a)}(a)$, there exist bijections $\pi_{i} : \Delta^{\Gmc_{1}} \to
    \Delta^{\Gmc_{i}}$, $2 \leq i \leq k$, such that
    \begin{itemize}
    \item $\pi_{i}(P_{1}(Y)) = P_{i}(Y)$, for every $Y\in\Ymc$, and
    \item $\pi_{i}(a^{\Gmc_{1}}) = a^{\Gmc_{i}}$ for every $a\in\OI(\B)$.
    \end{itemize}
    Hence, we can assume that all $\Gmc_{i}$ have the same domain and interpret individual names and
    rigid concept names in the same way.
  \end{claimproof}

  \noindent
  The \SHOQ-BKB $\smash{\widehat{\Bmf}_{X_{i}}}$ is of size polynomial in the size of \B and can be
  constructed in time exponential in the size of \B. We can check if
  $\smash{\widehat{\Bmf}_{X_{i}}}$ has a model that respects $(\OCR(\B),\Ymc)$ in time exponential
  in the size of $\smash{\widehat{\Bmf}_{X_{i}}}$ \cite{Lip-PhD14}, and thus in time exponential in
  the size of~\Bmf.
\end{proof}

\noindent
Together with the lower bounds shown in Theorem~\ref{thm:alcel-nexp-hard-rigid-concepts} and~\ref{thm:elalc-nexp-hard-rigid-concepts}, we
obtain \NExpTime-completeness for the consistency problem in \LMLO for \LM and \LO being DLs between
\EL and \SHOIQ, excluding \ELEL, if $\OCR \neq \emptyset$ and $\ORR = \emptyset$.

Summing up the results, we obtain the following corollary.

\begin{corollary}
  For \LMLO between \ELALC and \SHOQSHOQ, the consistency problem in \LMLO is
  \begin{itemize}
  \item \ExpTime-complete if  $\OCR=\emptyset$ and $\ORR=\emptyset$,
  \item \NExpTime-complete if $\OCR\ne\emptyset$ and $\ORR=\emptyset$, and
  \item \TwoExpTime-complete if $\ORR\ne\emptyset$.
  \end{itemize}
  For \SHOIQLO with \LO between \ALC and \SHOQ, the consistency problem in \SHOIQLO is
  \begin{itemize}
  \item \NExpTime-complete if $\ORR=\emptyset$, and
  \item \TwoExpTime-complete if $\ORR\neq\emptyset$.
  \end{itemize}
  For \LMSHOIQ with \LM between \EL and \SHOIQ, the consistency problem in \LMSHOIQ is
  \begin{itemize}
  \item \NExpTime-complete if  $\OCR=\emptyset$ and $\ORR=\emptyset$,
  \item \NExpTime-hard and in \TwoNExpTime if $\OCR\ne\emptyset$ and $\ORR=\emptyset$, and
  \item \TwoExpTime-hard and in \TwoNExpTime if $\ORR\ne\emptyset$.
  \end{itemize}
\end{corollary}

\section{Contextualised Description Logics Involving \texorpdfstring{\EL}{EL}}
\label{sec:case-el} 

In this section, we give some complexity results for context DLs
\LMLO where~\LM or~\LO is~\EL.
%
In Section~\ref{sec:dlinner-el}, we consider \LMEL where \LM is between \ALC and \SHOIQ.  Then, in
Section~\ref{sec:dlouter-el}, we consider the remaining context DLs \ELLO where \LO is between \ALC
and \SHOIQ.

For \EL as \LM, instead of considering \ELLO-BKBs, we allow only \ELLO-ontologies, i.e.\
\emph{conjunctions} of m-axioms. It seems unnatural to allow axiom negation for a logic which does
not have concept negation. Furthermore, this would miss any practical motivation. For
\ELEL-ontologies the consistency problem becomes trivial since all \ELEL-ontologies are consistent,
as \EL lacks to express contradictions.
%
This restriction, however, does not yield a better complexity in the cases of \ELLO, where \LO is
between \ALC and \SHOQ.

\subsection{The Contextualised Description Logics \texorpdfstring{\LMEL}{LM[EL]}}
\label{sec:dlinner-el}

In this section, we consider \LMEL where \LM is between \ALC
and \SHOIQ.
%
We start with the lower bounds for \ALCEL and \SHOIQEL in the case without
rigid names.

\begin{theorem}\label{thm:alcel-exp-hard-no-rigid}
  The consistency problem in \ALCEL is \ExpTime-hard if no rigid names are allowed, i.e.\ $\OCR = \ORR = \emptyset$.
\end{theorem}

\begin{proof}
    Deciding whether a given conjunction of \ALC-axioms~\B is consistent is
    \ExpTime-hard~\cite{Sch-IJCAI91}.  Obviously, \B is also an \ALCEL-BKB.
\end{proof}

\begin{theorem}\label{thm:shoiqel-lower-no-rigid}
  The consistency problem in \SHOIQEL is \NExpTime-hard if no rigid names are allowed, i.e.\ $\OCR = \ORR = \emptyset$.
\end{theorem}

\begin{proof}
    Deciding whether a given conjunction of \ALCOIQ-axioms~\B is consistent is
    \NExpTime-complete~\cite{Tob-JAIR00}.  Obviously, \B is also an \SHOIQEL-BKB.
\end{proof}

\noindent
For the cases of rigid names, the lower bounds of \NExpTime are obtained by a
careful reduction of the satisfiability problem in the temporalised DL
\EL-LTL~\cite{BoTh-IJCAI15,BoTh-LTCS-15-07}, which is a fragment of \ALC-LTL introduced
in~\cite{BaGL-KR08,BaGL-ToCL12}.
%
For the sake of completeness, we recall the basic definitions of \Lmc-LTL here,
where \Lmc is a DL.

\begin{definition}[Syntax of \Lmc-LTL]
  \emph{\Lmc-LTL-formulas over \Osig} are defined by induction:
  \begin{itemize}
  \item if $\alpha$ is an \Lmc-axiom over \Osig, then $\alpha$ is an \Lmc-LTL-formula, and
  \item if $\phi,\psi$ are \Lmc-LTL-formulas over \Osig, then so are $\lnot\phi$ (negation),
    $\phi\land\psi$ (conjunction), $\phi\until\psi$ (until), $\Next\phi$ (next), and
  \item nothing else is an \Lmc-LTL-formula. \qedhere
  \end{itemize}
\end{definition}

\noindent
As usual in temporal logics, we use the following abbreviations: 
\begin{itemize}
\item $\phi\lor\psi$ (disjunction) for $\lnot(\lnot\phi\land\lnot\psi)$,
\item $\mathsf{true}$ (tautology) for $A(a)\lor\lnot A(a)$ where $A\in\OC$ is arbitrary but fixed,
\item $\Diamond\phi$ (eventually) for $\mathsf{true}\until\phi$, and
\item $\Box\phi$ (always) for $\lnot\Diamond\lnot\phi$.
\end{itemize}
%
The semantics of \Lmc-LTL is based on DL-LTL-structures.  These are sequences of
\Osig-inter\-pre\-tations over the same non-empty domain that additionally respect
rigid names and the rigid individual assumption.

\begin{definition}[DL-LTL-structure]
    A \emph{DL-LTL-structure over \Osig} is a sequence $\Imf=(\I_i)_{i \geq 0}$ of
    \Osig-interpretations $(\Delta,\cdot^{\I_i})$ such that
    $x^{\I_{i}} = x^{\I_{j}}$ holds for all $x\in\OCR\cup\ORR\cup\OI$, $i,j>0$.
\end{definition}

\noindent
We are now ready to define the semantics of \Lmc-LTL.

\begin{definition}[Semantics of \Lmc-LTL]
  The validity of an \Lmc-LTL-formula~$\phi$ in a DL-LTL-structure $\Imf=(\I_i)_{i\ge 0}$ at
  time~$i\ge 0$, denoted by $\Imf,i\models\varphi$, is defined inductively:

  \vspace{\topsep}
  \begin{tabular}{l@{\quad}l@{\quad}l}
    $\Imf,i\models\alpha$        & iff & $\I_{i}\models\alpha$ where $\alpha$ is an \ALC-axiom over \Osig,\\
    $\Imf,i\models\phi\land\psi$      & iff & $\Imf,i\models\phi$ and $\Imf,i\models\psi$, \\
    $\Imf,i\models\lnot\phi$       & iff & not $\Imf,i\models\phi$, \\
    $\Imf,i\models\Next\phi$   & iff & $\Imf,i+1\models\phi$, \\
    $\Imf,i\models\phi\until\psi$ & iff & there is $k\geq i$ such that $\Imf,k\models\psi$ and \\
                      &     & $\Imf,j\models\phi$ for all $j$ with $i\leq j < k$.
  \end{tabular}

  \vspace{\topsep}\noindent We call an \Lmc-LTL-structure \Imf a \emph{model of~$\phi$} if
  $\Imf,0\models\phi$.  The \emph{satisfiability problem in \Lmc-LTL} is the question whether a
  given \Lmc-LTL-formula~$\phi$ has a model.
\end{definition}

\noindent
In~\cite{BoTh-IJCAI15,BoTh-LTCS-15-07}, it is shown that the satisfiability problem in \EL-LTL is
\NExpTime-hard as soon as rigid concept names are present.  We reduce the satisfiability problem
in \EL-LTL to the consistency problem in \ALCEL to obtain the lower bounds of \NExpTime, where we
use the fact that the lower bounds of~\cite{BoTh-IJCAI15,BoTh-LTCS-15-07} already hold for a
syntactically restricted fragment of \EL-LTL.

\begin{theorem}\label{thm:alcel-nexp-hard-rigid-concepts}
  The consistency problem in \ALCEL is \NExpTime-hard if $\OCR \neq \emptyset$ and
  $\ORR = \emptyset$.
\end{theorem}

\begin{proof}
  In fact, the lower bounds hold for \EL-LTL-formulas of the form $\Box\phi$ where $\phi$ is an
  \EL-LTL-formula that contains only \Next as temporal operator~\cite{BoTh-LTCS-15-07}.

  Let $\Box\phi$ be such an \EL-LTL-formula over~\Osig.  Now, we obtain the m-concept $C_\phi$
  from~$\phi$ by replacing \EL-axioms~$\alpha$ by \oalpha, $\land$ by $\sqcap$, and subformulas of
  the form $\Next\psi$ by $\forall t.\psi\sqcap\exists t.\psi$, where $t\in\MR$ is arbitrary but
  fixed.

  \begin{claim}
    $\Box\phi$ is satisfiable iff $\B=\top\sqsubseteq C_\phi\sqcap\exists t.\top$ is consistent.
  \end{claim}

  \begin{claimproof}
    First, assume that $\Box\phi$ is satisfiable.
    Take any DL-LTL-structure $\Imf=(\Delta,\cdot^{\I_i})_{i\ge 0}$ with $\Imf,0\models\Box\phi$.
    We define the nested interpretation \JJ as follows:
    \begin{align*}
      \Cbb & \coloneqq \{c_i \mid i \geq 0\},\\
      \Delta^{\J} & \coloneqq \Delta,\\
      \cdot^{\I_{c_{i}}} & \coloneqq \cdot^{\I_{i}},\\
      t^\J & \coloneqq \{(c_{i},c_{i+1}) \mid i \geq 0\}.
    \end{align*}
    
    \noindent
    We now show that for every $i\ge 0$, we have $\Imf,i\models\phi$ iff $c_i\in C_\phi^\J$ by
    induction on the structure of~$\phi$:
    
    \vspace{\topsep}\noindent
    \begin{tabularx}{\linewidth}{lX}
      $\phi = \alpha:$ & $\Imf,i\models\phi$ 
               iff $\I_{i} \models \alpha$ 
               iff $\I_{c_{i}} \models \alpha$
               iff $c_{i} \in \oax{\alpha}^{\J} = C_{\phi}^{\J}$, \\[1ex]
      $\phi = \lnot \psi$: &  $\Imf,i\models\phi$ 
               iff $\Imf,i\not\models\psi$ 
               iff $c_i\notin C_\psi^\J$ 
               iff $c_i\in(\lnot C_\psi)^\J=C_\phi^\J$, \\[1ex]
      $\phi = \psi_1\land\psi_2:$ & $\Imf,i\models\phi$ 
               iff $\Imf,i\models\psi_{1}$ and $\Imf,i\models\psi_{2}$ 
               iff $c_{i} \in C_{\psi_{1}}^{\J}$ and $c_{i} \in C_{\psi_{2}}^{\J}$ \\
             & \leavevmode\hphantom{$\Imf,i\models\phi$} iff $c_{i} \in (C_{\psi_{1}} \sqcap C_{\psi_{2}})^{\J}
               = C_{\phi}^{\J}$, and\\[1ex]
      $\phi = \Next\psi$: & $\Imf,i\models\phi$
               iff $\Imf,i+1\models\psi$
               iff $c_{i+1} \in C_{\psi}^{\J}$
               iff $c_{i} \in (\forall t.C_{\psi} \sqcap \exists t.C_{\psi})^{\J} = C_{\phi}^{\J}$,
    \end{tabularx}

    \vspace{\topsep}\noindent
    where $\alpha$ is an \EL-axiom over~\Osig.
    %
    It follows that $\Imf,0\models\Box\phi$ iff $\J\models\top\sqsubseteq C_\phi$.  Furthermore,
    since $(c_{i},c_{i+1})\in t^\J$, we have $c_{i}\in(\exists t.\top)^\J$.  Thus,
    $\J\models\top\sqsubseteq\exists t.\top$.

    For the `if' direction, take any nested interpretation \JJ that is a model of
    $\top\sqsubseteq C_\phi\sqcap\exists t.\top$.  Let $\Psf$ be an infinite path
    $\Psf=c_0 c_1 \dots$ with $c_i\in\Cbb$ and $(c_{i},c_{i+1}) \in t^\J$ for every $i \geq 0$.
    Such a path exists, because $\J\models\top\sqsubseteq\exists t.\top$.  We define the nested
    interpretation
    $\J_{\Psf}\coloneqq(\{c_i\mid i\ge 0\},\cdot^{\J_{\Psf}},\Delta^{\J},(\cdot^{\I_{c_{i}}})_{i
      \geq 0})$ where $\cdot^{\J_{\Psf}}$ is the restriction of $\cdot^{\J}$ to the domain
    $\{c_i\mid i\ge 0\}$.
        
    By construction we have that $\J_{\Psf}\models\top\sqsubseteq\exists t.\top$.  We show by a
    simple case distinction that $\J_{\Psf}\models\top\sqsubseteq C_\phi$.
    %
    If $C_\phi$ does not contain any role name $r\in\MR$, the restriction on the set of worlds
    preserves the entailment relation.
    %
    Otherwise, $C_\phi$ is of the form $\forall t.C_\psi\sqcap\exists t.C_\psi$.  Since
    $\J_{\Psf}\models\top\sqsubseteq\exists t.\top$, $\J_{\Psf}\models\top\sqsubseteq C_\psi$, and
    there is only one $t$-successor, we have $\J_{\Psf}\models\top\sqsubseteq C_\phi$.  Hence,
    $\J_{\Psf}\models\top\sqsubseteq C_{\phi}\sqcap\exists t.\top$.

    We define the DL-LTL-structure \Imf over~\Osig as
    $\Imf\coloneqq(\Delta^{\J},\cdot^{\I_i})_{i\ge 0}$ where
    $\cdot^{\I_{i}}\coloneqq\cdot^{\I_{c_{i}}}$.
    %
    Again, we show that for every $i \geq 0$, that we have $c_i \in C_{\phi}^{\J_{\Psf}}$ iff
    $\Imf,i\models\phi$ by induction on the structure of~$\phi$:

    \noindent
    \begin{tabularx}{\linewidth}{@{}l@{ }X@{}}
      $\phi = \alpha:$ & $c_{i} \in C_{\phi}^{\J_{\Psf}} = \oax{\alpha}^{\J_{\Psf}}$
               iff $\I_{c_{i}} \models \alpha$
               iff $\I_{i} \models \alpha$ 
               iff $\Imf,i \models \phi$,\\[1ex]
      $\phi = \lnot \psi$: &  $c_{i} \in C_{\phi}^{\J_{\Psf}} = (\lnot C_{\psi})^{\J_{\Psf}}$
               iff $c_{i} \notin C_{\psi}^{\J_{P}}$
               iff $\Imf,i\not\models\psi$
               iff $\Imf,i\models\phi$,\\[1ex]
      $\phi = \psi_1\land\psi_2:$ & $c_{i} \in C_{\phi}^{\J_{\Psf}} = (C_{\psi_{1}} \sqcap C_{\psi_{2}})^{\J_{\Psf}}$
               iff $c_{i} \in C_{\psi_{1}}^{\J_{\Psf}}$ and $c_{i} \in C_{\psi_{1}}^{\J_{\Psf}}$ 
               iff $\Imf,i\models\psi_{1}$ and $\Imf,i\models\psi_{2}$ \\
             & \leavevmode\hphantom{$c_{i} \in C_{\phi}^{\J_{\Psf}}$} iff $\Imf,i\models\phi$, and\\[1ex]
      $\phi = \Next\psi$: & $c_{i} \in C_{\phi}^{\J_{\Psf}} = (\forall t.C_{\psi} \sqcap \exists t.C_{\psi})^{\J_{\Psf}} $
               iff $c_{i+1} \in C_{\psi}^{\J_{\Psf}}$
               iff $\Imf,i+1\models\psi$
               iff $\Imf,i\models\phi$,
    \end{tabularx}
    %
    \vspace{\topsep}
    where $\alpha$ is an \EL-axiom over~\Osig.  It follows that
    $\J_{P}\models\top\sqsubseteq C_\phi$ iff $\Imf,0\models\Box\phi$.
    \end{claimproof}

    This claim yields the lower bound of \NExpTime for the consistency problem
    in \ALCEL if $\OCR\ne\emptyset$.
\end{proof}

Next, we prove the upper bound of \NExpTime for the consistency problem in the
case of rigid names.

\begin{theorem}\label{thm:shoiqel-in-nexp-rigid-roles}
  The consistency problem in \SHOIQEL is in \NExpTime if $\ORR \neq \emptyset$.
\end{theorem}

\begin{proof}
  Let $\Bmf = (\Bmc,\emptyset,\RM)$ be a \SHOIQEL-BKB and $\Bmfb = (\Bb,\emptyset)$ its outer
  abstraction.  We again use Lemma~\ref{lem:admissible-and-outerConsistent} to decide consistency of
  \Bmf.  First, we non-deterministically guess a set $\Xmc \subseteq \powerset{\ran(\bsf)}$. By
  Lemma~\ref{lem:shoiq-outer-consisteny-nexptime}, we can decide outer consistency of \Bb w.r.t.\
  \Xmc non-deterministically in time exponential in the size of~\Bb and linear in the size of~\Xmc.

  To check \Xmc for admissibility, we construct the \EL-BKB $\B_\Xmc$ over~\Osig as in the proof of
  Theorem~\ref{thm:shoiqshoq-with-rigid-names-twoexptime}. This actually is a conjunction of
  \EL-literals over~\Osig, i.e.\ a conjunction of (negated) \EL-axioms over~\Osig.  The following
  claim shows that consistency of~$\B_\Xmc$ can be reduced to consistency of a conjunction of
  \ELObot-axioms over~\Osig, where \ELObot is the extension of \EL with nominals and the bottom
  concept.

  \begin{claim}
    For every conjunction of \EL-literals \B over~\Osig, there exists an equisatisfiable conjunction
    $\B'$ of \ELObot-axioms over~\Osig which is of size polynomial in the size of \B.
  \end{claim}

  \begin{claimproof}
    Let $\B$ be a conjunction of \EL-literals over~\Osig, i.e.
    \begin{align*}
      \B & = \alpha_{1} \land \dots \land \alpha_{n} \land \lnot\beta_{1} \land \dots \land \lnot\beta_{m} \\
    \intertext{where $\alpha_{i}$, $1\leq i \leq n$, $\beta_{j}$, $1\leq j \leq m$ are \EL-axioms over~\Osig. We define $\B'$ as follows:}
      \B' & = \alpha_{1} \land \dots \land \alpha_{n} \land \gamma_{1} \land \dots \land \gamma_{m}, \\
    \intertext{where} 
      \gamma_{i} & \coloneqq
        \begin{cases}
          C(a_{i}) \land D'(a_{i}) \land D \sqcap D' \sqsubseteq \bot & \text{if $\beta_{i} = C \sqsubseteq D$,}\\
          A'(a) \land A \sqcap A' \sqsubseteq \bot & \text{if $\beta_{i} = A(a)$, and}\\
          \{a\} \sqcap \exists r.\{b\} \sqsubseteq \bot & \text{if $\beta_{i} = r(a,b)$}
        \end{cases}
    \end{align*}
    with $A',D'$ being fresh concept names and $a_{i}$ being fresh individual names.  It is easy to
    see that if an \Osig-interpretation \I is a model of
    $\lnot\beta_{1}\land\dots\land\lnot\beta_{m}$, there exists an extension of \I that is a model
    of $\gamma_{1}\land\dots\land\gamma_{m}$.  Conversely, if an \Osig-interpretation $\I'$ is a
    model of $\gamma_{1}\land\dots\land\gamma_{m}$, it is also a model of
    $\lnot\beta_{1}\land\dots\land\lnot\beta_{m}$.  Hence \B and $\B'$ are equisatisfiable. Clearly,
    $\B'$ is of size polynomial in the size of \B.
    \end{claimproof}

    \noindent Since $\Bmc_{\Xmc}$ is at most exponential in \Bmc and the fact that the consistency
    of conjunctions of \ELObot-axioms can be decided in polynomial time~\cite{BaBL-IJCAI05}, we can
    check whether $\Bmc_{\Xmc}$ is consistent in time polynomial in the size of $\Bmc_{\Xmc}$ and,
    thus, in time exponential in the size of \B.

    Overall, this yields the claimed upper bound.
\end{proof}

Summming up the results of this section, we obtain the following corollary.

\begin{corollary}
  For all \LM between \ALC and \SHOIQ, the consistency problem in \LMEL is
  \begin{itemize}
  \item \ExpTime-complete if $\OCR=\emptyset$, $\ORR=\emptyset$ and \LM is between \ALC and \SHOQ,
    and
  \item \NExpTime-complete otherwise.
  \end{itemize}
\end{corollary}

\subsection{The Contextualised Description Logics \texorpdfstring{\ELLO}{EL[LO]}}
\label{sec:dlouter-el}

In this section, we consider \ELLO where \LO is between \ALC and \SHOQ.
%
First, we show the lower bounds for the case without rigid names.

\begin{theorem}\label{thm:elalc-exp-hard-no-rigid}
  The consistency problem in \ELALC is \ExpTime-hard if no rigid names are allowed, i.e.\ $\OCR = \ORR = \emptyset$.
\end{theorem}

\begin{proof}
  Deciding whether a given conjunction $\B = \alpha_{1} \land \dots \land \alpha_{n}$ of \ALC-axioms
  is consistent is \ExpTime-hard\cite{Sch-IJCAI91}.  Obviously, \B is consistent iff the \ELALC-BKB
  $(\oax{\alpha_{1}} \sqcap \dots \sqcap \oax{\alpha_{n}})(c)$ is consistent, where $c\in\MI$.
\end{proof}

\begin{theorem}\label{thm:elshoiq-lower-no-rigid}
  The consistency problem in \ELSHOIQ is \NExpTime-hard if no rigid names are allowed, i.e.\ $\OCR = \ORR = \emptyset$.
\end{theorem}

\begin{proof}
  Deciding whether a given conjunction $\B = \alpha_{1} \land \dots \land \alpha_{n}$ of \ALCOIQ-axioms
  is consistent is \NExpTime-complete~\cite{Tob-JAIR00}.  Obviously, \B is consistent iff the \ELSHOIQ-BKB
  $(\oax{\alpha_{1}} \sqcap \dots \sqcap \oax{\alpha_{n}})(c)$ is consistent, where $c\in\MI$.
\end{proof}

\noindent
For the case of rigid role names, we have lower bounds of \TwoExpTime.

\begin{theorem}\label{thm:elalc-2exp-hard-rigid-roles}
  The consistency problem in \ELALC is \TwoExpTime-hard if $\ORR \neq \emptyset$.
\end{theorem}

\begin{proof}
  To show the lower bound, we adapt the proof ideas of~\cite{BaGL-KR08,BaGL-ToCL12}, and reduce the
  word problem for exponentially space-bounded alternating Turing machines (i.e.~is a given word~$w$
  accepted by the machine~$M$) to the consistency problem in \ELALC with rigid roles,
  i.e.~$\ORR\ne\emptyset$.
  %
  In~\cite{BaGL-KR08,BaGL-ToCL12}, a reduction was provided to show \TwoExpTime-hardness for the
  temporalised DL \ALC-LTL in the presence of rigid roles.
  %
  Here, we mimic the properties of the time dimension that are important for the reduction using a
  role name $t\in\MR$.

  Our \ELALC-BKB is the conjunction of the \ELALC-BKBs introduced below.
  %
  First, we ensure that we never have a \emph{last} time point:
  \begin{gather*}
    \top\sqsubseteq\exists t.\top
  \end{gather*}
  Note that in the corresponding model, we do not enforce a $t$-chain since cycles are not
  prohibited.  This, however, is not important in the reduction.

  The \ALC-LTL-formula obtained in the reduction of~\cite{BaGL-KR08,BaGL-ToCL12} is a conjunction of
  \ALC-LTL-formulas of the form $\Box\phi$, where $\phi$ is an \ALC-LTL-formula.  This makes sure
  that $\phi$ holds in all (temporal) worlds.  For the cases where $\phi$ is an \ALC-axiom, we can
  simply express this by:
  \begin{gather*}
    \top\sqsubseteq\oax{\phi}
  \end{gather*}
  %
  This captures all except for two conjuncts of the \ALC-LTL-formula of the reduction
  of~\cite{BaGL-KR08,BaGL-ToCL12}.  There, a $k$-bit binary counter using concept names
  $A_0',\dots,A_{k-1}'$ was attached to the individual name $a$, which is incremented along the
  temporal dimension.  We can express something similar in \ELALC. Instead of incrementing the
  counter values along a sequence of $t$-successors, we have to go backwards since \EL does allow
  for branching but does not allow for value restrictions, i.e.~we cannot make sure that all
  $t$-successors behave the same.  More precisely, if the counter value~$n$ is attached to~$a$ in
  context~$c$, the value $n+1$ (modulo $2^k-1$) must be attached to~$a$ in \emph{all} of $c$'s
  $t$-predecessors.
  %
  First, we ensure which bits must be flipped:
  \begin{align*}
    \bigwedge_{i<k} \Bigl( \exists t.\bigl(\oax{A_{0}'(a)} \sqcap \dots \sqcap \oax{A_{i-1}'(a)} \sqcap \oax{A_{i}'(a)}\bigr)
    & \ \sqsubseteq\ \oax{(\lnot A_{i}')(a)}\Bigr)\\ 
    \bigwedge_{i<k}\Bigl( \exists t.\bigl(\oax{A_{0}'(a)}\sqcap\ldots\sqcap \oax{A_{i-1}'(a)} \sqcap \oax{(\lnot A_{i}')(a)}\bigr)
    & \ \sqsubseteq\ \oax{A_{i}'(a)}\Bigr)
  \end{align*}
  %
  Next, we ensure that all other bits stay the same:
  %
  \begin{align*}
    \bigwedge_{0<i<k}\ \bigwedge_{j<i}\Bigl(\exists t.\bigl(\oax{(\lnot A_{j}')(a)}\sqcap\oax{A_{i}'(a)}\bigr)
    & \ \sqsubseteq\ \oax{A_{i}'(a)}\Bigr)\\
    \bigwedge_{0<i<k}\ \bigwedge_{j<i}\Bigl(\exists t.\bigl(\oax{(\lnot A_{j}')(a)}\sqcap\oax{(\lnot A_{i}')(a)}\bigr)
    & \ \sqsubseteq\ \oax{(\lnot A_{i}')(a)}\Bigr)
  \end{align*}
  %
  Note that due to the first m-axiom above, we enforce that every context has a $t$-successor.  By
  the other m-axioms, we make sure that we enforce a $t$-chain of length $2^k$.
  % 
  As in~\cite{BaGL-KR08,BaGL-ToCL12}, it is not necessary to initialize the counter.  Since we
  decrement the counter along the $t$-chain (modulo $2^k-1$), every value between $0$ and $2^k-1$
  is reached.
  
  The conjunction of all the \ELALC-BKBs above yields an \ELALC-BKB~\B that is consistent iff the
  given word~$w$
  is accepted by the machine~$M$.
\end{proof}

\noindent
Finally, we obtain a lower bound of \NExpTime in the case of rigid concept names
only.

\begin{theorem}\label{thm:elalc-nexp-hard-rigid-concepts}
  The consistency problem in \ELALC is \NExpTime-hard if $\OCR \neq \emptyset$ and
  $\ORR = \emptyset$.
\end{theorem}

\begin{proof}
  To show the lower bound, we again adapt the proof ideas of~\cite{BaGL-KR08,BaGL-ToCL12}, and
  reduce an exponentially bounded version of the domino problem to the consistency problem in \ELALC
  with rigid concepts, i.e.~$\OCR\ne\emptyset$ and $\ORR=\emptyset$.
  %
  In~\cite{BaGL-KR08,BaGL-ToCL12}, a reduction was provided to show \NExpTime-hardness for the
  temporalised DL \ALC-LTL in the presence of rigid concepts.
  %
  As in the proof of Theorem~\ref{thm:elalc-2exp-hard-rigid-roles}, we mimic the properties of the
  time dimension that are important for the reduction using a role name $t\in\MR$.
  
  \noindent
  Our \ELALC-BKB is the conjunction of the \ELALC-BKBs introduced below.  We proceed in a similar
  way as in the proof of Theorem~\ref{thm:elalc-2exp-hard-rigid-roles}.
  %
  First, we ensure that we never have a \emph{last} time point:
  \begin{gather*}
    \top\sqsubseteq\exists t.\top
  \end{gather*}
  Note that in the corresponding model, we do not enforce a $t$-chain since cycles are not
  prohibited.  As in the reduction in the proof of Theorem~\ref{thm:elalc-2exp-hard-rigid-roles},
  this is not important in the reduction here.

  Next, note that since the $\Box$-operator distributes over conjunction, most of the conjuncts of
  the \ALC-LTL-formula of the reduction of~\cite{BaGL-KR08,BaGL-ToCL12} can be rewritten as
  conjunctions of \ALC-LTL-formulas of the form $\Box\alpha$, where $\alpha$ is an \ALC-axiom.  As
  already argued in the proof of Theorem~\ref{thm:elalc-2exp-hard-rigid-roles}, this can equivalently be expressed
  by $\top\sqsubseteq\oalpha$.

  In~\cite{BaGL-KR08,BaGL-ToCL12}, a $(2n+2)$-bit binary counter is employed using concept names
  $Z_0,\dots,Z_{2n+1}$.  This counter is attached to an individual name~$a$, which is incremented
  along the temporal dimension.  This can be expressed in \ELALC as shown in the proof of
  Theorem~\ref{thm:elalc-2exp-hard-rigid-roles}:

  \begin{align*}
    \bigwedge_{i<2n+2} \Bigl( \exists t.\bigl(\oax{Z_{0}(a)} \sqcap\ldots\sqcap \oax{Z_{i-1}(a)} \sqcap \oax{Z_{i}(a)}\bigr)
    & \ \sqsubseteq\ \oax{(\lnot Z_{i})(a)}\Bigr)\\ 
    \bigwedge_{i<2n+2}\Bigl( \exists t.\bigl(\oax{Z_{0}(a)}\sqcap\ldots\sqcap \oax{Z_{i-1}(a)} \sqcap \oax{(\lnot Z_{i})(a)}\bigr)
    & \ \sqsubseteq\ \oax{Z_{i}(a)}\Bigr)\\
    \bigwedge_{0<i<2n+2}\ \bigwedge_{j<i}\Bigl(\exists t.\bigl(\oax{(\lnot Z_{j})(a)}\sqcap\oax{Z_{i}(a)}\bigr)
    & \ \sqsubseteq\ \oax{Z_{i}(a)}\Bigr)\\
    \bigwedge_{0<i<2n+2}\ \bigwedge_{j<i}\Bigl(\exists t.\bigl(\oax{(\lnot Z_{j})(a)}\sqcap\oax{(\lnot Z_{i})(a)}\bigr)
    & \ \sqsubseteq\ \oax{(\lnot Z_{i})(a)}\Bigr)
  \end{align*}
  %
  Note that due to the first m-axiom above, we enforce that every context has a $t$-successor.  By
  the other m-axioms, we make sure that we enforce a $t$-chain of length $2^{2n+2}$.
  %
  As in~\cite{BaGL-KR08,BaGL-ToCL12}, it is not necessary to initialize the counter.  Since we
  decrement the counter along the $t$-chain (modulo $2^{2n+1}$), every value between $0$ and
  $2^{2n+1}$ is reached.

  In~\cite{BaGL-KR08,BaGL-ToCL12}, an \ALC-LTL-formula is used to express that the value of the
  counter is shared by all domain elements belonging to the current (temporal) world.  This is
  expressed using a disjunction, which we can simulate as follows:
  \begin{gather*}
    \bigwedge_{0\le i\le 2n+1} \Bigl(\oax{Z_i(a)}\sqsubseteq\oax{\top\sqsubseteq Z_i}\ \land\
    \oax{(\lnot Z_i)(a)}\sqsubseteq\oax{Z_i\sqsubseteq\bot}\Bigr)
  \end{gather*}
  %
  Next, there is a concept name~$N$, which is required to be non-empty in every (temporal) world.  We
  express this using a role name $r\in\OR$:
  \begin{gather*}
    \top\sqsubseteq\oax{(\exists r.N)(a)}
  \end{gather*}
  %
  It is only left to express the following \ALC-LTL-formula of~\cite{BaGL-KR08,BaGL-ToCL12} that
  states that every world gets one domino type:
  \begin{gather*}
    \Box\Bigl(\bigvee_{d\in D} (\top\sqsubseteq d')\Bigr)
  \end{gather*}
  For readability, let $D=\{d_1,\dots,d_k\}$.  We use non-convexity of \ALC as follows to express
  this:
  \begin{gather*}
    \top\sqsubseteq\oax{(d_1'\sqcup\dots\sqcup d_k')(a)}\ \land\ \bigwedge_{1\le i\le k}
    \Bigl(\oax{d_i'(a)}\sqsubseteq\oax{\top\sqsubseteq d_i'}\Bigr)
  \end{gather*}
  %
  The conjunction of all the \ELALC-BKBs above yields an \ELALC-BKB~\B that is consistent iff the
  exponentially bounded version of the domino problem has a solution.
\end{proof}

% \noindent
% Summing up the results of this section together with the upper bounds of
% Section~\ref{sec:complexity-consis-problem}, we obtain the following corollary.

% \begin{corollary}
%   For all \LO between \ALC and \SHOIQ, the consistency problem in \ELLO is
%   \begin{itemize}
%   \item \ExpTime-complete if $\OCR=\emptyset$, $\ORR=\emptyset$ and \LO is between \ALC and \SHOQ,
%   \item \NExpTime-complete if $\OCR=\emptyset$, $\ORR=\emptyset$ and $\LO=\SHOIQ$  \\
%     \hphantom{\NExpTime-complete} \llap{or} if $\OCR\ne\emptyset$ and $\ORR=\emptyset$,
%   \item \TwoExpTime-complete if $\ORR\ne\emptyset$ and \LO is between \ALC and
%     \SHOQ, and
%   \item \TwoExpTime-hard and in \TwoNExpTime if $\ORR\ne\emptyset$ and $\LO = \SHOIQ$.
%   \end{itemize}
% \end{corollary}



%%% Local Variables:
%%% mode: latex
%%% TeX-master: "../thesis"
%%% reftex-default-bibliography: ("../references.bib")
%%% End:

%  LocalWords:  Löwenheim Skolem Logics logics iff temporalised Quasimodel quasimodel BKB Klarman
%  LocalWords:  DL et al GreenBayPackers AaronRodgers DLs bijections equisatisfiable satisfiability
%  LocalWords:  ontologies
