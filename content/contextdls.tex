%\chapter{The Contextualized Description Logic \SHOIQSHOIQ}

\chapter{Test und jetzt wird das eine etwas laengere Ueberschrift}

\label{cha:context-dls}

\begin{definition}[Syntax and Semantics of \SHOIQSHOIQ]
  Let \MC, \MR, \MI, \OC, \OR and \OI be non-empty, pairwise disjoint sets of \emph{meta concept}, \emph{role} and
  \emph{individual names} and\emph{ object concept}, \emph{role} and \emph{individual
    names}. Furthermore, let $\Msig \coloneqq (\MC,\MR,\MI)$,  $\Osig \coloneqq (\OC,\OR,\OI)$, and
  let $\OCR\subseteq\OC$ be the set of \emph{rigid object concept names} and $\ORR\subseteq\OR$ be the set of \emph{rigid object role names}.
\end{definition}

Note that we often refer to \OCR and \ORR simply as \emph{rigid concept names} and \emph{rigid role
  names} as there is no such notion on the meta level.



%%% Local Variables:
%%% mode: latex
%%% TeX-master: "../thesis"
%%% End:
