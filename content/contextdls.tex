\chapter{The Contextualized Description Logic LMLO}
\label{cha:context-dls}

\todo[inline]{somewhere a few paragraphs about existing cdls and why they are not feasible in my
  setting.}


Some introduction

As mentioned in the introduction classical description logics lack the expressive power to capture
knowledge about \ldots

mention somewhere what \LM and \LO are.

\todo[inline]{introductory paragraphs to this chapter}


\section{Syntax and Semantics of Contextualized Description Logics}
\label{sec:syn-seman-cdl}

conceptually we start with an description logic \LM on the meta level. Here we can state our
knowledge about contexts. We enrich this with \ldots 

\todo[inline]{basic ideas how the syntax works}


Throughout this chapter, let $\Msig = (\MC, \MR, \MI)$ and $\Osig = (\OC, \OR, \OI)$ denote the
signatures for \LM and \LO. Thus, we call \MC, \MR, \MI, \OC, \OR and \OI, respectively, the set of
\emph{meta concept}, \emph{role} and \emph{individual names} and \emph{object concept}, \emph{role}
and \emph{individual names}.

\begin{definition}[Syntax of \LMLO]\label{def:syntax-cdls}
  A \emph{concept of the object logic~\DLinner (o-concept)} is an \DLinner-concept over~\Osig.  An
  \emph{o-axiom} is an \DLinner-GCI over~\Osig, an \DLinner-concept assertion over~\Osig, or an
  \DLinner-role assertion over~\Osig.

  The set of \emph{concepts of the meta logic~\DLouter (m-concepts)} is the smallest set such that
  \begin{itemize}
  \item for all $A\in\MC$, $A$ is a meta concept (\emph{\todo{here I'm struggling with names. I need
        some distinction between ``pure meta concept'' and ``referring meta concepts'' as I need the
        second name very often}???}),
  \item for all o-axioms $\alpha$, \oalpha is a meta concept (\emph{???}), and
  \item all complex concepts that can be built with the concept constructors allowed in \LM are meta
    concepts.
  \end{itemize}
  
  An \emph{m-axiom} is \todo{what exactly is an m-axiom?}

  Analogously a \emph{Boolean m-axiom formula} is defined inductively as follows:
  \begin{itemize}
  \item every m-axiom is a Boolean m-axiom formula,
  \item if $\B_1,\B_2$ are Boolean m-axiom formulas, then so are $\lnot\B_1$ and $\B_1\land\B_2$,
    and
  \item nothing else is a Boolean m-axiom formula.
  \end{itemize}
    %
  Finally, a \emph{Boolean \LMLO-knowledge base (\LMLO-BKB)} is a triple $\Bmf=(\B,\RO,\RM)$ where
  \RO is an \LO-RBox over~\Osig, \RM an \LM-RBox over~\Msig, and \B is a Boolean m-axiom formula. An
  \emph{\LMLO-ontology} is an \LMLO-BKB, where only axiom conjunction and no axiom negation is
  allowed in the Boolean m-axiom formula.
\end{definition}

For the same reasons as mentioned in section \ref{sec:description-logics}, role inclusions
over~\Osig and transitivity axioms over~\Osig are not allowed to constitute m-concepts.  However, we
fix an RBox~\RO over~\Osig that contains such o-axioms and holds in \emph{all} contexts.  The same
applies to role inclusions over~\Msig and transitivity axioms over~\Msig, which are only allowed to
occur in a RBox~\RM over~\Msig.

Again, we use the usual abbreviations (for disjunctions etc.) for m-concepts and
Boolean m-axiom formulas.

The semantics of \LMLO is defined by the notion of \emph{nested interpretations}.  These consist of
\Osig-interpretations for the specific contexts and an \Msig-interpretation for the relational
structure between them.  We assume that all contexts speak about the same non-empty domain
(\emph{constant domain assumption}). \todo{here something about constant domain assumption,
  references.}

As argued earlier, sometimes it is desired that concepts or roles in the object logic are
interpreted the same in all contexts. Therefore we introduce \emph{rigid names}. Let
$\OCR\subseteq\OC$ be the set of \emph{rigid object concept names} and $\ORR\subseteq\OR$ be the set
of \emph{rigid object role names}.
%
Often, we refer to \OCR and \ORR simply as \emph{rigid concepts} and \emph{rigid roles} as there is
no such notion on the meta level.
%
We call concept names and role names in $\OC\setminus\OCR$ and $\OR\setminus\ORR$ \emph{flexible}.
%
Moreover, we assume that individuals of the object logic are always interpreted the same in all
contexts (\emph{rigid individual assumption}). \todo{reference to RIA}

\begin{definition}[Nested interpretation]\label{def:nested-interpretation}
  A \emph{nested interpretation} is a tuple \todo{not nice!}\\ \JJ, where \Cbb is a non-empty set (called
  \emph{contexts}) and $(\Cbb,\cdot^\J)$ is an \Msig-interpretation.
  %
  Moreover, for every $c\in\Cbb$, $\I_c\coloneqq(\Delta^{\J},\cdot^{\I_c})$ is an \Osig-interpretation
  such that we have for all $c,c'\in\Cbb$ that $x^{\I_{c}}=x^{\I_{c'}}$ for every
  $x\in\OI\cup\OCR\cup\ORR$.
\end{definition}

We are now ready to define the semantics of \LMLO.

\begin{definition}[Semantics of \LMLO]
  Let \JJ be a nested interpretation.  The mapping $\cdot^\J$ is extended to o-axioms
  \todo{referenced meta concept} as follows: $\oalpha^\J:=\{c\in\Cbb\mid\I_c\models\alpha\}$.

    Moreover, \J is a model of the m-axiom $\beta$ if $(\Cbb,\cdot^\J)$ is a model
    of $\beta$.  This is extended to Boolean m-axiom formulas inductively as
    follows:
    \begin{itemize}
        \item \J is a model of $\lnot\B_1$ if it is not a model of $\B_1$, and
        \item \J is a model of $\B_1\land\B_2$ if it is a model of both $\B_1$
            and $\B_2$.
    \end{itemize}
    We write $\J\models\B$ if \J is a model of the Boolean m-axiom formula~\B.
    Furthermore, \J is a model of~\RM (written $\J\models\RM$) if
    $(\Cbb,\cdot^\J)$ is a model of~\RM, and \J is a model of~\RO (written
    $\J\models\RO$) if $\I_{c}$ is a model of~\RO for all $c\in\Cbb$.
    
    Finally, \J is a model of the \con{\DLouter}{\DLinner}-BKB \BB (written
    $\J\models\Bmf$) if \J is a model of~\B, \RO, and~\RM.  We call~\Bmf
    \emph{consistent} if it has a model.

    The \emph{consistency problem in \LMLO} is the problem of deciding whether a given
    \LMLO-BKB is consistent.
\end{definition}



\todo[inline]{some words about the next example.}

\begin{example}
  
\end{example}

\section{Complexity of the Consistency Problem}
\label{sec:complexity-consis-problem}




\begin{definition}[\mbox{\Nsig-interpretation (weakly) respects $(\Umc,\Ymc)$}]
  \label{def:int-respects-D} 
  Let $\Umc\subseteq\NC$ and let $\Ymc\subseteq\powerset{\Umc}$.  The \Nsig-interpretation \II
  \emph{respects} $(\Umc,\Ymc)$ if $\Zmc = \Ymc$ where
  \begin{align*}
    \Zmc & \coloneqq\{Y\subseteq\Umc\mid\text{there is some $d\in\Delta^\I$ with $d\in(C_{\Umc,Y})^\I$}\}
    \intertext{and}
    C_{\Umc,Y}& \coloneqq\bigsqcap_{A \in Y} A\sqcap \bigsqcap_{A\in\Umc\setminus Y} \lnot A.
  \end{align*}

    It \emph{weakly respects} $(\Umc,\Ymc)$ if $\Zmc \subseteq \Ymc$.
\end{definition}

\begin{definition}[Outer abstraction]
    Let \BB be an \con{\DLouter}{\DLinner}-BKB.  Let \bsf be the bijection
    mapping every m-concept of the form \oalpha occurring in~\B to the concept
    name $A_{\oalpha}\in\MC$, where we assume w.l.o.g.\ that $A_{\oalpha}$ does
    not occur in~\B.
    \begin{enumerate}
        \item The Boolean \DLouter-axiom formula~\Bb over \Msig is obtained
            from~\B by replacing every occurrence of an m-concept of the form
            \oalpha by $\bsf(\oalpha)$.  We call the \DLouter-BKB
            $\Bmfb=(\Bb,\RM)$ the \emph{outer abstraction of~\Bmf}.
        \item Given \JJ, its \emph{outer abstraction} is the
            \Msig-interpretation $\Jb=(\Cbb,\cdot^{\Jb})$ where
            \begin{itemize}
                \item for every $x\in\MR\cup\MI\cup(\MC\setminus\ran(\bsf))$, we
                    have $x^{\Jb}=x^\J$, and
                \item for every $A\in\ran(\bsf)$, we have
                    $A^{\Jb}=(\bsf^{-1}(A))^\J$,
            \end{itemize}
            where $\ran(\bsf)$ denotes the image of~\bsf. \qedhere
    \end{enumerate}
\end{definition}


\blindtext

\begin{definition}[Admissibility]\label{def:admissibility}
  Let $\X=\{X_1,~\dots,\ X_k\}\subseteq\powerset{\ran(\bsf)}$.  We call \X \emph{admissible} if
  there exist \Osig-interpretations $\I_1=(\Delta,\cdot^{\I_1})$,~\dots,
  $\I_k=(\Delta,\cdot^{\I_k})$ such that
  \begin{itemize}
  \item $x^{\I_i}=x^{\I_j}$ for all $x\in\OI\cup\OCR\cup\ORR$ and all $i,j\in\{1,\dots,k\}$, and
  \item every $\I_i$, $1\le i\le k$, is a model of the \DLinner-BKB $\Bmf_{X_{i}}= (\B_{X_i},\RO)$
    over~\Osig where
    \begin{align*}
      \B_{X_i}:=\bigwedge_{\bsf(\oalpha)\in X_i}\alpha\ \land
      \bigwedge_{\bsf(\oalpha)\in\ran(\bsf)\setminus X_i}\lnot\alpha.
    \end{align*}
  \end{itemize}
  \vspace{-\baselineskip}
\end{definition}


\blindtext

\begin{definition}[Outer consistency]\label{def:outer-sat}
  Let $\X\subseteq\powerset{\ran(\bsf)}$.  We call the \DLouter-BKB~\Bmfb over \Msig
  \emph{outer consistent w.r.t.~\X{}} if there exists a model of~\Bmfb that weakly
  respects~$(\ran(\bsf),\X)$.
\end{definition}

\blindtext

\begin{lemma}\label{lem:admissible-and-outerConsistent}
    The \con{\DLouter}{\DLinner}-BKB~\Bmf is consistent iff there is a set
    $\X=\{X_1,\dots,X_k\}\subseteq\powerset{\ran(\bsf)}$ such that
    \begin{enumerate}
        \item \X is admissible, and
        \item \Bmfb is outer consistent w.r.t.~\X.
    \end{enumerate}
\end{lemma}
\missingproof

%%% Local Variables:
%%% mode: latex
%%% TeX-master: "../thesis"
%%% End:
