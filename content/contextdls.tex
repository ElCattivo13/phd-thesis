\chapter{The Contextualized Description Logic LMLO}
\label{cha:context-dls}

\todo[inline]{somewhere a few paragraphs about existing cdls and why they are not feasible in my
  setting.}


Some introduction

As mentioned in the introduction classical description logics lack the expressive power to capture
knowledge about \ldots

mention somewhere what \LM and \LO are.

\todo[inline]{introductory paragraphs to this chapter}


\section{Syntax and Semantics of Contextualized Description Logics}
\label{sec:syn-seman-cdl}

conceptually we start with an description logic \LM on the meta level. Here we can state our
knowledge about contexts. We enrich this with \ldots 

\todo[inline]{basic ideas how the syntax works}


Throughout this chapter, let $\Msig = (\MC, \MR, \MI)$ and $\Osig = (\OC, \OR, \OI)$ denote the
signatures for \LM and \LO. Thus, we call \MC, \MR, \MI, \OC, \OR and \OI, respectively, the set of
\emph{meta concept}, \emph{role} and \emph{individual names} and \emph{object concept}, \emph{role}
and \emph{individual names}.

\begin{definition}[Syntax of \LMLO]\label{def:syntax-cdls}
  A \emph{concept of the object logic~\LO (o-concept)} is an \LO-concept over~\Osig.  An
  \emph{o-axiom} is an \LO-GCI over~\Osig, an \LO-concept assertion over~\Osig, or an
  \LO-role assertion over~\Osig.

  The set of \emph{concepts of the meta logic~\LM (m-concepts)} is the smallest set such that
  \begin{itemize}
  \item for all $A\in\MC$, $A$ is a meta concept (\emph{\todo{here I'm struggling with names. I need
        some distinction between ``pure meta concept'' and ``referring meta concepts'' as I need the
        second name very often}???}),
  \item for all o-axioms $\alpha$, \oalpha is a meta concept (\emph{???}), and
  \item all complex concepts that can be built with the concept constructors allowed in \LM are meta
    concepts.
  \end{itemize}
  
  An \emph{m-axiom} is \todo{what exactly is an m-axiom?}

  Analogously a \emph{Boolean m-axiom formula} is defined inductively as follows:
  \begin{itemize}
  \item every m-axiom is a Boolean m-axiom formula,
  \item if $\B_1,\B_2$ are Boolean m-axiom formulas, then so are $\lnot\B_1$ and $\B_1\land\B_2$,
    and
  \item nothing else is a Boolean m-axiom formula.
  \end{itemize}
    %
  Finally, a \emph{Boolean \LMLO-knowledge base (\LMLO-BKB)} is a triple $\Bmf=(\B,\RO,\RM)$ where
  \RO is an \LO-RBox over~\Osig, \RM an \LM-RBox over~\Msig, and \B is a Boolean m-axiom formula. An
  \emph{\LMLO-ontology} is an \LMLO-BKB, where only axiom conjunction and no axiom negation is
  allowed in the Boolean m-axiom formula.
\end{definition}

For the same reasons as mentioned in section \ref{sec:description-logics}, role inclusions
over~\Osig and transitivity axioms over~\Osig are not allowed to constitute m-concepts.  However, we
fix an RBox~\RO over~\Osig that contains such o-axioms and holds in \emph{all} contexts.  The same
applies to role inclusions over~\Msig and transitivity axioms over~\Msig, which are only allowed to
occur in a RBox~\RM over~\Msig.

Again, we use the usual abbreviations (for disjunctions etc.) for m-concepts and
Boolean m-axiom formulas.

The semantics of \LMLO is defined by the notion of \emph{nested interpretations}.  These consist of
\Osig-interpretations for the specific contexts and an \Msig-interpretation for the relational
structure between them.  We assume that all contexts speak about the same non-empty domain
(\emph{constant domain assumption}). \todo{here something about constant domain assumption,
  references.}

As argued earlier, sometimes it is desired that concepts or roles in the object logic are
interpreted the same in all contexts. Therefore we introduce \emph{rigid names}. Let
$\OCR\subseteq\OC$ be the set of \emph{rigid object concept names} and $\ORR\subseteq\OR$ be the set
of \emph{rigid object role names}.
%
Often, we refer to \OCR and \ORR simply as \emph{rigid concepts} and \emph{rigid roles} as there is
no such notion on the meta level.
%
We call concept names and role names in $\OC\setminus\OCR$ and $\OR\setminus\ORR$ \emph{flexible}.
%
Moreover, we assume that individuals of the object logic are always interpreted the same in all
contexts (\emph{rigid individual assumption}). \todo{reference to RIA}

\begin{definition}[Nested interpretation]\label{def:nested-interpretation}
  A \emph{nested interpretation} is a tuple \todo{not nice!}\\ \JJ, where \Cbb is a non-empty set (called
  \emph{contexts}) and $(\Cbb,\cdot^\J)$ is an \Msig-interpretation.
  %
  Moreover, for every $c\in\Cbb$, $\I_c\coloneqq(\Delta^{\J},\cdot^{\I_c})$ is an \Osig-interpretation
  such that we have for all $c,c'\in\Cbb$ that $x^{\I_{c}}=x^{\I_{c'}}$ for every
  $x\in\OI\cup\OCR\cup\ORR$.
\end{definition}

We are now ready to define the semantics of \LMLO.

\begin{definition}[Semantics of \LMLO]
  Let \JJ be a nested interpretation.  The mapping $\cdot^\J$ is extended to o-axioms
  \todo{referenced meta concept} as follows: $\oalpha^\J:=\{c\in\Cbb\mid\I_c\models\alpha\}$.

    Moreover, \J is a model of the m-axiom $\beta$ if $(\Cbb,\cdot^\J)$ is a model
    of $\beta$.  This is extended to Boolean m-axiom formulas inductively as
    follows:
    \begin{itemize}
        \item \J is a model of $\lnot\B_1$ if it is not a model of $\B_1$, and
        \item \J is a model of $\B_1\land\B_2$ if it is a model of both $\B_1$
            and $\B_2$.
    \end{itemize}
    We write $\J\models\B$ if \J is a model of the Boolean m-axiom formula~\B.
    Furthermore, \J is a model of~\RM (written $\J\models\RM$) if
    $(\Cbb,\cdot^\J)$ is a model of~\RM, and \J is a model of~\RO (written
    $\J\models\RO$) if $\I_{c}$ is a model of~\RO for all $c\in\Cbb$.
    
    Finally, \J is a model of the \LMLO-BKB \BB (written
    $\J\models\Bmf$) if \J is a model of~\B, \RO, and~\RM.  We call~\Bmf
    \emph{consistent} if it has a model.

    The \emph{consistency problem in \LMLO} is the problem of deciding whether a given
    \LMLO-BKB is consistent.
\end{definition}



\todo[inline]{some words about the next example.}

\begin{example}
  
\end{example}

\section{Complexity of the Consistency Problem}
\label{sec:complexity-consis-problem}

\todo[inline]{somewhere a few words about top down view, can't go up, because of that its possible
  to divide reasoning problem into two subtasks, later on }

\todo[inline]{results later}

\todo[inline]{paragraph that we will handle lower bounds in 3.3 where we talk about EL}

For the upper bounds, let in the following \BB be an \LMLO-BKB.  We proceed
similar to what was done for \ALC-LTL in~\cite{BaGL-KR08,BaGL-ToCL12} (and
\SHOQ-LTL in~\cite{Lip-PhD14}) and reduce the consistency problem to two separate
decision problems.

For the first problem, we consider the so-called \emph{outer abstraction}, which is the \LM-BKB
over~\Msig obtained by replacing each \todo{referring m-concept?} m-concept of the form \oalpha
occurring in~\B by a fresh concept name such that there is a 1--1 relationship between them.

\begin{definition}[Outer abstraction]
  Let \BB be an \LMLO-BKB.  Let \bsf be the bijection mapping every m-concept of the form \oalpha
  occurring in~\B to the concept name $A_{\oalpha}\in\MC$, where we assume w.l.o.g.\ that
  $A_{\oalpha}$ does not occur in~\B.
  \begin{enumerate}
  \item The \LM-concept $C^{\bsf}$ over \Msig is obtained from the m-concept~$C$ by replacing every occurrence of
    \oalpha by $\bsf(\oalpha)$.
  \item The Boolean \LM-axiom formula~\Bb over \Msig is obtained from~\B by replacing every
    m-concept $C$ occurring in~\B with~$C^{\bsf}$.  We call the \LM-BKB $\Bmfb=(\Bb,\RM)$ the
    \emph{outer abstraction of~\Bmf}.
        \item Given \JJ, its \emph{outer abstraction} is the
            \Msig-interpretation $\Jb=(\Cbb,\cdot^{\Jb})$ where
            \begin{itemize}
                \item for every $x\in\MR\cup\MI\cup(\MC\setminus\ran(\bsf))$, we
                    have $x^{\Jb}=x^\J$, and
                \item for every $A\in\ran(\bsf)$, we have
                    $A^{\Jb}=(\bsf^{-1}(A))^\J$,
            \end{itemize}
            where $\ran(\bsf)$ denotes the image of~\bsf. \qedhere
    \end{enumerate}
\end{definition}

For simplicity, for $\Bmf'=(\B',\RO,\RM)$ where $\B'$ is a subformula of~\B, we
denote by $(\Bmf')^\bsf$ the outer abstraction of~$\Bmf'$ that is obtained by
restricting \bsf to the m-concepts occurring in~$\B'$.
%
Now let us consider the following small example.


\begin{example}\label{ex:outer-abstraction}
  Let $\Bmf_{\text{ex}} = (\B_{\text{ex}},\emptyset,\emptyset)$ with $\B_{\text{ex}}\coloneqq
  C\sqsubseteq(\oax{A\sqsubseteq\bot})\ \land\ (C\sqcap\oax{A(a)})(c)$ be an \ALCALC-BKB.  Then,
  \bsf maps $\oax{A\sqsubseteq\bot}$ to $A_{\oax{A\sqsubseteq\bot}}$ and $\oax{A(a)}$ to
  $A_{\oax{A(a)}}$.  Thus, we have that
  \begin{align*}
    \Bmf_{\text{ex}}^\bsf\coloneqq \Big(C\sqsubseteq(A_{\oax{A\sqsubseteq\bot}})\ \land\ (C\sqcap
    A_{\oax{A(a)}})(c),\ \emptyset\Big)
  \end{align*}
  is the outer abstraction of $\Bmf_\text{ex}$.
\end{example}

The following lemma makes the relationship between \Bmf and its outer abstraction
\Bmfb explicit.  It is proved by induction on the structure of~\B.

\begin{lemma}\label{lem:interpretation-outer-abstraction}
  Let \J be a nested interpretation such that \J is a model of \RO.  Then, \J is a model of \Bmf iff
  $\Jb$ is a model of~\Bmfb.
\end{lemma}

\begin{proof}
  Since $r^{\J}=r^{\Jb}$ for all \LM-role $r$ over \Msig, we have that \J is a model of \RM iff \Jb is a model of
  \RM. Thus, it is only left to show that for any m-axiom $\gamma$ occurring in \B, it holds that
  $\J \models \gamma$ iff $\Jb \models \gamma^{\bsf}$.

  \begin{claim}
    For any $x \in \Cbb$ it holds that $x \in C^{\J}$ iff
    $x \in (C^{\bsf})^{\Jb}$.
  \end{claim}

  \begin{claimproof}
    We prove the claim by induction on the structure of $C$: \todo{müssen hier noch inverse rollen
      betrachtet werden? eigentlich nicht.}

    \begin{tabularx}{\linewidth}{@{}l@{ }X@{}}
      $C = A \in \MC\!\setminus\!\ran(\bsf)$: 
      & $x \in A^{\J}$ 
        iff $x \in (A^{\bsf})^{\Jb}$ by definition of $\Jb$ and since $A = A^{\bsf}$ 
      \\[1ex]
      $C = \oalpha$:
      & $x \in \oalpha^{\J}$
        iff $x \in (A_{\oalpha})^{\Jb}$
        iff $x \in (\oalpha^{\bsf})^{\Jb}$
      \\[1ex] 
      $C = \lnot D$:
      & $x \in (\lnot D)^{\J}$ 
        iff $x \notin D^{\J}$ 
        iff, by induction hypothesis, $x \notin (D^{\bsf})^{\Jb}$ 
        iff $x \in (\lnot D^{\bsf})^{\Jb}$ 
        iff $x \in ((\lnot D)^{\bsf})^{\Jb}$ 
      \\[1ex]
      $C = D \sqcap E$: 
      & $x \in (D \sqcap E)^{\J}$
        iff $x \in D^{\J}$ and $x \in E^{\J}$ 
        iff, by induction hypothesis, $x \in (D^{\bsf})^{\Jb}$ and $x \in
        (E^{\bsf})^{\Jb}$
        iff $x \in (D^{\bsf} \sqcap E^{\bsf})^{\Jb}$
        iff $x \in ((D \sqcap E)^{\bsf})^{\Jb}$ 
      \\[1ex]
      $C = \exists r.D$: 
      & $x \in (\exists r.D)^{\J}$
        iff there exists $y \in \Cbb$ \suth $(x,y) \in r^{\J}$ and $y \in D^{\J}$
        iff there exists $y \in \Cbb$ \suth $(x,y) \in r^{\Jb}$ and $y \in (D^{\bsf})^{\Jb}$
        iff $x \in (\exists r.D^{\bsf})^{\Jb}$ 
        iff $x \in ((\exists r.D)^{\bsf})^{\Jb}$ 
      \\[1ex]
      $C = \{a\}$:
      & $x\in\{a\}^{\J}$ 
        iff $x\in(\{a\}^{\bsf})^{\Jb}$ by definition of $\Jb$ and since $\{a\} = \{a\}^{\bsf}$ 
      \\[1ex]
      $C =\ \atleast{n}{r}{D}$:
      & $x \in (\atleast{n}{r}{D})^{\J}$
        iff there are at least $n$ elements $y \in \Cbb$ s.t.\ $(x,y) \in r^{\J}$ and $y \in D^{\J}$
        iff there are at least $n$ elements $y \in \Cbb$ s.t.\ $(x,y) \in r^{\Jb}$ and $y \in (D^{\bsf})^{\Jb}$
        iff $x \in (\atleast{n}{r}{D^{\bsf}})^{\Jb}$
        iff $x \in ((\atleast{n}{r}{D})^{\bsf})^{\Jb}$ 
    \end{tabularx}

    \vspace{-2.0\baselineskip}
  \end{claimproof}

  If $\gamma$ is of the form $C \sqsubseteq D$, we have that $\J \models C
  \sqsubseteq D$ iff $x \in C^{\J}$ implies $x \in D^{\J}$
  iff (by claim) $x
  \in (C^{\bsf})^{\Jb}$ implies $x \in (D^{\bsf})^{\Jb}$ iff
  $\Jb \models C^{\bsf} \sqsubseteq D^{\bsf}$.

  If $\gamma$ is of the form $C(a)$, we have that $\J \models C(a)$ iff
  $a^{\J} \in C^{\J}$ iff (by claim) $a^{\Jb} \in
  (C^{\bsf})^{\Jb}$ iff $\Jb \models C^{\bsf}(a)$.

  If $\gamma$ is of the form $r(a,b)$, we have that $\J \models r(a,b)$ iff
  $(a^{\J}, b^{\J}) \in r^{\J}$ iff $(a^{\Jb},
  b^{\Jb}) \in r^{\Jb}$ iff $\Jb \models
  r(a,b)$.

  If \B is of the form $\lnot\B_{1}$, we have that $\J\models\B$ iff not
  $\J\models\B_{1}$ iff not $\Jb\models\Bb_1$ iff $\Jb\models\Bb$.

  If \B is of the form $\B_{1}\land\B_{2}$, we have that $\J\models\B$ iff $\J\models\B_{1}$ and
  $\B_{2}$ iff $\Jb\models\Bb_{1}$ and $\Jb\models\Bb_{2}$ iff $\Jb\models\Bb$.

  Since $\J\models\RO$, $\J\models\RM$ iff $\Jb\models\RM$ and $\J\models\B$ iff $\Jb\models\Bb$, we
  have $\J\models\Bmf$ iff $\Jb\models\Bmfb$.
\end{proof}

Note that this lemma yields that consistency of~\Bmf implies consistency of~\Bmfb.  Thus, the
consistency of~\Bmfb is a necessary condition for the consistency of~\Bmf.  However, it is not
sufficient since the converse does not hold as the following example shows.

\begin{example}\label{ex:outer-abstraction-continued}
  Consider again $\Bmf_\text{ex}$ of Example~\ref{ex:outer-abstraction}.
  %
  Take any \Msig-interpretation $\Hmc=(\Delta^{\Hmc},\cdot^\Hmc)$ with $\Delta^{\Hmc}=\{e\}$,
  $d^\Hmc=e$, and $C^\Hmc = A_{\oax{A\sqsubseteq\bot}}^\Hmc = A_{\oax{A(a)}}^\Hmc = \{e\}$.

  Clearly, \Hmc is a model of~$\Bmf_\text{ex}^{\bsf}$.  But there is no nested interpretation~\JJ
  with $\J\models\Bmf_\text{ex}$ since this would imply $\Cbb=\Delta^{\Hmc}$, and that $\I_e$ is a model of
  both $A\sqsubseteq\bot$ and $A(a)$, which is not possible.
\end{example}

The above example illustrates that there exist implicit restrictions on the interpretation of the
meta level as certain combinations of concept names in $\ran(\bsf)$ are not allowed.  Therefore, we
need to ensure that these are not treated independently.  For expressing such a restriction on the
model~\Hmc of~\Bmfb, we adapt a notion of~\cite{BaGL-KR08,BaGL-ToCL12}. It is also worth noting that
this problem occurs also in much less expressive DLs such as \ELbot (i.e.~\EL extended with the
bottom concept).

\begin{definition}[\mbox{\Nsig-interpretation (weakly) respects $(\Umc,\Ymc)$}]
  \label{def:int-respects-D} 
  Let $\Umc\subseteq\NC$, let $\Ymc\subseteq\powerset{\Umc}$ and let \II be an \Nsig-interpretation.
  The \emph{type of $d\in\Delta^{\I}$ w.r.t.~\Umc} is defined as
  $\mathsf{type}_{\Umc}(d) \coloneqq \{A \in \Umc \mid d \in A^{\I}\}$.  The interpretation \I
  \emph{respects} $(\Umc,\Ymc)$ if $\Zmc = \Ymc$ where
  \begin{align*}
    \Zmc & \coloneqq\{Y\subseteq\Umc\mid\text{there is some $d\in\Delta^\I$ with
           $\mathsf{type}_{\Umc}(d) = Y$}\}
  \end{align*}

    It \emph{weakly respects} $(\Umc,\Ymc)$ if $\Zmc \subseteq \Ymc$.
\end{definition}

\todo[inline]{a few words about the meaning of the definition, explanation. Types usual definition}


The second decision problem that we use for deciding consistency is needed to make sure that such a
set of concept names is admissible in the following sense.

\begin{definition}[Admissibility]\label{def:admissibility}
  Let $\Xmc=\{X_1,~\dots,\ X_k\}\subseteq\powerset{\ran(\bsf)}$.  We call \Xmc \emph{admissible} if
  there exist \Osig-interpretations $\I_1=(\Delta,\cdot^{\I_1})$,~\dots,
  $\I_k=(\Delta,\cdot^{\I_k})$ such that
  \begin{itemize}
  \item $x^{\I_i}=x^{\I_j}$ for all $x\in\OI\cup\OCR\cup\ORR$ and all $i,j\in\{1,\dots,k\}$, and
  \item every $\I_i$, $1\le i\le k$, is a model of the \LO-BKB $\Bmf_{X_{i}}= (\B_{X_i},\RO)$
    over~\Osig where
    \begin{align*}
      \B_{X_i}:=\bigwedge_{\bsf(\oalpha)\in X_i}\alpha\ \land
      \bigwedge_{\bsf(\oalpha)\in\ran(\bsf)\setminus X_i}\lnot\alpha.
    \end{align*}
  \end{itemize}
  \vspace{-1.7\baselineskip}
\end{definition}

Note that any subset $\Xmc'\subseteq\Xmc$ is admissible if \Xmc is admissible.
%
Intuitively, the sets $X_i$ in an admissible set \Xmc consist of concept names \todo{referring concept
  names} such that the corresponding o-axioms \enquote{fit together}.  Consider again
Example~\ref{ex:outer-abstraction-continued}.  Clearly, the set
$\{A_{\oax{A\sqsubseteq\bot}},A_{\oax{A(a)}}\}\in\powerset{\ran(\bsf)}$ \emph{cannot} be contained
in any admissible set~\Xmc.  

The next definition captures the above mentioned restriction on the model~\Hmc
of~\Bmfb.

\todo[inline]{explanation of outer consistency}

\begin{definition}[Outer consistency]\label{def:outer-consistency}
  Let $\Xmc \subseteq \powerset{\ran(\bsf)}$.  We call the \LM-BKB~\Bmfb over \Msig \emph{outer
    consistent w.r.t.~\Xmc} if there exists a model of~\Bmfb that weakly respects~$(\ran(\bsf),\Xmc)$.
\end{definition}

The next two lemmas show that the consistency problem in \LMLO can be decided by checking whether
there is an admissible set~\Xmc and the outer abstraction of the given \LMLO-BKB is outer consistent
w.r.t.~\Xmc.

\begin{lemma}\label{lem:model-equivalent-to-admissible}
  For every \Msig-interpretation \HH, the following two statements are equivalent:
  \begin{enumerate}
  \item There exists a model~\J of~\Bmf with $\Jb=\Hmc$.
  \item \Hmc is a model of~\Bmfb and the set $\{X_d\mid d\in\Gamma\}$ is admissible, where $X_d$ is
    defined as $X_{d}:=\{A\in\ran(\bsf)\mid d\in A^\Hmc\}$.
  \end{enumerate}
\end{lemma}

\begin{proof}
  (1 $\Rightarrow$ 2): Let \JJ be a model of~\Bmf with $\Jb=\Hmc$.  Since $\Jb=\Hmc$, we have that
  $\Cbb=\Delta^{\Hmc}$.  By Lemma~\ref{lem:interpretation-outer-abstraction}, we have that \Hmc is a
  model of~\Bmfb.
    %
  Moreover, since \bsf is a bijection between m-concepts of the form \oalpha occurring in~\Bmf and
  concept names of~\MC, we have that $\ran(\bsf)$ is finite, and thus also the set
  $\Xmc \coloneqq \{X_d\mid d \in \Delta^{\Hmc} \} \subseteq \powerset{\ran(\bsf)}$ is finite.  Let
  $\Xmc = \{Y_1, \dots, Y_k\}$.  Since $\Cbb = \Delta^{\Hmc}$, there exists an index function
  $\nu\colon\Cbb\to\{1,\dots,k\}$ such that $X_c = Y_{\nu(c)}$ for every $c\in\Cbb$, i.e.
  \begin{align*}
    Y_{\nu(c)} & = \bigl\{\bsf(\oalpha)\mid\text{\oalpha occurs in~\Bmf and}\
                 c\in\oalpha^\Hmc\bigr\} \\
               & =  \bigl\{\bsf(\oalpha)\mid\text{\oalpha occurs in~\Bmf and}\ \I_c\models\alpha\bigr\}.
  \end{align*}
  Conversely, for every $\mu\in\{1,\dots,k\}$, there is an element $c\in\Cbb$ such that
  $\nu(c)=\mu$.
    % 
  The \Osig-interpretations for showing admissibility of~\Xmc are obtained as follows.  Take
  $c_1,\dots,c_k \in \Cbb$ such that $\nu(c_1) = 1$,~\dots, $\nu(c_k) = k$.  Now, for every~$i$,
  $1 \leq i \leq k$, we define the \Osig-interpretation $\Gmc_i:=(\Delta,\cdot^{\I_{c_i}})$.
  Clearly, we have that $Gmc_i\models\B_{Y_i}$ and since $\J\models\RO$, we have that
  $Gmc_i\models\Bmf_{Y_i}$.  Moreover, the definition of a nested interpretation yields that
  $x^{\Gmc_i}=x^{\Gmc_j}$ for all $x\in\OI\cup\OCR\cup\ORR$ and all $i,j \in \{1,\dots,k\}$.  Hence,
  the \Osig-interpretations $\Gmc_1, \dots, \Gmc_k$ attest admissibility of~\Xmc.

  (2 $\Rightarrow$ 1): Assume that \HH is a model of~\Bmfb and that the set
  $\Xmc \coloneqq \{X_d\mid d\in\Gamma\}$ is admissible.  Again, since $\ran(\bsf)$ is finite, we
  have that $\Xmc \subseteq \powerset{\ran(\bsf)}$ is finite.  Let $\Xmc = \{Y_1,\dots,Y_k\}$.
  Since \Xmc is admissible, there are \Osig-interpretations $\Gmc_1=(\Delta,\cdot^{\Gmc_1})$,~\dots,
  $\Gmc_k=(\Delta,\cdot^{\Gmc_k})$ such that $\Gmc_i\models\Bmf_{Y_i}$ and $x^{\Gmc_i}=x^{\Gmc_j}$
  for all $x\in\OI\cup\OCR\cup\ORR$ and all $i,j\in\{1,\dots,k\}$.
    %
  Furthermore, there exists an index function $\nu\colon\Gamma\to\{1,\dots,k\}$ such that
  $Y_{\nu(d)}=X_d$ for every $d\in\Gamma$.
    %
  We define a nested interpretation \JJ as follows:
  \begin{itemize}
  \item $\Cbb \coloneqq \Delta^{\Hmc}$;
  \item $x^\J \coloneqq x^\Hmc$ for every $x\in\MC\cup\MR\cup\MI$; and
  \item $x^{\I_c}:=x^{\Gmc_{\nu(c)}}$ for every $x \in \OC \cup \OR \cup \OI$ and every $c \in \Cbb$.
  \end{itemize}
    %
  By construction of \J, we have that $x^{\Jb} = x^\Hmc$ for every
  $x \in \MR \cup \MI \cup (\MC \setminus \ran(\bsf))$.
    %
  Let $A \in \ran(\bsf)$, and let $\bsf^{-1}(A) = \oalpha$.  We have for every $d \in \Gamma = \Cbb$
  that $d\in A^{\Jb}$ iff $d\in(\bsf^{-1}(A))^\J$ iff $d \in \oalpha^\J$ iff $\I_d \models \alpha$
  iff $\Gmc_{\nu(d)}\models\alpha$ iff $\bsf(\oalpha)=A\in Y_{\nu(d)}$ (since
  $\Gmc_{\nu(d)}\models\B_{Y_{\nu(d)}}$) iff $A\in X_d$ iff $d\in A^\Hmc$.
    %
  Hence, we have $\Jb=\Hmc$.
    %
  Since \Hmc is a model of~\Bmfb and, by construction of \J, \J is a model of \RO, we have by
  Lemma~\ref{lem:interpretation-outer-abstraction} that \J is a model of~\Bmf.

  \todo[inline]{noch nicht drüber gelesen, hier müssen sicher noch einige Formelzeichen angepasst
    werden. Und ist er verständlich???}
\end{proof}

The following lemma is a consequence of the previous one.

\todo[inline]{a few more words? :/}

\begin{lemma}\label{lem:admissible-and-outerConsistent}%
  The \LMLO-BKB~\Bmf is consistent iff there is a set
  $\Xmc\subseteq\powerset{\ran(\bsf)}$ such that
  \begin{enumerate}
  \item \Xmc is admissible, and
  \item \Bmfb is outer consistent w.r.t.~\Xmc.
  \end{enumerate}
\end{lemma}

\begin{proof}
  \onlyifdirection Let \J be a model of \Bmf, and let $\Jb=(\Cbb,\cdot^{\Jb})$.  By
  Lemma~\ref{lem:model-equivalent-to-admissible}, we have that $\Jb$ is a model of~\Bmfb, and the
  set $\Xmc \coloneqq \{X_c \mid c \in \Cbb\}$ is admissible.  By construction, $\Jb$ weakly
  respects $(\ran(\bsf),\Xmc)$, and hence \Bmfb is outer consistent w.r.t.~\Xmc.
    
  \ifdirection Let $\Xmc = \{X_1,\dots,X_k\}\subseteq\powerset{\ran(b)}$ such that \Xmc is
  admissible and \Bmfb is outer consistent w.r.t.~\Xmc.  Hence there is a model
  $\Gmc=(\Cbb,\cdot^\Gmc)$ of~\Bmfb that weakly respects $(\ran(\bsf),\Xmc)$.
    %
  We define $\Xmc' \coloneqq \{Y_c \mid c\in\Cbb\}$, where
  $Y_c \coloneqq \{A \in \ran(b) \mid c \in A^\Gmc\}$.  Since \Gmc weakly respects
  $(\ran(\bsf),\Xmc)$ and $c \in (C_{\ran(b),Y_c})^\Gmc$ for every $c \in \Cbb$, we have that
  $\Xmc' \subseteq \Xmc$.  Since \Xmc is admissible, this yields admissibility of~$\Xmc'$.
  Lemma~\ref{lem:model-equivalent-to-admissible} yields now consistency of~\Bmf.
  %
  \todo[inline]{noch nicht drübergelesen!}
\end{proof}


\todo[inline]{some text}

\begin{lemma}\label{lem:shoq-outer-consisteny-exptime}
  Deciding whether a \cSHOQ-BKB \Bmfb is outer consistent w.r.t.~\Xmc can be done in time
  exponential in the size of~\Bmfb and linear in size of~\Xmc.
\end{lemma}

\begin{proof}
  It is enough to show that deciding whether~\Bmfb has a model that weakly respects
  $(\ran(\bsf),\Xmc)$ can be done in time exponential in the size of~\Bmfb and linear in the size
  of~\Xmc.  It is not hard to see that we can adapt the notion of a quasimodel respecting a pair
  $(\Umc,\Ymc)$ of~\cite{Lip-PhD14} to a quasimodel \emph{weakly} respecting $(\Umc,\Ymc)$.  Indeed,
  one just has to drop Condition~(i) in Definition~3.25 of~\cite{Lip-PhD14}.  Then, the proof of
  Lemma~3.26 there can be adapted such that our claim follows.  This is done by dropping one check
  in Step~4 of the algorithm of~\cite{Lip-PhD14}.

    \todo[inline]{complete rewrite necessary}
\end{proof}

\begin{lemma}\label{lem:shoiq-outer-consisteny-exptime}
  Deciding whether a \cSHOIQ-BKB \Bmfb is outer consistent w.r.t.~\Xmc can be non-deterministically
  done in time exponential in the size of~\Bmfb and linear in size of~\Xmc.
\end{lemma}

\missingproof


\subsection{Consistency without rigid names}
\label{sec:cons-without-rigid}
In this section, we consider the case where no rigid concept names or role names
are allowed. So we fix $\OCR=\ORR=\emptyset$.

\begin{lemma}\label{lem:shoqshoq-without-rigid-exptime}
  The consistency problem in \SHOQSHOQ is in \ExpTime if $\OCR=\ORR=\emptyset$.
\end{lemma}

\begin{proof}
  Let \Bmf be a \SHOQSHOQ-BKB and \Bmfb its outer abstraction.  We can decide consistency of~\Bmf
  using Lemma~\ref{lem:admissible-and-outerConsistent}.  We define
  $\Xmc \coloneqq \{ X \subseteq \ran(\bsf) \mid \Bmf_{X}\ \text{is consistent}\}$ where $\Bmf_{X}$
  is defined as in Definition~\ref{def:admissibility}.
  %
  We first show that $\Xmc = \{X_1, \dots, X_k\}$ is admissible.  Let $\I_i$ be a model
  of~$\Bmf_{X_i}$, which exists since $\Bmf_{X_i}$ is consistent.  Due to the Löwenheim-Skolem
  theorem, we can assume that all models $\I_i$, $1 \leq i \leq k$, have a countably infinite
  domain. Thus, w.l.o.g.\ we can assume that all models have the same domain~$\Delta$.  Furthermore,
  we can assume that individual names are interpreted the same.  Since $\OCR=\ORR=\emptyset$, the
  set~\Xmc fulfills all conditions of Definition~\ref{def:admissibility} for admissibility.

  Thus, if \Bmfb is outer consistent w.r.t.~\Xmc, then we have by
  Lemma~\ref{lem:admissible-and-outerConsistent} that \Bmf is consistent.
  %
  Conversely, assume that \Bmf is consistent.  Then, by
  Lemma~\ref{lem:admissible-and-outerConsistent}, there is an admissible set
  $\Xmc' \subseteq \powerset{\ran(\bsf)}$ and \Bmfb is outer consistent w.r.t.~$\Xmc'$.  Since \Xmc
  is the maximal admissible subset of $\powerset{\ran(\bsf)}$, we have $\Xmc' \subseteq \Xmc$.  If
  \Bmfb is outer consistent w.r.t.~$\Xmc'$, it is also outer consistent w.r.t.~\Xmc.  Hence, \Bmf is
  consistent iff \Bmfb is outer consistent w.r.t.~\Xmc, which yields a decision procedure for the
  consistency problem in \SHOQSHOQ.

  It remains to analyze the complexity.  There are exponentially many
  $Xmc \in \powerset{\ran(\bsf)}$, but each \SHOQ-BKB~$\Bmf_{X}$ can be constructed in time
  polynomial in the size of~\Bmf.  We can decide consistency of~$\Bmf_{X}$ in time
  exponential~\cite{Lip-PhD14}.  Thus, the set~\Xmc can be constructed in time exponential in the size
  of~\Bmf and it is of exponential size.  Due to Lemma \ref{lem:shoq-outer-consisteny-exptime},
  deciding whether \Bmfb is outer consistent w.r.t.~\Xmc can be done in time exponential in the size
  of~\Bmfb and linear in the size of~\Xmc.  Thus, overall we can decide the consistency problem in
  exponential time.
\end{proof}

Together with the lower bounds shown in Section~\ref{sec:case-el}, we obtain \ExpTime-completeness
for the consistency problem in \LMLO for \LM and \LO being DLs between \ALC and \SHOQ if
$\OCR=\ORR=\emptyset$.

\begin{lemma}\label{lem:shoiqshoiq-without-rigid-exptime}
  The consistency problem in \SHOIQSHOIQ is in \NExpTime if $\OCR=\ORR=\emptyset$.
\end{lemma}
\missingproof

\subsection{Consistency with rigid concept and role names}
\label{sec:cons-with-rigid}




\subsection{Consistency with only rigid concept names}
\label{sec:cons-with-only}



\section{The Case of \EL}
\label{sec:case-el}

For the sake of completeness 

\subsection{The Contextualized Description Logics LMEL}
\label{sec:con-dl-lm-el}

\begin{theorem}
  The consistency problem in \condl{\ALC}{\EL} is \ExpTime-hard if $\OCR = \ORR = \emptyset$.
\end{theorem}
\missingproof

\begin{theorem}
  The consistency problem in \condl{\ALC}{\EL} is \NExpTime-hard if $\OCR \neq \emptyset$ and
  $\ORR = \emptyset$.
\end{theorem}
\missingproof

\begin{theorem}
  The consistency problem in \condl{\SHOQ}{\EL} is in \NExpTime if $\OCR \neq \emptyset$ and
  $\ORR = \emptyset$.
\end{theorem}
\missingproof

\subsection{The Contextualized Description Logic ELLO}
\label{sec:con-dl-el-lo}

\begin{theorem}
  The consistency problem in \condl{\EL}{\ALC} is \ExpTime-hard if $\OCR = \ORR = \emptyset$.
\end{theorem}
\missingproof

\begin{theorem}
  The consistency problem in \condl{\EL}{\ALC} is \TwoExpTime-hard if $\OCR \neq \emptyset$ and
  $\ORR \neq \emptyset$.
\end{theorem}
\missingproof

\begin{theorem}
  The consistency problem in \condl{\EL}{\ALC} is \NExpTime-hard if $\OCR \neq \emptyset$ and
  $\ORR = \emptyset$.
\end{theorem}
\missingproof


\section{Consistency of Boolean SHOIQ knowledge bases}
\label{sec:consistency-shoiq-bkb}



%%% Local Variables:
%%% mode: latex
%%% TeX-master: "../thesis"
%%% reftex-default-bibliography: ("../references.bib")
%%% End:

%  LocalWords:  Löwenheim Skolem
