
In this section we discuss a possible extension to our contextualised description logic.  We start
with a comparison to temporal description logics. Both DL-LTL-structures and nested DL
interpretations use a \emph{possible worlds semantics}. Single time points or contexts are
represented in a meta dimension and for each such a meta element, or possible world, there exists
one DL interpretation on the object level. Important for both the expressivity and the complexity of
reasoning problems is how these two dimensions can interact. Syntactically, it is a question where
these \emph{meta} operators, i.e.\ temporal or contextual, can be used.

\begin{table}
  \caption{Classification of different two-dimensional temporal and contextual description logics
    (\cite{LuWZ-TIME08,BaGL-ToCL12,KG-JELIA10}) }
  \centering
  \begin{tabular}{lM{0.1\linewidth}M{0.25\linewidth}M{0.17\linewidth}N}
    \toprule
    \multicolumn{2}{c}{\multirow{2}{5cm}[-1ex]{\centering Temporal or contextual operators}}
    & \multicolumn{2}{c}{in front of/around axioms}\\
    & & yes & no &\\[10pt]
    \midrule
    & yes & temporal $\text{LTL}_{\ALC}$, \klarALC, \LMLOplus & $\text{LTL}_{\ALC}$, \hspace{2cm}$\emptyset$ &\\[15pt]
    \cmidrule{2-4}
    \multirow{-2}{*}[2.4ex]{
    \centering inside concepts
    }& no & \ALC-LTL, \hspace{2cm}\LMLO & \ALC &\\[15pt]
    \bottomrule
  \end{tabular}
  \label{tab:classification-tdl-cdl}
\end{table}

Table~\ref{tab:classification-tdl-cdl} gives an overview over some two-dimensional DLs. This is not
a complete overview, but it illustrates some common properties about the complexity of the
consistency problem. Note first that neither having meta operators in front of axioms nor having
meta operators inside concepts is strictly more expressive than the other. Meta operators in front
of axioms can handle general knowledge that holds in some or all worlds.  On the other hand, meta
operators inside object concepts allow to access the extension of concepts in other worlds.
%
We illustrate this by an example taken from \cite{LuWZ-TIME08}.
\begin{gather*}
  \Diamond\Box(\mathsf{European\_country} \sqsubseteq \mathsf{EU\_country})\\
  \mathsf{Independent\_country} \sqsubseteq \Box\mathsf{Independent\_country}
\end{gather*}
The first temporalised GCI states that eventually, i.e.\ at some time point $t$ in the future, all
European countries will forever be EU members, i.e.\ in every time point after $t$. The second axiom
says that the extension of the concept $\mathsf{Independent\_country}$ does not decrease.

In \LMLO, we only have the contextual operator $\oax{\cdot}$ around axioms. We can express that
certain axioms must hold in some worlds, but we cannot access the extension of a concept from
another world, i.e.~we cannot express the set of domain elements which belong to some concept in
another context.  The idea is to overcome this lack of expressive power by introducing a new
contextual operator inside object concepts.

Before extending \LMLO, we analyse some general behaviour of two-dimensional DLs.  The first common
property that we want to focus on is the computational complexity if rigid roles are present.
If meta operators are allowed within object concepts, the consistency problem becomes
undecidable. This holds for all logics in the first row of
Table~\ref{tab:classification-tdl-cdl}. We prove this negative result for \LMLOplus below. If meta operators
are only allowed in front of axioms, we may retain decidability, but at the cost that the
consistency problem is one exponential harder.
%
If no rigid names are allowed, the complexity of the consistency problem increases for
logics with meta operators both in front of axioms and in object concepts: from ExpTime-completeness for \ALC
to \ExpSpace-completeness for temporal $\text{LTL}_{\ALC}$ TBoxes and to \TwoExpTime-completeness
for \klarALC knowledge bases. If only one kind of meta operators is allowed, the complexity
class stays the same, i.e.\ the
consistency problem is ExpTime-complete in \ALCALC, \ALC-LTL, and $\text{LTL}_{\ALC}$.

A setting in which only rigid concepts, but no rigid roles are allowed, is only interesting if meta
operators are not allowed inside object concepts. Otherwise, rigid concepts can easily be
simulated. In $\text{LTL}_{\ALC}$, this can be done by adding $C\sqsubseteq\Box C$ and
$\lnot C\sqsubseteq\Box\lnot C$ to the TBox.  We show below how rigid concepts can be simulated in
\LMLOplus.
%
The contextualised description logic \LMLOplus is an extension of \LMLO in which we additionally allow
contextualised object concepts. Therefore, we update the definition of o-concepts from
Def. \ref{def:syntax-lmlo}.  

\begin{definition}[Object concepts of \LMLOplus]
The set of \emph{concepts of the object logic \LO (o-concepts)} is the smallest set such that
\begin{itemize}
\item for all $A\in\OC$, $A$ is an object concept,
\item if $D$ is an object concept, $\mathbf{C}\in\MC$, $\mathbf{r}\in\MR$, then $\ocont{C}[r]{D}$ is
  also an object concept, and
\item all complex concepts that can be built with the concept constructors allowed in \LO are
  object concepts.
\end{itemize}

Furthermore, for a nested interpretation \JJ the mapping $\cdot^{\I_{c}}$ is extended to
$\ocont{C}[r]{D}$ as follows: $(\ocont{C}[r]{D})^{\I_{c}}\!\coloneqq\!\{d \in \Delta^{\J} \mid \text{there is
some $c'\in \mathbf{C}^{\J}$ s.t.\ $(c,c')\in \mathbf{r}^{\J}$ and $d \in D^{\I_{c'}}$}\}$.
\end{definition}

Following customs of modal logic, we use $\ocont*{C}[r]{D}$ as an abbreviation for
$\lnot\ocont{C}[r]{(\lnot D)}$. Intuitively, in a context $c$ the concept $\ocont{C}[r]{D}$ denotes
the set of all object domain elements that belong to the concept $D$ in some other context $c'$
which belongs to the meta concept $\mathbf{C}$ and is related to $c$ via $\mathbf{r}$. An object
domain element is in the extension of concept $\ocont*{C}[r]{D}$ in context $c$, if it belongs to $D$ in all
contexts $c'$ that belong to the meta concept $\mathbf{C}$ and are related to $c$ via $\mathbf{r}$.

Thus, within a context we can talk about object elements that belong to some object concept in some
other context. This is somehow similar to $\Next C$ in $\text{LTL}_{\ALC}$, which denotes the set of all
elements that are in $C$ \emph{in the next time point}.
\begin{example}\label{ex:alcalc-plus}
  Going back to Example~\ref{ex:nfl-with-contexts}, the following meta concept assertion states that someone who
  plays quarterback for the Green Bay Packers must work as coach in a junior training camp that is organised by
  Green Bay:
  \begin{align*}
    & \oax{\exists\plays.\mathsf{Quarterback} \sqsubseteq
    \ocont{\mathsf{JuniorFootballClinic}}[\mathsf{organises}]{\mathsf{Coach}}}(\text{\textit{GreenBayPackers}}).
  \end{align*}
  Note here, that Green Bay Packers and the junior training camp are two different contexts and that
  this cannot be expressed in \LMLO.
\end{example}

The contextualised description logic \ALCALCplus without rigid names is a syntactical variant of
\klarALC~\cite{KG-JELIA10,KG16}. Consequently, the consistency problem in \ALCALCplus has
the same complexity.

\begin{theorem}\label{thm:alcalcplus-without-rigid-twoexptime}
  The consistency problem in \ALCALCplus is \TwoExpTime-complete if $\OCR = \emptyset$ and
  $\ORR = \emptyset$.
\end{theorem}

\begin{proof}[Sketch]
  We can prove the theorem by a mutual reduction of an \klarALC and an \ALCALCplus knowledge base.
  Without introducing the complete syntax of \klarALC, we show how to map an \klarALC ontology into
  \ALCALCplus.

  Table~\ref{tab:syntax-klarALC} shows in the upper part the two special constructors for object
  concepts available in \klarALC. The middle part provides the syntax and semantics of \emph{object
    formulas} which in turn constitute the \emph{object ontology axioms}, shown in lower part. The
  rightmost column defines the mapping $\tau$ which translates terms from \klarALC to
  \ALCALCplus. The following example shows how an object ontology axiom is mapped.
  \begin{align*}
    \mathbf{C}:\langle \mathbf{C} \rangle_{\mathbf{r}}(\langle \mathbf{C}\rangle_{\mathbf{r}}D
    \sqsubseteq A) 
    \qquad\leadsto\qquad
    C\sqsubseteq\exists r.\left(C\sqcap\oax{\ocont{C}[r]{D}\sqsubseteq A}\right)
  \end{align*}

  \begin{table}[t]    
    \caption{Syntax and semantics of \klarALC, and the mapping $\tau$ to \ALCALCplus}
    \centering
    \begin{tabularx}{0.96\linewidth}{ll@{ iff }X@{}l}
      \toprule
      Syntax   & \multicolumn{2}{l}{Semantics} 
               & mapping $\tau(x)$ \\
      \midrule
      $\langle \mathbf{C} \rangle_{\mathbf{r}}D$ 
               & \multicolumn{2}{l}{
                 $\cdot^{\I_{c}} = \{d\in\Delta\mid\text{there is $c'$ s.t.\
                 $(c,c')\in\mathbf{r}^{\J}$, $c'\in C^{\J}$, $d\in D^{\I_{c'}}$}\}$} 
               & $\ocont{C}[r]{D}$ \\
      $[\mathbf{C}]_{\mathbf{r}}D$ 
               & \multicolumn{2}{l}{
                 $\cdot^{\I_{c}} = \{d\in\Delta\mid\text{$(c,c')\in\mathbf{r}^{\J}$ and $c' \in
                 C^{\J}$ imply $d\in D^{\I_{c'}}$}\}$}
               & $\ocont*{C}[r]{D}$ \\
      \midrule
      $B \sqsubseteq D$  & $\I_{c}\models B \sqsubseteq D$
               & $B^{\I_{c}}\subseteq D^{\I_{c}}$ 
               & $\oax{B \sqsubseteq D}$\\
      $D(a)$   & $\I_{c}\models D(a)$
               & $a^{\I_{c}}\in D^{\I_{c}}$
               & $\oax{D(a)}$\\
      $s(a,b)$ & $\I_{c}\models s(a,b)$
               & $(a^{\I_{c}}, b^{\I_{c}})\in s^{\I_{c}}$
               & $\oax{s(a,b)}$\\
      $\lnot\varphi$     & $\I_{c}\models\lnot\varphi $
               & $\I_{c}\not\models\varphi$
               & $\lnot\tau(\varphi)$\\
      $\varphi\land\psi$    & $\I_{c}\models\varphi\land\psi $
               & $\I_{c}\models\varphi$ and $\I_{c}\models\psi$
               & $\tau(\varphi) \sqcap \tau(\psi)$\\
      $\langle \mathbf{C}\rangle_{\mathbf{r}}\varphi$ 
               & $\I_{c}\models\langle\mathbf{C}\rangle_{\mathbf{r}}\varphi$
               & \text{there is $c'\in \mathbf{C}^{\J}$ s.t.\ $(c,c')\in\mathbf{r}^{\J}$, $\I_{c'}\models\varphi$}
               & $\exists r.(C\sqcap\tau(\varphi))$\\
      $[\mathbf{C}]_{\mathbf{r}}\varphi$ 
               & $\I_{c}\models [\mathbf{C}]_{\mathbf{r}}\varphi$
               & \text{every $c'\in \mathbf{C}^{\J}$, $(c,c')\in\mathbf{r}^{\J}$ implies $\I_{c'}\models\varphi$ }
               & $\forall r.(\lnot C \sqcup \tau(\varphi))$\\
      \midrule
      $\mathbf{a} : \varphi$ 
               & $\J\models \mathbf{a} : \varphi$
               & $\I_{c}\models\varphi$ with $c=\mathbf{a}^{\J}$
               & $(\tau(\varphi))(a)$\\
      $\mathbf{C} : \varphi$
               & $\J\models \mathbf{C} : \varphi$ 
               & $\I_{c}\models\varphi$ for every $c$ with $c\in\mathbf{C}^{\J}$
               & $C \sqsubseteq \tau(\varphi)$\\
      \bottomrule
    \end{tabularx}
    \label{tab:syntax-klarALC}
  \end{table}

  An \klarALC ontology $\Kmc = (\Cmc, \Omc)$ consists of a context ontology \Cmc, which is in fact a
  standard \ALC ontology, and of an object ontology. Given \Kmc, let us define the \ALCALC ontology
  $\Bmc_{\Kmc} \coloneqq \Cmc \land \tau(\Omc)$. It is easy to verify that a nested interpretation
  \J is a model of \Kmc if and only if it is a model of $\Bmc_{\Kmc}$.

  Conversely, for an \ALCALC ontology \Bmc, we take the outer abstraction \Bb as context ontology
  and for each \oalpha in \Bmc, we add $(\mathbf{A_{\oalpha}} : \alpha)$ and
  $(\lnot\mathbf{A_{\oalpha}} : \lnot\alpha)$ to the object ontology~$\Omc_{\Bmc}$. Again, it is
  easy to show that~\J models~\Bmc iff~\J models $\Kmc_{\Bmc} = (\Bb,\Omc_{\Bmc})$.
\end{proof}



The more interesting setting with rigid roles behaves much worse. One can easily show that the consistency
problem becomes undecidable.

\begin{theorem}\label{thm:elalcplus-with-rigid-undecidable}
  The consistency problem in \ELALCplus is undecidable if $\ORR \neq \emptyset$.
\end{theorem}

\begin{proof}
  Similar to the idea of \cite{LuWZ-TIME08}, we proof the claim by reduction of a well-known
  undecidable problem, namely the \emph{domino problem} \cite{Ber-66}: given a triple
  $\Dmc = (D, H, V)$ with a set of domino types $D=\{d_{1}, \dots, d_{n}\}$, a horizontal
  compatibility relation $H \subseteq D \times D$ and a vertical compatibility relation
  $V \subseteq D \times D$, decide whether there exists a solution to cover the
  $\nat\times\nat$-grid with these domino types respecting the compatibility relations, i.e.\ does
  there exist a \emph{tiling} $\mathsf{t}:\nat\times\nat\to D$ s.t.\
  $(\mathsf{t}(i,j),\mathsf{t}(i+1,j))\in H$ and $(\mathsf{t}(i,j),\mathsf{t}(i,j+1))\in V$?

  {\setlength{\jot}{7pt}
  We encode this problem in \ELALCplus with a single rigid role $v\in\ORR$. Let $\Bmc_{\Dmc}$ be the
  conjunction of the following meta axioms. Each context represents a \emph{column} of the
  grid. Using a meta role $h \in \MR$ and a rigid object role $v \in \ORR$, we ensure the existence
  of a grid:
  \vspace{0.19ex}
  \begin{align*}
    \top & \sqsubseteq \exists h.\top \tag{$\alpha_{1}$}\\
    \top & \sqsubseteq \oax{\top \sqsubseteq \exists v.\top} \tag{$\alpha_{2}$}
  \end{align*}
  \vspace{0.19ex}
  Let $A_{0}, \dots, A_{n} \in \OC$ be object concept names representing the given domino types. To
  ensure that a single domino type is assigned to each grid position, we use
  \vspace{0.19ex}
  \begin{align*}
    \top & \sqsubseteq \oax{\top \sqsubseteq (A_{1} \sqcup \dots \sqcup A_{n})
           \sqcap \bigsqcap_{1 \leq i < j \leq n} \lnot(A_{i} \sqcap A_{j})}.
           \tag{$\alpha_{3}$}
  \end{align*}
  \vspace{0.19ex}
  To enforce the compatibility relations, we use
  \vspace{0.19ex}
  \begin{align*}
    \top & \sqsubseteq \bigsqcap_{i=1}^{n} 
           \oax{A_{i} \sqsubseteq \forall v.(\bigsqcup_{(d_{i},d_{j})\in V} A_{j})}, \text{ and} \tag{$\alpha_{4}$}\\
    \top & \sqsubseteq \bigsqcap_{i=1}^{n} 
           \oax{A_{i} \sqsubseteq \bigsqcup_{(d_{i},d_{j})\in H} \ocont*{\top}[h]{A_{j}}}.\tag{$\alpha_{5}$}
  \end{align*}
  }

  \begin{claim}
    $\Bmc_{\Dmc}$ is consistent iff \Dmc has a
    solution.
  \end{claim}

  \begin{claimproof}
    Assume that $\mathsf{t}$ is a solution for \Dmc. Then, based on $\mathsf{t}$ we define the
    nested interpretation
    $\J_{\mathsf{t}} = (\nat, \cdot^{\J_{\mathsf{t}}}, \nat, (\cdot^{\I_{x}})_{x\in\nat})$ with
    \begin{align*}
      h^{\J_{\mathsf{t}}} & \coloneqq \{(k,k+1) \mid k \geq 0\}\\
      v^{\I_{x}} & \coloneqq \{(l, l+1) \mid l \geq 0\} \qquad \text{for all $n \geq 0$}\\
      %
      A_{i}^{\I_{x}} & \coloneqq \{ y \in \nat \mid \mathsf{t}(x,y) = d_{i}\}
    \end{align*}
    By definition, $\J_{\mathsf{t}}$ models $\alpha_{1}$ and $\alpha_{2}$. For each object domain
    element $y \in \nat$ and each $\I_{x}$, $x \geq 0$, we have that $y \in {A_{i}}^{\I_{x}}$ and
    $y \notin {A_{j}}^{\I_{x}}$, $1 \leq j \leq n, i\neq j$, for $\mathsf{t}(x,y)=d_{i}$. Hence,
    $\J_{\mathsf{t}} \models \alpha_{3}$. By definition, $y+1$ is the only $v$-successor of $y$. If
    $y \in A_{i}^{\I_{x}}$ we know that
    $y+1 \in \smash{\big(\bigsqcup_{(d_{i},d_{j})\in V} A_{j}\big)^{\I_{x}}}$ because $\mathsf{t}$
    is a solution and $(\mathsf{t}(x,y),\mathsf{t}(x,y+1))\in V$. Analogously, $x+1$ is the only
    $h$-successor of $x$ and if $y \in A_{i}^{\I_{x}}$, we know that
    $y \in \smash{\big(\bigsqcup_{(d_{i},d_{j})\in H} A_{j}\big)^{\I_{x+1}}}$ because $\mathsf{t}$
    is a solution and $(\mathsf{t}(x,y),\mathsf{t}(x+1,y))\in H$. Hence,
    $\J_{\mathsf{t}} \models \alpha_{4} \land \alpha_{5}$. Thus, $\Bmc_{\Dmc}$ is consistent.

    Assume that \JJ is a model of $\Bmc_{\Dmc}$. Let $\Psf_{\Msf}$ and $\Psf_{\Osf}$ be infinite
    paths $\Psf_{\Msf} = c_{0}c_{1}\dots$ and $\Psf_{\Osf} = o_{0}o_{1}\dots$ with $c_{i} \in \Cbb$,
    $(c_{i}, c_{i+1}) \in h^{\J}$, $o_{i} \in \Delta^{\J}$ and $(o_{i}, o_{i+1}) \in v^{\I_{c}}$ for
    some $c \in \Cbb$. Such paths exists, because $\J\models\alpha_{1}$, $\J\models\alpha_{2}$ and
    $v$ is a rigid role. We define the nested interpretation
    $\J_{\Psf} \coloneqq (\{c_{i} \mid i \geq 0\}, \cdot^{\J_{\Psf}}, \{o_{i} \mid i \geq 0\},
    (\cdot^{\I_{\Psf,c_{i}}})_{i \geq 0})$, where $\cdot^{\J_{\Psf}}$ is the restriction of
    $\cdot^{\J}$ to the domain ${c_{i} \mid i \geq 0}$, and $\cdot^{\I_{\Psf,c_{i}}}$ is the restriction of
    $\cdot^{\I_{c_{i}}}$ to the domain ${o_{i} \mid i \geq 0}$.

    By construction, $\J_{\Psf}$ is a model of $\alpha_{1}$ and $\alpha_{2}$. Since $\alpha_{3}$ to
    $\alpha_{5}$ do not contain any existential or at-least restrictions, the restriction of the
    meta and the object domain preserves the entailment relation, and
    $\J_{\Psf}\models\Bmc_{\Dmc}$. We define the tiling $\mathsf{t}$ as follows:
    \begin{align*}
      t(x,y) = d_{i} \quad\text{ if $o_{y} \in A_{i}^{\I_{\Psf,c_{x}}}$}.
    \end{align*}
    The tiling $\mathsf{t}$ is a total function and well-defined, due to $\alpha_{3}$. Let
    $o_{j} \in {A_{k_{1}}}^{\!\!\!\I_{\Psf,c_{i}}}$,
    $o_{j+1} \in {A_{k_{2}}}^{\!\!\!\I_{\Psf,c_{i}}}$ and
    $o_{j} \in {A_{k_{3}}}^{\I_{\Psf,c_{i+1}}}$. Thus, we have $\mathsf{t}(i,j) = d_{k_{1}}$,
    $\mathsf{t}(i,j+1) = d_{k_{2}}$ and $\mathsf{t}(i+1,j) = d_{k_{3}}$. By $\alpha_{4}$ we know
    that
    $o_{j} \in \smash{\forall v.\big(\bigsqcup_{(d_{k_{1}},d_{j})\in V}
      A_{j}\big)^{\I_{\Psf,c_{i}}}}$ and, hence, $(d_{k_{1}}, d_{k_{2}})\in V$. Analogously by
    $\alpha_{5}$, we get $(d_{k_{1}}, d_{k_{3}})\in H$. Thus, \Dmc has a solution.
  \end{claimproof}

  Thus, deciding whether $\Bmc_{\Dmc}$ is consistent is undecidable.
\end{proof}

In Section~\ref{sec:complexity-consis-problem}, we discussed three different settings depending on
whether rigid concept and role names are admitted. We already obtained the complexity results for
\LMLOplus for Setting~(i), i.e.~no rigid names are allowed, and for Setting~(iii), i.e.~rigid roles
are allowed.  With some assumptions, in \LMLOplus, however, Setting~(ii) that allows rigid concept
names but no rigid roles coincides with Setting~(i), since rigid concepts can be simulated. Hence,
we obtain the following result:

\begin{theorem}
  The consistency problem for \SHOISHOIplus{\kern-0.1em}-ontologies is \TwoExpTime-complete if $\ORR=\emptyset$.
\end{theorem}

\begin{proof}
  First, we have to simulate the universal role $u$. With $u$, we can simulate rigid concepts and,
  thus, reduce the consistency problem to the case without rigid names.

  Let $\Omf = (\Omc,\RO, \RM)$ be an \SHOISHOIplus-ontology. We first have to simulate a universal
  role $u$ on the meta level. We can assume w.l.o.g.\ that $u\in\MR$ does not occur in \Omf. We
  obtain $\Omf_{u}$ from \Omf by adding the following axioms to \Omf:
  \begin{itemize}
  \item $u^{-} \sqsubseteq u$,
    % \item $\trans{u}$,
  \item $r \sqsubseteq u$, for all $r\in\MR$ occurring in \Omf,
  \item $u(a,b)$ for all $a,b\in\MI$ occurring in \Omf.
  \end{itemize}
  Note that $\Omf_{u}$ is of size polynomial in the size of \Omf.
  %
  Assume there exists an unnamed context in a model of \Omf, i.e.~$c\in\Cbb$ such that there is
  no~$a\in\MI$ occurring in~\Omf with~$a^{\J}=c$, and $c$ is not connected to any named context by
  some path. Then, we can always remove that unnamed context from the interpretation and still have
  a model.  Hence, we can assume that every model of~$\Omf_{u}$ is connected and that every context
  can be reached by any other context via a path of $u$-edges.
  %
  Note that we restrict~\Omf to~\SHOISHOIplus-ontologies, since in an \SHOISHOIplus-BKB, a negated
  meta GCI can enforce the existence of an unnamed context that is not necessarily connected to the
  rest of the model.

  With the help of $u$ and contextualised concepts, we can express that $A\in\OC$ is rigid by the
  following two axioms:
  \begin{align*}
    \top & \sqsubseteq \oax{A \sqsubseteq \ocont*{\top}[u]{A}} \\
    \top & \sqsubseteq \oax{\lnot A \sqsubseteq \ocont*{\top}[u]{\lnot A}}
  \end{align*}
  Due to these axioms, the extension of $A$ in a context $c$ are exactly these elements which are in
  $A$ in every context $c'$ that is related to $c$ via $u$. Thus, the extension of $A$ is equal in
  every context, i.e.\ $A$ is rigid.

  In~\cite{KG16}, Theorem~6 states that deciding the consistency problem in \klarSHOI without rigid
  names is \TwoExpTime-complete. We can again use the same translation as shown in the proof of
  Theorem~\ref{thm:alcalcplus-without-rigid-twoexptime} and, hence, obtain \TwoExpTime-completeness
  for \SHOISHOIplus-ontologies with rigid concepts.
\end{proof}

To sum up, we showed that adding contextualised concepts dramatically increases the complexity. In
the presence of rigid roles, the consistency problem for even less expressive \ELALCplus-ontologies
becomes undecidable. Without rigid roles for any contextualised DL between \ALCALCplus and
\SHOISHOIplus deciding the consistency of an ontology is \TwoExpTime-complete.



%%% Local Variables:
%%% mode: latex
%%% TeX-master: "../thesis"
%%% reftex-default-bibliography: ("../references.bib")
%%% End:

%  LocalWords:  logics DL LTL expressivity temporalised ontologies