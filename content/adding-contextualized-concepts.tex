
\todo[inline, caption={intro paragraph for adding contextualized concepts}]{
\ALCALC without rigid names sublogic of \klarALC, relation to \klarALC

explain a bit more the restriction of \LMLO, what cannot be expressed.

introduce new o-concept constructor

undecidability result
}

The contextualized description logic \LMLOplus is an extension of \LMLO where we additionally allow
contextualized object concepts. Therefore, we update the definition of o-concepts from
Def. \ref{def:syntax-cdls}.  

\begin{definition}[Object concepts of \LMLOplus]
The set of \emph{concepts of the object logic \LO (o-concepts)} is the smallest set such that
\begin{itemize}
\item for all $A\in\OC$, $A$ is an object concept,
\item if $D$ is an object concept, $\mathbf{C}\in\MC$, $\mathbf{r}\in\MR$, then $\ocont{C}[r]{D}$ is
  also an object concept, and
\item all complex concepts that can be built with the concept constructors allowed in \LO are
  object concepts.
\end{itemize}

Furthermore, for a nested interpretation \JJ the mapping $\cdot^{\I_{c}}$ is extended to
$\ocont{C}[r]{D}$ as follows: $(\ocont{C}[r]{D})^{\I_{c}} \coloneqq \{d \in \Delta \mid \text{there is
some $c'$ s.t.\ $(c,c')\in r^{\J}$ and $d \in D^{\I_{c'}}$}\}$.
\end{definition}

Following customs of modal logic, we use $\ocont*{C}[r]{D}$ as an abbreviation for
$\lnot\ocont{C}[r]{(\lnot D)}$. Intuitively, in a context $c$ the concept $\ocont{C}[r]{D}$ denotes
the set of all object domain elements that belong to the concept $D$ in some other context $c'$
which belongs to the meta concept $\mathbf{C}$ and is related to $c$ via $\mathbf{r}$. An object
domain element is in the concept $\ocont*{C}[r]{D}$ in context $c$, if it belongs to $D$ in all
contexts $c'$ that belong to the meta concept $\mathbf{C}$ and are related to $c$ via $\mathbf{r}$.

\todo[inline]{example}
\begin{example}\label{ex:alcalc-plus}
  
\end{example}

The contextualized description logic \ALCALCplus without rigid names is a syntactical variant of
\klarALC \textbf{[??]}\todo{cite klarman}. Consequently, the consistency problem in \ALCALCplus has
the same complexity.

\begin{theorem}\label{thm:alcalcplus-without-rigid-twoexptime}
  The consistency problem in \ALCALCplus is \TwoExpTime-complete if $\OCR = \emptyset$ and
  $\ORR = \emptyset$.
\end{theorem}

mutual reduction. 

\missingproof


The more interesting case of rigid roles behaves much worse. One can even show that the consistency
problem becomes undecidable.

\begin{theorem}\label{thm:elalcplus-with-rigid-undecidable}
  The consistency problem in \ELALCplus is undecidable if $\ORR \neq \emptyset$.
\end{theorem}

\begin{proof}
  Similar to the idea of \cite{LuWZ-TIME08} we proof the claim by reduction of a well-known
  undecidable problem, namely the \emph{domino problem} \cite{Ber-66}: given a triple
  $\Dmc = (D, H, V)$ with a set of domino types $D=\{d_{1}, \dots, d_{n}\}$, a horizontal
  compatibility relation $H \subseteq D \times D$ and a vertical compatibility relation
  $V \subseteq D \times D$, decide whether there exists a solution to cover the
  $\nat\times\nat$-grid with these domino types respecting the compatibility relations, i.e.\ does
  there exist a \emph{tiling} $\mathsf{t}:\nat\times\nat\to D$ s.t.\
  $(\mathsf{t}(i,j),\mathsf{t}(i+1,j))\in H$ and $(\mathsf{t}(i,j),\mathsf{t}(i,j+1))\in V$?

  We encode this problem in \ELALCplus with a single rigid role $v\in\ORR$. Let $\Bmc_{\Dmc}$ be the
  conjunction of the following meta axioms. Each context represents a \emph{column} of the
  grid. Using a meta role $h \in \MR$ and a rigid object role $v \in \ORR$ we build the grid:
  \begin{align*}
    \top & \sqsubseteq \exists h.\top \tag{$\alpha_{1}$}\\
    \top & \sqsubseteq \oax{\top \sqsubseteq \exists v.\top} \tag{$\alpha_{2}$}
  \end{align*}
  Let $A_{0}, \dots, A_{n} \in \OC$ be object concept names representing the given domino types. To
  ensure that a single domino type is assigned to each grid position, we use
  \begin{align*}
    \top & \sqsubseteq \oax{\top \sqsubseteq (A_{1} \sqcup \dots \sqcup A_{n})
           \sqcap \bigsqcap_{1 \leq i < j \leq n} \lnot(A_{i} \sqcap A_{j})}.
           \tag{$\alpha_{3}$}
  \end{align*}
  To enforce the compatibility relations, we have
  \begin{align*}
    \top & \sqsubseteq \bigsqcap_{i=1}^{n} 
           \oax{A_{i} \sqsubseteq \forall v.(\bigsqcup_{(d_{i},d_{j})\in V} A_{j})}, \text{ and} \tag{$\alpha_{4}$}\\
    \top & \sqsubseteq \bigsqcap_{i=1}^{n} 
           \oax{A_{i} \sqsubseteq \bigsqcup_{(d_{i},d_{j})\in H} \ocont*{\top}[h]{A_{j}}}.\tag{$\alpha_{5}$}
  \end{align*}

  \begin{claim}
    $\Bmc_{\Dmc}$ is consistent iff \Dmc has a
    solution.
  \end{claim}

  \begin{claimproof}
    Assume that $\mathsf{t}$ is a solution for \Dmc. Then, based on $\mathsf{t}$ we define the
    nested interpretation
    $\J_{\mathsf{t}} = (\nat, \cdot^{\J_{\mathsf{t}}}, \nat, (\cdot^{\I_{x}})_{x\in\nat})$ with
    \begin{align*}
      h^{\J_{\mathsf{t}}} & \coloneqq \{(k,k+1) \mid k \geq 0\}\\
      v^{\I_{x}} & \coloneqq \{(l, l+1) \mid l \geq 0\} \qquad \text{for all $n \geq 0$}\\
      %
      A_{i}^{\I_{x}} & \coloneqq \{ y \in \nat \mid \mathsf{t}(x,y) = d_{i}\}
    \end{align*}
    By definition, $\J_{\mathsf{t}}$ models $\alpha_{1}$ and $\alpha_{2}$. For each object domain
    element $y \in \nat$ and each $\I_{x}$, $x \geq 0$, we have that $y \in {A_{i}}^{\I_{x}}$ and
    $y \notin {A_{j}}^{\I_{x}}$, $1 \leq j \leq n, i\neq j$, for $\mathsf{t}(x,y)=d_{i}$. Hence,
    $\J_{\mathsf{t}} \models \alpha_{3}$. By definition, $y+1$ is the only $v$-successor of $y$. If
    $y \in A_{i}^{\I_{x}}$ we know that
    $y+1 \in \smash{\big(\bigsqcup_{(d_{i},d_{j})\in V} A_{j}\big)^{\I_{x}}}$ because $\mathsf{t}$
    is a solution and $(\mathsf{t}(x,y),\mathsf{t}(x,y+1))\in V$. Analogously, $x+1$ is the only
    $h$-successor of $x$ and if $y \in A_{i}^{\I_{x}}$, we know that
    $y \in \smash{\big(\bigsqcup_{(d_{i},d_{j})\in H} A_{j}\big)^{\I_{x+1}}}$ because $\mathsf{t}$
    is a solution and $(\mathsf{t}(x,y),\mathsf{t}(x+1,y))\in H$. Hence,
    $\J_{\mathsf{t}} \models \alpha_{4} \land \alpha_{5}$. Thus, $\Bmc_{\Dmc}$ is consistent.

    Assume that \JJ is a model of $\Bmc_{\Dmc}$. Let $\Psf_{\Msf}$ and $\Psf_{\Osf}$ be infinite
    paths $\Psf_{\Msf} = c_{0}c_{1}\dots$ and $\Psf_{\Osf} = o_{0}o_{1}\dots$ with $c_{i} \in \Cbb$,
    $(c_{i}, c_{i+1}) \in h^{\J}$, $o_{i} \in \Delta^{\J}$ and $(o_{i}, o_{i+1}) \in v^{\I_{c}}$ for
    some $c \in \Cbb$. Such paths exists, because $\J\models\alpha_{1}$, $\J\models\alpha_{2}$ and
    $v$ is a rigid role. We define the nested interpretation
    $\J_{\Psf} \coloneqq (\{c_{i} \mid i \geq 0\}, \cdot^{\J_{\Psf}}, \{o_{i} \mid i \geq 0\},
    (\cdot^{\I_{\Psf,c_{i}}})_{i \geq 0})$ where $\cdot^{\J_{\Psf}}$ is the restriction of
    $\cdot^{\J}$ to the domain ${c_{i} \mid i \geq 0}$ and $\cdot^{\I_{\Psf,c_{i}}}$ is the restriction of
    $\cdot^{\I_{c_{i}}}$ to the domain ${o_{i} \mid i \geq 0}$.

    By construction, $\J_{\Psf}$ is a model of $\alpha_{1}$ and $\alpha_{2}$. Since $\alpha_{3}$ to
    $\alpha_{5}$ do not contain any existential or at-least restrictions, the restriction of the
    meta and the object domain does preserve the entailment relation and
    $\J_{\Psf}\models\Bmc_{\Dmc}$. We define the tiling $\mathsf{t}$ as follows:
    \begin{align*}
      t(x,y) = d_{i} \quad\text{ if $o_{y} \in A_{i}^{\I_{\Psf,c_{x}}}$}.
    \end{align*}
    The tiling $\mathsf{t}$ is a total function and well-defined, due to $\alpha_{3}$. Let
    $o_{j} \in {A_{k_{1}}}^{\!\!\!\I_{\Psf,c_{i}}}$,
    $o_{j+1} \in {A_{k_{2}}}^{\!\!\!\I_{\Psf,c_{i}}}$ and
    $o_{j} \in {A_{k_{3}}}^{\I_{\Psf,c_{i+1}}}$. Thus, we have $\mathsf{t}(i,j) = d_{k_{1}}$,
    $\mathsf{t}(i,j+1) = d_{k_{2}}$ and $\mathsf{t}(i+1,j) = d_{k_{3}}$. By $\alpha_{4}$ we know
    that
    $o_{j} \in \smash{\forall v.\big(\bigsqcup_{(d_{k_{1}},d_{j})\in V}
      A_{j}\big)^{\I_{\Psf,c_{i}}}}$ and, hence, $(d_{k_{1}}, d_{k_{2}})\in V$. Analogously by
    $\alpha_{5}$, we get $(d_{k_{1}}, d_{k_{3}})\in H$. Thus, \Dmc has a solution.
  \end{claimproof}

  Thus, deciding whether $\Bmc_{\Dmc}$ is consistent is undecidable.
\end{proof}

Besides the negative result about rigid roles, it is still unclear how \LMLOplus behaves in the case
when only rigid concepts are allowed. But based on the necessity to express rigid roles when
modelling RoSIs \todo{reference to section 2.2} as discussed in Section \ref{sec:rosiroles}, we did
not look into this setting any further and it is left open as future work when appropriate use cases
emerge.


%%% Local Variables:
%%% mode: latex
%%% TeX-master: "../thesis"
%%% reftex-default-bibliography: ("../references.bib")
%%% End:
