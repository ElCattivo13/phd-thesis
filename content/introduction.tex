

\listoftodos

\chapter{Introduction}
\label{ch:introduction}

Nowadays, we are surrounded by software systems literally everywhere. With regard to current
developments there will be even more in the future. Not only the amount of systems increases, but also the
requirements and expectations users impose on current software steadily rise. Modern software
systems should cope with very complex scenarios. This includes the ability of context-awareness and
self-adaptability. For example, a robot in a smart factory should recognize when a human coworker
approaches and switches to a different, human-friendly working mode accordingly. Similarly, software
in autonomous cars or an the area of smart homes needs to adapt to various situations of which some
are not even stated explicitly. 
%
Furthermore, software must be easily maintainable and, when necessary, changes on the system should be realized
without much down time which, for example, in a smart factory is very costly. 

% To sum up, current software systems long-lasting, high complexity, easy to maintain.



% \bulletItem{Aufzählung, Beispiele}. Requirements
% users impose on these ... \bulletItem{long-lasting (legacy systems in COBOL, especially in banking
%   systems), easy to maintain, but high complexity, (self-)adapting}




\section{Role-Based Systems}
\label{sec:intro-role-based-systems}

In order to achieve all these goals, the concept of \emph{roles} is very promising. First introduced
by Bachman~\cite{BaD-VLDB77}, roles appeared over the last decades in several fields of computer
science\todo{citations}. The relational or context-dependent properties and behaviour of objects is
transferred into the roles that object plays in a certain context. These paradigm also supports
Dijkstra's separation of concerns~\cite{Dij-SelWrCom82} which simplifies development and maintenance
of such systems.  Due to the use of roles, \emph{role-based systems} can cleaner and more structured
model application domains since ontologically different entities are modelled by different concepts.

Consider, for example, the concepts of $\mathsf{Person}$ and $\mathsf{Customer}$. With an
object-oriented approach of inheritance as a specialization relation, we could model
$\mathsf{Customer}$ as a subclass of $\mathsf{Person}$, as obviously not every person is a customer.
On the other hand, if we restrict our domain to a business context and add the concept of a
$\mathsf{Company}$, the inheritance relation would flip and we also have $\mathsf{Company}$ as
subclass of $\mathsf{Customer}$.  This conflict can be resolved when we realize that
$\mathsf{Person}$ and $\mathsf{Company}$ are context-independent basic concepts, so-called
\emph{natural types} and $\mathsf{Customer}$ is actually a role a person or company can play in a
business context. Here, it becomes also apparent, that the concept of a \emph{context} is closely
related to a role.

While role-based modelling provides the means to handle and model complex and context-dependent
domains in a well-structured and modular way, the process can still be tedious, hard and
error-prone. Due to the sophisticated semantics of roles, contexts and many different kinds of
constraints, unintended implications or even inconsistencies can easily be hidden within such a
model. Since it is nearly impossible to uncover all inferences, it becomes imperative for domain
analysts to reason on role-based models to find such implicit knowledge.  Here, a feasible logical
formalism is needed.

\section{Description Logics}
\label{sec:intro-description-logics}

Description Logics (DLs) \cite{DLhandbook-07} are a well-known formalism for knowledge
representation. They possess formal semantics and allow to define a variety of reasoning problems.

The basic building blocks in description logics are so-called \emph{concept names} and \emph{role
  names}. Concept names denote sets of domain elements. For example the concept names
$\mathsf{Person}$ or $\mathsf{Bank}$ denote the sets of all persons or banks in a domain. Relational
structures are represented by so-called DL \emph{role names}, which are essentially binary relations
on the the domain. The term ``role'' originates from the early knowledge representation system
KL-ONE~\cite{WoS-CMA133} and has only little in common with roles of role-based systems except that
it reflects the relational property of a role. A person which is related to a bank via a DL role
$\mathsf{customer}$ could be seen as someone playing the role of a customer in the context of a
bank. Besides that, DL roles are merely binary relations. With the help of \emph{concept} and
\emph{role constructors}, complex concepts and roles can be defined.  Which constructors are allowed
depends on the specific DL. Complex concepts can be used as descriptions  or to
classify domain elements, e.g.\ the complex concept
\begin{align}
  \label{eq:1-1}
  \mathsf{NFL\_Player} \sqcap \mathsf{Healthy} \sqcap \exists.\mathsf{wins}(\mathsf{NFL\_Game})
\end{align}
describes the set of all healthy NFL players who win NFL games.

With the help of concepts we can express our knowledge about a domain through \emph{DL
  axioms}. General knowledge is phrased via \emph{general concept inclusions (GCIs)} which state
that one concept is a sub-concept of another. For example the GCI
\begin{align}
  \label{eq:1-2}
  \mathsf{NFL\_Player} \sqcap \mathsf{Healthy} \sqcap
  \exists.\mathsf{wins}(\mathsf{NFL\_Game}) \sqsubseteq \mathsf{Happy\_NFL\_Player}
\end{align}
states that a healthy NFL player who wins NFL games is a happy NFL player. Conversely, it does not say
anything whether every happy NFL player is healthy or wins games. Facts about a domain can be
expressed via \emph{concept} and \emph{role assertions}.  To express facts, we also introduce
\emph{individual names} which denote single domain elements. As an example consider the following axioms:
\begin{gather}
  (\mathsf{NFL\_Player} \sqcap \mathsf{Healthy})(\text{\textit{AaronRodgers}})\label{eq:1-3}\\
  \mathsf{NFL\_Game}(\text{\textit{SuperBowlXLV}}),\label{eq:1-4}\\
  \mathsf{\mathsf{wins}}(\text{\textit{AaronRodgers}},\text{\textit{SuperBowlXLV}})\label{eq:1-5}
\end{gather}
The first two concept assertions~\eqref{eq:1-3} and~\eqref{eq:1-4} state that Aaron Rodgers is a healthy NFL player and that Super Bowl~XLV is
an NFL game, while~\eqref{eq:1-5} expresses that he won Super Bowl~XLV. So he is also a happy
NFL player, even if not stated explicitly.  A \emph{DL knowledge base} is a set of such axioms.

The semantics of DLs are defined in a model-theoretic way and capture exactly the above mentioned
intentions.  An interpretation~\I consists of a domain and an interpretation function~$\cdot^{\I}$
which maps concept, role and individual names, respectively, to subsets, binary relations and elements of the
domain.  From there, it is exactly defined how complex concepts must be
interpreted. For example, $(A\sqcap B)^{\I}$ is the intersection of $A^{\I}$ and $B^{\I}$.  The
concept name $\mathsf{NFL\_Player}$ itself has no meaning and an interpretation must make sure that
$\mathsf{NFL\_Player}^{\I}$ actually is the set of all NFL players.

Now, the most interesting reasoning problems are the \emph{consistency problem} and the \emph{entailment
problem}. A knowledge base is consistent if there exists some interpretation that \emph{models} the
knowledge base, i.e.\ an interpretation that fulfills all the axioms. An axiom is \emph{entailed} by
a knowledge base if every model of the knowledge base also models that axiom. For example, the following axiom is
entailed by~\eqref{eq:1-2} to~\eqref{eq:1-5}:
\begin{gather}
  \mathsf{Happy\_NFL\_Player}(\text{\textit{AaronRodgers}}).\label{eq:1-6}
\end{gather}


\begin{figure}
  \centering
  \begin{tikzpicture}
    \node[node, label={[align=right]180:\textsf{AaronRodgers},\\ $\mathsf{Healthy}$,\\
      $\mathsf{NFL\_Player}$,\\$\mathsf{Happy\_NFL\_Player}$}] (ar) at(-2,0) {}; 
    \node[node, label={[align=left]0:\textsf{SuberBowlXLV},\\ $\mathsf{NFL\_Game}$}] (sb45) at (2,0)
    {};
    \draw[edge] (ar) to[bend left=10] node{$\mathsf{wins}$} (sb45);
  \end{tikzpicture}
  \caption{Interpretation that models axioms~\eqref{eq:1-2} to~\eqref{eq:1-6}.}
  \label{fig:dl-example-intro}
\end{figure}

\noindent
Figure~\ref{fig:dl-example-intro} depicts an interpretation which is a model of axioms~\eqref{eq:1-2} to~\eqref{eq:1-6}.


\section{Contextualized Description Logics}
\label{sec:intro-contextualized-description-logics}




\todo[inline]{ConDL with top down view, first describe contexts, then what's going on inside. from
  within we can't refer to other contexts, trade-off between rigid roles and contextual concepts}

\blindtext

\section{Ontology generator}
\label{sec:zweite-section}

Besides capability of contextualized description logics, it is rather hard for domain analysts who
in general are not experts in the area of DLs to capture the precise semantics of the ontology and model the
contextual ontology correctly. Therefore, it would be ideal 

evolved and elaborate semantics of condl

we present mapping which captures most parts of CROM

forfeit expressiveness, conDL can express more than CROM


\blindtext 


\section{A Reasoner for Contextualized Description Logics}
\label{sec:intro-reasoner}


\blindtext




\clearpage

\section{General ToDos}
\label{sec:todos}



%\todo[inline]{check for iff}
%\todo[inline]{check for \textasciitilde{} before \textbackslash{}cite}
%\todo[inline]{check for \textasciitilde{} before \Bmc and so on}
%\todo[inline]{check for \textasciitilde{} before \textbackslash{}ref}
%\todo[inline]{check for ``i.e.'' with correct spacing, see \url{https://en.wikibooks.org/wiki/LaTeX/Tips_and_Tricks}}
%\todo[inline]{all referenced chapters, sections, subsections, definitions, lemmas, theorems,
  examples capitalized?}
%\todo[inline]{Capitalized letters in title of definitions (only first letter)}
\todo[inline]{Spacing after macros, i.e.\ dl names and \OCR, etc.}
\todo[inline]{paragraph in chapter 2 about NI, UNA, nominals and  $\top\sqsubseteq\{a\}$ is inconsistent.}
\todo[inline]{search source code for \textbackslash{}hl macro}
\todo[inline]{use \Hmc if it is an \Msig-interpretaion}
\todo[inline]{In section \ref{sec:case-el}, are the \EL-LTL papers correctly cited?}
\todo[inline]{References: Months?}
\todo[inline]{define fresh names in preliminaries}
\todo[inline]{range or image -- be consistent!}
\todo[inline]{natural numbers with 0 or without 0, should that be stated somewhere?}
\todo[inline]{indent after def or proof environment or not}
\todo[inline]{on the object level, on the meta level}
\todo[inline]{allows to falsche}
\todo[inline]{optimised}
\todo[inline]{bei defs immer nur ersten buchstaben gross}
\todo[inline]{References überprüfen}
\todo[inline]{suche nach -ize, analyze -> analyse}
\todo[inline]{Role-Based immer klein geschrieben, außer in title}
\todo[inline]{a and an überprüfen}
\todo[inline]{role und rosirole überprüfen}
\todo[inline]{chapter 4 -> bild role type}



%%% Local Variables:
%%% mode: latex
%%% TeX-master: "../thesis"
%%% End:



%%% Local Variables:
%%% mode: latex
%%% TeX-master: "../thesis"
%%% reftex-default-bibliography: ("../references.bib")
%%% End:

%  LocalWords:  logics Bachman XLV DLs
