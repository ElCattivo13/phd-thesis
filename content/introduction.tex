
\chapter{Introduction}
\label{ch:introduction}

Nowadays, we are literally everywhere surrounded by software systems. Current developments indicate
a continuing growth in the future.  Not only the amount of systems increases, but also the
requirements and expectations users impose on current software steadily rise. Modern software
systems should cope with very complex scenarios. This includes the ability of context-awareness and
self-adaptability. For example, a robot in a smart factory should recognise when a human co-worker
approaches and switch to a different, human-friendly working mode accordingly. Similarly, software
in autonomous cars or in the area of smart homes needs to adapt to various situations of which some
are not even stated explicitly.
%
Furthermore, software must be easily maintainable and, when necessary, changes on the system should
be realised without much down time which, for example, in a smart factory is very costly.

\section{Role-Based Systems}
\label{sec:intro-role-based-systems}

In order to achieve all these goals, the concept of \emph{roles} is very promising. First introduced
by Bachman~\cite{BaD-VLDB77}, roles appeared over the last decades in several fields of computer
science. Most prominent is the \emph{role-based access
  control}~\cite{FeKC-RBAC03,AlFe-ECS11,SaCF-IEEE96}, albeit it is only a special application for
roles with a narrow scope. Roles are also introduced, for example, in data
modelling~\cite{Ha-ORM2006}, conceptual modelling~\cite{Stei-DKE00,Gui-PHD05,Stei-AO07} and
programming languages~\cite{BaBT-SAC06,Herr-AO07,BaGE-ECOOP07}.

The relational or context-dependent properties and behaviour of objects are transferred into the
roles that object plays in a certain context. This paradigm also supports Dijkstra's separation of
concerns~\cite{Dij-SelWrCom82} which simplifies development and maintenance of such systems.  Due to
the use of roles, \emph{role-based systems} can model application domains cleaner and more
structured, since ontologically different entities are modelled by different concepts.

Let us consider, for example, the concepts of $\mathsf{Person}$ and $\mathsf{Customer}$. With an
object-oriented approach of inheritance as a specialisation relation, we could model
$\mathsf{Customer}$ as a subclass of $\mathsf{Person}$, as not every person is a customer.
On the other hand, if we restrict our domain to a business context and add the concept of a
$\mathsf{Company}$, the inheritance relation would flip and we also have $\mathsf{Company}$ as
subclass of $\mathsf{Customer}$.  This conflict can be resolved by recognising
$\mathsf{Person}$ and $\mathsf{Company}$ as context-independent basic concepts, so-called
\emph{natural types} and $\mathsf{Customer}$ as a role a person or company can play in a
business context. Here, it becomes also apparent that the concept of a \emph{context} is closely
related to a role.

While role-based modelling provides the means to handle and model complex and context-dependent
domains in a well-structured and modular way, the process can still be tedious, hard and
error-prone. Due to the sophisticated semantics of roles, contexts and many different kinds of
constraints, unintended implications or even inconsistencies can easily be hidden within such a
model. Since it is nearly impossible to uncover all inferences, it becomes imperative for domain
analysts to reason on role-based models to find such implicit knowledge.  Here, a feasible logical
formalism is needed.

\section{Description Logics}
\label{sec:intro-description-logics}

Description Logics (DLs) \cite{DLhandbook-07} are a well-known formalism for knowledge
representation. They possess formal semantics and allow to define a variety of reasoning problems.

The basic building blocks in description logics are so-called \emph{concept names} and \emph{role
  names}. Concept names denote sets of domain elements. For example, the concept names
$\mathsf{Person}$ or $\mathsf{Bank}$ denote the sets of all persons or banks in a domain. Relational
structures are represented by so-called DL \emph{role names}, which are essentially binary relations
on the domain. The term ``role'' originates from the early knowledge representation system
KL-ONE~\cite{WoS-CMA133} and has only little in common with roles of role-based systems except that
it reflects the relational property of a role. A person which is related to a bank via a DL role
$\mathsf{customer}$ could be seen as someone playing the role of a customer in the context of a
bank. Besides that, DL roles are merely binary relations. With the help of \emph{concept} and
\emph{role constructors}, complex concepts and roles can be defined.  Which constructors are allowed
depends on the specific DL. Complex concepts can be used as descriptions and to classify domain
elements, e.g.\ the complex concept
\begin{align}
  \label{eq:1-1}
  \mathsf{NFL\_Player} \sqcap \mathsf{Healthy} \sqcap \exists.\mathsf{wins}(\mathsf{NFL\_Game})
\end{align}
describes the set of all healthy NFL players who win NFL games.

With the help of concepts, we can express our knowledge about a domain through \emph{DL
  axioms}. General knowledge is phrased via \emph{general concept inclusions (GCIs)}, which state
that one concept is a sub-concept of another. For example the GCI
\begin{align}
  \label{eq:1-2}
  \mathsf{NFL\_Player} \sqcap \mathsf{Healthy} \sqcap
  \exists.\mathsf{wins}(\mathsf{NFL\_Game}) \sqsubseteq \mathsf{Happy\_NFL\_Player}
\end{align}
states that a healthy NFL player who wins NFL games is a happy NFL player. Conversely, it does not say
anything on whether every happy NFL player is healthy or wins games. Facts about a domain can be
expressed via \emph{concept} and \emph{role assertions}.  To express facts, we also introduce
\emph{individual names} which denote single domain elements. As an example consider the following axioms:
\begin{gather}
  (\mathsf{NFL\_Player} \sqcap \mathsf{Healthy})(\text{\textit{AaronRodgers}})\label{eq:1-3}\\
  \mathsf{NFL\_Game}(\text{\textit{SuperBowlXLV}}),\label{eq:1-4}\\
  \mathsf{\mathsf{wins}}(\text{\textit{AaronRodgers}},\text{\textit{SuperBowlXLV}})\label{eq:1-5}
\end{gather}
The first two concept assertions~\eqref{eq:1-3} and~\eqref{eq:1-4} state that Aaron Rodgers is a healthy NFL player and that Super Bowl~XLV is
an NFL game, while~\eqref{eq:1-5} expresses that he won Super Bowl~XLV. So he is also a happy
NFL player, even if not stated explicitly.  A \emph{DL knowledge base} is a set of such axioms.

The semantics of DLs are defined in a model-theoretic way and capture exactly the above mentioned
intentions.  An interpretation~\I consists of a domain and an interpretation function~$\cdot^{\I}$
which maps concept, role and individual names, respectively, to subsets, binary relations and elements of the
domain.  From there, it is exactly defined how complex concepts must be
interpreted. For example, $(A\sqcap B)^{\I}$ is the intersection of $A^{\I}$ and $B^{\I}$.  The
concept name $\mathsf{NFL\_Player}$ itself has no meaning and an interpretation must make sure that
$\mathsf{NFL\_Player}^{\I}$ actually is the set of all NFL players.

Now, the most interesting reasoning problems are the \emph{consistency problem} and the \emph{entailment
problem}. A knowledge base is consistent if there exists some interpretation that \emph{models} the
knowledge base, i.e.\ an interpretation that fulfils all the axioms. An axiom is \emph{entailed} by
a knowledge base if every model of the knowledge base also models that axiom. For example, the following axiom is
entailed by~\eqref{eq:1-2} to~\eqref{eq:1-5}:
\begin{gather}
  \mathsf{Happy\_NFL\_Player}(\text{\textit{AaronRodgers}}).\label{eq:1-6}
\end{gather}


\begin{figure}
  \centering
  \begin{tikzpicture}
    \node[node, label={[align=right]180:\textsf{AaronRodgers},\\ $\mathsf{Healthy}$,\\
      $\mathsf{NFL\_Player}$,\\$\mathsf{Happy\_NFL\_Player}$}] (ar) at(-2,0) {}; 
    \node[node, label={[align=left]0:\textsf{SuperBowlXLV},\\ $\mathsf{NFL\_Game}$}] (sb45) at (2,0)
    {};
    \draw[edge] (ar) to[bend left=10] node{$\mathsf{wins}$} (sb45);
  \end{tikzpicture}
  \caption{Interpretation that models axioms~\eqref{eq:1-2} to~\eqref{eq:1-6}.}
  \label{fig:dl-example-intro}
\end{figure}

\noindent
Figure~\ref{fig:dl-example-intro} depicts an interpretation which is a model of axioms~\eqref{eq:1-2} to~\eqref{eq:1-6}.

However, classical DLs lack expressive power to formalise that some individuals satisfy certain
concepts and relate to other individuals depending on a \emph{specific context} which is needed to
reason on role-based systems.

\section{Contextualised Description Logics}
\label{sec:intro-contextualised-description-logics}

To overcome that deficiency in expressiveness of classical DLs, often two-dimensional DLs are
employed~\cite{KG-JELIA10,KLGu-DL-11,KlGu-AAAI11,KG16}. This approach uses one DL \LM as the
\emph{meta logic} to represent the contexts and their relationships to each other, and combines it
with the \emph{object logic} \LO that captures the relational structure within each context.
%
Moreover, while some pieces of information depend on the context, other pieces of information are
shared throughout all contexts.  For instance, the name of a person typically stays the same
independent of the actual context.  Expressing this context-independent information requires that
some concepts and roles are designated to be \emph{rigid}, i.e.~they are required to be interpreted
the same in all contexts.  Unfortunately, if rigid roles are admitted, reasoning in the above
mentioned two-dimensional DLs of context turns out to be undecidable; see~\cite{KG-JELIA10}.

We propose and investigate a family of two-dimensional context DLs \LMLO that
meets the above requirements, but is a restricted form of the ones defined in~\cite{KG-JELIA10} in
the sense that we limit the interaction of \LM and \LO.  More precisely, in our family of
context DLs the meta logic can refer to the internal structure of each context, but not vice versa.
That means that information is viewed in a top-down manner, i.e.~information of different contexts
is strictly capsuled and can only be accessed from the meta level.  Hence, we cannot
express, for instance, that everybody who is employed by a company has a certain property in the
context of private life.  We show that reasoning in \LMLO stays decidable with
such a restriction, even in the presence of rigid roles.
%
In some sense this restriction is similar to what is done in~\cite{BaGL-KR08,BaGL-ToCL12,Lip-PhD14}
to obtain a decidable temporalised DL with rigid roles.  

To provide a better intuition on how our formalism works, we examine the following example.  Consider these axioms:
\begin{align}
  \top & \sqsubseteq \oax{\exists\mathsf{worksFor}.\{\text{\textit{Siemens}}\} \sqsubseteq \exists\mathsf{hasAccessRights}.\{\text{\textit{Siemens}}\}}\label{eq:1-7} \\
  \mathsf{Work} & \sqsubseteq \oax{\mathsf{worksFor}(\text{\textit{Bob}},\text{\textit{Siemens}})} \label{eq:1-8}\\
  \oax{(\exists\mathsf{worksFor}.\top)(\text{\textit{Bob}})}\
  & \sqsubseteq\ \exists\mathsf{related}.(\mathsf{Private} \sqcap \oax{\mathsf{HasMoney}(\text{\textit{Bob}})})\label{eq:1-9} \\
  \top\
  & \sqsubseteq\ \oax{\exists\mathsf{isCustomerOf}.\top \sqsubseteq\mathsf{HasMoney}} \label{eq:1-10}\\
  \mathsf{Private}\
  & \sqsubseteq\ \oax{\mathsf{isCustomerOf}(\text{\textit{Bob}},\text{\textit{Siemens}})}\label{eq:1-11}\\
  \mathsf{Private} \sqcap \mathsf{Work}\
  & \sqsubseteq\ \bot\label{eq:1-12}\\
  \lnot \mathsf{Work}\
  & \sqsubseteq\ \oax{\exists\mathsf{worksFor}.\top\sqsubseteq\bot}\label{eq:1-13}
\end{align}
%
The \enquote{outside} or meta GCIs like $\top\sqsubseteq\oax{\ldots}$ express knowledge about the meta dimension whereas
the axioms inside \oax{\ldots} refer to knowledge in the object level. The complex meta concept
\oalpha describes all contexts in which the object axiom $\alpha$ holds.
%
In detail, the first axiom states that in all contexts somebody who
works for Siemens also has access rights to certain data.  The second axiom
states that Bob is an employee of Siemens in any work context.  Furthermore,
Axioms~\eqref{eq:1-9} and~\eqref{eq:1-10} say intuitively that if Bob has a job, he will earn money,
which he can spend as a customer.  Axiom~\eqref{eq:1-11} formalises that Bob is a customer
of Siemens in any private context.  Moreover, Axiom~\eqref{eq:1-12} ensures that the private
contexts are disjoint from the work contexts.  Finally, Axiom~\eqref{eq:1-13} states that
the $\mathsf{worksFor}$ relation only exists in work contexts.

\begin{figure}[t]
  \centering
  \begin{tikzpicture}[auto]
    %\draw[thin,gray] (0,-2) grid (15,3);
    \node[rectangle,
          draw,
          rounded corners= 5mm, 
          %minimum height = 4cm, 
          %minimum width = 6cm,
          label={120:$\mathsf{Work}$}] (a) at (3.5,0){
            \begin{tikzpicture}
              \node[node,label={[align=right]180:\textit{Bob},\\ $\mathsf{Person}$}] (bob) at (0,3){};
              \node[node,label={60:$\mathsf{SSN}$}] (ssn) at (3.7,3){};
              \draw[edge] (bob) to[bend left=10] node{$\mathsf{hasSSN}$} (ssn);
              \node[node,label={[align=right]180:\textit{Siemens},\\ $\mathsf{Company}$}] (siem) at (1,0.5){};
              \draw[edge] (bob) to[bend right=15] node[mylabel,swap]{$\mathsf{worksFor}$} (siem);
              \draw[edge] (bob) to[bend left=15] node[mylabel]{$\mathsf{hasAccessRights}$} (siem);
              \node[node,label={[align=right]north east:$\mathsf{Person}$}] (ceo) at (3.2,0.5){};
              \draw[edge] (siem) to[bend left=10] node{$\mathsf{hasCEO}$} (ceo);
              \node at (1.8,0.2){$\ddots$};
            \end{tikzpicture}
          };
    \node[rectangle,
          draw,
          rounded corners= 5mm,
          %minimum height = 4cm,
          %minimum width = 6cm,
          label={120:$\mathsf{Private}$}] (b) at (11.0,0){
            \begin{tikzpicture}
              \node[node,label={[align=right]180:\textit{Bob},\\ $\mathsf{Person}$,\\ $\mathsf{HasMoney}$}] (bob) at (0,3){};
              \node[node,label={60:$\mathsf{SSN}$}] (ssn) at (3,3){};
              \draw[edge] (bob) to[bend left=10] node{$\mathsf{hasSSN}$} (ssn);
            \node[node,label={[align=right]180:\textit{Siemens},\\ $\mathsf{Company}$}] (siem) at (1,0.5){};
              \draw[edge] (bob) to[bend left=15] node[mylabel]{$\mathsf{isCustomerOf}$} (siem);
              \node[node,label={[align=right]north east:$\mathsf{Person}$}] (ceo) at (2.5,0.5){};
            \end{tikzpicture}
          };
    \draw[edge] (a.east) to[bend left] node{$\mathsf{related}$} (b.west);
  \end{tikzpicture}
  \caption{Nested interpretation that models of Axioms \eqref{eq:1-7}--~\eqref{eq:1-13}}
  \label{fig:example-interpretation}
\end{figure}

\section{An Ontology Generator}
\label{sec:zweite-section}

Besides the capability of context description logics to formalise role-based models, it is rather
hard for domain analysts---who in general are not experts in DLs---to grasp the precise
semantics of the ontology, and to define the contextualised ontology in a way that all entities and
constraints appearing in the role-based model are mapped correctly. Therefore, it would be ideal to
have an algorithm that translates role-based models into context DL knowledge bases.  In this thesis, we
present exactly such a mapping.

First of all, we have to decide how to represent role-based models. Here, we focus on two essential
properties of the modelling language. It is very important that the role-based model is already
equipped with a formal semantics. Otherwise, we cannot ensure that the intended meaning of every
model is correctly translated into the ontology. Furthermore, the modelling language needs to be
expressive enough to express all concepts needed in the role-based model.
%
The \emph{Compartment Role Object Model (CROM)}\cite{KuLG-SLE14,KBG-SLE15} emerged to be a capable
candidate that meets exactly the above mentioned requirements.

Next, there is some freedom on how to express roles and role-playing in an ontology. While it is
important to consider the ontological nature of roles such as identity or rigidity, we also have to
consider practical reasons. Whether some constraints of a role-based model can be expressed in an ontology
highly depends on how roles and other predicates are translated.

On the one hand, we can and will prove the semantical correctness of our translation from role-based
models into an \LMLO ontology. On the other hand, for practical use there must exist some
implementation of the mapping. We based our implementation on the reference implementation for
CROM\footnote{\url{https://github.com/Eden-06/CROM}}, which in turn can be used by
FRaMED\footnote{\url{https://github.com/leondart/FRaMED}}, a graphical editor allowing the
specification of role-based models. In the end, our implemented mapping produces an ontology which
is specially formatted in the Web Ontology Language (OWL).  This leads to the last open part in the
overall workflow.

\section{A Reasoner for Contextualised Description Logics}
\label{sec:intro-reasoner}

A contextualised DL capable of formalising role-based models and an automated mapping from
role-based models into an ontology still helps only very little in practice, if there es no DL reasoner
available which can process such ontologies. 
%
Usually DL reasoners use OWL as language for the input ontology. But OWL in general does not
have the syntactical means to express contextualised DL axioms. However, OWL enables us to annotate
axioms which we use to encode \LMLO-axioms.

Although the reasoning tasks in DLs have a high complexity, DLs have been successfully introduced as
a formalism for knowledge representation. One reason for the success of DLs is the availability of
highly optimised reasoners which makes drawing logical inferences feasible. A black-box approach for
deciding consistency of an \LMLO-ontology could benefit from these optimised reasoners. The main
idea is to divide the consistency problem into separate subproblems each of which can be processed
by a standard DL reasoner.
%
Taking into account the special form of the axioms of the generated ontologies further optimisations
are possible.


% In total, with the work presented in this thesis, it is possible for a domain analyst modelling a
% domain of interest as a role-based model in CROM, to automatically check for inconsistencies of
% unintended implications.

\section{Outline of the Thesis}
\label{sec:outline-thesis}

In the following, we give a short outline how the thesis is structured.

In the first section of Chapter~\ref{cha:preliminaries}, we present the basic definitions of
description logics which we will use throughout the thesis. We define the syntax and semantics of
concepts, axioms and knowledge bases and state specific DLs together with the respective
complexities of the consistency problem.
%
We then present an ontological overview of the notion of a role and introduce
a syntactical variant of the Compartment Role Object Model, the modelling language we use for
role-based modelling.

Chapter~\ref{cha:context-dls} starts with an discussion about the requirements for a logical
formalism in order to be feasible in our setting, and we then introduce the contextualised
description logic \LMLO. We again start with the definition of the syntax and semantics. For the
complexity analysis of the consistency problem, the upper bounds are investigated first. We consider
the case of the lightweight description logic \EL in a separate section, which at the same time
yields the lower bounds of the complexity for more expressive contextualised DLs. For the complexity
analysis, we consider three different settings which result in different complexity results. In the
first setting, we do not allow any rigid names.  In the second setting, we assume the presence of
rigid concepts, but forbid any rigid roles. In the last and complexity-wise hardest setting, we
allow both rigid concepts and rigid names.
%
At last, we introduce an extension of \LMLO, and show that the consistency problem becomes
undecidable in the presence of rigid roles. The main results of this chapter, namely the complexity
results excluding the description logic \SHOIQ, have already been published in~\cite{BoLi-FroCoS15,
  BoLi-DL15, BoLi-LTCS-15-04}:
\begin{itemize}
\item \fullcite{BoLi-FroCoS15}
\item \fullcite{BoLi-DL15}
\item \fullcite{BoLi-LTCS-15-04}
\end{itemize}


In Chapter~\ref{cha:mapping}, we present the mapping from role-based models into \LMLO-ontologies.
We explain in detail how the different assumptions of a CROM are translated and proof that the
mapping algorithm preserves the semantics of the role-based model.  We conclude the chapter with
some thoughts on how to express constraints for role-based models that go beyond CROM.  Both the
formal role-based modelling language CROM and the mapping algorithm have already been published
in~\cite{KBG-SLE15,BoKu-JIST17}:
\begin{itemize}
\item \fullcite{KBG-SLE15}
\item \fullcite{BoKu-JIST17}
\end{itemize}


Finally, in Chapter~\ref{cha:jconht}, we present JConHT, our reasoner for \LMLO. To implement a
decision procedure for the consistency problem, we have to adapt some ideas of
Chapter~\ref{cha:context-dls}, which are discussed in
Section~\ref{sec:blackbox-approach}. Section~\ref{sec:using-hypertableau} covers the analysis of the
contextualised hypertableau algorithm. We finish the chapter with some notes on the implementation
and an evaluation of our implementation. Both the contextualised hypertableau algorithm and a system
description of JConHT are also published in~\cite{BoKu-JIST17}.


% \clearpage

% \section{General ToDos}
% \label{sec:todos}

% 

%\todo[inline]{check for iff}
%\todo[inline]{check for \textasciitilde{} before \textbackslash{}cite}
%\todo[inline]{check for \textasciitilde{} before \Bmc and so on}
%\todo[inline]{check for \textasciitilde{} before \textbackslash{}ref}
%\todo[inline]{check for ``i.e.'' with correct spacing, see \url{https://en.wikibooks.org/wiki/LaTeX/Tips_and_Tricks}}
%\todo[inline]{all referenced chapters, sections, subsections, definitions, lemmas, theorems,
  examples capitalized?}
%\todo[inline]{Capitalized letters in title of definitions (only first letter)}
\todo[inline]{Spacing after macros, i.e.\ dl names and \OCR, etc.}
\todo[inline]{paragraph in chapter 2 about NI, UNA, nominals and  $\top\sqsubseteq\{a\}$ is inconsistent.}
\todo[inline]{search source code for \textbackslash{}hl macro}
\todo[inline]{use \Hmc if it is an \Msig-interpretaion}
\todo[inline]{In section \ref{sec:case-el}, are the \EL-LTL papers correctly cited?}
\todo[inline]{References: Months?}
\todo[inline]{define fresh names in preliminaries}
\todo[inline]{range or image -- be consistent!}
\todo[inline]{natural numbers with 0 or without 0, should that be stated somewhere?}
\todo[inline]{indent after def or proof environment or not}
\todo[inline]{on the object level, on the meta level}
\todo[inline]{allows to falsche}
\todo[inline]{optimised}
\todo[inline]{bei defs immer nur ersten buchstaben gross}
\todo[inline]{References überprüfen}
\todo[inline]{suche nach -ize, analyze -> analyse}
\todo[inline]{Role-Based immer klein geschrieben, außer in title}
\todo[inline]{a and an überprüfen}
\todo[inline]{role und rosirole überprüfen}
\todo[inline]{chapter 4 -> bild role type}



%%% Local Variables:
%%% mode: latex
%%% TeX-master: "../thesis"
%%% End:



%%% Local Variables:
%%% mode: latex
%%% TeX-master: "../thesis"
%%% reftex-default-bibliography: ("../references.bib")
%%% End:

%  LocalWords:  logics Bachman XLV DLs CROM ontologies DL FRaMED workflow Dls subproblems
