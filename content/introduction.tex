

\listoftodos

\chapter{Introduction}
\label{ch:introduction}

Nowadays software systems a literally everywhere. \bulletItem{Aufzählung, Beispiele}. Requirements
users impose on these ... \bulletItem{long-lasting (legacy systems in COBOL, especially in banking
  systems), easy to maintain, but high complexity, (self-)adapting}

The concept of \emph{Roles} can solve these problems. \bulletItem{What is a role} Role-based
systems. When investigating the ontological nature of roles, closely connected to the concept of a
\emph{context}. \bulletItem{What is a context} 


\blindtext

\section{Role-Based Systems}
\label{sec:intro-role-based-systems}

\blindtext

\section{Description Logics}
\label{sec:intro-description-logics}

Description Logics (DLs) are a well-known formalism for knowledge representation

\bulletItem{prominent, DL handbook, example, decidable reasoning tasks, connection to FOL, }

\section{Contextualized Description Logics}
\label{sec:intro-contextualized-description-logics}

\blindtext

\section{Ontology generator}
\label{sec:zweite-section}

\blindtext 

\begin{definition}
  This is a test definition.
\end{definition}

Here is some more text

\begin{definition}[Definition Titel]
  This is some other test definition.
\end{definition}

\begin{theorem*}[thmtitel]
  Her comes a named theorem.
\end{theorem*}

\section{A Reasoner for Contextualized Description Logics}
\label{sec:intro-reasoner}


\blindtext

\begin{theorem}
  And a normal (unnamed) theorem.
\end{theorem}


\begin{proof}
  A very long proof to test the claim-environment.

  \blindtext

  \begin{claim}
    Here comes the claim
  \end{claim}

  \begin{claimproof}{}
    and the proof of the claim.
  \end{claimproof}
  
  the proof continues
  \blindtext
  and END!
\end{proof}

\blindtext

\begin{proof}[proofheading]
 some named proof 
\end{proof}


\clearpage

\section{General ToDos}
\label{sec:todos}



%\todo[inline]{check for iff}
%\todo[inline]{check for \textasciitilde{} before \textbackslash{}cite}
%\todo[inline]{check for \textasciitilde{} before \Bmc and so on}
%\todo[inline]{check for \textasciitilde{} before \textbackslash{}ref}
%\todo[inline]{check for ``i.e.'' with correct spacing, see \url{https://en.wikibooks.org/wiki/LaTeX/Tips_and_Tricks}}
%\todo[inline]{all referenced chapters, sections, subsections, definitions, lemmas, theorems,
  examples capitalized?}
%\todo[inline]{Capitalized letters in title of definitions (only first letter)}
\todo[inline]{Spacing after macros, i.e.\ dl names and \OCR, etc.}
\todo[inline]{paragraph in chapter 2 about NI, UNA, nominals and  $\top\sqsubseteq\{a\}$ is inconsistent.}
\todo[inline]{search source code for \textbackslash{}hl macro}
\todo[inline]{use \Hmc if it is an \Msig-interpretaion}
\todo[inline]{In section \ref{sec:case-el}, are the \EL-LTL papers correctly cited?}
\todo[inline]{References: Months?}
\todo[inline]{define fresh names in preliminaries}
\todo[inline]{range or image -- be consistent!}
\todo[inline]{natural numbers with 0 or without 0, should that be stated somewhere?}
\todo[inline]{indent after def or proof environment or not}
\todo[inline]{on the object level, on the meta level}
\todo[inline]{allows to falsche}
\todo[inline]{optimised}
\todo[inline]{bei defs immer nur ersten buchstaben gross}
\todo[inline]{References überprüfen}
\todo[inline]{suche nach -ize, analyze -> analyse}
\todo[inline]{Role-Based immer klein geschrieben, außer in title}
\todo[inline]{a and an überprüfen}
\todo[inline]{role und rosirole überprüfen}
\todo[inline]{chapter 4 -> bild role type}



%%% Local Variables:
%%% mode: latex
%%% TeX-master: "../thesis"
%%% End:



%%% Local Variables:
%%% mode: latex
%%% TeX-master: "../thesis"
%%% End:

%  LocalWords:  logics
