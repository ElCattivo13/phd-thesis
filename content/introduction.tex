

\listoftodos

\chapter{Introduction}
\label{ch:introduction}

Nowadays, we are surrounded by software systems literally everywhere. With regard to current
developments there will be even more in the future. Not only the amount increases, but also the
requirements and expectations users impose on current software steadily rise. Modern software
systems should cope with complex scenarios. This includes the ability of context-awareness and
self-adaptability. For example, a robot in a smart factory should recognize when a human coworker
approaches and switches to a different, human-friendly working mode accordingly. Similarly, software
in autonomous cars or an the area of smart homes needs to adapt 


Besides that there is still plenty of software running in legacy code. Especially in areas
where the correctness of the software is utterly important, e.g., in banking applications, a large
amount of old 

These software system are very hard to maintain


To sum up, current software systems long-lasting, high complexity, easy to maintain.



\bulletItem{Aufzählung, Beispiele}. Requirements
users impose on these ... \bulletItem{long-lasting (legacy systems in COBOL, especially in banking
  systems), easy to maintain, but high complexity, (self-)adapting}




\section{Role-Based Systems}
\label{sec:intro-role-based-systems}

The concept of \emph{Roles} was first imposed


could solve these problems. \bulletItem{What is a role} Role-based
systems. When investigating the ontological nature of roles, closely connected to the concept of a
\emph{context}. \bulletItem{What is a context} 


Role-based


While role-based modelling provides the means to handle and model complex and context-dependent
domains, ...

Besides the means of role-based models to aid modelling complex and context-dependent
domains, the process is still tedius and hard. 

Many constraints, unintended implications or even unconsistencies. imparative for domain analysts to reason

reasoning

Here, a feasible logical formalism is needed.

\section{Description Logics}
\label{sec:intro-description-logics}

Description Logics (DLs) are a well-known formalism for knowledge representation

\bulletItem{prominent, DL handbook, example, decidable reasoning tasks, connection to FOL, }


The basic building blocks in description logics are so-called \emph{concepts}. 

As described

used as description, classify a set of objects
\todo[inline]{what are concepts and for what are they good}

\section{Contextualized Description Logics}
\label{sec:intro-contextualized-description-logics}

\todo[inline]{ConDL with top down view, first describe contexts, then what's going on inside. from
  within we can't refer to other contexts, trade-off between rigid roles and contextual concepts}

\blindtext

\section{Ontology generator}
\label{sec:zweite-section}

Besides capability of contextualized description logics, it is rather hard for domain analysts who
in general are not experts in the area of DLs to capture the precise semantics of the ontology and model the
contextual ontology correctly. Therefore, it would be ideal 

evolved and elaborate semantics of condl

we present mapping which captures most parts of CROM

forfeit expressiveness, conDL can express more than CROM


\blindtext 


\section{A Reasoner for Contextualized Description Logics}
\label{sec:intro-reasoner}


\blindtext




\clearpage

\section{General ToDos}
\label{sec:todos}



%\todo[inline]{check for iff}
%\todo[inline]{check for \textasciitilde{} before \textbackslash{}cite}
%\todo[inline]{check for \textasciitilde{} before \Bmc and so on}
%\todo[inline]{check for \textasciitilde{} before \textbackslash{}ref}
%\todo[inline]{check for ``i.e.'' with correct spacing, see \url{https://en.wikibooks.org/wiki/LaTeX/Tips_and_Tricks}}
%\todo[inline]{all referenced chapters, sections, subsections, definitions, lemmas, theorems,
  examples capitalized?}
%\todo[inline]{Capitalized letters in title of definitions (only first letter)}
\todo[inline]{Spacing after macros, i.e.\ dl names and \OCR, etc.}
\todo[inline]{paragraph in chapter 2 about NI, UNA, nominals and  $\top\sqsubseteq\{a\}$ is inconsistent.}
\todo[inline]{search source code for \textbackslash{}hl macro}
\todo[inline]{use \Hmc if it is an \Msig-interpretaion}
\todo[inline]{In section \ref{sec:case-el}, are the \EL-LTL papers correctly cited?}
\todo[inline]{References: Months?}
\todo[inline]{define fresh names in preliminaries}
\todo[inline]{range or image -- be consistent!}
\todo[inline]{natural numbers with 0 or without 0, should that be stated somewhere?}
\todo[inline]{indent after def or proof environment or not}
\todo[inline]{on the object level, on the meta level}
\todo[inline]{allows to falsche}
\todo[inline]{optimised}
\todo[inline]{bei defs immer nur ersten buchstaben gross}
\todo[inline]{References überprüfen}
\todo[inline]{suche nach -ize, analyze -> analyse}
\todo[inline]{Role-Based immer klein geschrieben, außer in title}
\todo[inline]{a and an überprüfen}
\todo[inline]{role und rosirole überprüfen}
\todo[inline]{chapter 4 -> bild role type}



%%% Local Variables:
%%% mode: latex
%%% TeX-master: "../thesis"
%%% End:



%%% Local Variables:
%%% mode: latex
%%% TeX-master: "../thesis"
%%% End:

%  LocalWords:  logics
