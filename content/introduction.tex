

\listoftodos

\chapter{Introduction}
\label{ch:introduction}

Nowadays, we are surrounded by software systems literally everywhere. With regard to current
developments there will be even more in the future. Not only the amount of systems increases, but also the
requirements and expectations users impose on current software steadily rise. Modern software
systems should cope with very complex scenarios. This includes the ability of context-awareness and
self-adaptability. For example, a robot in a smart factory should recognize when a human coworker
approaches and switches to a different, human-friendly working mode accordingly. Similarly, software
in autonomous cars or an the area of smart homes needs to adapt to various situations of which some
are not even stated explicitly. 
%
Furthermore, software must be easily maintainable and, when necessary, changes on the system should be realized
without much down time which, for example, in a smart factory is very costly. 

% To sum up, current software systems long-lasting, high complexity, easy to maintain.



% \bulletItem{Aufzählung, Beispiele}. Requirements
% users impose on these ... \bulletItem{long-lasting (legacy systems in COBOL, especially in banking
%   systems), easy to maintain, but high complexity, (self-)adapting}




\section{Role-Based Systems}
\label{sec:intro-role-based-systems}

In order to achieve all these goals, the concept of \emph{roles} is very promising. First introduced
by Bachman~\cite{BaD-VLDB77}, roles appeared over the last decades in several fields of computer
science\todo{citations}. The relational or context-dependent properties and behaviour of objects is
transferred into the roles that object plays in a certain context. These paradigm also supports
Dijkstra's separation of concerns~\cite{Dij-SelWrCom82} which simplifies development and maintenance
of such systems.  Due to the use of roles, \emph{role-based systems} can cleaner and more structured
model application domains since ontologically different entities are modelled by different concepts.

Consider, for example, the concepts of $\mathsf{Person}$ and $\mathsf{Customer}$. With an
object-oriented approach of inheritance as a specialization relation, we could model
$\mathsf{Customer}$ as a subclass of $\mathsf{Person}$, as obviously not every person is a customer.
On the other hand, if we restrict our domain to a business context and add the concept of a
$\mathsf{Company}$, the inheritance relation would flip and we also have $\mathsf{Company}$ as
subclass of $\mathsf{Customer}$.  This conflict can be resolved when we realize that
$\mathsf{Person}$ and $\mathsf{Company}$ are context-independent basic concepts, so-called
\emph{natural types} and $\mathsf{Customer}$ is actually a role a person or company can play in a
business context. Here, it becomes also apparent, that the concept of a \emph{context} is closely
related to a role.

While role-based modelling provides the means to handle and model complex and context-dependent
domains in a well-structured and modular way, the process can still be tedious, hard and
error-prone. Due to the sophisticated semantics of roles, contexts and many different kinds of
constraints, unintended implications or even inconsistencies can easily be hidden within such a
model. Since it is nearly impossible to uncover all inferences, it becomes imperative for domain
analysts to reason on role-based models to find such implicit knowledge.  Here, a feasible logical
formalism is needed.

\section{Description Logics}
\label{sec:intro-description-logics}

Description Logics (DLs) \cite{DLhandbook-07} are a well-known formalism for knowledge
representation. They possess formal semantics and allow to define a variety of reasoning problems.

The basic building blocks in description logics are so-called \emph{concept names} and \emph{role
  names}. Concept names denote sets of domain elements. For example the concept names
$\mathsf{Person}$ or $\mathsf{Bank}$ denote the sets of all persons or banks in a domain. Relational
structures are represented by so-called DL \emph{role names}, which are essentially binary relations
on the the domain. The term ``role'' originates from the early knowledge representation system
KL-ONE~\cite{WoS-CMA133} and has only little in common with roles of role-based systems except that
it reflects the relational property of a role. A person which is related to a bank via a DL role
$\mathsf{customer}$ could be seen as someone playing the role of a customer in the context of a
bank. Besides that, DL roles are merely binary relations. With the help of \emph{concept} and
\emph{role constructors}, complex concepts and roles can be defined.  Which constructors are allowed
depends on the specific DL. Complex concepts can be used as descriptions  or to
classify domain elements, e.g.\ the complex concept
\begin{align}
  \label{eq:1-1}
  \mathsf{NFL\_Player} \sqcap \mathsf{Healthy} \sqcap \exists.\mathsf{wins}(\mathsf{NFL\_Game})
\end{align}
describes the set of all healthy NFL players who win NFL games.

With the help of concepts we can express our knowledge about a domain through \emph{DL
  axioms}. General knowledge is phrased via \emph{general concept inclusions (GCIs)} which state
that one concept is a sub-concept of another. For example the GCI
\begin{align}
  \label{eq:1-2}
  \mathsf{NFL\_Player} \sqcap \mathsf{Healthy} \sqcap
  \exists.\mathsf{wins}(\mathsf{NFL\_Game}) \sqsubseteq \mathsf{Happy\_NFL\_Player}
\end{align}
states that a healthy NFL player who wins NFL games is a happy NFL player. Conversely, it does not say
anything whether every happy NFL player is healthy or wins games. Facts about a domain can be
expressed via \emph{concept} and \emph{role assertions}.  To express facts, we also introduce
\emph{individual names} which denote single domain elements. As an example consider the following axioms:
\begin{gather}
  (\mathsf{NFL\_Player} \sqcap \mathsf{Healthy})(\text{\textit{AaronRodgers}})\label{eq:1-3}\\
  \mathsf{NFL\_Game}(\text{\textit{SuperBowlXLV}}),\label{eq:1-4}\\
  \mathsf{\mathsf{wins}}(\text{\textit{AaronRodgers}},\text{\textit{SuperBowlXLV}})\label{eq:1-5}
\end{gather}
The first two concept assertions~\eqref{eq:1-3} and~\eqref{eq:1-4} state that Aaron Rodgers is a healthy NFL player and that Super Bowl~XLV is
an NFL game, while~\eqref{eq:1-5} expresses that he won Super Bowl~XLV. So he is also a happy
NFL player, even if not stated explicitly.  A \emph{DL knowledge base} is a set of such axioms.

The semantics of DLs are defined in a model-theoretic way and capture exactly the above mentioned
intentions.  An interpretation~\I consists of a domain and an interpretation function~$\cdot^{\I}$
which maps concept, role and individual names, respectively, to subsets, binary relations and elements of the
domain.  From there, it is exactly defined how complex concepts must be
interpreted. For example, $(A\sqcap B)^{\I}$ is the intersection of $A^{\I}$ and $B^{\I}$.  The
concept name $\mathsf{NFL\_Player}$ itself has no meaning and an interpretation must make sure that
$\mathsf{NFL\_Player}^{\I}$ actually is the set of all NFL players.

Now, the most interesting reasoning problems are the \emph{consistency problem} and the \emph{entailment
problem}. A knowledge base is consistent if there exists some interpretation that \emph{models} the
knowledge base, i.e.\ an interpretation that fulfills all the axioms. An axiom is \emph{entailed} by
a knowledge base if every model of the knowledge base also models that axiom. For example, the following axiom is
entailed by~\eqref{eq:1-2} to~\eqref{eq:1-5}:
\begin{gather}
  \mathsf{Happy\_NFL\_Player}(\text{\textit{AaronRodgers}}).\label{eq:1-6}
\end{gather}


\begin{figure}
  \centering
  \begin{tikzpicture}
    \node[node, label={[align=right]180:\textsf{AaronRodgers},\\ $\mathsf{Healthy}$,\\
      $\mathsf{NFL\_Player}$,\\$\mathsf{Happy\_NFL\_Player}$}] (ar) at(-2,0) {}; 
    \node[node, label={[align=left]0:\textsf{SuberBowlXLV},\\ $\mathsf{NFL\_Game}$}] (sb45) at (2,0)
    {};
    \draw[edge] (ar) to[bend left=10] node{$\mathsf{wins}$} (sb45);
  \end{tikzpicture}
  \caption{Interpretation that models axioms~\eqref{eq:1-2} to~\eqref{eq:1-6}.}
  \label{fig:dl-example-intro}
\end{figure}

\noindent
Figure~\ref{fig:dl-example-intro} depicts an interpretation which is a model of axioms~\eqref{eq:1-2} to~\eqref{eq:1-6}.


\section{Contextualized Description Logics}
\label{sec:intro-contextualized-description-logics}

However, classical DLs lack expressive power to formalize furthermore that some individuals satisfy
certain concepts and relate to other individuals depending on a specific context which is needed to
reason role-based systems.  Therefore, often
two-dimensional DLs are employed~\cite{KG-JELIA10,KLGu-DL-11,KlGu-AAAI11,KG16}. The approach is to
have one DL \LM as the \emph{meta logic} to represent the contexts and their relationships to each
other. This logic is combined with the \emph{object logic} \LO that captures the relational
structure within each of the contexts.
%
Moreover, while some pieces of information depend on the context, other pieces of information are
shared throughout all contexts.  For instance, the name of a person typically stays the same
independent of the actual context.  To be able to express that, some concepts and roles are designated
to be \emph{rigid}, i.e.~they are required to be interpreted the same in all contexts.
Unfortunately, if rigid roles are admitted, reasoning in the above mentioned two-dimensional DLs of
context turns out to be undecidable; see~\cite{KG-JELIA10}.

We propose and investigate a family of two-dimensional context DLs \LMLO that
meets the above requirements, but is a restricted form of the ones defined in~\cite{KG-JELIA10} in
the sense that we limit the interaction of \LM and \LO.  More precisely, in our family of
context DLs the meta logic can refer to the internal structure of each context, but not vice versa.
That means that information is viewed in a top-down manner, i.e.~information of different contexts
is strictly capsuled and can only be accessed from the meta level.  Hence, we cannot
express, for instance, that everybody who is employed by Siemens has a certain property in the
context of private life.  Interestingly, reasoning in \LMLO stays decidable with
such a restriction, even in the presence of rigid roles.
%
In some sense this restriction is similar to what was done in~\cite{BaGL-KR08,BaGL-ToCL12,Lip-PhD14}
to obtain a decidable temporalized DL with rigid roles.  

For providing better intuition on how our formalism works, we examine the above mentioned example a bit
further.  Consider the following axioms:
\begin{align}
  \top & \sqsubseteq \oax{\exists\mathsf{worksFor}.\{\text{\textit{Siemens}}\} \sqsubseteq \exists\mathsf{hasAccessRights}.\{\text{\textit{Siemens}}\}}\label{eq:1-7} \\
  \mathsf{Work} & \sqsubseteq \oax{\mathsf{worksFor}(\text{\textit{Bob}},\text{\textit{Siemens}})} \label{eq:1-8}\\
  \oax{(\exists\mathsf{worksFor}.\top)(\text{\textit{Bob}})}\
  & \sqsubseteq\ \exists\mathsf{related}.(\mathsf{Private} \sqcap \oax{\mathsf{HasMoney}(\text{\textit{Bob}})})\label{eq:1-9} \\
  \top\
  & \sqsubseteq\ \oax{\exists\mathsf{isCustomerOf}.\top \sqsubseteq\mathsf{HasMoney}} \label{eq:1-10}\\
  \mathsf{Private}\
  & \sqsubseteq\ \oax{\mathsf{isCustomerOf}(\text{\textit{Bob}},\text{\textit{Siemens}})}\label{eq:1-11}\\
  \mathsf{Private} \sqcap \mathsf{Work}\
  & \sqsubseteq\ \bot\label{eq:1-12}\\
  \lnot \mathsf{Work}\
  & \sqsubseteq\ \oax{\exists\mathsf{worksFor}.\top\sqsubseteq\bot}\label{eq:1-13}
\end{align}
%
The first axiom states that in all contexts somebody who
works for Siemens also has access rights to certain data.  The second axiom
states that Bob is an employee of Siemens in any work context.  Furthermore,
Axioms~\eqref{eq:1-9} and~\eqref{eq:1-10} say intuitively that if Bob has a job, he will earn money,
which he can spend as a customer.  Axiom~\eqref{eq:1-11} formalises that Bob is a customer
of Siemens in any private context.  Moreover, Axiom~\eqref{eq:1-12} ensures that the private
contexts are disjoint from the work contexts.  Finally, Axiom~\eqref{eq:1-13} states that
the $\mathsf{worksFor}$ relation only exists in work contexts.

\begin{figure}[t]
  \centering
  \begin{tikzpicture}[auto]
    %\draw[thin,gray] (0,-2) grid (15,3);
    \node[rectangle,
          draw,
          rounded corners= 5mm, 
          %minimum height = 4cm, 
          %minimum width = 6cm,
          label={120:$\mathsf{Work}$}] (a) at (3.5,0){
            \begin{tikzpicture}
              \node[node,label={[align=right]180:\textit{Bob},\\ $\mathsf{Person}$}] (bob) at (0,3){};
              \node[node,label={60:$\mathsf{SSN}$}] (ssn) at (3.7,3){};
              \draw[edge] (bob) to[bend left=10] node{$\mathsf{hasSSN}$} (ssn);
              \node[node,label={[align=right]180:\textit{Siemens},\\ $\mathsf{Company}$}] (siem) at (1,0.5){};
              \draw[edge] (bob) to[bend right=15] node[mylabel,swap]{$\mathsf{worksFor}$} (siem);
              \draw[edge] (bob) to[bend left=15] node[mylabel]{$\mathsf{hasAccessRights}$} (siem);
              \node[node,label={[align=right]north east:$\mathsf{Person}$}] (ceo) at (3.2,0.5){};
              \draw[edge] (siem) to[bend left=10] node{$\mathsf{hasCEO}$} (ceo);
              \node at (1.8,0.2){$\ddots$};
            \end{tikzpicture}
          };
    \node[rectangle,
          draw,
          rounded corners= 5mm,
          %minimum height = 4cm,
          %minimum width = 6cm,
          label={120:$\mathsf{Private}$}] (b) at (11.0,0){
            \begin{tikzpicture}
              \node[node,label={[align=right]180:\textit{Bob},\\ $\mathsf{Person}$,\\ $\mathsf{HasMoney}$}] (bob) at (0,3){};
              \node[node,label={60:$\mathsf{SSN}$}] (ssn) at (3,3){};
              \draw[edge] (bob) to[bend left=10] node{$\mathsf{hasSSN}$} (ssn);
            \node[node,label={[align=right]180:\textit{Siemens},\\ $\mathsf{Company}$}] (siem) at (1,0.5){};
              \draw[edge] (bob) to[bend left=15] node[mylabel]{$\mathsf{isCustomerOf}$} (siem);
              \node[node,label={[align=right]north east:$\mathsf{Person}$}] (ceo) at (2.5,0.5){};
            \end{tikzpicture}
          };
    \draw[edge] (a.east) to[bend left] node{$\mathsf{related}$} (b.west);
  \end{tikzpicture}
  \caption{Nested interpretation that models of Axioms \eqref{eq:1-7}--~\eqref{eq:1-13}}
  \label{fig:example-interpretation}
\end{figure}

\section{An Ontology Generator}
\label{sec:zweite-section}

Besides capability of context description logics, it is rather hard for domain analysts who
in general are not experts in the area of DLs to grasp the precise semantics of the ontology and define the
contextual ontology in a way that all entities and constraints appearing in the role-based model are
mapped correctly. Therefore, it would be ideal to have a mapping algorithm from role-based models
into context DL knowledge bases.  In this thesis, we present exactly such a mapping.

First of all, we have to decide how to represent role-based models. Here, we focus on two essential
properties of the representation. It is very important that the original representation of the
role-based model is already equipped with a formal semantics. Otherwise, we cannot ensure that the
intended meaning of some model is correctly translated into the ontology. Furthermore, the
representation needs the expressive power to actually grasp all concepts needed in the role-based
model. 

The \emph{Compartment Object Role Model (CROM)}\cite{KuLG-SLE14,KBG-SLE15} emerged to be a potent candidate
that meets exactly the above mentioned requirements.   


There exists graphical editor for CROM, FRAMED, 

We implemented the mapping within 


specially formatted OWL ontology


\section{A Reasoner for Contextualized Description Logics}
\label{sec:intro-reasoner}




\clearpage

% \section{General ToDos}
% \label{sec:todos}



\todo[inline]{check for iff}
\todo[inline]{check for \textasciitilde{} before \textbackslash{}cite}
\todo[inline]{check for \textasciitilde{} before \Bmc and so on}
\todo[inline]{check for \textasciitilde{} before \textbackslash{}ref}
\todo[inline]{check for ``i.e.'' with correct spacing, see \url{https://en.wikibooks.org/wiki/LaTeX/Tips_and_Tricks}}
\todo[inline]{all referenced chapters, sections, subsections, definitions, lemmas, theorems,
  examples capitalized?}
\todo[inline]{Capitalized letters in title of definitions (only first letter)}
\todo[inline]{Spacing after macros, i.e.\ dl names and \OCR, etc.}
\todo[inline]{paragraph in chapter 2 about NI, UNA, nominals and  $\top\sqsubseteq\{a\}$ is inconsistent.}
\todo[inline]{search source code for \textbackslash{}hl macro}
\todo[inline]{use \Hmc if it is an \Msig-interpretaion}
\todo[inline]{In section \ref{sec:case-el}, are the \EL-LTL papers correctly cited?}
\todo[inline]{References: Months?}
\todo[inline]{define fresh names in preliminaries}
\todo[inline]{range or image -- be consistent!}
\todo[inline]{natural numbers with 0 or without 0, should that be stated somewhere?}
\todo[inline]{indent after def or proof environment or not}
\todo[inline]{on the object level, on the meta level}
\todo[inline]{allows to falsche}
\todo[inline]{optimised}
\todo[inline]{bei defs immer nur ersten buchstaben gross}

%%% Local Variables:
%%% mode: latex
%%% TeX-master: "../thesis"
%%% End:



%%% Local Variables:
%%% mode: latex
%%% TeX-master: "../thesis"
%%% reftex-default-bibliography: ("../references.bib")
%%% End:

%  LocalWords:  logics Bachman XLV DLs CROM
