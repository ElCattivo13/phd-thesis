\chapter{Conclusions}
\label{cha:conclusions}


\section{Major Contributions}
\label{sec:major-contributions}

In this thesis we presented an overall workflow to reason on role-based models. Proper formalization
of contexts is crucial for role-based systems, but logical formalisms able to express these, easily
tend to become undecidable.  We introduced a novel family of description logics which is capable of
expressing contextual knowledge, even in the presence of rigid roles, i.e.\ relational knowledge
which is context-independent.  For these contextual description logics we did a thorough analysis on
the complexity of the consistency problem where we investigated different settings depending on
whether rigid role names or rigid concept names are admitted. It could be shown that for the least
expressive setting, where no rigid names are allowed at all, the complexity class of the consistency
problem does not increase compared to the non-contextual version of that DL, namely the consistency
problem is \ExpTime-complete up to \SHOQSHOQ and \NExpTime-complete for \SHOIQSHOIQ. On the other
hand, allowing rigid roles, which often causes undecidability in other approaches, only increases
the complexity by one exponential.
%
We also looked into a broad variety of description logics
ranging from the lightweight DL \EL up to the very expressive DL \SHOIQ and, hence, obtained a
nearly complete map of complexity results.

But the purpose of this thesis was not only to theoretically investigate a logical formalism capable
of reasoning on role-based models. We also presented a mapping from the formal role-based modelling
language CROM into contextual DL ontologies. An implementation of this mapping is part of the CROM
implementation\todo{cite}. We proved that the formal semantics of the role-based model is preserved
by the mapping and introduced further constraints that go beyond the current capabilities of CROM.

Finally, we implemented a reasoner for our contextual description logics which is based on the
highly optimized existing DL reasoner HermiT~\cite{GHM-JAR14}.  While analyzing the complexity of
the consistency problem, we could show that deciding consistency could be split up into two
subtasks. We further refined these subproblems, so they could be processed by HermiT.  Due to the
special form of the context ontology derived from CROM models, we could introduce a further
optimization step and showed its semantic correctness.




\section{Future Work}
\label{sec:future-work}

As mentioned earlier we focused in this thesis on the overall workflow, but we also see several
linking points for future research. When investigating \LMLO we mainly focused on the consistency
problem as most important for our goal. But besides that, interesting work in the direction of query
answering awaits to be investigated. Recently, there has been done a lot of work in the area of
temporal query answering, and as our approach shares a similar setting, we are sure, that there is
plenty motivation, applications and methods available to analyze contextual query answering. There
is also current work on role-based databases, and due to the close connection of query answering and
database theory, there also might arise interesting collaborations.

Going back to the consistency problem, we still think one can narrow the gap to undecidability. We
added contextual concepts to \LMLO which results in undecidability in the presence of rigid
roles. Quite certain, there are other, probably more restrictive, means to further extend the
expressive power of the logic and stay decidable.

Another extension to \LMLO could be in the area of temporal logics. The combinations of DLs with
temporal logics, point-based or interval, are well understood. So why not also combine temporal
logics with contextual DLs. On a more abstract level, it might be even possible to analyze common
properties of these combinations and deduce an abstract combination of DLs with itself or with other
logics. Then temporal DLs or contextual DLs could be an instance of that abstract combination.

The mapping for role-based models we presented is based on the \emph{Compartment Role Object Model
  (CROM)}. Since CROM might change in future, for example to add new kinds of constraints, there is
always some future work to analyse these upcoming features and investigate whether they can also be
represented by a contextual DL ontology.  Besides that, an investigation where contextual DLs can be
used except for CROM will help to detect any missing expressiveness of \LMLO if existent.

The last starting point for future work would be the reasoner. While we used a black box approach,
combined or integrated (tableaux) algorithms for deciding consistency are conceivable. A different
optimization would be an even more target oriented reasoner, which behaves especially well for
contextual ontologies produced from CROM models. Since, for example, validation of role groups is
rather of combinatorial nature which is quite hard for DL reasoners, it might be useful to use SAT
solvers internally to improve overall performance.




%%% Local Variables:
%%% mode: latex
%%% TeX-master: "../thesis"
%%% reftex-default-bibliography: ("../references.bib")
%%% End:

%  LocalWords:  workflow logics DL DLs ontologies LocalWords subtasks HermiT 
