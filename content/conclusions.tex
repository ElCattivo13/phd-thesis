\chapter{Conclusions}
\label{cha:conclusions}


\section{Major Contributions}
\label{sec:major-contributions}

In this thesis, we presented an overall workflow to reason on role-based models. Proper
formalization of contexts is crucial for role-based systems, but logical formalisms able to express
these, easily tend to become undecidable.  We introduced a novel family of description logics that
is capable of expressing contextual knowledge, even in the presence of rigid roles, i.e.\ relational
knowledge that is context-independent.  For these contextualised description logics we did a
thorough analysis on the complexity of the consistency problem, for which we investigated different
settings depending on whether rigid role names or rigid concept names are admitted. We showed that
for the least expressive setting, in which no rigid names are allowed, the complexity class of the
consistency problem does not increase compared to the non-contextual version of that DL, namely the
consistency problem is \ExpTime-complete up to \SHOQSHOQ and \NExpTime-complete for \SHOIQSHOIQ. On
the other hand, allowing rigid roles, which often causes undecidability in other approaches, only
increases the complexity by one exponential.
%
We also looked into a broad variety of description logics
ranging from the lightweight DL \EL up to the very expressive DL \SHOIQ and, hence, obtained a
nearly complete map of complexity results.

But the purpose of this thesis was not only to theoretically investigate a logical formalism capable
of reasoning on role-based models. We also presented a mapping from the formal role-based modelling
language CROM into contextual DL ontologies. An implementation of this mapping is part of the CROM
implementation. We proved that the formal semantics of the role-based model is preserved
by the mapping and introduced further constraints that go beyond the current capabilities of CROM.

Finally, we implemented a reasoner for our contextual description logics that is based on the highly
optimised existing DL reasoner HermiT~\cite{GHM-JAR14}.  During our analysis of the complexity of
the consistency problem, we showed that deciding consistency could be split up into two
subtasks. This idea was also used in our implementation.  We further refined these subproblems, so
they could be processed by HermiT.  Due to the special form of the context ontology derived from
CROM models, we could introduce a further optimization step and showed its semantic correctness.




\section{Future Work}
\label{sec:future-work}

As mentioned earlier, the focus of this thesis was on the overall workflow of modelling and
reasoning about context-based domain models, but we also observe several linking points for future
research. When investigating \LMLO, we mainly focused on the consistency problem, which was central
for our goal. Besides that, query answering with context DLs would be an interesting direction for
further investigations. Recently, there has been a lot of work in the area of temporal query
answering. As our approach shares a similar setting, we are sure that there is plenty of motivation,
applications and methods available to analyse contextual query answering. Furthermore, there is
current work on role-based databases~\cite{JaKV-ADBIS16}, i.e. database systems that are based on a
conceptual, role-based data model to natively represent complex data. Due to the close connection of
query answering and database theory, it might be worth investigating query answering involving
role-based database systems.

Going back to the consistency problem, we still think one can narrow the gap to undecidability. We
added contextual concepts to \LMLO which results in undecidability in the presence of rigid
roles. Quite certain, there are other, probably more restrictive, means to further extend the
expressive power of the logic and stay decidable.

Another extension to \LMLO could be towards temporal logics. The combinations of DLs with
temporal logics, point-based or interval, are well understood. As both temporal and contextual DLs
adopt a possible worlds semantics, it seems natural to also combine temporal logics with contextual
DLs. On a more abstract level, it might be even possible to analyse common properties of these
combinations and deduce an abstract combination of DLs with itself or with other logics. Then
temporal DLs or contextual DLs could be an instance of that abstract combination.

The presented mapping for role-based models is based on the \emph{Compartment Role Object Model
  (CROM)}. As CROM might be extended in the future, for example by new kinds of constraints,
there is always some future work to analyse these upcoming features and to investigate whether they can
also be represented by a contextualised DL ontology.  Besides that, an investigation where contextual
DLs can be used except for CROM will help to detect any missing expressiveness of \LMLO if existent.

The last starting point for future work would be the reasoner. While we used a black box approach,
combined or integrated (tableaux) algorithms for deciding consistency are conceivable. A different
optimisation would be an even more goal-oriented reasoner, which behaves especially well for
contextual ontologies produced from CROM models. Since, for example, the validation of role groups is
rather of combinatorial nature which is quite hard for DL reasoners, it might be useful to use SAT
solvers internally to improve overall performance.




%%% Local Variables:
%%% mode: latex
%%% TeX-master: "../thesis"
%%% reftex-default-bibliography: ("../references.bib")
%%% End:

%  LocalWords:  workflow logics DL DLs ontologies LocalWords subtasks HermiT  natively
