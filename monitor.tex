\chapter[Runtime Verification Using \texorpdfstring{\SHOQ-LTL}{SHOQ-LTL}]{%
    Runtime Verification Using the Temporalised Description Logic \texorpdfstring{\SHOQ-LTL}{SHOQ-LTL}}\label{ch:monitor}

Runtime verification deals with the problem of verifying properties about the
behaviour of observed systems.
%
In this chapter, we investigate runtime verification using the temporalised
description logic \SHOQ-LTL\@.  We show how monitors for \SHOQ-LTL-formulas can
be constructed and establish complexity results for related decision problems.
%
Some of the results of this chapter have already been published
in~\cite{BaLi-LTCS-14-01}.

This chapter is organised as follows.  In Section~\ref{sec:monitor-ltl}, we
first consider propositional runtime verification, and revisit known results to
be able to compare them to one we establish.  Then, in
Section~\ref{sec:ba-for-shoq-ltl} we show how to construct Büchi-automata for
\SHOQ-LTL-formulas.  This section is very related to Chapter~\ref{ch:shoq-ltl}
where we established complexity results for the satisfiability problem in
\SHOQ-LTL\@.  After that, we consider the actual monitor construction for
\SHOQ-LTL in Section~\ref{sec:monitor}.  Then, in
Section~\ref{sec:liveness-monitorability}, we show some complexity results about
the important related decision problems \enquote{liveness} and
\enquote{monitorability}.  Finally, Section~\ref{sec:monitor-summary} gives a
brief summary of the main results of this chapter.


\section{Runtime Verification Using Propositional LTL}\label{sec:monitor-ltl}

In propositional runtime verification~\cite{BaLS-JLC10,BaLS-ToSEM11}, one
observes the actual behaviour of the given system since it started, which at any
point in time can be described by a finite word~$u$ over~$\Sigma_\Pmc$.  Here
\Pmc is a finite set of propositional variables whose truth values at any point
in time can be determined by observing the system.
%
Given such a word $u=u_0u_1\dots u_t\in\Sigma_\Pmc^*$, we say that the
propositional LTL-structure $\Wmf=(w_i)_{i\ge 0}$ \emph{extends~$u$} if
$u_i=w_i$ for every~$i$, $0\le i\le t$.  In this case, we also call~\Wmf an
\emph{extension of~$u$}.
%
In principle, a monitor for a propositional LTL-formula~$\phi$ needs to realise
the following \emph{monitoring function}
$\msf_\phi\colon\Sigma_\Pmc^*\to\{\top,\bot,{?}\}$:
\[\msf_\phi(u):=\begin{cases}
        \top &\text{if $\Wmf,0\models\phi$ for every propositional LTL-structure~\Wmf that extends~$u$;}\\
        \bot &\text{if $\Wmf,0\models\lnot\phi$ for every propositional LTL-structure~\Wmf that extends~$u$; and}\\
        {?}  &\text{otherwise.}
    \end{cases}\]

\noindent
As mentioned above, this function should not be computed from scratch whenever a
new observation $\sigma\in\Sigma_\Pmc$ is added.  In particular, the time needed
for computing the next function value $\msf_\phi(u\sigma)$ should not depend on
the length of the already observed word~$u$.  This can be achieved by
constructing a deterministic Moore-automaton as monitor.

\begin{definition}[Deterministic Moore-automaton]
    A \emph{deterministic Moore-automaton} is a tuple
    $\Mmc=(S,\Sigma,\delta,s_0,\Gamma,\lambda)$ consisting of a finite set of
    states~$S$, a finite input alphabet~$\Sigma$, a transition function
    $\delta\colon S\times\Sigma\to S$, an initial state $s_0\in S$, a finite
    output alphabet~$\Gamma$, and an output function $\lambda\colon S\to\Gamma$.

    The transition function and the output function can be extended to functions
    $\delta^*\colon S\times\Sigma^*\to S$ and $\lambda^*\colon\Sigma^*\to\Gamma$
    as follows:
    \begin{itemize}
        \item $\delta^*(s,\varepsilon):=s$ where $\varepsilon$ denotes the empty
            word;
        \item $\delta^*(s,u\sigma):=\delta(\delta^*(s,u),\sigma)$ where
            $u\in\Sigma^*$ and $\sigma\in\Sigma$; and
    \end{itemize}
    for every $u\in\Sigma^*$, $\lambda^*(u):=\lambda(\delta^*(s_0,u))$.
\end{definition}

\noindent
Such a deterministic Moore-automaton is a monitor for~$\phi$ if its extended
output function is the monitoring function for~$\phi$.

\begin{definition}[Monitor for propositional LTL-formula]
    Let~\Pmc be a finite set of propositional variables, and let~$\phi$ be a
    propositional LTL-formula over~\Pmc.
    %
    The deterministic Moore-automaton
    $\Mmc=(S,\Sigma_\Pmc,\delta,s_0,\{\top,\bot,{?}\},\lambda)$
    is a \emph{monitor for~$\phi$} if $\lambda^*(u)=\msf_\phi(u)$ holds for
    every $u\in\Sigma_\Pmc^*$.
\end{definition}

\noindent
Given a propositional LTL-formula~$\phi$ over~\Pmc, a monitor for~$\phi$ can
effectively be computed~\cite{BaLS-ToSEM11}.  Basically, on proceeds as follows.
First, one computes Büchi-automata for~$\phi$ and~$\lnot\phi$.  These automata
are then determinised (viewed as finite automata rather than Büchi-automata).
Finally, one builds the product of the two deterministic automata.  The output
function is computed using reachability tests in the Büchi-automata
(see~\cite{BaLS-ToSEM11} and the monitor construction for \SHOQ-LTL in
Section~\ref{sec:monitor} below for details).  The monitor obtained this way is
in the worst case of doubly exponential size and be be computed in doubly
exponential time.

One can actually show that this doubly exponential blow-up in the construction
of the monitor cannot be avoided.  Such a doubly exponential lower bound was
already claimed in~\cite{BaLS-ToSEM11}, referring to a result of Kupferman and
Vardi~\cite{KuVa-FMSD01}.  However, a closer look at Theorem~3.3
in~\cite{KuVa-FMSD01} shows that it only yields a lower bound
of~$2^{2^{\sqrt{n}}}$.  Fortunately, a more recent result by Kupferman and
Rosenberg~\cite{KuRo-MoChArt10} can be used to show a lower bound of~$2^{2^n}$.
We include a proof of this tight lower bound for the monitor construction here
for the sake of completeness.  This lower bound can also be used to show
optimality of our monitor constructions for \SHOQ-LTL\@.

Kupferman and Rosenberg show (see Theorem~3 in~\cite{KuRo-MoChArt10}) that there
exists a sequence $(L_n)_{n\ge 1}$ of $\omega$-languages and a sequence
$(\phi_n)_{n\ge 1}$ of propositional LTL-formulas such that the following holds
for every $n\ge 1$:
\begin{enumerate}
    \item the $\omega$-language~$L_n$ can be accepted by a deterministic
        Büchi-automaton, but the number of states of any deterministic
        Büchi-automaton accepting $L_n$ is at least~$2^{2^n}$; and
    \item $L_n=L_\omega(\phi_n)$ and the size of~$\phi_n$ is linear in~$n$.
\end{enumerate}

\noindent
Using an argument similar to the one employed in~\cite{KuRo-MoChArt10}, we can
show that the number of states of any monitor for~$\phi_n$ is at
least~$2^{2^n}$.  For this purpose, we first recall the definition of the
languages~$L_n$ from~\cite{KuRo-MoChArt10}.

For every $n\ge 0$, we consider the alphabet
$\Sigma_n:=\{a_1,\dots,a_n\}\cup\{b_1,\dots,b_n\}\cup\{\#,\$\}$, and define
\begin{align*}
    T_n &:=\{a_1,b_1\}\cdot\dotso\cdot\{a_n,b_n\},\\[1ex]
    S_n &:=\{\#\}\cdot(T_n\cdot\{\#\})^*\cdot\{\$\}\cdot T_n\cdot\{\#\}^\omega,\\[1ex]
    R_n &:=\bigcup_{w\in T_n}\Sigma_n^*\cdot\{\#\}\cdot\{w\}\cdot\{\#\}\cdot\Sigma_n^*\cdot\{\$\}\cdot\Sigma_n^*\cdot\{w\}\cdot\{\#\}^\omega,\\
    L_n &:=S_n\cap R_n.
\end{align*}

\noindent
Thus, the language~$T_n$ consist of the words of length~$n$ such that the letter
at position~$i$ is~$a_i$ or~$b_i$.  Obviously, there are $2^n$ such words.
%
The $\omega$-language~$S_n$ consists of $\omega$-words that start with a finite
sequence of elements of~$T_n$, which are separated by the $\#$-symbol.  This
sequence is terminated by the $\$$-symbol, which is followed by exactly one
element of~$T_n$.  Then comes an infinite sequence of $\#$-symbols.
%
Intersecting the $\omega$-language~$S_n$ with the $\omega$-language~$R_n$ has
the following effect: it ensures that the element~$w$ of~$T_n$ that follows the
$\$$-symbol has already occurred in the sequence of elements of~$T_n$ before the
$\$$-symbol.

The propositional LTL-formulas~$\phi_n$ representing the
$\omega$-languages~$L_n$ are built over sets of propositional variables with
$2n+2$ elements, i.e.~over $\Pmc_n:=\{p_1,\dots,p_{2n+2}\}$.
%
Recall that such a propositional LTL-formula defines a $\omega$-language over
the alphabet~$\Sigma_{\Pmc_n}$, whose letters are the subsets of~$\Pmc_n$.  Of
these exponentially many letters, the language~$L_n$ uses only the (linearly
many) singleton sets, where
\begin{itemize}
    \item $\{p_i\}$ represents the letter~$a_i$ for~$i$, $1\le i\le n$;
    \item $\{p_{n+i}\}$ represents the letter~$b_i$ for~$i$, $1\le i\le n$;
    \item $\{p_{2n+1}\}$ represents the letter~$\#$; and
    \item $\{p_{2n+2}\}$ represents the letter~$\$$.
\end{itemize}

\noindent
In order to increase readability, we continue to use the letters from~$\Sigma_n$
rather than these singleton sets in our argument below.

\begin{theorem}\label{thm:monitor-size}
    There is a sequence $(\phi_n)_{n\ge 1}$ of propositional LTL-formulas of
    size linear in~$n$ such that the number of states of any monitor
    for~$\phi_n$ is at least~$2^{2^n}$.
\end{theorem}

\begin{proof}
    Let $(\phi_n)_{n\ge 1}$ be the sequence of propositional LTL-formulas
    constructed in the proof of Theorem~3 of~\cite{KuRo-MoChArt10}.  It is shown
    in that proof that $L_n=L_\omega(\phi_n)$ and that the size of~$\phi_n$ is
    linear in~$n$.

    Now assume that
    $\Mmc_n=(S_n,\Sigma_n,\delta_n,s_{0,n},\{\top,\bot,{?}\},\lambda_n)$ is a
    monitor for~$\phi_n$ with less than $2^{2^n}$ states.
    %
    Given a set $T\subseteq T_n$, we enumerate its elements in
    lexicographic order (where $a_i$ comes before $b_i$).  Assume that
    $w_1,\dots,w_m$ is the enumeration of the elements of~$T$ in this order.
    Then we define
    \[\wsf(T):=\#w_1\#\dots\#w_m\#.\]
    %
    Moreover, let $\ssf(T)$ be the state reached in~$\Mmc_n$ with
    input~$\wsf(T)$ when starting at the initial state~$s_{0,n}$.
    %
    Since there are $2^{2^n}$ different subsets of~$T_n$, but less than
    $2^{2^n}$ states, there must be two different such subsets $T,T'$ such that
    $\ssf(T)=\ssf(T')$.  We assume without loss of generality that
    there is a word $w\in T\setminus T'$.\footnote{%
        The case where $T'\setminus T$ is non-empty can be treated
        symmetrically.}
    %
    Now consider the state~$s$ reached from~$s_{0,n}$ on input~$\wsf(T)w\#$.
    Since $\ssf(T)=\ssf(T')$, this is the same state as the one reached from
    $s_{0,n}$ on input~$\wsf(T')w\#$.  Since $\Mmc_n$ is a monitor for~$\phi_n$,
    this implies that
    \[\msf_{\phi_n}(\wsf(T)w\#)=\lambda_n(s)=\msf_{\phi_n}(\wsf(T')w\#).\]

    \noindent
    This, however, yields a contradiction since actually we have
    \[\msf_{\phi_n}(\wsf(T)w\#)={?}\ne\bot=\msf_{\phi_n}(\wsf(T')w\#).\]
    %
    In fact, we can extend $\wsf(T)w\#$ to an $\omega$-word belonging to~$L_n$
    (and thus satisfying~$\phi_n$) by adding an infinite sequence of
    $\#$-symbols.  Any other extension of $\wsf(T)w\#$ does not belong to~$L_n$.
    This shows that $\msf_{\phi_n}(\wsf(T)w\#)={?}$.  The word $\wsf(T')w\#$,
    however, cannot be extended to an element of~$L_n$ since $w$ does not occur
    in the sequence before the $\$$-symbol.  This shows that
    $\msf_{\phi_n}(\wsf(T')w\#)=\bot$.

    Summing up, we have seen that our assumption that there is a monitor
    for~$\phi_n$ with less than $2^{2^n}$ states leads to a contradiction, which
    shows that any monitor for~$\phi_n$ must have at least $2^{2^n}$ states.
\end{proof}

\noindent
Before building a monitor for a propositional LTL-formula~$\phi$, it makes sense
to check whether the monitor will actually be able to give reasonable answers.
For example, a monitor that always, i.e.~for every finite word, returns the
answer~${?}$ is clearly useless.  Similarly, when running the monitor, it makes
sense to check whether, according to what has been seen of the system's
behaviour until now, i.e.~the finite word read by the monitor until now, it
makes sense to continue running the monitor.  This leads to the following
definition of monitorability~\cite{PnZa-FM06,FaFM-RV09,Bau-CoRR10}.

\begin{definition}[Monitorability]\label{def:monitorability-ltl}
    Let $\phi$ be a propositional LTL-formula over~\Pmc, and let $u$ be a finite
    word over~$\Sigma_\Pmc$.  We say that $\phi$ is \emph{$u$-monitorable} if
    there is a finite word $v\in\Sigma^*$ such that $\msf_\phi(uv)\ne{?}$.
    Moreover, we call $\phi$ \emph{monitorable} if it is $u$-monitorable for
    every finite word $u\in\Sigma_\Pmc^*$.
\end{definition}

\noindent
Given a monitor~\Mmc for~$\phi$, one can easily decide monitorability through
reachability tests in~\Mmc.  We call a state in~\Mmc \emph{good} if one can
reach from it a state whose output is different to~${?}$.  Then
\begin{itemize}
    \item $\phi$ is $u$-monitorable iff the state reached from the initial state
        with input~$u$ is good; and
    \item $\phi$ is monitorable iff every state reachable from the initial state
        is good.
\end{itemize}
%
This shows that monitorability can be decided in time doubly exponential in the
size of the input formula.  More precisely, one can obtain an upper bound of
\ExpSpace by constructing the relevant parts of the monitor on-the-fly while
performing the reachability test.\footnote{%
    This is the same idea underlying the automata-based \PSpace satisfiability
    check for propositional LTL~\cite{SiCl-JACM85,LiPZ-CLP85} mentioned in
    Section~\ref{sec:aut-for-ltl}.}
%
To the best of our knowledge, it is an open problem whether this upper
bound of \ExpSpace is tight.  The only known lower
bound is one of \PSpace, which can be obtained using a reduction from the
satisfiability problem.\footnote{%
    Note that the proof of an upper bound of \PSpace given in~\cite{Bau-CoRR10}
    actually does not go through.}
%
Interestingly, the same is true for the related but simpler-looking problem of
liveness~\cite{AlSc-IPL85}.

\begin{definition}[Liveness]\label{def:liveness-ltl}
    Let $\phi$ be a propositional LTL-formula over~\Pmc.  We say that $\phi$
    \emph{expresses a liveness property} if every finite word
    $u\in\Sigma_\Pmc^*$ has an extension to an $\omega$-word that
    satisfies~$\phi$.
\end{definition}

\noindent
Using the monitoring function, liveness of~$\phi$ can thus be expressed as
follows: $\phi$ expresses a liveness property iff $\msf_\phi(u)\ne\bot$ for
every $u\in\Sigma_\Pmc^*$.  Consequently, given a monitor for~$\phi$, liveness
of~$\phi$ can again be tested by checking reachability in the monitor, which
yields an upper bound of \ExpSpace.  Again, it is open whether
this upper bound is tight.  The only known lower bound is again one of \PSpace,
which can again be obtained using a reduction from satisfiability.\footnote{%
    Note that the proof of an upper bound of \PSpace sketched
    in~\cite{UlWo-CoRR01} actually does not go through.}

In the next sections, we show how to extend the notions of this section to
obtain monitors for \SHOQ-LTL-formulas.


\section{Büchi-Automata for \texorpdfstring{\SHOQ-LTL}{SHOQ-LTL}-Formulas}\label{sec:ba-for-shoq-ltl}

The decision procedures in Section~\ref{sec:complexity-shoq-ltl} for the
satisfiability problem in \SHOQ-LTL are not based on Büchi-automata.  We show in
this section, however, that the ideas underlying these decision procedures can
be used to obtain automata-based decision procedures.  The Büchi-automata
constructed in this section will be the building blocks of our monitors.

In the following, let~\Rmc be an RBox, and let~$\phi$ be a \SHOQ-LTL-formula
w.r.t.~\Rmc.  In principle, we want to construct a
Büchi-automaton~$\Nmc_{\phi,\Rmc}$ that accepts exactly the models of~$\phi$
w.r.t.~\Rmc.  This is very similar to what is done for propositional LTL\@; see
Definition~\ref{def:ba-for-ltl}.  Since there are infinitely many
interpretations, we would end up with an infinite alphabet for this
Büchi-automaton.  For this reason, we abstract from specific interpretations and
consider only the axioms in~$\phi$ that they satisfy.

For a given interpretation~\Imc, we denote by $\tau_\phi(\Imc)$ the set of all
axioms in $\Ax(\phi)$ that \Imc is a model of, i.e.\
\[\tau_\phi(\Imc):=\{\alpha\in\Ax(\phi)\mid\Imc\models\alpha\}.\]
%
Note that if $\Imc\models\Rmc$, we have that \Imc is a model of the Boolean
knowledge base
\[\Biggl(\bigwedge_{\alpha\in\tau_\phi(\Imc)}\alpha\land%
    \bigwedge_{\alpha\in\Ax(\phi)\setminus\tau_\phi(\Imc)}\lnot\alpha,\quad\Rmc\Biggr).\]
%
This motivates the following definition.

\begin{definition}[Axiom type]
    The set of axioms~$T$ is an \emph{axiom type for~$\phi$ w.r.t.~\Rmc} if the
    following two properties are satisfied:
    \begin{itemize}
        \item $T\subseteq\Ax(\phi)$; and
        \item the Boolean knowledge base
            \[\Bmc_T:=\Biggl(\bigwedge_{\alpha\in T}\alpha\land%
                \bigwedge_{\alpha\in\Ax(\phi)\setminus T}\lnot\alpha,\quad\Rmc\Biggr)\]
            is consistent.
    \end{itemize}
\end{definition}

\noindent
We denote the set of all axiom types for~$\phi$ w.r.t.~\Rmc
with~$\Tmf_{\phi,\Rmc}$.
%
The following lemma is an easy consequence of this definition.

\begin{lemma}\label{lem:interpretation-axiom-type}
    Let \Imc be an interpretation, and let $T$ be a set of axioms.
    \begin{enumerate}
        \item If \Imc is a model of~\Rmc, then $\tau_\phi(\Imc)$ is an axiom
            type for~$\phi$ w.r.t.~\Rmc.
        \item If $T$ is any axiom type for~$\phi$ w.r.t.~\Rmc, then there is a
            model~\Jmc of~\Rmc such that $T=\tau_\phi(\Jmc)$.
    \end{enumerate}
\end{lemma}

\begin{proof}
    For Part~1 of the lemma, assume that \Imc is a model of~\Rmc.  We have
    $\tau_\phi(\Imc)\subseteq\Ax(\phi)$ by definition, and as argued above that
    $\Imc\models\Bmc_{\tau_\phi(\Imc)}$.

    For Part~2 of the lemma, assume that $T$ is an axiom type for~$\phi$
    w.r.t.~\Rmc.  Hence, $\Bmc_T$ is consistent.  Let~\Jmc be a model
    of~$\Bmc_T$.  Note that we have $\Jmc\models\Rmc$.  Furthermore, it is easy
    to see that, by construction of~$\Bmc_T$, we have that $T=\tau_\phi(\Jmc)$.
\end{proof}

\noindent
In the following, for $\Imc\models\Rmc$, we call $\tau_\phi(\Imc)$ the
\emph{axiom type of~\Imc}.  This notion can be further extended to
DL-LTL-structures.  For a given DL-LTL-structure $\Imf=(\Imc_i)_{i\ge 0}$, we
define
\[\tau_\phi(\Imf):=\tau_\phi(\Imc_0)\tau_\phi(\Imc_1)\tau_\phi(\Imc_2)\dots,\]
and, for $\Imf\models\Rmc$, we call $\tau_\phi(\Imf)$ the \emph{axiom type
of~\Imf}.  Note that the axiom type of~\Imf is an $\omega$-word over the
alphabet~$\Tmf_{\phi,\Rmc}$.

Whether a given DL-LTL-structure~\Imf with $\Imf\models\Rmc$ is a model
of~$\phi$ w.r.t.~\Rmc only depends on its axiom type.  This is stated formally
in the following lemma.

\begin{lemma}\label{lem:equal-type}
    Let \Imf and \Jmf be DL-LTL-structures such that $\Imf\models\Rmc$,
    $\Jmf\models\Rmc$, and $\tau_\phi(\Imf)=\tau_\phi(\Jmf)$.  Then, \Imf is a
    model of~$\phi$ w.r.t.~\Rmc iff \Jmf is a model of~$\phi$ w.r.t.~\Rmc.
\end{lemma}

\begin{proof}
    Let $\Imf=(\Imc_i)_{i\ge 0}$ and $\Jmf=(\Jmc_i)_{i\ge 0}$ be
    DL-LTL-structures such that $\Imf\models\Rmc$, $\Jmf\models\Rmc$, and
    $\tau_\phi(\Imf)=\tau_\phi(\Jmf)$.  It is enough to
    prove that $\Imf,i\models\phi$ iff $\Jmf,i\models\phi$ for every~$i\ge 0$,
    which we show by induction on the structure of~$\phi$.

    For the case where $\phi$ is an axiom, we have for every~$i\ge 0$ that
    $\Imf,i\models\phi$ \emph{iff} $\Imc_i\models\phi$ \emph{iff}
    $\phi\in\tau_\phi(\Imc_i)$ \emph{iff} $\phi\in\tau_\phi(\Jmc_i)$ \emph{iff}
    $\Jmc_i\models\phi$ \emph{iff} $\Jmf,i\models\phi$.

    If $\phi$ is of the form $\lnot\phi_1$, we have for every~$i\ge 0$ that
    $\Imf,i\models\lnot\phi_1$ \emph{iff} $\Imf,i\not\models\phi_1$ \emph{iff}
    $\Jmf,i\not\models\phi_1$ \emph{iff} $\Jmf,i\models\lnot\phi_1$.

    If $\phi$ is of the form $\phi_1\land\phi_2$, we have for every~$i\ge 0$
    that $\Imf,i\models\phi_1\land\phi_2$ \emph{iff} $\Imf,i\models\phi_1$ and
    $\Imf,i\models\phi_2$ \emph{iff} $\Jmf,i\models\phi_1$ and
    $\Jmf,i\models\phi_2$ \emph{iff} $\Jmf,i\models\phi_1\land\phi_2$.

    If $\phi$ is of the form $\Next\phi_1$, we have for every~$i\ge 0$ that
    $\Imf,i\models\Next\phi_1$ \emph{iff} $\Imf,i+1\models\phi_1$ \emph{iff}
    $\Jmf,i+1\models\phi_1$ \emph{iff} $\Jmf,i\models\Next\phi_1$.

    If $\phi$ is of the form $\Previous\phi_1$, we have for every~$i\ge 0$ that
    $\Imf,i\models\Previous\phi_1$ \emph{iff} $i>0$ and $\Imf,i-1\models\phi_1$
    \emph{iff} $i>0$ and $\Jmf,i-1\models\phi_1$ \emph{iff}
    $\Jmf,i\models\Previous\phi_1$.

    If $\phi$ is of the form $\phi_1\Until\phi_2$, we have for every~$i\ge 0$
    that $\Imf,i\models\phi_1\Until\phi_2$ \emph{iff} there is some $k\ge i$
    such that $\Imf,k\models\phi_2$, and $\Imf,j\models\phi_1$ for every~$j$,
    $i\le j<k$ \emph{iff} there is some $k\ge i$ such that
    $\Jmf,k\models\phi_2$, and $\Jmf,j\models\phi_1$ for every~$j$, $i\le j<k$
    \emph{iff} $\Jmf,i\models\phi_1\Until\phi_2$.

    Finally, if $\phi$ is of the form $\phi_1\Since\phi_2$, we have for
    every~$i\ge 0$ that $\Imf,i\models\phi_1\Since\phi_2$ \emph{iff} there is
    some $k$, $0\le k\le i$, such that $\Imf,k\models\phi_2$, and
    $\Imf,j\models\phi_1$ for every~$j$, $k<j\le i$ \emph{iff} there is some
    $k$, $0\le k\le i$, such that $\Jmf,k\models\phi_2$, and
    $\Jmf,j\models\phi_1$ for every~$j$, $k<j\le i$ \emph{iff}
    $\Jmf,i\models\phi_1\Since\phi_2$.
\end{proof}

\noindent
This lemma justifies considering Büchi-automata that receive axiom types of
DL-LTL-structures as input rather than the DL-LTL-structures themselves.  The
next definition is very similar to the one for propositional LTL\@; see
Definition~\ref{def:ba-for-ltl}.

\begin{definition}[Büchi-automaton for \SHOQ-LTL-formula]
    Let \Rmc be an RBox, let $\phi$ be a \SHOQ-LTL-formula w.r.t.~\Rmc, and let
    \Nmc be a Büchi-automaton working on the alphabet~$\Tmf_{\phi,\Rmc}$.  We
    define
    \[L_\omega(\phi,\Rmc):=\bigl\{\tau_\phi(\Imf)\in\Tmf_{\phi,\Rmc}^\omega\mid%
        \Imf=(\Imc_i)_{i\ge 0}\ \text{is a model of~$\phi$ w.r.t.~\Rmc}\bigr\},\]
    and say that \Nmc is a \emph{Büchi-automaton for~$\phi$ w.r.t.~\Rmc} if
    $L_\omega(\Nmc)=L_\omega(\phi,\Rmc)$.
\end{definition}

\noindent
Instead of constructing such Büchi-automata directly for \SHOQ-LTL-formulas, we
build their propositional abstractions and then reuse the known construction for
the propositional case.\footnote{%
    One could also define the Büchi-automata directly as done
    in~\cite{BaBL-FroCoS09,Lip-Dipl09}, but the approach developed below is more
    modular and also easier to implement.  In fact, the approach does not depend
    on a specific algorithm for generating Büchi-automata for propositional
    LTL-formulas.  Thus, any existing tool for transforming a propositional
LTL-formula into a Büchi-automaton can be used.}
%
For that, we reuse also the results shown in Chapter~\ref{ch:shoq-ltl}.  In the
following, let $\psf\colon\Ax(\phi)\to\Pmc_\phi$ be a bijection.
%
It turns out that it is convenient to define the notion of r-satisfiability (see
Definition~\ref{def:r-sat}) on the level of axiom types.

\begin{definition}[R-consistency]\label{def:r-cons}
    Let $\Tmf=\{T_1,\dots,T_k\}\subseteq\Tmf_{\phi,\Rmc}$.  We call \Tmf
    \emph{r-consistent} if there exist interpretations
    $\Imc_1=(\Delta,\cdot^{\Imc_1})$,~\dots, $\Imc_k=(\Delta,\cdot^{\Imc_k})$
    such that
    \begin{itemize}
        \item $a^{\Imc_i}=a^{\Imc_j}$ holds for every $a\in\NI$ and all $i,j$,
            $1\le i<j\le k$;
        \item $A^{\Imc_i}=A^{\Imc_j}$ holds for every $A\in\NRC$ and all $i,j$,
            $1\le i<j\le k$;
        \item $r^{\Imc_i}=r^{\Imc_j}$ holds for every $r\in\NRR$ and all $i,j$,
            $1\le i<j\le k$; and
        \item $\Imc_i\models\Rmc$ and $\tau_\phi(\Imc_i)=T_i$ holds for
            every~$i$, $1\le i\le k$.
    \end{itemize}
\end{definition}

\noindent
Note that any subset of an r-consistent set of axiom types for~$\phi$
w.r.t.~\Rmc is again r-consistent.  In particular, the empty set is always
r-consistent.

We denote the set of all r-consistent sets of axiom types for~$\phi$ w.r.t.~\Rmc
with~$\Cmf_{\phi,\Rmc}$.
%
Moreover, we denote by $\psf(T)$, for $T\subseteq\Ax(\phi)$, the following
subset of~$\Pmc_\phi$:
\[\psf(T):=\bigl\{\psf(\alpha)\mid\alpha\in T\bigr\}.\]
Conversely, we denote by $\psf^{-1}(w)$, for $w\subseteq\Pmc_\phi$, the
following subset of~$\Ax(\phi)$:
\[\psf^{-1}(w):=\bigl\{\psf^{-1}(p)\mid p\in w\bigr\}.\]
%
It is easy to see that for any DL-LTL-structure $\Imf=(\Imc_i)_{i\ge 0}$, we
have that $\Imf^\psf=(\psf(\tau_\phi(\Imc_i)))_{i\ge 0}$.
%
Moreover, we obtain the following relationship between r-satisfiability and
r-consistency.

\begin{lemma}\label{lem:r-cons-r-sat}
    Let $\Tmf=\{T_1,\dots,T_k\}\subseteq\Tmf_{\phi,\Rmc}$, and let
    $\Wmc=\{X_1,\dots,X_k\}\subseteq 2^{\Pmc_\phi}$.
    \begin{enumerate}
        \item If \Tmf is r-consistent, then
            $\bigl\{\psf(T_1),\dots,\psf(T_k)\bigr\}$ is r-satisfiable
            w.r.t.~\Rmc.
        \item If \Wmc is r-satisfiable w.r.t.~\Rmc, then
            $\bigl\{\psf^{-1}(X_1),\dots,\psf^{-1}(X_k)\bigr\}$ is r-consistent.
    \end{enumerate}
\end{lemma}

\begin{proof}
    For Part~1 of the lemma, assume that
    $\Tmf=\{T_1,\dots,T_k\}\subseteq\Tmf_{\phi,\Rmc}$ is r-consistent.  Then
    there are interpretations $\Imc_1=(\Delta,\cdot^{\Imc_1})$,~\dots,
    $\Imc_k=(\Delta,\cdot^{\Imc_k})$ such that the conditions of
    Definition~\ref{def:r-cons} are satisfied.  Note that the first three
    conditions coincide with the ones of Definition~\ref{def:r-sat}.  Thus, to
    show that $\bigl\{\psf(T_1),\dots,\psf(T_k)\bigr\}$ is r-satisfiable
    w.r.t.~\Rmc, it only remains to show that every~$\Imc_i$, $1\le i\le k$, is
    a model of the Boolean knowledge base~$\Bmc_{\psf(T_i)}$, which is defined
    as in Definition~\ref{def:r-sat}.  This is satisfied, since for every~$i$,
    $1\le i\le k$, we have by Definition~\ref{def:r-cons} and the arguments
    above that~$\Imc_i$ is a model of
    \[\Bmc_{\psf(T_i)}:=%
        \Biggl(\bigwedge_{p\in\psf(T_i)}\psf^{-1}(p)\land%
        \bigwedge_{p\in\Pmc_\phi\setminus\psf(T_i)}\lnot\psf^{-1}(p),\quad\Rmc\Biggr)=%
        \Biggl(\bigwedge_{\alpha\in\tau_\phi(\Imc_i)}\alpha\land%
        \bigwedge_{\alpha\in\Ax(\phi)\setminus\tau_\phi(\Imc_i)}\lnot\alpha,\quad\Rmc\Biggr).\]

    For Part~2 of the lemma, assume that
    $\Wmc=\{X_1,\dots,X_k\}\subseteq 2^{\Pmc_\phi}$ is r-satisfiable
    w.r.t.~\Rmc.  Then there are interpretations
    $\Imc_1=(\Delta,\cdot^{\Imc_1})$,~\dots, $\Imc_k=(\Delta,\cdot^{\Imc_k})$
    such that the conditions of Definition~\ref{def:r-sat} are satisfied.  Note
    again that the first three conditions coincide with the ones of
    Definition~\ref{def:r-cons}.  Thus, to show that
    $\bigl\{\psf^{-1}(X_1),\dots,\psf^{-1}(X_k)\bigr\}$ is r-consistent, it only
    remains to show that for every~$i$, $1\le i\le k$, we have that
    $\Imc_i\models\Rmc$ and $\tau_\phi(\Imc_i)=\psf^{-1}(X_i)$.  This is
    satisfied, since for every~$i$, $1\le i\le k$, we have by
    Definition~\ref{def:r-sat} that~$\Imc_i$ is a model of
    \[\Bmc_{X_i}:=%
        \Biggl(\bigwedge_{p\in X_i}\psf^{-1}(p)\land%
        \bigwedge_{p\in\Pmc_\phi\setminus X_i}\lnot\psf^{-1}(p),\quad\Rmc\Biggr)=%
        \Biggl(\bigwedge_{\alpha\in\psf^{-1}(X_i)}\alpha\land%
        \bigwedge_{\alpha\in\Ax(\phi)\setminus\psf^{-1}(X_i)}\lnot\alpha,\quad\Rmc\Biggr),\]
    which yields that $\Imc_i\models\Rmc$ and
    $\tau_\phi(\Imc_i)=\{\alpha\in\Ax(\phi)\mid\Imc_i\models\alpha\}=\psf^{-1}(X_i)$.
\end{proof}

\noindent
We are now ready to define Büchi-automata for \SHOQ-LTL-formulas.
%
We consider here only the case without rigid names, i.e.~$\NRC=\NRR=\emptyset$,
and the case with rigid concept and role names, i.e.~$\NRC\ne\emptyset$ and
$\NRR\ne\emptyset$.  These two cases are treated in
Sections~\ref{sec:ba-no-rigid} and~\ref{sec:ba-rigid}.
%
The intermediate case where only concept names are allowed to be rigid,
i.e.~$\NRC\ne\emptyset$ and $\NRR=\emptyset$, is not considered.  As we will see
below, the corresponding monitors are of size doubly exponential in the size of
the input \SHOQ-LTL-formula (and the RBox) irrespective of the fact whether
rigid names are allowed or not.


\subsection{The Case without Rigid Names}\label{sec:ba-no-rigid}

In this section, we consider the case where neither concept names nor role names
are allowed to be rigid, i.e.~$\NRC=\NRR=\emptyset$.  With the lemmas above, we
are able to establish the following result.

\begin{theorem}\label{thm:ba-shoq-ltl-no-rigid}
    Let \Rmc be an RBox, let $\phi$ a \SHOQ-LTL-formula w.r.t.~\Rmc, and let
    $\psf\colon\Ax(\phi)\to\Pmc_\phi$ be a bijection.
    %
    If $\NRC=\NRR=\emptyset$ and
    $\Nmc_{\phi^\psf}=(Q,\Sigma_{\Pmc_\phi},\Delta,Q_0,F)$ is a Büchi-automaton
    for the propositional abstraction~$\phi^\psf$ of~$\phi$, then
    $\Nmc_{\phi,\Rmc}:=(Q,\Tmf_{\phi,\Rmc},\Delta',Q_0,F)$ with
    \[\Delta':=\bigl\{(q,T,q')\mid%
        (q,\psf(T),q')\in\Delta\ \text{and}\ T\in\Tmf_{\phi,\Rmc}\bigr\}\]
    is a Büchi-automaton for~$\phi$ w.r.t.~\Rmc.
\end{theorem}

\begin{proof}
    We have to show that $L_\omega(\Nmc_{\phi,\Rmc})=L_\omega(\phi,\Rmc)$.

    For the direction~\enquote{$\subseteq$}, assume that
    $T_0T_1T_2\dotso\in L_\omega(\Nmc_{\phi,\Rmc})$.  The definition
    of~$\Nmc_{\phi,\Rmc}$ yields that
    $\psf(T_0)\psf(T_1)\psf(T_2)\dotso\in L_\omega(\Nmc_{\phi^\psf})$.  Since
    $\Nmc_{\phi^\psf}$ is a Büchi-automaton for~$\phi^\psf$, we obtain that
    $\Wmf=(\psf(T_i))_{i\ge 0}$ is a model of~$\phi^\psf$.
    %
    We define $\Tmf:=\{T_i\mid i\ge 0\}$.  Since for every $i\ge 0$, we have
    $T_i\in\Tmf_{\phi,\Rmc}$, it follows that
    $\Tmf=\{T_1',\dots,T_k'\}\subseteq\Tmf_{\phi,\Rmc}$.  By
    Lemma~\ref{lem:interpretation-axiom-type}, we have that for every~$i$,
    $1\le i\le k$, there is a model~$\Jmc_i$ of~\Rmc such that
    $T_i'=\tau_\phi(\Jmc)$.  We can assume w.l.o.g.\ that all of these models
    have the same domain since we can assume w.l.o.g.\ that their domains are
    countably infinite due to the Löwenheim-Skolem
    theorem~\cite{Loe-MA15,Sko-VS20}.  Moreover, we can assume w.l.o.g.\ that
    all individual names are interpreted by the same domain elements in all
    models.  Since $\NRC=\NRR=\emptyset$, this yields that \Tmf is r-consistent.
    By Lemma~\ref{lem:r-cons-r-sat}, we have that
    $\Wmc:=\{\psf(T_1'),\dots,\psf(T_k')\}=\{\psf(T_i)\mid i\ge 0\}$ is
    r-satisfiable w.r.t.~\Rmc.  Then, Lemma~\ref{lem:ltl-structure-r-sat} yields
    that there is a model \Imf of~$\phi$ w.r.t.~\Rmc with $\Imf^\psf=\Wmf$.
    Consequently, $\tau_\phi(\Imf)=T_0T_1T_2\dotso\in L_\omega(\phi,\Rmc)$.

    For the direction~\enquote{$\supseteq$}, assume that
    $T=T_0T_1T_2\dotso\in L_\omega(\phi,\Rmc)$.  Then there is a model
    $\Imf=(\Imc_i)_{i\ge 0}$ of~$\phi$ w.r.t.~\Rmc with $\tau_\phi(\Imf)=T$.  By
    Lemma~\ref{lem:interpretation-axiom-type}, for every $i\ge 0$, the
    letter~$T_i$ is an axiom type for~$\phi$ w.r.t.~\Rmc,
    i.e.~$T_i\in\Tmf_{\phi,\Rmc}$.
    %
    By Lemma~\ref{lem:ltl-structure-r-sat}, we have that
    $\Imf^\psf=(\psf(T_i))_{i\ge 0}$ is a model of~$\phi^\psf$, and thus the
    $\omega$-word $\psf(T_0)\psf(T_1)\psf(T_2)\dots$ is accepted
    by~$\Nmc_{\phi^\psf}$.  Consequently, we have
    $T_0T_1T_2\dotso\in L_\omega(\Nmc_{\phi,\Rmc})$.
\end{proof}

\noindent
As an immediate consequence of this theorem, the satisfiability problem in
\SHOQ-LTL (for the case $\NRC=\NRR=\emptyset$) can be reduced to the emptiness
problem for Büchi-automata.

\begin{corollary}
    If $\NRC=\NRR=\emptyset$, the \SHOQ-LTL-formula~$\phi$ is satisfiable
    w.r.t.\ the RBox~\Rmc iff $L_\omega(\Nmc_{\phi,\Rmc})\ne\emptyset$.
\end{corollary}

\noindent
It remains to analyse the complexity of the decision procedure for the
satisfiability problem obtained by this reduction.

The size of the Büchi-automaton~$\Nmc_{\phi,\Rmc}$ is obviously exponential in
the size of~$\phi$ (and~\Rmc) since the Büchi-automaton~$\Nmc_{\phi^\psf}$
for~$\phi^\psf$ is of size exponential in the size of~$\phi^\psf$ (and thus
exponential in the size of~$\phi$ since the size of~$\phi^\psf$ is linearly
bounded by the size of~$\phi$) as discussed in Section~\ref{sec:aut-for-ltl}.
%
In addition, the Büchi-automaton~$\Nmc_{\phi,\Rmc}$ can be constructed in
exponential time.  As shown in Section~\ref{sec:aut-for-ltl}, a Büchi-automaton
for a propositional LTL-formula can be constructed in time exponential in the
size of the input formula.  Thus, we can construct $\Nmc_{\phi^\psf}$ in time
exponential in the size of~$\phi^\psf$ (and thus in time exponential in the size
of~$\phi$).  To obtain $\Nmc_{\phi,\Rmc}$ from~$\Nmc_{\phi^\psf}$, we basically
have to remove all transitions labelled with a letter~$\sigma$ such that
$\psf^{-1}(\sigma)\notin\Tmf_{\phi,\Rmc}$.  For the remaining transitions, we
then simply replace the letter~$\sigma$ with $\psf^{-1}(\sigma)$.  In order to
check whether $\psf^{-1}(\sigma)$ belongs to~$\Tmf_{\phi,\Rmc}$, we need to
check the Boolean knowledge base~$\Bmc_{\psf^{-1}(\sigma)}$ for consistency.  By
Corollary~\ref{cor:cons-boolean-shoq-kb}, this can be done in time exponential
in the size of this Boolean knowledge base.  Since there are exponentially many
letters~$\sigma$, but the size of each Boolean knowledge
base~$\Bmc_{\psf^{-1}(\sigma)}$ is linearly bounded by the size of~$\phi$
and~\Rmc, we need to perform exponentially many checks (where each needs
exponential time), which yields an overall complexity of exponential time for
the construction of~$\Nmc_{\phi,\Rmc}$.

Since the emptiness problem for Büchi-automata can be solved in time polynomial
in the size of the Büchi-automaton~\cite{VaWo-IC94}, this yields an alternative
proof of the fact that the satisfiability problem in \SHOQ-LTL is in \ExpTime if
$\NRC=\NRR=\emptyset$ (see Theorem~\ref{thm:upper-bound-shoq-ltl-no-rigid}).


\subsection{The Case of Rigid Concept and Role Names}\label{sec:ba-rigid}

In this section, we consider the case where both concept and role names may be
rigid, i.e.~$\NRC\ne\emptyset$ and $\NRR\ne\emptyset$.
%
If rigid names are allowed, the Büchi-automaton needs to check whether the set
of axiom types seen within a run are r-consistent.  This is achieved by using
tuples $(q_1,q_2)$ as states, where $q_1$ is a state of the
Büchi-automaton~$\Nmc_{\phi,\Rmc}$ introduced in
Theorem~\ref{thm:ba-shoq-ltl-no-rigid}, and $q_2$ is an r-consistent set of
axiom types for~$\phi$ w.r.t.~\Rmc.

\begin{theorem}\label{thm:ba-for-shoq-ltl-rigid}
    Let \Rmc be an RBox, let $\phi$ a \SHOQ-LTL-formula w.r.t.~\Rmc, and let
    $\psf\colon\Ax(\phi)\to\Pmc_\phi$ be a bijection.
    %
    If $\Nmc_{\phi^\psf}=(Q,\Sigma_{\Pmc_\phi},\Delta,Q_0,F)$ is a
    Büchi-automaton for the propositional abstraction~$\phi^\psf$ of~$\phi$,
    then
    $\Nmc_{\phi,\Rmc}^\text{r}:=(Q\times\Cmf_{\phi,\Rmc},\Tmf_{\phi,\Rmc},\Delta',Q_0\times\{\emptyset\},F\times\Cmf_{\phi,\Rmc})$
    with
    \[\Delta':=\bigl\{((q_1,q_2),T,(q_1',q_2'))\mid%
        (q_1,\psf(T),q_1')\in\Delta\ \text{and}\ q_2'=q_2\cup\{T\}\in\Cmf_{\phi,\Rmc}\bigr\}\]
    is a Büchi-automaton for~$\phi$ w.r.t.~\Rmc.
\end{theorem}

\begin{proof}
    We have to show that $L_\omega(\Nmc_{\phi,\Rmc}^\text{r})=L_\omega(\phi,\Rmc)$.

    For the direction~\enquote{$\subseteq$}, assume that
    $T=T_0T_1T_2\dotso\in L_\omega(\Nmc_{\phi,\Rmc}^\text{r})$, and let
    \[(q_1^{(0)},q_2^{(0)})(q_1^{(1)},q_2^{(1)})(q_1^{(2)},q_2^{(2)})\dots\]
    be an accepting run of~$\Nmc_{\phi,\Rmc}^\text{r}$ on~$T$.
    %
    It is easy to see that the projection $q_1^{(0)}q_1^{(1)}q_1^{(2)}\dots$ of
    this run to the first component is an accepting run of~$\Nmc_{\phi,\Rmc}$
    on~$T$.
    %
    As argued in the proof of Theorem~\ref{thm:ba-shoq-ltl-no-rigid}, we have
    that $\psf(T_0)\psf(T_1)\psf(T_2)\dotso\in L_\omega(\Nmc_{\phi^\psf})$, and
    that $\Wmf=(\psf(T_i))_{i\ge 0}$ is a model of~$\phi^\psf$.  We define
    $\Tmf:=\{T_i\mid i\ge 0\}$.  Since for every $i\ge 0$, we have
    $T_i\in\Tmf_{\phi,\Rmc}$, it follows that
    $\Tmf=\{T_1',\dots,T_k'\}\subseteq\Tmf_{\phi,\Rmc}$.  It remains to show
    that \Tmf is r-consistent.  In fact, once this is shown,
    Lemma~\ref{lem:r-cons-r-sat} yields that
    $\Wmf:=\{\psf(T_1'),\dots,\psf(T_k')\}=\{\psf(T_i)\mid i\ge 0\}$ is
    r-satisfiable w.r.t.~\Rmc.  Then, Lemma~\ref{lem:ltl-structure-r-sat} yields
    that there is a model \Imf of~$\phi$ w.r.t.~\Rmc with $\Imf^\psf=\Wmf$.
    Consequently, $\tau_\phi(\Imf)=T_0T_1T_2\dotso\in L_\omega(\phi,\Rmc)$.
    %
    To see that \Tmf is r-consistent, we note that the second components
    $q_2^{(j)}$, $j\ge 0$, of the states in the run are r-consistent sets of
    axiom types for~$\phi$ w.r.t.~\Rmc satisfying
    $q_2^{(j)}=\{T_0,\dots,T_{j-1}\}$.  Since there are only finitely many axiom
    types for~$\phi$ w.r.t.~\Rmc, there is an index $\ell\ge 1$ such that
    $q_2^{(k)}=\{T_i\mid i\ge 0\}$, and thus this set is r-consistent.

    For the direction~\enquote{$\supseteq$}, assume that
    $T_0T_1T_2\dotso\in L_\omega(\phi,\Rmc)$.  Then there is a model
    $\Imf=(\Imc_i)_{i\ge 0}$ of~$\phi$ w.r.t.~\Rmc with $\tau_\phi(\Imf)=T$.
    Using the arguments in the proof of Lemma~\ref{thm:ba-shoq-ltl-no-rigid}
    (direction~\enquote{$\supseteq$}), we obtain that
    $T_0T_1T_2\dotso\in L_\omega(\Nmc_{\phi,\Rmc})$.  Let
    $q_1^{(0)}q_1^{(1)}q_1^{(2)}\dots$ be an accepting run of~$\Nmc_{\phi,\Rmc}$
    on~$T$.
    %
    Moreover, by Lemma~\ref{lem:ltl-structure-r-sat}, we obtain that
    $\Imf^\psf=(\psf(T_i))_{i\ge 0}$ is a model of~$\phi^\psf$ and
    $\Wmc=\{X_1,\dots,X_k\}:=\{\psf(T_i)\mid i\ge 0\}$ is r-satisfiable
    w.r.t.~\Rmc.  By Lemma~\ref{lem:r-cons-r-sat}, we obtain that the set
    $\{\psf^{-1}(X_1),\dots,\psf^{-1}(X_k)\}=\{T_i\mid i\ge 0\}$ is
    r-consistent.  If we define $q_2^{(j)}:=\{T_0,\dots,T_{j-1}\}$ for every
    $j\ge 0$, then these sets are r-consistent since they are subsets of the
    r-consistent set $\{T_i\mid i\ge 0\}$.  Consequently,
    \[(q_1^{(0)},q_2^{(0)})(q_1^{(1)},q_2^{(1)})(q_1^{(2)},q_2^{(2)})\dots\]
    is an accepting run of~$\Nmc_{\phi,\Rmc}^\text{r}$ on~$T$.
\end{proof}

\noindent
As an immediate consequence of this theorem, we obtain that the satisfiability
problem in \SHOQ-LTL (even for the case where $\NRC\ne\emptyset$ and
$\NRR\ne\emptyset$) can be reduced to the emptiness problem for Büchi-automata.

\begin{corollary}
    The \SHOQ-LTL-formula~$\phi$ is satisfiable w.r.t.\ the RBox~\Rmc iff
    $L_\omega(\Nmc_{\phi,\Rmc}^\text{r})\ne\emptyset$.
\end{corollary}

\noindent
The complexity of the decision procedure for the satisfiability problem obtained
by this reduction is, however, higher than the complexity of the decision
procedure for the case without rigid names.

The size of the Büchi-automaton~$\Nmc_{\phi,\Rmc}^\text{r}$ is doubly
exponential in the size of~$\phi$ and~\Rmc.  This is due to the fact that the
set~$\Cmf_{\phi,\Rmc}$ of all r-consistent sets of axiom types for~$\phi$
w.r.t.~\Rmc may contain doubly exponentially many elements since these sets are
subsets of the exponentially large set~$\Tmf_{\phi,\Rmc}$ of all axiom types
for~$\phi$ w.r.t.~\Rmc.  Each element of~$\Cmf_{\phi,\Rmc}$ may be of
exponential size.

Next, we show that the Büchi-automaton~$\Nmc_{\phi,\Rmc}^\text{r}$ can be
constructed in doubly exponential time.  In addition to constructing the
Büchi-automaton~$\Nmc_{\phi,\Rmc}$, i.e.~the Büchi-automaton constituting the
first component of~$\Nmc_{\phi,\Rmc}^\text{r}$, we must also compute the
set~$\Cmf_{\phi,\Rmc}$.  For this, we consider all sets of axiom types
for~$\phi$ w.r.t.~\Rmc.  There are doubly exponentially many such sets, each of
size at most exponential in the size of~$\phi$ and~\Rmc.  By
Lemma~\ref{lem:r-cons-r-sat}, checking such a set $\Tmf=\{T_1,\dots,T_k\}$ for
r-consistency amounts to check the set
$\Wmc:=\bigl\{\psf(T_1),\dots,\psf(T_k)\bigr\}$ for r-satisfiability
w.r.t.~\Rmc.  By Lemma~\ref{lem:bmc-wmc}, this amounts to checking the Boolean
knowledge base~$\Bmc_\Wmc$ (as defined in Section~\ref{sec:sat-rigid}) for
consistency.  Since the size of~$\Bmc_\Wmc$ is exponential in the size of~$\phi$
and~\Rmc, we obtain by Corollary~\ref{cor:cons-boolean-shoq-kb} that this can be
done in time doubly exponential in the size of~$\phi$ and~\Rmc.  Overall, the
computation of~$\Cmf_{\phi,\Rmc}$ requires doubly exponentially many such tests,
each requiring doubly exponential time.  This shows that~$\Cmf_{\phi,\Rmc}$, and
thus also the Büchi-automaton~$\Nmc_{\phi,\Rmc}^\text{r}$ can be constructed in
doubly exponential time.

Since the emptiness problem for Büchi-automata can be solved in time polynomial
in the size of the Büchi-automaton~\cite{VaWo-IC94}, this yields an alternative
proof of the fact that the satisfiability problem in \SHOQ-LTL is in \TwoExpTime
(see Theorem~\ref{thm:upper-bound-shoq-ltl-rigid-roles}).


\section{Monitoring \texorpdfstring{\SHOQ-LTL}{SHOQ-LTL}-Formulas}\label{sec:monitor}

In this section, we extend existing definitions and results for runtime
verification from propositional LTL to \SHOQ-LTL\@.  We restrict the attention
to the case with rigid names since the complexity of the monitor construction
for this more general case is actually the same (doubly exponential) as for the
case without rigid names.  Thus, it does not make sense to treat the restricted
case separately.  In addition to considering a more expressive logic, our notion
of monitoring extends the one for propositional logic in two directions.

On the one hand, we do not assume that the monitor has complete knowledge about
the states of the system.  In the propositional case, as introduced in
Section~\ref{sec:monitor-ltl}, at each point in time the monitor \enquote{knows}
which of the propositional variables are true at this point and which are not.
In our setting, \SHOQ-axioms take the place of propositional variables, but we
do not assume that we have complete knowledge about their truth value.  For some
of the relevant axioms, we may know that they are true, for others that they are
false, but it also may be the case that we have no information regarding the
truth status of a certain axiom.

On the other hand, we take background knowledge about the working of the system
into account.  This background knowledge could, for example, be a global TBox,
i.e.~a finite set of GCIs that are known to hold for every state of the system.
In this case, the formula describing the background knowledge is of the form
$\psi=\Box\bigwedge\Tmc$, where \Tmc is a TBox.  The presence of background
knowledge enables the monitor to give more often definite answers (i.e.~$\top$
or~$\bot$) rather than the answer~${?}$.


\subsection{Basic Definitions}

In the following, we extend the notion of a monitoring function and a monitor,
as introduced in Section~\ref{sec:monitor-ltl} for propositional LTL-formulas,
to the case of \SHOQ-LTL-formulas~$\phi$ w.r.t.\ an RBox~\Rmc.  We assume that
the background knowledge is described by an additional \SHOQ-LTL-formula~$\psi$
w.r.t.~\Rmc, and that the monitor receives the information about the current
state of the system in the form of a Boolean knowledge base~\Omc (observations)
that provides (partial) information about the truth values of certain axioms
from a fixed finite set of axioms.  Without loss of generality, we can also
assume that the axioms occurring in~$\psi$ and~\Omc also occur in~$\phi$.  This
assumption is indeed without loss of generality since for every such
axiom~$\alpha$, which does not occur in~$\phi$, we can define
$\phi':=\phi\land(\alpha\lor\lnot\alpha)$.  Obviously, every model of~$\phi$ is
also a model of~$\phi'$, and vice versa.  We make this assumption throughout
this section without explicitly mentioning it.

To simplify the subsequent definitions, we introduce the following notation.  A
\emph{literal of~$\phi$} is either an axiom $\alpha\in\Ax(\phi)$
(\emph{positive} literal) or the negation $\lnot\alpha$ of an axiom
$\alpha\in \Ax(\phi)$ (\emph{negative} literal).

\begin{definition}[Partial axiom type]
    A finite conjunction $\Omc=L_1\land\dots\land L_m$ of literals of~$\phi$ is
    a \emph{partial axiom type for~$\phi$ w.r.t.~\Rmc} if the Boolean knowledge
    base $(\Omc,\Rmc)$ is consistent.
\end{definition}

\noindent
We denote the set of all partial axiom types for~$\phi$ w.r.t.~\Rmc
by~$\Pmf_{\phi,\Rmc}$.
%
Note that, up to equivalence, there are at most exponentially many (in the size
of~$\phi$ and~\Rmc) partial axiom types for~$\phi$ w.r.t.~\Rmc.

Given an axiom type~$T$ for~$\phi$ w.r.t.~\Rmc, the first component of the
corresponding Boolean knowledge base~$\Bmc_T=(\Omc_T,\Rmc)$, i.e.\
\[\Omc_T:=\bigwedge_{\alpha\in T}\alpha\land\bigwedge_{\alpha\in\Ax(\phi)\setminus T}\lnot\alpha,\]
is a partial axiom type for~$\phi$ w.r.t.~\Rmc.  In this case, every axiom
of~$\phi$ occurs either positively or negatively in~$\Omc_T$.  For an arbitrary
partial axiom type~\Omc for~$\phi$ w.r.t.~\Rmc, this need not be the case.  Some
axioms of~$\phi$ may not occur at all in~\Omc.

\begin{definition}[Extensions]\label{def:extensions}
    Let \Rmc be an RBox, $\phi$ and $\psi$ be \SHOQ-LTL-formulas w.r.t.~\Rmc,
    and $\Omf=\Omc_0\Omc_1\dots\Omc_t$ be a \emph{finite} sequence of partial
    axioms types for~$\phi$ and~\Rmc.
    \begin{enumerate}
        \item We say that the DL-LTL-structure $\Imf=(\Imc_i)_{i\ge 0}$
            \emph{extends \Omf w.r.t.~$\psi$ and~\Rmc} if \Imf is a model
            of~$\psi$ w.r.t.~\Rmc, and $\Imc_i\models\Omc_i$ for every~$i$,
            $0\le i\le t$.
        \item\label{def:modelsex}
            We write $\Omf,\psi,\Rmc\modelsex\phi$ if there is a
            DL-LTL-structure~\Imf that extends~\Omf w.r.t.~$\psi$ and~\Rmc, and
            is a model of~$\phi$ w.r.t.~\Rmc.  If this is not the case, we write
            $\Omf,\psi,\Rmc\notmodelsex\phi$.
        \item\label{def:modelsal}
            We write $\Omf,\psi,\Rmc\modelsal\phi$ if every
            DL-LTL-structure~\Imf extending~\Omf w.r.t.~$\psi$ and~\Rmc is a
            model of~$\phi$ w.r.t.~\Rmc.  If this is not the case, we write
            $\Omf,\psi,\Rmc\notmodelsal\phi$.
    \end{enumerate}
\end{definition}

\noindent
The notions introduced in Part~\ref{def:modelsex} and Part~\ref{def:modelsal} of
this definition are dual to each other in the following sense:
\[\Omf,\psi,\Rmc\modelsex\phi\ \text{iff}\ \Omf,\psi,\Rmc\notmodelsal\lnot\phi%
    \qquad\text{and}\qquad%
    \Omf,\psi,\Rmc\modelsal\phi\ \text{iff}\ \Omf,\psi,\Rmc\notmodelsex\lnot\phi.\]

We assume that our system actually respects rigid names and satisfies the
background knowledge~$\psi$ and the RBox~\Rmc in the sense that any run of the
system corresponds to a DL-LTL-structure that is a model of~$\psi$ w.r.t.~\Rmc.
Thus, if the monitor receives information about the partial axiom types of a
finite prefix of such a run, this finite sequence of partial axiom types can
actually be extended to a DL-LTL-structure satisfying~$\psi$ w.r.t.~\Rmc.  In
this case, the following lemma holds.

\begin{lemma}\label{lem:modelsal}
    Let \Rmc be an RBox, $\phi$ and $\psi$ be \SHOQ-LTL-formulas w.r.t.~\Rmc,
    and \Omf be a finite sequence of partial axiom types for~$\phi$ w.r.t.~\Rmc
    such that there is a DL-LTL-structure extending \Omf w.r.t.~$\psi$ and~\Rmc.
    Then $\Omf,\psi,\Rmc\modelsal\phi$ and $\Omf,\psi,\Rmc\modelsal\lnot\phi$
    cannot both be true.
\end{lemma}

\begin{proof}
    Let \Imf be a DL-LTL-structure extending \Omf w.r.t.~$\psi$ and~\Rmc.  Then
    we have $\Imf,0\models\phi$ or $\Imf,0\models\lnot\phi$.  In the first case,
    we have $\Omf,\psi,\Rmc\notmodelsal\lnot\phi$, and in the second one, we
    have $\Omf,\psi,\Rmc\notmodelsal\phi$.
\end{proof}

\noindent
The monitoring function receives as input a finite sequence of partial axiom
types for~$\phi$ w.r.t.~\Rmc, i.e.~a finite word over the
alphabet~$\Pmf_{\phi,\Rmc}$.

\begin{definition}[Monitoring function]\label{def:monitoring-function}
    Let \Rmc be an RBox, and let $\phi$ and $\psi$ be \SHOQ-LTL-formulas
    w.r.t.~\Rmc.  The \emph{monitoring function for~$\phi$ w.r.t.~$\psi$
    and~\Rmc} is defined to be the function
    $\msf_{\phi,\psi,\Rmc}\colon\Pmf_{\phi,\Rmc}^*\to\{\top,\bot,{?},\lightning\}$
    with
    \[\msf_{\phi,\psi,\Rmc}(\Omf):=\begin{cases}
        \top       &\text{if}\ \Omf,\psi,\Rmc\modelsal\phi\ \text{and}\ \Omf,\psi,\Rmc\notmodelsal\lnot\phi;\\
        \bot       &\text{if}\ \Omf,\psi,\Rmc\notmodelsal\phi\ \text{and}\ \Omf,\psi,\Rmc\modelsal\lnot\phi;\\
        {?}        &\text{if}\ \Omf,\psi,\Rmc\notmodelsal\phi\ \text{and}\ \Omf,\psi,\Rmc\notmodelsal\lnot\phi;\ \text{and}\\
        \lightning &\text{if}\ \Omf,\psi,\Rmc\modelsal\phi\ \text{and}\ \Omf,\psi,\Rmc\modelsal\lnot\phi.
    \end{cases}\]
\end{definition}

\noindent
Compared to the definition of the monitoring function in the propositional
setting, we have added the fourth possible output~$\lightning$ in order to have
a well-defined value also for sequences $\Omf\in\Pmf_{\phi,\Rmc}^*$ that have no
extension w.r.t.~$\psi$ and~\Rmc.  In fact, if there is no DL-LTL-structure
extending~\Omf w.r.t.~$\psi$ and~\Rmc, the monitoring function yields the
value~$\lightning$.  In practice, this value should not be encountered since we
assume that the observed system actually respects rigid names and satisfies the
background knowledge~$\psi$ and~\Rmc.  Thus, no finite sequence of partial axiom
types obtained by observing the system can yields this case.\footnote{%
    If it does, then the modelling of the properties of the system
    using~$\psi$, \Rmc, and the rigididy of symbols was incorrect, or the
    sensors that generated the sequence~\Omf were faulty.}
%
The monitoring function returns the value~$\top$ if there is at least one
extension of~\Omf w.r.t.~$\psi$ and~\Rmc (expressed by
$\Omf,\psi,\Rmc\notmodelsal\lnot\phi$), and all such extensions satisfy~$\phi$
(expressed by $\Omf,\psi,\Rmc\modelsal\phi$).  Similarly, it returns the
value~$\bot$ if there is at least one extension of~\Omf w.r.t.~$\psi$ and~\Rmc,
and all such extensions satisfy~$\lnot\phi$.  Finally, it returns he value~${?}$
if there is an extension of~\Omf w.r.t.~$\psi$ and~\Rmc that satisfies~$\phi$,
and there is another extension of~\Omf w.r.t.~$\psi$ and~\Rmc that
satisfies~$\lnot\phi$.

We are interested in constructing a monitor that realises the monitoring
function defined above.  As in the propositional case, this monitor is a
deterministic Moore-automaton whose output function is equal to the monitoring
function.

\begin{definition}[Monitor for \SHOQ-LTL-formula]\label{def:monitor-shoq-ltl}
    Let \Rmc be an RBox, and $\phi,\psi$ be \SHOQ-LTL-formulas w.r.t.~\Rmc.  The
    deterministic Moore-automaton
    $\Mmc=(S,\Pmf_{\phi,\Rmc},\delta,s_0,\{\top,\bot,{?},\lightning\},\lambda)$
    is a \emph{monitor for~$\phi$ w.r.t.~$\psi$ and~\Rmc} if
    $\lambda^*(\Omf)=\msf_{\phi,\psi,\Rmc}(\Omf)$ holds for every
    $\Omf\in\Pmf_{\phi,\Rmc}^*$.
\end{definition}

\noindent
Before we construct the monitor, we need an auxiliary automaton that we define
next.


\subsection{An Auxiliary Deterministic Finite Automaton}\label{sec:aux-dfa}

In this section, we define a \emph{deterministic} finite automaton that accepts
exactly those sequences of partial axiom types $\Omf\in\Pmf_{\phi,\Rmc}^*$ such
that $\Omf,\psi,\Rmc\modelsal\phi$.  We know that requiring
$\Omf,\psi,\Rmc\modelsal\phi$ is the same as requiring
$\Omf,\psi,\Rmc\notmodelsex\lnot\phi$.  Thus, the automaton needs to accept all
words $\Omf\in\Pmf_{\phi,\Rmc}^*$ that have no extension w.r.t.~$\psi$ and~\Rmc
that satisfies~$\lnot\phi$.
%
To construct this automaton, we take the Büchi-automaton
$\Nmc_{\lnot\phi\land\psi,\Rmc}^\text{r}$ for $\lnot\phi\land\psi$ w.r.t.~\Rmc
as defined in Theorem~\ref{thm:ba-for-shoq-ltl-rigid}, and make it deterministic
by applying an appropriate modification of the power-set construction to the
first components of the states of $\Nmc_{\lnot\phi\land\psi,\Rmc}^\text{r}$.
The second component of a state of $\Nmc_{\lnot\phi\land\psi,\Rmc}^\text{r}$
collects the axiom types encountered on the path leading to this state, which
enables the automaton to check whether this collection of axiom types is
r-consistent.  Instead, our deterministic automaton collects the partial axiom
types encountered on a path, and checks whether this set is related in an
appropriate way to an r-consistent set of axiom types.

Before we can define this relation, we need to introduce some notation.  Given a
partial axiom type $\Omc=L_1\land\dots\land L_m$, we define
$\Pos(\Omc):=\{L_i\mid 1\le i\le m,\ L_i\ \text{is positive}\}$
and
$\Neg(\Omc):=\{\alpha_i\mid 1\le i\le m,\ L_i=\lnot\alpha_i\ \text{is negative}\}$.
%
Given an axiom type~$T$ for~$\phi$ w.r.t.~\Rmc and a partial axiom type~\Omc
for~$\phi$ w.r.t.~\Rmc, we define
\[\Omc<^{\phi,\Rmc}T\quad\text{iff}\quad%
    \Pos(\Omc)\subseteq T\ \text{and}\ \Neg(\Omc)\cap T=\emptyset.\]

\noindent
If $T=\tau_\phi(\Imc)$ for a model~\Imc of~\Rmc, then we obviously have
$\Imc\models(\Omc,\Rmc)$ iff $\Omc<^{\phi,\Rmc}T$.  We now lift the
relation~$<^{\phi,\Rmc}$ from (partial) axiom types to sets of (partial) axiom
types.

\begin{definition}[Realisation]
    Let \Tmf be a set of axiom types for~$\phi$ w.r.t.~\Rmc, and let \Pmf be a
    set of partial axiom types for~$\phi$ w.r.t.~\Rmc.
    %
    We say that \emph{\Tmf realises~\Pmf} and write $\Pmf\prec^{\phi,\Rmc}\Tmf$
    if the following property is satisfied: for every $\Omc\in\Pmf$, there is a
    $T\in\Tmf$ such that $\Omc<^{\phi,\Rmc}T$.
\end{definition}

\noindent
This relation can be used to characterise r-consistency of a set of partial
axiom types for~$\phi$ w.r.t.~\Rmc.

\begin{definition}[R-consistency of partial axiom types]
    Let $\Pmf=\{\Omc_1,\dots,\Omc_k\}$ be a set of partial axiom types
    for~$\phi$ w.r.t.~\Rmc.  We call \Pmf \emph{r-consistent} if there exist
    interpretations $\Imc_1=(\Delta,\cdot^{\Imc_1})$,~\dots,
    $\Imc_k=(\Delta,\cdot^{\Imc_k})$ such that
    \begin{itemize}
        \item $a^{\Imc_i}=a^{\Imc_j}$ holds for every $a\in\NI$ and all $i,j$,
            $1\le i<j\le k$;
        \item $A^{\Imc_i}=A^{\Imc_j}$ holds for every $A\in\NRC$ and all $i,j$,
            $1\le i<j\le k$;
        \item $r^{\Imc_i}=r^{\Imc_j}$ holds for every $r\in\NRR$ and all $i,j$,
            $1\le i<j\le k$; and
        \item $\Imc_i\models(\Omc_i,\Rmc)$ holds for every~$i$, $1\le i\le k$.
    \end{itemize}
\end{definition}

\noindent
We denote the set of all r-consistent sets of partial axiom types for~$\phi$
w.r.t.~\Rmc with~$\Cmf_{\phi,\Rmc}^\text{p}$.

\begin{lemma}\label{lem:r-cons-partial}
    The set \Pmf of partial axiom types for~$\phi$ w.r.t.~\Rmc is r-consistent
    iff there is an r-consistent set~\Tmf of axiom types for~$\phi$ w.r.t.~\Rmc
    such that $\Pmf\prec^{\phi,\Rmc}\Tmf$.
\end{lemma}

\begin{proof}
    For the \enquote{only if} direction, assume that
    $\Pmf=\{\Omc_1,\dots,\Omc_k\}$ is r-consistent.
    %
    Then there are interpretations $\Imc_1,\dots,\Imc_k$ that share the same
    domain, coincide on the individual names and the rigid concept and role
    names, and satisfy $\Imc_i\models(\Omc_i,\Rmc)$ for every~$i$,
    $1\le i\le k$.  If we define
    $\Tmf:=\bigl\{\tau_\phi(\Imc_1),\dots,\tau_\phi(\Imc_k)\bigr\}$, then this
    set of axiom types is obviously r-consistent, and we have
    $\Omc_i<^{\phi,\Rmc}\tau_\phi(\Imc_i)$ for every~$i$, $1\le i\le k$.  This
    shows $\Pmf\prec^{\phi,\Rmc}\Tmf$.

    Conversely, for the \enquote{if} direction, let
    $\Pmf=\{\Omc_1,\dots,\Omc_k\}$ and assume that $\Tmf=\{T_1,\dots,T_m\}$ is
    an r-consistent set of axiom types for~$\phi$ w.r.t.~\Rmc such that
    $\Pmf\prec^{\phi,\Rmc}\Tmf$.
    %
    Then there are interpretations $\Imc_1,\dots,\Imc_m$ that share the same
    domain, coincide on the individual names and the rigid concept and role
    names, and satisfy $\Imc_i\models\Rmc$ and $T_i=\tau_\phi(\Imc_i)$ for
    every~$i$, $1\le i\le m$.  In addition, for every $j$, $1\le j\le k$, there
    is an index $\nu_j$, $1\le\nu_j\le m$, such that
    $\Omc_j<^{\phi,\Rmc}T_{\nu_j}$.  The interpretations
    $\Imc_{\nu_1},\dots,\Imc_{\nu_k}$ share the same domain, coincide on the
    individual names and the rigid concept and role names, and satisfy
    $\Imc_{\nu_j}\models(\Omc_j,\Rmc)$ for every~$j$, $1\le j\le k$.  This shows
    that \Pmf is r-consistent.
\end{proof}

\noindent
We are now ready to define a deterministic finite automaton that accepts exactly
those sequences of partial axiom types $\Omf\in\Pmf_{\phi,\Rmc}^*$ such that
$\Omf,\psi,\Rmc\modelsal\phi$.  But first, for the sake of completeness, let us
recall the definition of a deterministic finite automaton.

\begin{definition}[Deterministic finite automaton]
    A \emph{deterministic finite automaton} is a tuple
    $\Dmc=(S,\Sigma,\delta,s_0,E)$ consisting of a finite set of states~$S$, a
    finite input alphabet~$\Sigma$, a transition function
    $\delta\colon S\times\Sigma\to S$, an initial state $s_0\in S$, and a set of
    final states $E\subseteq S$.

	The transition function can be extended to a function
	$\delta^*\colon S\times\Sigma^*\to S$ as follows:
	\begin{itemize}
        \item $\delta^*(s,\varepsilon):=s$ where $\varepsilon$ denotes the empty
            word; and
        \item $\delta^*(s,u\sigma):=\delta(\delta^*(s,u),\sigma)$ where
            $u\in\Sigma^*$ and $\sigma\in\Sigma$.
	\end{itemize}
    %
    The \emph{language $L(\Dmc)$ accepted by~\Dmc} is defined as as
    \[L(\Dmc):=\{u\in\Sigma^*\mid\delta^*(s_0,u)\in E\}.\]
\end{definition}

\noindent
As mentioned above, the deterministic finite automaton to be defined is based on
the Büchi-automaton $\Nmc_{\lnot\phi\land\psi,\Rmc}^\text{r}$ for the
\SHOQ-LTL-formula $\lnot\phi\land\psi$ w.r.t.~\Rmc as introduced in
Theorem~\ref{thm:ba-for-shoq-ltl-rigid}.  Recall that, according to our
assumption, all the axioms occurring in~$\psi$ already occur in~$\phi$.  Thus,
the alphabet of this Büchi-automaton is actually $\Tmf_{\phi,\Rmc}$ and the
second components of the states are r-consistent sets of axiom types for~$\phi$
w.r.t.~\Rmc, i.e.~we have
\[\Nmc_{\lnot\phi\land\psi,\Rmc}^\text{r}=%
    (Q\times\Cmf_{\phi,\Rmc},\Tmf_{\phi,\Rmc},\Delta,Q_0\times\{\emptyset\},F\times\Cmf_{\phi,\Rmc}).\]

\noindent
Given a state $(q,\Tmf)$ of $\Nmc_{\lnot\phi\land\psi,\Rmc}^\text{r}$, we denote
the Büchi-automaton obtained from this automaton by replacing the set of initial
states with $\{(q,\Tmf)\}$ by $\Nmc_{\lnot\phi\land\psi,\Rmc}^\text{r}(q,\Tmf)$.
%
The deterministic finite automaton
$\Dmc_{\phi,\psi,\Rmc}=(S,\Pmf_{\phi,\Rmc},\delta,s_0,E)$ is defined as follows:
\begin{itemize}
    \item $S:=2^Q\times\Cmf_{\phi,\Rmc}^\text{p}$;
    \item $s_0:=(Q_0,\emptyset)$;
    \item $\delta\colon S\times\Pmf_{\phi,\Rmc}\to S$ is defined as follows:
        \begin{itemize}
            \item if $\Pmf\cup\{\Omc\}\notin\Cmf_{\phi,\Rmc}^\text{p}$, then
                $\delta((P,\Pmf),\Omc):=(\emptyset,\emptyset)$;
            \item if $\Pmf\cup\{\Omc\}\in\Cmf_{\phi,\Rmc}^\text{p}$, then
                $\delta((P,\Pmf),\Omc):=(P',\Pmf\cup\{\Omc\})$ where
                \begin{align*}
                    P':=\bigcup_{q\in P}\bigl\{q'\in Q\mid{}%
                    &\text{there is}\ ((q,\Tmf),T,(q',\Tmf\cup\{T\}))\in\Delta\ \text{such that}\\[-2ex]
                    &\Omc<^{\phi,\Rmc}T,\ \Pmf\prec^{\phi,\Rmc}\Tmf,\ \text{and}\
                        L_\omega(\Nmc_{\lnot\phi\land\psi,\Rmc}^\text{r}(q',\Tmf\cup\{T\}))\ne\emptyset\bigr\};
                \end{align*}
        \end{itemize}
    \item $E:=\{\emptyset\}\times\Cmf_{\phi,\Rmc}^\text{p}$.
\end{itemize}

\noindent
Final states are those whose first component is the empty set.  Note that these
states reproduce themselves: states whose first component is the empty set have
only successor states for which this is again the case.  There are two possible
reasons for reaching such a state with letter~\Omc from a state~$(P,\Pmf)$ whose
first component~$P$ is non-empty.  Either the set $\Pmf\cup\{\Omc\}$ is not
r-consistent, or there are no states $q'\in Q$ satisfying the conditions in the
definition of~$P'$.

The following lemma states that this deterministic finite automaton behaves as
intended.

\begin{lemma}\label{lem:aux-dfa}
    For every finite sequence of partial axiom types
    $\Omf\in\Pmf_{\phi,\Rmc}^*$, we have $\Omf,\psi,\Rmc\modelsal\phi$ iff
    $\Omf\in L(\Dmc_{\phi,\psi,\Rmc})$.
\end{lemma}

\begin{proof}
    For the \enquote{if} direction, assume to the contrary that
    $\Omf=\Omc_0\Omc_1\dots\Omc_t\in L(\Dmc_{\phi,\psi,\Rmc})$ and
    $\Omf,\psi,\Rmc\notmodelsal\phi$.  Then we have
    $\Omf,\psi,\Rmc\modelsex\lnot\phi$, i.e.~there is a DL-LTL-structure
    $\Imf=(\Imc_i)_{i\ge 0}$ that extends~\Omf w.r.t.~$\psi$ and~\Rmc, and is a
    model of~$\lnot\phi$ w.r.t.~\Rmc.  This means that \Imf is a model
    of~$\lnot\phi\land\psi$ w.r.t.~\Rmc, and $\Imc_i\models\Omc_i$ for
    every~$i$, $0\le i\le t$.
    %
    Thus, $\tau_\phi(\Imf)\in L_\omega(\lnot\phi\land\psi,\Rmc)$, and since
    $\Nmc_{\lnot\phi\land\psi,\Rmc}^\text{r}$ is a Büchi-automaton for
    $\lnot\phi\land\psi$ w.r.t.~\Rmc, we have
    $\tau_\phi(\Imf)\in L_\omega(\Nmc_{\lnot\phi\land\psi,\Rmc}^\text{r})$.
    %
    This means that there is an accepting run $(q_0,\Tmf_0)(q_1,\Tmf_1)\dots$
    of~$\Nmc_{\lnot\phi\land\psi,\Rmc}^\text{r}$ on~$\tau_\phi(\Imf)$.  In
    particular, this yields
    $L_\omega(\Nmc_{\lnot\phi\land\psi,\Rmc}^\text{r}(q_i,\Tmf_i))\ne\emptyset$
    for every~$i\ge 0$.

    Moreover, we have by the construction
    of~$\Nmc_{\lnot\phi\land\psi,\Rmc}^\text{r}$ that
    $\Tmf_i=\{\tau_\phi(\Imc_j)\mid 0\le j<i\}$ for every $i\ge 0$.  We define
    $\Pmf_i:=\{\Omc_j\mid 0\le j<i\}$ for every~$i$, $0\le i\le t+1$.  Note that
    we have $\Omc_i<^{\phi,\Rmc}\tau_\phi(\Imc_i)$ for every~$i$,
    $0\le i\le t+1$.  Hence, $\Pmf_i\prec^{\phi,\Rmc}\Tmf_i$ holds for
    every~$i$, $0\le i\le t+1$.  By the definition
    of~$\Nmc_{\lnot\phi\land\psi,\Rmc}^\text{r}$, the sets~$\Tmf_i$ are
    r-consistent, and thus Lemma~\ref{lem:r-cons-partial} yields that $\Pmf_i$
    is r-consistent for every~$i$, $0\le i\le t+1$.  Thus, we have
    $\delta^*(s_0,\Omc_0\dots\Omc_i)=(P_{i+1},\Pmf_{i+1})$ with $q_{i+1}\in
    P_{i+1}$ for every~$i$, $0\le i\le t$.  In particular,
    $\delta^*(s_0,\Omf)=(P_{t+1},\Pmf_{t+1})$ with $q_{t+1}\in P_{t+1}$, which
    shows that $P_{t+1}\ne\emptyset$.  Consequently,
    $(P_{t+1},\Pmf_{t+1})\notin E$, which is a contradiction to the assumption
    that $\Omf\in L(\Dmc_{\phi,\psi,\Rmc})$.

    For the \enquote{only if} direction, assume to the contrary that
    $\Omf=\Omc_0\Omc_1\dots\Omc_t\notin L(\Dmc_{\phi,\psi,\Rmc})$ and
    $\Omf,\psi,\Rmc\modelsal\phi$, i.e.~every DL-LTL-structure
    $\Imf=(\Imc_i)_{i\ge 0}$ that extends~\Omf w.r.t.~$\psi$ and~\Rmc is a model
    of~$\phi$ w.r.t.~\Rmc.

    The first assumption implies that
    $\delta^*(s_0,\Omf)\notin E$,
    i.e.~$\delta^*(s_0,\Omf)=(P_{t+1},\Pmf_{t+1})\in 2^Q\times\Cmf_{\phi,\Rmc}^\text{p}$
    with $P_{t+1}\ne\emptyset$.
    %
    This yields intermediate states
    $(P_i,\Pmf_i)\in 2^Q\times\Cmf_{\phi,\Rmc}^\text{p}$, $0\le i\le t$, such
    that $P_0=Q_0$, $\Pmf_0=\emptyset$, and
    $\delta^*(s_0,\Omc_0\dots\Omc_i)=(P_{i+1},\Pmf_{i+1})\in 2^Q\times\Cmf_{\phi,\Rmc}^\text{p}$
    with $\Pmf_{i+1}=\Pmf_i\cup\{\Omc_i\}$ and $P_{i+1}\ne\emptyset$ for
    every~$i$, $0\le i\le t$.
    %
    Moreover, we have that there are for every~$i$, $0\le i\le t$, a state
    $q_i\in P_i$, an axiom type $T_i\in\Tmf_{\phi,\Rmc}$, and an r-consistent set
    of axiom types $\Tmf_i\in\Cmf_{\phi,\Rmc}$ such that
    $((q_i,\Tmf_i),T_i,(q_{i+1},\Tmf_{i+1}))\in\Delta$,
    $\Tmf_{i+1}=\Tmf_i\cup\{T_i\}$, $\Omc_i<^{\phi,\Rmc}T_i$,
    $\Pmf_i\prec^{\phi,\Rmc}\Tmf_i$, and
    $L_\omega(\Nmc_{\lnot\phi\land\psi,\Rmc}^\text{r}(q_{i+1},\Tmf_{i+1}))\ne\emptyset$.
    Note that $q_0\in Q_0$ since $q_0\in P_0$ and $P_0=Q_0$.

    We define $\Tmf_i':=\{T_j\mid 0\le j<i\}$ for every~$i$, $0\le i\le t+1$.
    Obviously, we then have $\Tmf_i'\subseteq\Tmf_i$ for every~$i$,
    $0\le i\le t+1$.  Since every subset of an r-consistent set of axiom types
    is again r-consistent, this shows $\Tmf_i'\in\Cmf_{\phi,\Rmc}$ for
    every~$i$, $0\le i\le t+1$.  Moreover, since
    $\Pmf_i=\{\Omc_j\mid 0\le j<i\}$ for every~$i$, $0\le i\le t+1$, the fact
    that $\Omc_i<^\phi T_i$ for every~$i$, $0\le i\le t+1$, implies
    $\Pmf_i\prec^{\phi,\Rmc}\Tmf_i'$ for every~$i$, $0\le i\le t+1$.  In
    addition, we have
    $((q_i,\Tmf_i'),T_i,(q_{i+1},\Tmf_{i+1}'))\in\Delta$ for every~$i$,
    $0\le i\le t$.

    Since
    $L_\omega(\Nmc_{\lnot\phi\land\psi,\Rmc}^\text{r}(q_{t+1},\Tmf_{t+1}))\ne\emptyset$,
    there is a $\omega$-word $T\in\Tmf_{\phi,\Rmc}^\omega$ such that there is an
    accepting run
    of~$\Nmc_{\lnot\phi\land\psi,\Rmc}^\text{r}(q_{t+1},\Tmf_{t+1})$ on~$T$.
    %
    Using similar arguments as above, we can transform this run into an
    accepting run
    of~$\Nmc_{\lnot\phi\land\psi,\Rmc}^\text{r}(q_{t+1},\Tmf_{t+1}')$ on~$T$.
    Hence, we have that
    $T\in L_\omega(\Nmc_{\lnot\phi\land\psi,\Rmc}^\text{r}(q_{t+1},\Tmf_{t+1}'))$.
    Overall, we obtain that the $\omega$-word $T_0T_1\dots T_t\cdot T$ is in
    $L_\omega(\Nmc_{\lnot\phi\land\psi,\Rmc}^\text{r})$.  Since
    $\Nmc_{\lnot\phi\land\psi,\Rmc}^\text{r}$ is a Büchi-automaton
    for~$\lnot\phi\land\psi$ w.r.t.~\Rmc, this shows that there exists a
    DL-LTL-structure $\Imf=(\Imc_i)_{i\ge 0}$ such that
    $\tau_\phi(\Imf)=T_0T_1\dots T_t\cdot T$ and \Imf is a model of
    $\lnot\phi\land\psi$ w.r.t.~\Rmc.

    For every~$i$, $0\le i\le t$, we have
    $\Omc_i<^{\phi,\Rmc}\tau_\phi(\Imc_i)$ since $\tau_\phi(\Imc_i)=T_i$.  This
    yields $\Imc_i\models\Omc_i$ for every~$i$, $0\le i\le t$.  Since \Imf is a
    model of~$\psi$ w.r.t.~\Rmc, we obtain that \Imf extends~\Omf w.r.t.~$\psi$
    and~\Rmc.  Hence, there is a DL-LTL-structure, namely~\Imf, extending~\Omf
    w.r.t.~$\psi$ and~\Rmc that is a model of~$\lnot\phi$ w.r.t.~\Rmc, which
    contradicts our assumption that $\Omf,\psi,\Rmc\modelsal\phi$.
\end{proof}

\noindent
It remains to analyse the complexity of the construction
of the deterministic finite automaton~$\Dmc_{\phi,\psi,\Rmc}$.  The size
of~$\Dmc_{\phi,\psi,\Rmc}$ is doubly exponential in the size of~$\phi$, $\psi$,
and~\Rmc.  This is due to the fact that the size of~$Q$ may be exponential and
the fact that the set $\Cmf_{\phi,\Rmc}^\text{p}$ of all r-consistent partial
axiom types for~$\phi$ w.r.t.~\Rmc may contain doubly exponentially many
elements since these sets are subsets of the exponentially large
set~$\Pmf_{\phi,\Rmc}$ of all partial axiom types for~$\phi$ w.r.t.~\Rmc.  Each
element of~$\Cmf_{\phi,\Rmc}^\text{p}$ may be of exponential size.

Next, we show that $\Dmc_{\phi,\psi,\Rmc}$ can be constructed in doubly
exponential time.  In addition to constructing the
Büchi-automaton~$\Nmc_{\lnot\phi\land\psi,\Rmc}^\text{r}$, we must also compute
the set~$\Cmf_{\phi,\Rmc}^\text{p}$.  As shown in Section~\ref{sec:ba-rigid},
the Büchi-automaton~$\Nmc_{\lnot\phi\land\psi,\Rmc}^\text{r}$, and thus also the
set~$\Cmf_{\phi,\Rmc}$, can be constructed in time doubly exponential in the
size of~$\phi$, $\psi$, and~\Rmc.  To compute $\Cmf_{\phi,\Rmc}^\text{p}$, we
use Lemma~\ref{lem:r-cons-partial}, which yields
\[\Cmf_{\phi,\Rmc}^\text{p}=\bigl\{\Pmf\subseteq\Pmf_{\phi,\Rmc}\mid\Pmf\prec^{\phi,\Rmc}\Tmf\
    \text{for some}\ \Tmf\in\Cmf_{\phi,\Rmc}\bigr\}.\]
%
We consider all sets of partial axiom types for~$\phi$ w.r.t.~\Rmc.  There are
doubly exponentially many such sets, each of size at most exponential in the
size of~$\phi$ and~\Rmc.  For each such set $\Pmf=\{\Omc_1,\dots,\Omc_k\}$, we
need to check whether there is a set $\Tmf=\{T_1,\dots,T_m\}\in\Cmf_{\phi,\Rmc}$
such that $\Pmf\prec^{\phi,\Rmc}\Tmf$.  Since $\Cmf_{\phi,\Rmc}$ is of doubly
exponential size, there are at most doubly exponentially many such tests for
each~\Pmf.  The test $\Pmf\prec^{\phi,\Rmc}\Tmf$ itself amounts to checking for
each~$\Omc_i$, $1\le i\le k$, whether there is a~$T_j$, $1\le j\le m$, such that
$\Pos(\Omc_i)\subseteq T_j$ and $\Neg(\Omc_i)\cap T_j=\emptyset$, which can be
done in exponential time since both $k$ and~$m$ are at most exponential in the
size of~$\phi$ and~\Rmc.  Overall, we can thus compute
$\Cmf_{\phi,\Rmc}^\text{p}$ is doubly exponential time.
%
Using these arguments, the fact that $\Nmc_{\lnot\phi\land\psi,\Rmc}^\text{r}$
can be constructed in doubly exponential time, and the fact that the emptiness
problem for Büchi-automata can be solved in time polynomial in the size of the
Büchi-automaton~\cite{VaWo-IC94}, it is easy to see that the transition
function~$\delta$ and the set of final states~$E$ can be computed in doubly
exponential time.
%
Overall, we have shown that $\Dmc_{\phi,\psi,\Rmc}$ can be constructed in time
doubly exponential in the size of~$\phi$, $\psi$, and~\Rmc.


\subsection{The Monitor Construction}

Given the construction of the deterministic finite automaton of the previous
section, it is now a simple exercise to construct the monitor for~$\phi$
w.r.t.~$\psi$ and~\Rmc.  Such a monitor is obtained by first constructing the
auxiliary deterministic finite automata~$\Dmc_{\phi,\psi,\Rmc}$
and~$\Dmc_{\lnot\phi,\psi,\Rmc}$, and then building the product of these two
automata.  The output of the monitor is determined by the final states of the
auxiliary automata.

\begin{theorem}\label{thm:monitor-shoq-ltl}
    Let \Rmc be an RBox, and let $\phi$ and~$\psi$ be \SHOQ-LTL-formulas
    w.r.t.~\Rmc.  If $\Dmc_{\phi,\psi,\Rmc}=(S,\Pmf_{\phi,\Rmc},\delta,s_0,E)$
    and~$\Dmc_{\lnot\phi,\psi,\Rmc}=(S',\Pmf_{\phi,\Rmc},\delta',s_0',E')$ are
    the deterministic finite automata introduced in Section~\ref{sec:aux-dfa},
    then
    $\Mmc_{\phi,\psi,\Rmc}:=(S\times S',\Pmf_{\phi,\Rmc},\deltah,(s_0,s_0'),\{\top,\bot,{?},\lightning\},\lambda)$
    with $\deltah((s,s'),\Omc):=(\delta(s,\Omc),\delta'(s',\Omc))$ and
    \[\lambda((s,s')):=\begin{cases}
            \top       &\text{if}\ s\in E\ \text{and}\ s'\notin E';\\
            \bot       &\text{if}\ s\notin E\ \text{and}\ s'\in E';\\
            {?}        &\text{if}\ s\notin E\ \text{and}\ s'\notin E';\ \text{and}\\
            \lightning &\text{if}\ s\in E\ \text{and}\ s'\in E'.
        \end{cases}\]
    is a monitor for~$\phi$ w.r.t.~$\psi$ and~\Rmc.
\end{theorem}

\begin{proof}
    We have to prove that for every $\Omf\in\Pmf_{\phi,\Rmc}^*$, we have
    $\lambda^*(\Omf)=\msf_{\phi,\psi,\Rmc}(\Omf)$.  This is an immediate
    consequence of the definition of the monitoring function
    (Definition~\ref{def:monitoring-function}) and the following facts:
    \begin{itemize}
        \item
            $\deltah^*((s_0,s_0'),\Omf)=(\delta^*(s_0,\Omf),(\delta')^*(s_0',\Omf))$
            for every $\Omf\in\Pmf_{\phi,\Rmc}^*$;
        \item $\delta^*(s_0,\Omf)\in E$ iff $\Omf,\psi,\Rmc\modelsal\phi$ for
            every $\Omf\in\Pmf_{\phi,\Rmc}^*$ (by Lemma~\ref{lem:aux-dfa}); and
        \item $(\delta')^*(s_0',\Omf)\in E'$ iff
            $\Omf,\psi,\Rmc\modelsal\lnot\phi$ for every
            $\Omf\in\Pmf_{\phi,\Rmc}^*$ (by Lemma~\ref{lem:aux-dfa}).
    \end{itemize}
    %
    To show the theorem formally, take any $\Omf\in\Pmf_{\phi,\Rmc}^*$.  Using
    the above facts, we have:
    \begingroup
    \allowdisplaybreaks
    \begin{align*}
         \lambda^*(\Omf)=\top\quad
        &\text{iff}\quad
         \lambda(\deltah^*((s_0,s_0'),\Omf))=\top\\
        &\text{iff}\quad
         \deltah^*((s_0,s_0'),\Omf)=(s,s')\ \text{with}\ s\in E\ \text{and}\ s'\notin E'\\
        &\text{iff}\quad
         \delta^*(s_0,\Omf)\in E\ \text{and}\ (\delta')^*(s_0',\Omf)\notin E'\\
        &\text{iff}\quad
         \Omf\in L(\Dmc_{\phi,\psi,\Rmc})\ \text{and}\ \Omf\notin L(\Dmc_{\lnot\phi,\psi,\Rmc})\\
        &\text{iff}\quad
         \Omf,\psi,\Rmc\modelsal\phi\ \text{and}\ \Omf,\psi,\Rmc\notmodelsal\lnot\phi\\
        &\text{iff}\quad
         \msf_{\phi,\psi,\Rmc}(\Omf)=\top.\\
      \intertext{Moreover, we have:}
         \lambda^*(\Omf)=\bot\quad
        &\text{iff}\quad
         \lambda(\deltah^*((s_0,s_0'),\Omf))=\bot\\
        &\text{iff}\quad
         \deltah^*((s_0,s_0'),\Omf)=(s,s')\ \text{with}\ s\notin E\ \text{and}\ s'\in E'\\
        &\text{iff}\quad
         \delta^*(s_0,\Omf)\notin E\ \text{and}\ (\delta')^*(s_0',\Omf)\in E'\\
        &\text{iff}\quad
         \Omf\notin L(\Dmc_{\phi,\psi,\Rmc})\ \text{and}\ \Omf\in L(\Dmc_{\lnot\phi,\psi,\Rmc})\\
        &\text{iff}\quad
         \Omf,\psi,\Rmc\notmodelsal\phi\ \text{and}\ \Omf,\psi,\Rmc\modelsal\lnot\phi\\
        &\text{iff}\quad
         \msf_{\phi,\psi,\Rmc}(\Omf)=\bot,\\
      \intertext{and also:}
         \lambda^*(\Omf)={?}\quad
        &\text{iff}\quad
         \lambda(\deltah^*((s_0,s_0'),\Omf))={?}\\
        &\text{iff}\quad
         \deltah^*((s_0,s_0'),\Omf)=(s,s')\ \text{with}\ s\notin E\ \text{and}\ s'\notin E'\\
        &\text{iff}\quad
         \delta^*(s_0,\Omf)\notin E\ \text{and}\ (\delta')^*(s_0',\Omf)\notin E'\\
        &\text{iff}\quad
         \Omf\notin L(\Dmc_{\phi,\psi,\Rmc})\ \text{and}\ \Omf\notin L(\Dmc_{\lnot\phi,\psi,\Rmc})\\
        &\text{iff}\quad
         \Omf,\psi,\Rmc\notmodelsal\phi\ \text{and}\ \Omf,\psi,\Rmc\notmodelsal\lnot\phi\\
        &\text{iff}\quad
         \msf_{\phi,\psi,\Rmc}(\Omf)={?}.
    \end{align*}
    \endgroup

    \noindent
    This shows that $\Mmc_{\phi,\psi,\Rmc}$ is indeed a monitor for~$\phi$
    w.r.t.~$\psi$ and~\Rmc.
\end{proof}

\noindent
It remains to analyse the complexity of the construction.  As shown in
Section~\ref{sec:aux-dfa}, the size of the auxiliary deterministic finite
automata $\Dmc_{\phi,\psi,\Rmc}$ and~$\Dmc_{\lnot\phi,\psi,\Rmc}$ is doubly
exponential in the size of~$\phi$, $\psi$, and~\Rmc.  Furthermore, they can be
constructed in doubly exponential time.  Hence, the size
of~$\Mmc_{\phi,\psi,\Rmc}$ is also doubly exponential in the size of~$\phi$,
$\psi$, and~\Rmc, and it can be constructed in doubly exponential time.

This doubly exponential blow-up in the construction of the monitor cannot be
avoided, since Theorem~\ref{thm:monitor-size} yields that such a blow-up is
unavoidable even for propositional LTL\@.


\section{The Complexity of Deciding Liveness and Monitorability in \texorpdfstring{\SHOQ-LTL}{SHOQ-LTL}}\label{sec:liveness-monitorability}

In this section, we extend the definitions and results about liveness and
monitorability from propositional LTL to \SHOQ-LTL\@.
%
In Section~\ref{sec:liveness}, we consider the simpler-looking problem of
liveness and in Section~\ref{sec:monitorability}, we consider monitorability.


\subsection{Deciding Liveness}\label{sec:liveness}

First, we extend the notion of liveness from propositional LTL (see
Definition~\ref{def:liveness-ltl}) to the temporalised description logic
\SHOQ-LTL and the presence of background knowledge.

\begin{definition}[Liveness]\label{def:liveness-shoq-ltl}
    Let \Rmc be an RBox, and $\phi$ and $\psi$ be \SHOQ-LTL-formulas
    w.r.t.~\Rmc.  We say that $\phi$ \emph{expresses a liveness property
    w.r.t.~$\psi$ and~\Rmc} if for every finite sequence of partial axiom types
    $\Omf\in\Pmf_{\phi,\Rmc}^*$ that has an extension w.r.t.~$\psi$ and~\Rmc, we
    have $\Omf,\psi,\Rmc\modelsex\phi$.
\end{definition}

\noindent
Note that, in this definition, we restrict ourselves to the finite sequences of
partial axiom types that have an extension w.r.t.~$\psi$ and~\Rmc.  In fact,
these are the sequences that we expect to see in practice since we assume that
the system satisfies~$\psi$, \Rmc, and respects rigid names.

As in the propositional case, liveness of~$\phi$ w.r.t.~$\psi$ and~\Rmc can be
expressed using the monitoring function.

\begin{lemma}\label{lem:liveness-char}
    Let \Rmc be an RBox, and $\phi$ and $\psi$ be \SHOQ-LTL-formulas
    w.r.t.~\Rmc.  Then $\phi$ expresses a liveness property w.r.t.~$\psi$
    and~\Rmc iff $\msf_{\phi,\psi,\Rmc}(\Omf)\ne\bot$ for every
    $\Omf\in\Pmf_{\phi,\Rmc}^*$.
\end{lemma}

\begin{proof}
    For the \enquote{only if} direction, assume that $\phi$ expresses a liveness
    property w.r.t.~$\psi$ and~\Rmc, and consider $\Omf\in\Pmf_{\phi,\Rmc}^*$.
    If \Omf does not have an extension w.r.t.~$\psi$ and~\Rmc, then
    $\msf_{\phi,\psi,\Rmc}(\Omf)=\lightning\ne\bot$.  Otherwise, the fact that
    $\phi$ expresses a liveness property w.r.t.~$\psi$ and~\Rmc implies that
    $\Omf,\psi,\Rmc\modelsex\phi$.  Consequently, we have
    $\Omf,\psi,\Rmc\notmodelsal\lnot\phi$, which yields
    $\msf_{\phi,\psi,\Rmc}(\Omf)\ne\bot$.

    For the \enquote{if} direction, assume that
    $\msf_{\phi,\psi,\Rmc}(\Omf)\ne\bot$ for every $\Omf\in\Pmf_{\phi,\Rmc}^*$.
    Consider a finite sequence of partial axiom types
    $\Omf\in\Pmf_{\phi,\Rmc}^*$ that has an extension w.r.t.~$\psi$ and~\Rmc.
    The existence of this extension implies that
    $\msf_{\phi,\psi,\Rmc}(\Omf)\ne\lightning$.  Thus, we know that
    $\msf_{\phi,\psi,\Rmc}(\Omf)\in\{\top,{?}\}$.  In both cases,
    $\Omf,\psi,\Rmc\notmodelsal\lnot\phi$ holds, which is equivalent to
    $\Omf,\psi,\Rmc\modelsex\phi$.
\end{proof}

\noindent
Consequently, given a monitor for~$\phi$ w.r.t.~$\psi$ and~\Rmc, liveness
of~$\phi$ w.r.t.~$\psi$ and~\Rmc can be tested by checking reachability in the
monitor, which yields an upper bound of \TwoExpTime.
%
The lower bound can be obtained by a reduction of the \emph{un}satisfiability
problem in \ALC-LTL~\cite{BaGL-ToCL12}.

\begin{lemma}\label{lem:liveness-lower}
    The problem of deciding whether a \SHOQ-LTL-formula~$\phi$ expresses a
    liveness property w.r.t.~a \SHOQ-LTL-formula~$\psi$ and an RBox~\Rmc is as
    hard as the unsatisfiability problem in \ALC-LTL\@.
\end{lemma}

\begin{proof}
    Let $\psi$ be an \ALC-LTL-formula (and thus a \SHOQ-LTL-formula w.r.t.\ the
    empty RBox $\Rmc:=\emptyset$).  We prove that $\psi$ is unsatisfiable iff
    the \SHOQ-LTL-formula $\phi:=\false$ expresses a liveness property
    w.r.t.~$\psi$ and~\Rmc.

    In fact, if $\psi$ is unsatisfiable, then there is \emph{no} sequence
    $\Omf\in\Pmf_{\phi,\Rmc}^*$ such that \Omf has extension w.r.t.~$\psi$
    and~\Rmc.  Consequently, the condition in the definition of liveness
    quantifies over the empty set of sequences, and is thus trivially true.

    Conversely, if $\psi$ is satisfiable, then there is some
    $\Omf\in\Pmf_{\phi,\Rmc}^*$ (e.g.~the empty sequence) that has an extension
    w.r.t.~$\psi$ and~\Rmc.  But then $\Omf,\psi,\Rmc\notmodelsex\phi$ since
    $\phi$ is unsatisfiable.  Hence, $\phi$ does not express a liveness property
    w.r.t.~$\psi$ and~\Rmc.
\end{proof}

\noindent
Due to the complexity results for the satisfiability problem in \ALC-LTL
(see~\cite{BaGL-ToCL12}), we obtain from these lemmas the following theorem.

\begin{theorem}
    The problem of deciding whether a \SHOQ-LTL-formula~$\phi$ expresses a
    liveness property w.r.t.~a \SHOQ-LTL-formula~$\psi$ and an RBox~\Rmc is
    \begin{itemize}
        \item \ExpTime-hard and in \TwoExpTime if $\NRC=\NRR=\emptyset$;
        \item \coNExpTime-hard and in \TwoExpTime if $\NRC\ne\emptyset$ and
            $\NRR=\emptyset$; and
        \item \TwoExpTime-complete if $\NRC\ne\emptyset$ and $\NRR\ne\emptyset$.
    \end{itemize}
\end{theorem}

\begin{proof}
    Regarding the upper bounds, Lemma~\ref{lem:liveness-char} implies that
    $\phi$ expresses a liveness property w.r.t.~$\psi$ and~\Rmc iff in the
    monitor $\Mmc_{\phi,\psi,\Rmc}$ no state with output~$\bot$ is reachable
    from the initial state.  Since this monitor is of doubly exponential size
    and reachability can be decided in linear time in the size of the automaton,
    this yields the required upper bounds of \TwoExpTime.

    The lower bounds follow immediately from Lemma~\ref{lem:liveness-lower} and
    the complexity results for the satisfiability problem in
    \ALC-LTL~\cite{BaGL-ToCL12}.  Indeed, this problem is \ExpTime-complete if
    $\NRC=\NRR=\emptyset$, \NExpTime-complete if $\NRC\ne\emptyset$ and
    $\NRR=\emptyset$, and \TwoExpTime-complete if $\NRC\ne\emptyset$ and
    $\NRR\ne\emptyset$.  Since both \ExpTime and \TwoExpTime are closed under
    complement, we obtain the complexity lower bounds of our theorem.
\end{proof}

\noindent
Note that only in the case $\NRC\ne\emptyset$ and $\NRR\ne\emptyset$, we have a
tight complexity result.  Recall, however, that the exact complexity for this
problem is not even known for propositional LTL\@.

Also, our hardness proof (Lemma~\ref{lem:liveness-lower}) strongly depends on
the presence of background knowledge.  Without background knowledge (i.e.~in the
case where $\psi=\true$ and $\Rmc=\emptyset$), we can only show an
\ExpTime-hardness result by a reduction of the satisfiability problem in
\ALC-LTL (without rigid names).

\begin{theorem}
    The problem of deciding whether a \SHOQ-LTL-formula~$\phi$ expresses a
    liveness property w.r.t.~the \SHOQ-LTL-formula~\true and the empty RBox is
    \ExpTime-hard.
\end{theorem}

\begin{proof}
    Consider an \ALC-LTL-formula $\phi$ and assume $\NRC=\NRR=\emptyset$.  We
    prove that $\phi$ is satisfiable iff $\Diamond\phi$ expresses a liveness
    property w.r.t.~\true and the empty RBox $\Rmc:=\emptyset$.  Since the
    satisfiability problem in \ALC-LTL is \ExpTime-complete if
    $\NRC=\NRR=\emptyset$ (see~\cite{BaGL-ToCL12}), this shows \ExpTime-hardness
    of liveness w.r.t.~\true and the empty RBox.

    If $\phi$ is unsatisfiable, then obviously $\Diamond\phi$ is unsatisfiable
    as well, and thus no sequence of partial axiom types can be extended to a
    model of $\Diamond\phi$.  In addition, there is a sequence of partial axiom
    types (e.g.~the empty sequence) that can be extended to a model of~\true
    and~\Rmc.  Consequently, $\Diamond\phi$ does not express a liveness property
    w.r.t.~\true and~\Rmc.

    Conversely, assume that $\phi$ is satisfiable, and let
    $\Omf=\Omc_0\Omc_1\dots\Omc_{t-1}\in\Pmf_{\phi,\Rmc}^*$ be a sequence of
    partial axiom types.  Satisfiability of~$\phi$ yields a model
    $\Imf=(\Imc_i)_{i\ge 0}$ of~$\phi$.  In addition, since partial axiom types
    are by definition consistent, there are interpretations $\Imc'_i$,
    $0\le i\le t-1$, such that $\Imc'_i\models\Omc_i$.  It is easy to see that
    the DL-LTL-structure
    \[\Jmf:=(\Jmc_i)_{i\ge 0}\ \text{with}\ \Jmc_i=\Imc'_i,\ 0\le i\le t-1,\
        \text{and}\ \Jmc_{i+t}=\Imc_i,\ i\ge 0,\]
    is a model of $\Diamond\phi$ that extends~\Omf w.r.t.~\true and~\Rmc,
    i.e.~$\Omf,\true,\Rmc\modelsex\phi$.  This shows that $\Diamond\phi$
    expresses a liveness property w.r.t.~\true and~\Rmc.
\end{proof}

\noindent
Unfortunately, the proof of this theorem does not go through in the presence of
rigid names.  In fact, the DL-LTL-structure~\Jmf constructed there need not
respect rigid names.


\subsection{Deciding Monitorability}\label{sec:monitorability}

We first extend the notion of monitorability from propositional LTL (see
Definition~\ref{def:monitorability-ltl}) to the temporalised description logic
\SHOQ-LTL and the presence of background knowledge.

\begin{definition}[Monitorability]
    Let \Rmc be an RBox, let $\phi$ and $\psi$ be \SHOQ-LTL-formulas
    w.r.t.~\Rmc, and let $\Omf\in\Pmf_{\phi,\Rmc}^*$.  We say that $\phi$ is
    \emph{\Omf-monitorable w.r.t.~$\psi$ and~\Rmc} if there is a finite word
    $\Omf'\in\Pmf_{\phi,\Rmc}^*$ such that
    $\msf_{\phi,\psi,\Rmc}(\Omf\cdot\Omf')\in\{\top,\bot\}$.  Moreover, we call
    $\phi$ \emph{monitorable w.r.t.~$\psi$ and~\Rmc} if it is \Omf-monitorable
    for every finite sequence of partial axiom types $\Omf\in\Pmf_{\phi,\Rmc}^*$
    that has an extension w.r.t.~$\psi$ and~\Rmc.
\end{definition}

\noindent
Monitorability can thus be expressed using the monitoring function as follows:
$\phi$ is monitorable w.r.t.~$\psi$ and~\Rmc iff for every finite sequence of
partial axiom types $\Omf\in\Pmf_{\phi,\Rmc}^*$ with
$\msf_{\phi,\psi,\Rmc}(\Omf)\ne\lightning$, there exists a finite sequence of
partial axiom types $\Omf'\in\Pmf_{\phi,\Rmc}^*$ satisfying
$\msf_{\phi,\psi,\Rmc}(\Omf\cdot\Omf')\in\{\top,\bot\}$.
%
This can again be checked using reachability tests in the monitor.

\begin{lemma}\label{lem:monitorability-upper}
    The problem of deciding monitorability of a \SHOQ-LTL-formula~$\phi$ w.r.t.\
    a \SHOQ-LTL-formula~$\psi$ and an RBox~\Rmc is in \TwoExpTime.
\end{lemma}

\begin{proof}
    To decide monitorability of $\phi$ w.r.t.~$\psi$ and~\Rmc, we construct the
    monitor $\Mmc_{\phi,\psi,\Rmc}$.  In this monitor, we compute all the states
    with output different from~$\lightning$ that are reachable from the initial
    state.  For each of these states, we then check whether a state with
    output~$\top$ or~$\bot$ is reachable.  If this is the case, then $\phi$ is
    monitorable w.r.t.~$\psi$ and~\Rmc.  Otherwise, i.e.~if there is a state
    reachable from the initial state such that every state reachable from it has
    output~${?}$ or~$\lightning$, then $\phi$ is not monitorable w.r.t.~$\psi$
    and~\Rmc.

    Since the monitor can be constructed in doubly exponential time and each of
    the doubly exponentially many reachability tests requires at most doubly
    exponential time, this yields a \TwoExpTime procedure for deciding
    monitorability.
\end{proof}

\noindent
For the lower bound, we again reduce the \emph{un}satisfiability problem in
\ALC-LTL\@.  For monitorability, such a reduction is possible even for the case
without background knowledge.

\begin{lemma}\label{lem:monitorability-lower}
    The problem of deciding monitorability of a \SHOQ-LTL-formula~$\phi$
    w.r.t.~\true and the empty RBox is as hard as the unsatisfiability problem
    in \ALC-LTL\@.
\end{lemma}

\begin{proof}
    Note that the lower bounds of the satisfiability problem in \ALC-LTL hold
    also for \ALC-LTL-formulas without past operators~\cite{BaGL-ToCL12}.
    %
    Thus, let $\psi$ be an \ALC-LTL-formula without past operators.  We define
    the \SHOQ-LTL-formula~$\phi$ as $\phi:=\Diamond\psi\land\Box\Diamond A(a)$
    where the flexible concept name~$A$ and the individual name~$a$ do not occur
    in~$\psi$.  We prove that $\psi$ is unsatisfiable iff $\phi$ is monitorable
    w.r.t.~\true and the empty RBox $\Rmc:=\emptyset$.

    If $\psi$ is unsatisfiable, we have that
    $\phi\equiv\Diamond\false\land\Box\Diamond A(a)\equiv\false\land\Box\Diamond A(a)\equiv\false$,
    i.e.~$\phi$ is also unsatisfiable.  Take now any $\Omf\in\Pmf_{\phi,\psi}^*$
    that has an extension w.r.t.~\true and~\Rmc.  Since $\phi$ is unsatisfiable,
    we have $\Omf,\true,\Rmc\modelsal\lnot\phi$.  Thus, Lemma~\ref{lem:modelsal}
    yields $\Omf,\true,\Rmc\notmodelsal\phi$, which shows that
    $\msf_{\phi,\true,\Rmc}(\Omf)=\bot$.  Consequently, $\phi$ is
    \Omf-monitorable w.r.t.~\true and~\Rmc (take $\Omf'$ to be the empty word).
    Since \Omf was an arbitrary element of~$\Pmf_{\phi,\Rmc}^*$ that has an
    extension w.r.t.~\true and~\Rmc, this shows that $\phi$ is monitorable
    w.r.t.~\true and~\Rmc.

    Conversely, if $\psi$ is satisfiable, then there is a model
    $\Imf=(\Imc_i)_{i\ge 0}$ of~$\psi$.  We define
    \[\Omc_i:=\bigwedge_{\alpha\in\tau_\psi(\Imc_i)}\alpha\land\bigwedge_{\alpha\in\Ax(\psi)\setminus\tau_\psi(\Imc_i)}\lnot\alpha\]
    for every $i\ge 0$.  Obviously, $\Imc_i$ is a model of~$\Omc_i$ and the
    empty RBox~\Rmc, and thus $\Omc_i$ is a partial axiom type,
    i.e.~$\Omc_i\in\Pmf_{\phi,\Rmc}$ for every $i\ge 0$.  Since there only
    finitely many partial axiom types, there are finitely many partial axiom
    types $\Omc_1',\dots,\Omc_k'$ such that
    $\{\Omc_1',\dots,\Omc_k'\}=\{\Omc_i\mid i\ge 0\}$.  Then there is a
    surjective function $\nu\colon\Nbb\to\{1,\dots,k\}$ such that
    $\Omc_i=\Omc_{\nu(i)}'$.

    To show that $\phi$ is not monitorable w.r.t.~\true and~\Rmc, we consider
    the finite sequence of partial axiom types $\Omf:=\Omc_1'\dots\Omc_k'$.
    Since the function~$\nu$ is surjective, any partial axiom type~$\Omc_i'$ in
    this sequence has at least one of the interpretation~$\Imc_j$ as model, and
    since $\Imf=(\Imc_i)_{i\ge 0}$ is a DL-LTL-structure, these models of
    $\Omc_1',\dots,\Omc_k'$ share the same domain and coincide on the individual
    names and the rigid concept and role names.  Also, these models satisfy the
    empty RBox~\Rmc. Consequently, the set $\{\Omc_1',\dots,\Omc_k'\}$ is
    r-consistent.  Obviously, this implies that \Omf has an extension
    w.r.t.~\true and~\Rmc.

    To disprove monitorability of~$\phi$ w.r.t.~\true and~\Rmc, it is thus
    sufficient to show that $\phi$ is not \Omf-monitorable w.r.t.~\true
    and~\Rmc. For this purpose, we take any finite sequence
    $\Omf'=\Omc_1''\dots\Omc_m''\in\Pmf_{\phi,\Rmc}^*$ and show that
    $\msf_{\phi,\true,\Rmc}(\Omf\cdot\Omf')\notin\{\top,\bot\}$.

    If $\Omf\cdot\Omf'$ does not have an extension w.r.t.~$\true$ and~\Rmc
    (which can happen due to the presence of rigid names), then
    $\Omf\cdot\Omf',\true,\Rmc\modelsal\phi$ and
    $\Omf\cdot\Omf',\true,\Rmc\modelsal\lnot\phi$, and thus
    $\msf_{\phi,\true,\Rmc}(\Omf\cdot\Omf')=\lightning\notin\{\top,\bot\}$ as
    required.

    Otherwise, let $\Jmf=(\Jmc_i)_{i\ge 0}$ be an extension of $\Omf\cdot\Omf'$
    w.r.t.~\true and~\Rmc.  We define a new DL-LTL-structure
    $\Jmf':=(\Jmc_i')_{i\ge 0}$ with
    \[\Jmc_i':=\begin{cases}
            \Jmc_i            &\text{if}\ 0\le i\le k+m-1;\ \text{and}\\
            \Jmc_{\nu(i-k-m)} &\text{otherwise.}
        \end{cases}\]
    %
    By definition, $\Jmf'$ consists of interpretations occurring in~\Jmf, and
    thus is indeed a DL-LTL-structure, i.e.~all interpretations occurring
    in~$\Jmf'$ share the same domain and coincide on the individual names and
    the rigid concept and role names.  Additionally, by definition $\Jmf'$
    coincides with~\Jmf on the first $k+m$ interpretations, which shows that it
    extends $\Omf\cdot\Omf'$ w.r.t.~\true and~\Rmc.  Moreover, since
    every~$\Omc_i'$, $1\le i\le k$, contains complete information about the
    axioms in~$\psi$, we have that
    \[\tau_\psi(\Imf)=\tau_\psi(\Jmc_{k+m}')\tau_\psi(\Jmc_{k+m+1}')\dots.\]
    By Lemma~\ref{lem:equal-type}, this shows that $(\Jmc_{k+m+i}')_{i\ge 0}$
    is a model of~$\psi$.  Since $\psi$ does not contain past operators, this
    implies that $\Jmf'$ is a model of~$\Diamond\psi$.
    %
    Since $\psi$ does not contain the concept name~$A$ and the individual
    name~$a$, this is independent on how~$A$ and~$a$ are interpreted.  In
    addition, since $A$ is flexible, changing its interpretation does not change
    the fact that rigid names are respected.

    Let now $\Jmf_A$ and $\Jmf_{\lnot A}$ be DL-LTL-structures such that:
    \begin{enumerate}
        \item $\Jmf_A$ and $\Jmf_{\lnot A}$ coincide for all points in time
            with~$\Jmf'$ on the interpretation domain as well as on the
            interpretation of all individual names, role names, and concept
            names different from~$A$.
        \item $\Jmf_A$ and $\Jmf_{\lnot A}$ coincide with~$\Jmf'$ for all points
            in time up to $k+m-1$ also on the interpretation of~$A$.
        \item In~$\Jmf_A$, the interpretation of~$A$ consists of the individual
            interpreting~$a$ at all points in time strictly after $k+m-1$.
        \item In~$\Jmf_{\lnot A}$, the interpretation of~$A$ is empty at all
            points in time strictly after $k+m-1$.
    \end{enumerate}
    %
    Obviously, both $\Jmf_A$ and $\Jmf_{\lnot A}$ are models of~$\Diamond\psi$
    and they extend $\Omf\cdot\Omf'$ w.r.t.~\true and~\Rmc.  However, only
    $\Jmf_A$ is also a model of $\Box\Diamond A(a)$.  Thus, $\Jmf_A$ is an
    extension of $\Omf\cdot\Omf'$ w.r.t.~\true and~\Rmc that satisfies~$\phi$,
    and $\Jmf_{\lnot A}$ is an extension of $\Omf\cdot\Omf'$ w.r.t.~\true
    and~\Rmc that satisfies~$\lnot\phi$.  This shows that we have
    $\Omf\cdot\Omf',\true,\Rmc\notmodelsal\phi$ and
    $\Omf\cdot\Omf',\true,\Rmc\notmodelsal\lnot\phi$.  Consequently,
    $\msf_{\phi,\true,\Rmc}(\Omf\cdot\Omf')\notin\{\top,\bot\}$, which finishes
    the proof that $\phi$ is not monitorable w.r.t.~\true and~\Rmc.
\end{proof}

\noindent
Putting the previous two lemmas together, we obtain the following theorem.

\begin{theorem}
    The problem of deciding monitorability of a \SHOQ-LTL-formula~$\phi$
    w.r.t.~a \SHOQ-LTL-formula~$\psi$ and an RBox~\Rmc is
    \begin{itemize}
        \item \ExpTime-hard and in \TwoExpTime if $\NRC=\NRR=\emptyset$;
        \item \coNExpTime-hard and in \TwoExpTime if $\NRC\ne\emptyset$ and
            $\NRR=\emptyset$; and
        \item \TwoExpTime-complete if $\NRC\ne\emptyset$ and $\NRR\ne\emptyset$.
    \end{itemize}
    %
    The lower bounds hold already for the special case where $\phi$ is an
    \ALC-LTL-formula, $\psi=\true$, and $\Rmc=\emptyset$.
\end{theorem}

\begin{proof}
    The upper bounds follow immediately from
    Lemma~\ref{lem:monitorability-upper}.
    The lower bounds follow immediately from
    Lemma~\ref{lem:monitorability-lower}, the fact that the
    \SHOQ-LTL-formula~$\phi$ constructed in the proof of this lemma is actually
    an \ALC-LTL-formula, and the complexity results for the satisfiability
    problem in \ALC-LTL~\cite{BaGL-ToCL12}.  Indeed, this problem is
    \ExpTime-complete if $\NRC=\NRR=\emptyset$, \NExpTime-complete if
    $\NRC\ne\emptyset$ and $\NRR=\emptyset$, and \TwoExpTime-complete if
    $\NRC\ne\emptyset$ and $\NRR\ne\emptyset$.  Since both \ExpTime and
    \TwoExpTime are closed under complement, we obtain the complexity lower
    bounds of our theorem.
\end{proof}

\noindent
Note again that only in the case $\NRC\ne\emptyset$ and $\NRR\ne\emptyset$, we
have a tight complexity result.  Recall, however, that the exact complexity for
this problem is not even known for propositional LTL\@.


\section{Summary}\label{sec:monitor-summary}

In this chapter, we have investigated runtime verification using the temporalised
description logic \SHOQ-LTL\@.  More precisely, we have shown how to construct
monitors for \SHOQ-LTL-formulas w.r.t.\ background knowledge that can deal with
incomplete knowledge in the form of partial \SHOQ-axiom types.
%
The complexity of the monitor construction is quite high.  We have seen that the
size of a monitor is doubly exponential in the size of the input.  However, this
cannot be avoided as this doubly exponential blow-up also occurs for
propositional LTL, which we have shown in Section~\ref{sec:monitor-ltl}.
%
It should be noted that the complexity of the monitor construction is a
worst-case complexity.  Minimisation of the intermediate Büchi-automata and the
auxiliary deterministic finite automata may lead to much smaller monitors than
the ones defined above.

Moreover, we have considered the decision problems of liveness and
monitorability.  For these problems, we have shown that they are as hard as
\emph{un}satisfiability in \ALC-LTL and in \TwoExpTime.  For liveness, the proof
of the lower bounds depends on the fact that background knowledge is available.
If this is not the case, we have shown that we obtain a lower bound of \ExpTime
for the liveness problem.  For the monitorability problem, we could prove the
lower bounds without using the background knowledge.  Overall, we have obtained
both liveness and monitorability are \ExpTime-hard if no rigid names are
available, \coNExpTime-hard if only rigid concept names are allowed, and
\TwoExpTime-complete if both concept names and role names may be rigid.
%
Unfortunately, this leaves gaps for the cases without rigid role names.
However, the precise complexity of those problems is unknown even for
propositional LTL\@.

Future work will include trying to close those gaps.  Note that since the
satisfiability problem in \ALC-LTL is harder than in propositional LTL, it may
be easier to come up with new complexity results in the case of \ALC-LTL and
\SHOQ-LTL\@.
%
Furthermore, an implementation of the monitor construction and a thorough
empirical evaluation of it is an important direction of future research.
