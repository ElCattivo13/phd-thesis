
\chapter*{Abstract}
In a modern world software systems are literally everywhere. These should cope with very
complex scenarios including the ability of context-awareness and self-adaptability. The concept of
roles provide the means to model such complex, context-dependent systems. In role-based systems, the
relational and context-dependent properties of objects are transferred into the roles that the
object plays in a certain context. However, even if the domain can be expressed in a well-structured
and modular way, role-based models can still be hard to comprehend due to the sophisticated
semantics of roles, contexts and different constraints. Hence, unintended implications or
inconsistencies may be overlooked. A feasible logical formalism is required here. In this setting
Description Logics (DLs) fit very well as a starting point for further considerations since as a
decidable fragment of first-order logic they have both an underlying formal semantics and decidable
reasoning problems. DLs are a well-understood family of knowledge representation formalisms which
allow to represent application domains in a well-structured way by DL-concepts, i.e. unary
predicates, and DL-roles, i.e. binary predicates. However, classical DLs lack expressive power to
formalise contextual knowledge which is crucial for formalising role-based systems.

We investigate a novel family of contextualised description logics that is capable of expressing
contextual knowledge and preserves decidability even in the presence of rigid DL-roles,
i.e. relational structures that are context-independent. For these contextualised description logics
we thoroughly analyse the complexity of the consistency problem.  Furthermore, we present a mapping
algorithm that allows for an automated translation from a formal role-based model, namely a
Compartment Role Object Model (CROM), into a contextualised DL ontology. We prove the semantical
correctness and provide ideas how features extending CROM can be expressed in our contextualised
DLs. As final step for a completely automated analysis of role-based models, we investigate a
practical reasoning algorithm and implement the first reasoner that can process contextual
ontologies.


%  LocalWords:  semantical DLs CROM ontologies DL unary

%%% Local Variables:
%%% mode: latex
%%% TeX-master: "thesis"
%%% End:
